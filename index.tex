% Options for packages loaded elsewhere
\PassOptionsToPackage{unicode}{hyperref}
\PassOptionsToPackage{hyphens}{url}
\PassOptionsToPackage{dvipsnames,svgnames,x11names}{xcolor}
%
\documentclass[
  letterpaper,
]{krantz}

\usepackage{amsmath,amssymb}
\usepackage{lmodern}
\usepackage{iftex}
\ifPDFTeX
  \usepackage[T1]{fontenc}
  \usepackage[utf8]{inputenc}
  \usepackage{textcomp} % provide euro and other symbols
\else % if luatex or xetex
  \usepackage{unicode-math}
  \defaultfontfeatures{Scale=MatchLowercase}
  \defaultfontfeatures[\rmfamily]{Ligatures=TeX,Scale=1}
\fi
% Use upquote if available, for straight quotes in verbatim environments
\IfFileExists{upquote.sty}{\usepackage{upquote}}{}
\IfFileExists{microtype.sty}{% use microtype if available
  \usepackage[]{microtype}
  \UseMicrotypeSet[protrusion]{basicmath} % disable protrusion for tt fonts
}{}
\makeatletter
\@ifundefined{KOMAClassName}{% if non-KOMA class
  \IfFileExists{parskip.sty}{%
    \usepackage{parskip}
  }{% else
    \setlength{\parindent}{0pt}
    \setlength{\parskip}{6pt plus 2pt minus 1pt}}
}{% if KOMA class
  \KOMAoptions{parskip=half}}
\makeatother
\usepackage{xcolor}
\setlength{\emergencystretch}{3em} % prevent overfull lines
\setcounter{secnumdepth}{5}
% Make \paragraph and \subparagraph free-standing
\ifx\paragraph\undefined\else
  \let\oldparagraph\paragraph
  \renewcommand{\paragraph}[1]{\oldparagraph{#1}\mbox{}}
\fi
\ifx\subparagraph\undefined\else
  \let\oldsubparagraph\subparagraph
  \renewcommand{\subparagraph}[1]{\oldsubparagraph{#1}\mbox{}}
\fi

\usepackage{color}
\usepackage{fancyvrb}
\newcommand{\VerbBar}{|}
\newcommand{\VERB}{\Verb[commandchars=\\\{\}]}
\DefineVerbatimEnvironment{Highlighting}{Verbatim}{commandchars=\\\{\}}
% Add ',fontsize=\small' for more characters per line
\usepackage{framed}
\definecolor{shadecolor}{RGB}{241,243,245}
\newenvironment{Shaded}{\begin{snugshade}}{\end{snugshade}}
\newcommand{\AlertTok}[1]{\textcolor[rgb]{0.68,0.00,0.00}{#1}}
\newcommand{\AnnotationTok}[1]{\textcolor[rgb]{0.37,0.37,0.37}{#1}}
\newcommand{\AttributeTok}[1]{\textcolor[rgb]{0.40,0.45,0.13}{#1}}
\newcommand{\BaseNTok}[1]{\textcolor[rgb]{0.68,0.00,0.00}{#1}}
\newcommand{\BuiltInTok}[1]{\textcolor[rgb]{0.00,0.23,0.31}{#1}}
\newcommand{\CharTok}[1]{\textcolor[rgb]{0.13,0.47,0.30}{#1}}
\newcommand{\CommentTok}[1]{\textcolor[rgb]{0.37,0.37,0.37}{#1}}
\newcommand{\CommentVarTok}[1]{\textcolor[rgb]{0.37,0.37,0.37}{\textit{#1}}}
\newcommand{\ConstantTok}[1]{\textcolor[rgb]{0.56,0.35,0.01}{#1}}
\newcommand{\ControlFlowTok}[1]{\textcolor[rgb]{0.00,0.23,0.31}{#1}}
\newcommand{\DataTypeTok}[1]{\textcolor[rgb]{0.68,0.00,0.00}{#1}}
\newcommand{\DecValTok}[1]{\textcolor[rgb]{0.68,0.00,0.00}{#1}}
\newcommand{\DocumentationTok}[1]{\textcolor[rgb]{0.37,0.37,0.37}{\textit{#1}}}
\newcommand{\ErrorTok}[1]{\textcolor[rgb]{0.68,0.00,0.00}{#1}}
\newcommand{\ExtensionTok}[1]{\textcolor[rgb]{0.00,0.23,0.31}{#1}}
\newcommand{\FloatTok}[1]{\textcolor[rgb]{0.68,0.00,0.00}{#1}}
\newcommand{\FunctionTok}[1]{\textcolor[rgb]{0.28,0.35,0.67}{#1}}
\newcommand{\ImportTok}[1]{\textcolor[rgb]{0.00,0.46,0.62}{#1}}
\newcommand{\InformationTok}[1]{\textcolor[rgb]{0.37,0.37,0.37}{#1}}
\newcommand{\KeywordTok}[1]{\textcolor[rgb]{0.00,0.23,0.31}{#1}}
\newcommand{\NormalTok}[1]{\textcolor[rgb]{0.00,0.23,0.31}{#1}}
\newcommand{\OperatorTok}[1]{\textcolor[rgb]{0.37,0.37,0.37}{#1}}
\newcommand{\OtherTok}[1]{\textcolor[rgb]{0.00,0.23,0.31}{#1}}
\newcommand{\PreprocessorTok}[1]{\textcolor[rgb]{0.68,0.00,0.00}{#1}}
\newcommand{\RegionMarkerTok}[1]{\textcolor[rgb]{0.00,0.23,0.31}{#1}}
\newcommand{\SpecialCharTok}[1]{\textcolor[rgb]{0.37,0.37,0.37}{#1}}
\newcommand{\SpecialStringTok}[1]{\textcolor[rgb]{0.13,0.47,0.30}{#1}}
\newcommand{\StringTok}[1]{\textcolor[rgb]{0.13,0.47,0.30}{#1}}
\newcommand{\VariableTok}[1]{\textcolor[rgb]{0.07,0.07,0.07}{#1}}
\newcommand{\VerbatimStringTok}[1]{\textcolor[rgb]{0.13,0.47,0.30}{#1}}
\newcommand{\WarningTok}[1]{\textcolor[rgb]{0.37,0.37,0.37}{\textit{#1}}}

\providecommand{\tightlist}{%
  \setlength{\itemsep}{0pt}\setlength{\parskip}{0pt}}\usepackage{longtable,booktabs,array}
\usepackage{calc} % for calculating minipage widths
% Correct order of tables after \paragraph or \subparagraph
\usepackage{etoolbox}
\makeatletter
\patchcmd\longtable{\par}{\if@noskipsec\mbox{}\fi\par}{}{}
\makeatother
% Allow footnotes in longtable head/foot
\IfFileExists{footnotehyper.sty}{\usepackage{footnotehyper}}{\usepackage{footnote}}
\makesavenoteenv{longtable}
\usepackage{graphicx}
\makeatletter
\def\maxwidth{\ifdim\Gin@nat@width>\linewidth\linewidth\else\Gin@nat@width\fi}
\def\maxheight{\ifdim\Gin@nat@height>\textheight\textheight\else\Gin@nat@height\fi}
\makeatother
% Scale images if necessary, so that they will not overflow the page
% margins by default, and it is still possible to overwrite the defaults
% using explicit options in \includegraphics[width, height, ...]{}
\setkeys{Gin}{width=\maxwidth,height=\maxheight,keepaspectratio}
% Set default figure placement to htbp
\makeatletter
\def\fps@figure{htbp}
\makeatother
\newlength{\cslhangindent}
\setlength{\cslhangindent}{1.5em}
\newlength{\csllabelwidth}
\setlength{\csllabelwidth}{3em}
\newlength{\cslentryspacingunit} % times entry-spacing
\setlength{\cslentryspacingunit}{\parskip}
\newenvironment{CSLReferences}[2] % #1 hanging-ident, #2 entry spacing
 {% don't indent paragraphs
  \setlength{\parindent}{0pt}
  % turn on hanging indent if param 1 is 1
  \ifodd #1
  \let\oldpar\par
  \def\par{\hangindent=\cslhangindent\oldpar}
  \fi
  % set entry spacing
  \setlength{\parskip}{#2\cslentryspacingunit}
 }%
 {}
\usepackage{calc}
\newcommand{\CSLBlock}[1]{#1\hfill\break}
\newcommand{\CSLLeftMargin}[1]{\parbox[t]{\csllabelwidth}{#1}}
\newcommand{\CSLRightInline}[1]{\parbox[t]{\linewidth - \csllabelwidth}{#1}\break}
\newcommand{\CSLIndent}[1]{\hspace{\cslhangindent}#1}

\usepackage{booktabs}
\usepackage{longtable}
\usepackage[bf,singlelinecheck=off]{caption}
\usepackage[scale=.77]{sourcecodepro}
\usepackage{hyperref}
\usepackage{emoji}

\usepackage{framed,color}
\definecolor{shadecolor}{RGB}{248,248,248}

\renewcommand{\textfraction}{0.05}
\renewcommand{\topfraction}{0.8}
\renewcommand{\bottomfraction}{0.8}
\renewcommand{\floatpagefraction}{0.75}

\renewenvironment{quote}{\begin{VF}}{\end{VF}}
\let\oldhref\href
\renewcommand{\href}[2]{#2\footnote{\url{#1}}}

\makeatletter
\newenvironment{kframe}{%
\medskip{}
\setlength{\fboxsep}{.8em}
 \def\at@end@of@kframe{}%
 \ifinner\ifhmode%
  \def\at@end@of@kframe{\end{minipage}}%
  \begin{minipage}{\columnwidth}%
 \fi\fi%
 \def\FrameCommand##1{\hskip\@totalleftmargin \hskip-\fboxsep
 \colorbox{shadecolor}{##1}\hskip-\fboxsep
     % There is no \\@totalrightmargin, so:
     \hskip-\linewidth \hskip-\@totalleftmargin \hskip\columnwidth}%
 \MakeFramed {\advance\hsize-\width
   \@totalleftmargin\z@ \linewidth\hsize
   \@setminipage}}%
 {\par\unskip\endMakeFramed%
 \at@end@of@kframe}
\makeatother

\renewenvironment{Shaded}{\begin{kframe}}{\end{kframe}}

\usepackage{makeidx}
\makeindex

\urlstyle{tt}

\usepackage{amsthm}
\makeatletter
\def\thm@space@setup{%
  \thm@preskip=8pt plus 2pt minus 4pt
  \thm@postskip=\thm@preskip
}
\makeatother

\frontmatter
\makeatletter
\makeatother
\makeatletter
\@ifpackageloaded{bookmark}{}{\usepackage{bookmark}}
\makeatother
\makeatletter
\@ifpackageloaded{caption}{}{\usepackage{caption}}
\AtBeginDocument{%
\ifdefined\contentsname
  \renewcommand*\contentsname{Tartalomjegyzék}
\else
  \newcommand\contentsname{Tartalomjegyzék}
\fi
\ifdefined\listfigurename
  \renewcommand*\listfigurename{Ábrák jegyzéke}
\else
  \newcommand\listfigurename{Ábrák jegyzéke}
\fi
\ifdefined\listtablename
  \renewcommand*\listtablename{Táblázatok jegyzéke}
\else
  \newcommand\listtablename{Táblázatok jegyzéke}
\fi
\ifdefined\figurename
  \renewcommand*\figurename{Ábra}
\else
  \newcommand\figurename{Ábra}
\fi
\ifdefined\tablename
  \renewcommand*\tablename{Táblázat}
\else
  \newcommand\tablename{Táblázat}
\fi
}
\@ifpackageloaded{float}{}{\usepackage{float}}
\floatstyle{ruled}
\@ifundefined{c@chapter}{\newfloat{codelisting}{h}{lop}}{\newfloat{codelisting}{h}{lop}[chapter]}
\floatname{codelisting}{Listing}
\newcommand*\listoflistings{\listof{codelisting}{List of Listings}}
\makeatother
\makeatletter
\@ifpackageloaded{caption}{}{\usepackage{caption}}
\@ifpackageloaded{subcaption}{}{\usepackage{subcaption}}
\makeatother
\makeatletter
\@ifpackageloaded{tcolorbox}{}{\usepackage[many]{tcolorbox}}
\makeatother
\makeatletter
\@ifundefined{shadecolor}{\definecolor{shadecolor}{rgb}{.97, .97, .97}}
\makeatother
\makeatletter
\makeatother
\ifLuaTeX
\usepackage[bidi=basic]{babel}
\else
\usepackage[bidi=default]{babel}
\fi
\babelprovide[main,import]{magyar}
% get rid of language-specific shorthands (see #6817):
\let\LanguageShortHands\languageshorthands
\def\languageshorthands#1{}
\ifLuaTeX
  \usepackage{selnolig}  % disable illegal ligatures
\fi
\IfFileExists{bookmark.sty}{\usepackage{bookmark}}{\usepackage{hyperref}}
\IfFileExists{xurl.sty}{\usepackage{xurl}}{} % add URL line breaks if available
\urlstyle{same} % disable monospaced font for URLs
\hypersetup{
  pdftitle={Többváltozós statisztika jamovi-ban és R-ben},
  pdfauthor={Abari Kálmán},
  pdflang={hu},
  colorlinks=true,
  linkcolor={blue},
  filecolor={Maroon},
  citecolor={Blue},
  urlcolor={Blue},
  pdfcreator={LaTeX via pandoc}}

\title{Többváltozós statisztika jamovi-ban és R-ben}
\author{Abari Kálmán}
\date{2023. 05. 07.}

\begin{document}
\maketitle
% you may need to leave a few empty pages before the dedication page

%\cleardoublepage\newpage\thispagestyle{empty}\null
%\cleardoublepage\newpage\thispagestyle{empty}\null
%\cleardoublepage\newpage
\thispagestyle{empty}

\begin{center}
A könyvet Máth János lektorálta, és Friss Kinga illusztrálta.
%\includegraphics{images/dedication.pdf}
\end{center}

\setlength{\abovedisplayskip}{-5pt}
\setlength{\abovedisplayshortskip}{-5pt}


\ifdefined\Shaded\renewenvironment{Shaded}{\begin{tcolorbox}[borderline west={3pt}{0pt}{shadecolor}, interior hidden, breakable, enhanced, frame hidden, boxrule=0pt, sharp corners]}{\end{tcolorbox}}\fi

\renewcommand*\contentsname{Tartalomjegyzék}
{
\hypersetup{linkcolor=}
\setcounter{tocdepth}{2}
\tableofcontents
}
\listoffigures
\listoftables
\bookmarksetup{startatroot}

\hypertarget{sec-eloszo}{%
\chapter*{Előszó}\label{sec-eloszo}}
\addcontentsline{toc}{chapter}{Előszó}

\markboth{Előszó}{Előszó}

A statisztika alapfogalmai nagyon jól szemléltethetők az egyváltozós
statisztikai eljárásokkal. Ezek az eljárások tipikusan egy (vagy két)
változó vizsgálatával járulnak hozzá az empirikus vizsgálatok során
felmerülő statisztikai jellegű kérdések megválaszolásához.

A kutatómunka során azonban szükség lehet egyszerre több változó
bevonására az elemzésbe, ezeket az eljárásokat többváltozós statisztikai
eljárásoknak nevezzük. Ilyen eljárás például:

\begin{itemize}
\tightlist
\item
  Lineáris regresszió (\ref{sec-linearis-regresszio})
\item
  Főkomponens elemzés (\ref{sec-fokomponens-elemzes})
\item
  Megbízhatóság elemzés (\ref{sec-megbizhatosag-elemzes})
\item
  Feltáró faktorelemzés (\ref{sec-feltaro-faktorelemzes})
\item
  Megerősítő faktorelemzés (\ref{sec-megerosito-faktorelemzes})
\item
  Többszempontos varianciaelemzés
  (\ref{sec-tobbszempontos-varianciaelemzes})
\item
  Klaszterelemzés (\ref{sec-klaszterelemzes})
\item
  Diszkriminancia elemzés (\ref{sec-diszkriminancia-elemzes})
\item
  Többváltozós varianciaelemzés
  (\ref{sec-tobbvaltozos-varianciaelemzes})
\item
  Logisztikus regresszióelemzés (\ref{sec-logisztikus-regresszio})
\item
  Többdimenziós skálázás (\ref{sec-tobbdimenzios-skalazas})
\end{itemize}

A jegyzet elkészítéséhez elsősorban a kurzus tankönyvét (Münnich és
mtsai., 2006) használtuk fel, de támaszkodtunk egyéb forrásokra is
(Csallner, 2015; Ketskeméty és Izsó, 2005; Malhotra és Simon, 2008;
Moksony, 2006; Sajtos és Mitev, 2007; Székelyi és Barna, 2002; Takács,
2017; Varga, 2019).

\mainmatter

\bookmarksetup{startatroot}

\hypertarget{sec-linearis-regresszio}{%
\chapter{Lineáris regresszió}\label{sec-linearis-regresszio}}

A korrelációszámítás két változó szimmetrikus kapcsolatának erősségét és
irányát vizsgálja csupán. Mivel az egyszerű lineáris regresszió két
változó függvényszerű kapcsolatát vizsgálja, ez már nem szimmetrikus
viszony, vagyis megkülönböztetjük a

\begin{itemize}
\tightlist
\item
  függő változót (célváltozót, \(Y\)-t), amely ``elszenvedi'' a
  független változó hatását, és a
\item
  független változót (magyarázó változót, \(X\)-et), amely befolyásolja
  a függő változót.
\end{itemize}

A többszörös lineáris regresszió annyiban tér el az egyszerű lineáris
regressziótól, hogy a független változók száma egynél több. Itt is
megkülönböztetjük a

\begin{itemize}
\tightlist
\item
  függő változót (célváltozót, \(Y\)-t), amelynek értékei a független
  változóktól függenek, és a
\item
  független változókat (magyarázó változókat, \(X_1,X_2,..,X_r\)-t)
  amelyek hatnak a függő változóra.
\end{itemize}

Két változó (\(X\) és \(Y\)) között nem feltétlenül van szisztematikus
kapcsolat, lehet a két változó független is egymástól. Ha van valamilyen
szisztematikus kapcsolat \(X\) és \(Y\) között, akkor az még számos
formában megvalósulhat, ezek egyike a lineáris kapcsolat,

\begin{itemize}
\tightlist
\item
  amely olyan függvényszerű kapcsolat, amely megmondja, hogy milyen
  mértékű változás várható az \(Y\) változóban, ha \(X\) adott
  mértéknyit változik.
\end{itemize}

\hypertarget{egyszerux171-lineuxe1ris-regressziuxf3}{%
\section{Egyszerű lineáris
regresszió}\label{egyszerux171-lineuxe1ris-regressziuxf3}}

Az egyszerű lineáris regressziós modell: \(Y=\beta_0+β_1 X+\epsilon\),
amely egy egyenessel (regressziós egyenes) írja le a két változó
függvényszerű kapcsolatát, ahol

\begin{itemize}
\tightlist
\item
  \(\beta_0\) -- tengelymeszet, a regressziós egyenes itt metszi az y
  tengelyt
\item
  \(\beta_1\) -- meredekség, a regressziós egyenes és az x tengely
  szögének tangense
\item
  \(\epsilon\) -- hibatag, amelyről feltételezzük, hogy normális
  eloszlású 0 várható értékkel.
\end{itemize}

A \(\beta_0\) és \(\beta_1\) populációbeli paramétereket a minta alapján
becsüljük a legkisebb négyzetek módszere segítségével, így kapjuk a
\(b_0\) és \(b_1\) becsléseket.

A regressziós egyenes birtokában tetszőleges \(X\) értékhez tudunk \(Y\)
értéket előre jelezni, vagyis jósolni bizonyos hibával:
\(\hat{Y}=b_0+b_1 X\).

Például egy fiktív adatbázison vizsgálhatjuk a fizetés és a munkahellyel
való elégedettség kapcsolatát (Münnich és mtsai., 2006).

\begin{Shaded}
\begin{Highlighting}[]
\NormalTok{d }\OtherTok{\textless{}{-}}\NormalTok{ rio}\SpecialCharTok{::}\FunctionTok{import}\NormalTok{(}\AttributeTok{file =} \StringTok{"adat/lin\_reg\_fizetes\_elegedettseg\_02.xlsx"}\NormalTok{)}
\FunctionTok{str}\NormalTok{(d)}
\CommentTok{\#\textgreater{} \textquotesingle{}data.frame\textquotesingle{}:    5 obs. of  2 variables:}
\CommentTok{\#\textgreater{}  $ fizetes     : num  44 66 89 155 130}
\CommentTok{\#\textgreater{}  $ elegedettseg: num  30 45 60 100 85}
\NormalTok{d}
\CommentTok{\#\textgreater{}   fizetes elegedettseg}
\CommentTok{\#\textgreater{} 1      44           30}
\CommentTok{\#\textgreater{} 2      66           45}
\CommentTok{\#\textgreater{} 3      89           60}
\CommentTok{\#\textgreater{} 4     155          100}
\CommentTok{\#\textgreater{} 5     130           85}
\end{Highlighting}
\end{Shaded}

\begin{Shaded}
\begin{Highlighting}[]
\NormalTok{lm\_1 }\OtherTok{\textless{}{-}} \FunctionTok{lm}\NormalTok{(elegedettseg }\SpecialCharTok{\textasciitilde{}}\NormalTok{ fizetes, }\AttributeTok{data =}\NormalTok{ d)}
\FunctionTok{summary}\NormalTok{(lm\_1)}
\CommentTok{\#\textgreater{} }
\CommentTok{\#\textgreater{} Call:}
\CommentTok{\#\textgreater{} lm(formula = elegedettseg \textasciitilde{} fizetes, data = d)}
\CommentTok{\#\textgreater{} }
\CommentTok{\#\textgreater{} Residuals:}
\CommentTok{\#\textgreater{}       1       2       3       4       5 }
\CommentTok{\#\textgreater{} {-}0.8423  0.3420  0.8983 {-}0.5488  0.1508 }
\CommentTok{\#\textgreater{} }
\CommentTok{\#\textgreater{} Coefficients:}
\CommentTok{\#\textgreater{}             Estimate Std. Error t value Pr(\textgreater{}|t|)    }
\CommentTok{\#\textgreater{} (Intercept) 3.210890   0.931791   3.446   0.0411 *  }
\CommentTok{\#\textgreater{} fizetes     0.627987   0.008873  70.774 6.22e{-}06 ***}
\CommentTok{\#\textgreater{} {-}{-}{-}}
\CommentTok{\#\textgreater{} Signif. codes:  }
\CommentTok{\#\textgreater{} 0 \textquotesingle{}***\textquotesingle{} 0.001 \textquotesingle{}**\textquotesingle{} 0.01 \textquotesingle{}*\textquotesingle{} 0.05 \textquotesingle{}.\textquotesingle{} 0.1 \textquotesingle{} \textquotesingle{} 1}
\CommentTok{\#\textgreater{} }
\CommentTok{\#\textgreater{} Residual standard error: 0.8077 on 3 degrees of fre...}
\CommentTok{\#\textgreater{} Multiple R{-}squared:  0.9994, Adjusted R{-}squared:  0...}
\CommentTok{\#\textgreater{} F{-}statistic:  5009 on 1 and 3 DF,  p{-}value: 6.216e{-}06}
\end{Highlighting}
\end{Shaded}

Jamovi-ban a \texttt{Regression\ /\ Linear\ Regression} menüpontot kell
használnunk.

\begin{figure}

{\centering \includegraphics{./images/lin_reg_fizetes_elegedettseg_02_kep.jpg}

}

\caption{Fizetés és elégedettésg kapcsolata (N=5): együtthatók}

\end{figure}

A fenti elemzés alapján például a \(\hat{Y}=b_0+b_1 X\) konkrét formája:

\begin{Shaded}
\begin{Highlighting}[]
\NormalTok{becsült elégedettség = 3,211+ 0,628 * fizetés}
\end{Highlighting}
\end{Shaded}

\begin{itemize}
\tightlist
\item
  A \(b_0\) értelmezése: a zérus \(X\)-hez tartozó \(Y\) érték.
\item
  A \(b_1\) értelmezése: az \(X\) egy egységnyi növekedéséhez ilyen
  nagyságú \(Y\) változás tartozik.
\end{itemize}

Tudjuk, hogy az \(r_{XY}\) Pearson-féle korrelációs együttható, az \(X\)
és \(Y\) változók közötti kapcsolat erősségét és irányát mutatja meg. A
\(b_1\) és \(r_{XY}\) kapcsolatban áll:

\begin{itemize}
\tightlist
\item
  azonos az előjelük,
\item
  az \(X\) egy szórásnyi növekedéséhez tartozó \(Y\) változás megegyezik
  az \(Y\) szórásának \(r_{XY}\) szeresével (rövidebben, a populációbeli
  paraméterekkel megfogalmazva:
  \(\beta_1=\frac{\sigma_Y}{\sigma_X}\rho_{XY}\)
\end{itemize}

A determinációs együttható (\(R^2\)) a korrelációs együttható négyzete
\((R^2=r_{XY}^2)\), amely szimmetrikus mutató, megmutatja, hogy \(Y\)
varianciájának mekkora hányadát magyarázza \(X\) varianciája, vagy
fordítva, \(X\) varianciájának mekkora hányadát magyarázza \(Y\)
varianciája.

A fenti példában látható, hogy 99\%-ban lehet a függő változó
varianciáját magyarázni a független változóval (az arányt legtöbbször
százalékos formában adjuk meg).

\begin{Shaded}
\begin{Highlighting}[]
\FunctionTok{summary}\NormalTok{(lm\_1)}\SpecialCharTok{$}\NormalTok{r.squared}
\CommentTok{\#\textgreater{} [1] 0.9994014}
\end{Highlighting}
\end{Shaded}

\begin{figure}

{\centering \includegraphics{./images/lin_reg_fizetes_elegedettseg_02_kep_02.jpg}

}

\caption{Fizetés és elégedettésg kapcsolata (N=5): determinációs
együttható}

\end{figure}

A \(\beta_0\) és \(\beta_1\) együtthatók értékét hipotézisvizsgálatokkal
vizsgálhatjuk:

\begin{itemize}
\tightlist
\item
  \(H_0\): \(\beta_0=0\), \(H_1: \beta_0 \neq 0\) Kérdés: origón átmenő
  a regresszió? (\(H_0\) megtartása esetén igen)
\item
  \(H_0\): \(\beta_1=0\), \(H_1: \beta_1 \neq 0\) Kérdés: \(Y\) függ
  \(X\)-től? (\(H_1\) elfogadása esetén igen)
\end{itemize}

A példában látható, hogy nem origón átmenő a regresszió, és az
elégedettség függ a fizetéstől.

\begin{Shaded}
\begin{Highlighting}[]
\FunctionTok{summary}\NormalTok{(lm\_1)}\SpecialCharTok{$}\NormalTok{coefficients}
\CommentTok{\#\textgreater{}              Estimate  Std. Error   t value     Pr(...}
\CommentTok{\#\textgreater{} (Intercept) 3.2108896 0.931790852  3.445934 4.10580...}
\CommentTok{\#\textgreater{} fizetes     0.6279867 0.008873152 70.773796 6.21643...}
\end{Highlighting}
\end{Shaded}

\begin{figure}

{\centering \includegraphics{./images/lin_reg_fizetes_elegedettseg_02_kep_02.jpg}

}

\caption{Fizetés és elégedettésg kapcsolata (N=5): hipotézisvizsgálat az
együtthatókra}

\end{figure}

\hypertarget{tuxf6bbszuxf6ruxf6s-lineuxe1ris-regressziuxf3}{%
\section{Többszörös lineáris
regresszió}\label{tuxf6bbszuxf6ruxf6s-lineuxe1ris-regressziuxf3}}

A többszörös lineáris regressziós modell:
\(Y=\beta_0+\beta_1 X_1+\beta_2 X_2+\dots + \beta_r X_r+\epsilon\).

Míg az egyszerű lineáris regresszió esetén a regressziós egyenes írta le
a két változó kapcsolatát, a többszörös lineáris regresszió esetén a
lineáris függvény egy \(r\) dimenziós sík az \(r+1\) dimenziós térben.

Az egyes \(\beta_i\) együtthatók becslése itt is a legkisebb négyzetek
elve alapján történik, így kapjuk a \(b_0, b_1, \dots, b_r\)
becsléseket.

A lineáris függvény birtokában tetszőleges \(X_1,X_2,\dots,X_r\)
értékekhez tudunk \(Y\) értéket előre jelezni, vagyis jósolni bizonyos
hibával: \(\hat{Y}=b_0+b_1 X_1+\dots+ b_r X_r\).

\begin{Shaded}
\begin{Highlighting}[]
\NormalTok{d }\OtherTok{\textless{}{-}}\NormalTok{ rio}\SpecialCharTok{::}\FunctionTok{import}\NormalTok{(}\AttributeTok{file =} \StringTok{"adat/lin\_reg\_fizetes\_eletkor\_elegedettseg\_01.xlsx"}\NormalTok{)}
\FunctionTok{str}\NormalTok{(d)}
\CommentTok{\#\textgreater{} \textquotesingle{}data.frame\textquotesingle{}:    5 obs. of  3 variables:}
\CommentTok{\#\textgreater{}  $ fizetes     : num  44 66 89 155 130}
\CommentTok{\#\textgreater{}  $ eletkor     : num  25 65 21 35 40}
\CommentTok{\#\textgreater{}  $ elegedettseg: num  37 36 61 92 76}
\NormalTok{d}
\CommentTok{\#\textgreater{}   fizetes eletkor elegedettseg}
\CommentTok{\#\textgreater{} 1      44      25           37}
\CommentTok{\#\textgreater{} 2      66      65           36}
\CommentTok{\#\textgreater{} 3      89      21           61}
\CommentTok{\#\textgreater{} 4     155      35           92}
\CommentTok{\#\textgreater{} 5     130      40           76}
\end{Highlighting}
\end{Shaded}

\begin{Shaded}
\begin{Highlighting}[]
\NormalTok{lm\_1 }\OtherTok{\textless{}{-}} \FunctionTok{lm}\NormalTok{(elegedettseg }\SpecialCharTok{\textasciitilde{}}\NormalTok{ fizetes }\SpecialCharTok{+}\NormalTok{ eletkor, }\AttributeTok{data =}\NormalTok{ d)}
\FunctionTok{summary}\NormalTok{(lm\_1)}
\CommentTok{\#\textgreater{} }
\CommentTok{\#\textgreater{} Call:}
\CommentTok{\#\textgreater{} lm(formula = elegedettseg \textasciitilde{} fizetes + eletkor, data...}
\CommentTok{\#\textgreater{} }
\CommentTok{\#\textgreater{} Residuals:}
\CommentTok{\#\textgreater{}        1        2        3        4        5 }
\CommentTok{\#\textgreater{}  0.28596  0.08556 {-}0.30015  0.71047 {-}0.78184 }
\CommentTok{\#\textgreater{} }
\CommentTok{\#\textgreater{} Coefficients:}
\CommentTok{\#\textgreater{}              Estimate Std. Error t value Pr(\textgreater{}|t|)    }
\CommentTok{\#\textgreater{} (Intercept) 21.508055   1.292166   16.64  0.00359 ** }
\CommentTok{\#\textgreater{} fizetes      0.519198   0.008847   58.69  0.00029 ***}
\CommentTok{\#\textgreater{} eletkor     {-}0.305549   0.023279  {-}13.12  0.00575 ** }
\CommentTok{\#\textgreater{} {-}{-}{-}}
\CommentTok{\#\textgreater{} Signif. codes:  }
\CommentTok{\#\textgreater{} 0 \textquotesingle{}***\textquotesingle{} 0.001 \textquotesingle{}**\textquotesingle{} 0.01 \textquotesingle{}*\textquotesingle{} 0.05 \textquotesingle{}.\textquotesingle{} 0.1 \textquotesingle{} \textquotesingle{} 1}
\CommentTok{\#\textgreater{} }
\CommentTok{\#\textgreater{} Residual standard error: 0.8047 on 2 degrees of fre...}
\CommentTok{\#\textgreater{} Multiple R{-}squared:  0.9995, Adjusted R{-}squared:  0...}
\CommentTok{\#\textgreater{} F{-}statistic:  1841 on 2 and 2 DF,  p{-}value: 0.000543}
\end{Highlighting}
\end{Shaded}

\begin{figure}

{\centering \includegraphics{./images/lin_reg_fizetes_eletkor_elegedettseg_01_kep_01.jpg}

}

\caption{Elégedettség kapcsolata a fizetéssel és az élekorral (N=5):
együtthatók}

\end{figure}

A fenti példában a lineáris regresszió futtatása után azt mondhatjuk:

\begin{Shaded}
\begin{Highlighting}[]
\NormalTok{becsült elégedettség = 21,05 {-}0,306*életkor + 0,519*fizetés}
\end{Highlighting}
\end{Shaded}

Más szavakkal a fizetés tekintetében a magasabb fizetés nagyobb mértékű
elégedettséggel jár, addig az életkor esetében az évek számának
növekedése a munkahellyel való elégedetlenséget vonja maga után.

\begin{itemize}
\tightlist
\item
  A \(b_0\) értelmezése: a csupa zérus \(X_1, X_2,\dots,X_r\)-ekhez
  tartozó \(Y\) érték.
\item
  A \(b_i\) \((i=1,\dots,r)\) értelmezése: az \(X_i\) hatása úgy, hogy a
  többi független változót is figyelembe vesszük.
\end{itemize}

A fenti többszörös lineáris regressziós együtthatók nem alkalmasak az
egyes magyarázó változóktól való függés erősségének mérésére, ugyanis a
nagyságuk függ a változó értékeinek nagyságától is. Ezért a standard
lineáris regressziós együtthatókat használjuk, amelyek már mértékegység
nélküli, egymással összehasonlítható arányszámok, így abszolút
értékeiket összevetve megtudhatjuk, milyen relatív fontossággal bírnak
az egyes független változók a függő változó magyarázásában.

\begin{Shaded}
\begin{Highlighting}[]
\NormalTok{lsr}\SpecialCharTok{::}\FunctionTok{standardCoefs}\NormalTok{(lm\_1)}
\CommentTok{\#\textgreater{}                  b       beta}
\CommentTok{\#\textgreater{} fizetes  0.5191980  0.9677518}
\CommentTok{\#\textgreater{} eletkor {-}0.3055489 {-}0.2164358}
\end{Highlighting}
\end{Shaded}

\begin{figure}

{\centering \includegraphics{./images/lin_reg_fizetes_eletkor_elegedettseg_01_kep_02.jpg}

}

\caption{Elégedettség kapcsolata a fizetéssel és az élekorral (N=5):
standardizált együtthatók}

\end{figure}

A fenti példában láthatjuk, hogy a fizetés erősebb kapcsolatban van az
elégedettséggel, hiszen a standardizált együtthatójának értéke abszolút
értékben nagyobb, mint az életkor standardizált együtthatójának abszolút
értéke.

Többszörös lineáris regresszió esetén több hipotézisvizsgálat végezhető:

\begin{itemize}
\tightlist
\item
  minden együtthatót külön tesztelhetünk t-próbákkal \((n-r-1)\)
  szabadsági fokkal

  \begin{itemize}
  \tightlist
  \item
    \(H_0:\beta_i=0\), \(H_1:\beta_i\neq0\), \(i=1,\dots,r\) Kérdés:
    \(Y\) függ \(X_i\)-től? (\(H_1\) elfogadása esetén igen)
  \end{itemize}
\item
  a teljes modellt tesztelhetjük F-próbával \((r,n-r-1)\) szabadsági
  fokkal

  \begin{itemize}
  \tightlist
  \item
    \(H_0: \text{minden } \beta_i=0\),
    \(H_1: \text{van olyan i, melyre } \beta_i\neq0\) Kérdés: a modell
    bír valamilyen bejósló erővel? (\(H_1\) elfogadása esetén igen)
  \end{itemize}
\end{itemize}

\begin{Shaded}
\begin{Highlighting}[]
\FunctionTok{summary}\NormalTok{(lm\_1)}\SpecialCharTok{$}\NormalTok{coefficients}
\CommentTok{\#\textgreater{}               Estimate Std. Error   t value     Pr(...}
\CommentTok{\#\textgreater{} (Intercept) 21.5080549 1.29216584  16.64496 0.00358...}
\CommentTok{\#\textgreater{} fizetes      0.5191980 0.00884672  58.68819 0.00029...}
\CommentTok{\#\textgreater{} eletkor     {-}0.3055489 0.02327903 {-}13.12550 0.00575...}
\FunctionTok{summary}\NormalTok{(lm\_1)}\SpecialCharTok{$}\NormalTok{fstatistic}
\CommentTok{\#\textgreater{}    value    numdf    dendf }
\CommentTok{\#\textgreater{} 1840.547    2.000    2.000}
\FunctionTok{pf}\NormalTok{(}\AttributeTok{q =} \FunctionTok{summary}\NormalTok{(lm\_1)}\SpecialCharTok{$}\NormalTok{fstatistic[}\DecValTok{1}\NormalTok{], }\AttributeTok{df1 =} \FunctionTok{summary}\NormalTok{(lm\_1)}\SpecialCharTok{$}\NormalTok{fstatistic[}\DecValTok{2}\NormalTok{],}
    \AttributeTok{df2 =} \FunctionTok{summary}\NormalTok{(lm\_1)}\SpecialCharTok{$}\NormalTok{fstatistic[}\DecValTok{3}\NormalTok{], }\AttributeTok{lower.tail =}\NormalTok{ F)}
\CommentTok{\#\textgreater{}        value }
\CommentTok{\#\textgreater{} 0.0005430217}
\end{Highlighting}
\end{Shaded}

\begin{figure}

{\centering \includegraphics{./images/lin_reg_fizetes_eletkor_elegedettseg_01_kep_03.jpg}

}

\caption{Elégedettség kapcsolata a fizetéssel és az élekorral (N=5):
hipotézisvizsgálatok}

\end{figure}

A lenti példában látható, hogy mindkét magyarázó változótól függ az
elégedettség (életkor p-értéke: 0,006, a fizetés p-értéke p \textless{}
0,001), és a teljes modell bír magyarázó erővel (p-érték: p \textless{}
0,001).

A függő változó és a független változók közötti korreláció erősségének
leírására több mennyiséget használhatunk

\begin{itemize}
\item
  többszörös korrelációs együttható: \(R\), amely a függő változó és a
  becsült értékek közötti korrelációs együttható értékével egyezik meg,
  azaz \(R(Y,X_1,X_2,…,X_r )=R(Y,\hat{Y})\). Valójában a lineáris
  regresszió ennek a korrelációs együtthatónak az értékét maximalizálja,
  mikor az \(\hat{Y}\)-t \(X\)-ek speciális lineáris kombinációjaként
  előállítja.
\item
  többszörös determinációs együttható: \(R^2\), amely a többszörös
  korrelációs együttható négyzete, és megmutatja, hogy a magyarázó
  változók a függő változó ingadozásának hányad részét magyarázzák.
\item
  korrigált determinációs együttható: \(R_{adj}^2\), amely kiküszöböli
  az \(R^2\) azon tulajdonságát, hogy a magyarázó változók számának
  növekedésével, függetlenül azok hatásától, nő az értéke. Így alkalmas
  több modell esetén a magyarázó erők összehasonlítására, akkor is, ha
  azok eltérő számú független változót használnak.
\end{itemize}

\begin{Shaded}
\begin{Highlighting}[]
\FunctionTok{summary}\NormalTok{(lm\_1)}\SpecialCharTok{$}\NormalTok{r.squared}
\CommentTok{\#\textgreater{} [1] 0.999457}
\FunctionTok{summary}\NormalTok{(lm\_1)}\SpecialCharTok{$}\NormalTok{adj.r.squared}
\CommentTok{\#\textgreater{} [1] 0.998914}
\end{Highlighting}
\end{Shaded}

\begin{figure}

{\centering \includegraphics{./images/lin_reg_fizetes_eletkor_elegedettseg_01_kep_04.jpg}

}

\caption{Elégedettség kapcsolata a fizetéssel és az élekorral (N=5):
magyarázó erő}

\end{figure}

A fenti példában látható mindhárom fenti mutató. Az \(R_{adj}^2\)
leolvasásával láthatjuk, hogy a két független változó, az életkor és a
fizetés a függő változó 99\%-át magyarázza.

\hypertarget{parciuxe1lis-korreluxe1ciuxf3s-egyuxfctthatuxf3}{%
\section{Parciális korrelációs
együttható}\label{parciuxe1lis-korreluxe1ciuxf3s-egyuxfctthatuxf3}}

Parciális korrelációs együttható: két változó (\(S_1,S_2\)) közötti
korreláció mértéke, miután változók egy halmazának
\((T_1,T_2,\dots,T_g)\) a két változó korrelációjára vonatkozó hatását
többszörös lineáris regresszióval kiküszöböljük:

\begin{itemize}
\tightlist
\item
  \(R( S_1,S_2 |T_1,T_2,\dots,T_g )= R(S_1- \hat{S_1},S_2- \hat{S_2})\),
  ahol \(\hat{S_1}\) és \(\hat{S_2}\) az \(S_1\) és \(S_2\) változó
  többszörös lineáris regresszióból származó becslése a
  \(T_1,T_2,\dots,T_g\) magyarázó változók esetén.
\end{itemize}

\begin{Shaded}
\begin{Highlighting}[]
\NormalTok{d }\OtherTok{\textless{}{-}}\NormalTok{ rio}\SpecialCharTok{::}\FunctionTok{import}\NormalTok{(}\AttributeTok{file =} \StringTok{"adat/lin\_reg\_intelligencia\_testmagassag\_eletkor\_01.xlsx"}\NormalTok{)}
\FunctionTok{str}\NormalTok{(d)}
\CommentTok{\#\textgreater{} \textquotesingle{}data.frame\textquotesingle{}:    5 obs. of  3 variables:}
\CommentTok{\#\textgreater{}  $ intelligencia: num  81 86 91 101 111}
\CommentTok{\#\textgreater{}  $ testmagassag : num  138 145 156 163 167}
\CommentTok{\#\textgreater{}  $ eletkor      : num  9 12 14 18 22}
\NormalTok{d}
\CommentTok{\#\textgreater{}   intelligencia testmagassag eletkor}
\CommentTok{\#\textgreater{} 1            81          138       9}
\CommentTok{\#\textgreater{} 2            86          145      12}
\CommentTok{\#\textgreater{} 3            91          156      14}
\CommentTok{\#\textgreater{} 4           101          163      18}
\CommentTok{\#\textgreater{} 5           111          167      22}
\end{Highlighting}
\end{Shaded}

\begin{Shaded}
\begin{Highlighting}[]
\FunctionTok{cor.test}\NormalTok{(d}\SpecialCharTok{$}\NormalTok{intelligencia, d}\SpecialCharTok{$}\NormalTok{testmagassag)}
\CommentTok{\#\textgreater{} }
\CommentTok{\#\textgreater{}  Pearson\textquotesingle{}s product{-}moment correlation}
\CommentTok{\#\textgreater{} }
\CommentTok{\#\textgreater{} data:  d$intelligencia and d$testmagassag}
\CommentTok{\#\textgreater{} t = 5.4629, df = 3, p{-}value = 0.01205}
\CommentTok{\#\textgreater{} alternative hypothesis: true correlation is not equ...}
\CommentTok{\#\textgreater{} 95 percent confidence interval:}
\CommentTok{\#\textgreater{}  0.4463631 0.9970093}
\CommentTok{\#\textgreater{} sample estimates:}
\CommentTok{\#\textgreater{}      cor }
\CommentTok{\#\textgreater{} 0.953235}
\NormalTok{RcmdrMisc}\SpecialCharTok{::}\FunctionTok{partial.cor}\NormalTok{(d, }\AttributeTok{tests =}\NormalTok{ T)}
\CommentTok{\#\textgreater{} }
\CommentTok{\#\textgreater{}  Partial correlations:}
\CommentTok{\#\textgreater{}               intelligencia testmagassag eletkor}
\CommentTok{\#\textgreater{} intelligencia       0.00000     {-}0.46317 0.97875}
\CommentTok{\#\textgreater{} testmagassag       {-}0.46317      0.00000 0.62856}
\CommentTok{\#\textgreater{} eletkor             0.97875      0.62856 0.00000}
\CommentTok{\#\textgreater{} }
\CommentTok{\#\textgreater{}  Number of observations: 5 }
\CommentTok{\#\textgreater{} }
\CommentTok{\#\textgreater{}  Pairwise two{-}sided p{-}values:}
\CommentTok{\#\textgreater{}               intelligencia testmagassag eletkor}
\CommentTok{\#\textgreater{} intelligencia               0.5368       0.0213 }
\CommentTok{\#\textgreater{} testmagassag  0.5368                     0.3714 }
\CommentTok{\#\textgreater{} eletkor       0.0213        0.3714              }
\CommentTok{\#\textgreater{} }
\CommentTok{\#\textgreater{}  Adjusted p{-}values (Holm\textquotesingle{}s method)}
\CommentTok{\#\textgreater{}               intelligencia testmagassag eletkor}
\CommentTok{\#\textgreater{} intelligencia               0.7429       0.0638 }
\CommentTok{\#\textgreater{} testmagassag  0.7429                     0.7429 }
\CommentTok{\#\textgreater{} eletkor       0.0638        0.7429}
\end{Highlighting}
\end{Shaded}

\includegraphics{./images/lin_reg_intelligencia_testmagassag_eletkor_01_kep_01.jpg}
A fenti példában látható, hogy míg szignifikáns erős pozitív kapcsolat
van az intelligencia és a magasság között (korrelációs együttható:
\(r=0,95; p=0,012\)), ez a kapcsolat eltűnik, ha figyelembe vesszük az
életkor változót is (parciális korreláció: \(r_{par}=-0,46; p=0,537\)).
Vagyis sikerült az intelligencia és a testmagasság közötti kapcsolat
erősségét megállapítani, miközben az életkor hatását erre a kapcsolatra
kiküszöböltük.

A többszörös lineáris regressziós modell
\((Y=\beta_0+\beta_1 X_1+\beta_2 X_2+\dots+\beta_r X_r+\epsilon)\)
becsült paraméterei \((b_1,b_2,…,b_r)\) nagyban hasonlítanak a parciális
korrelációs együtthatókra, mivel minden \(b_i\) az \(Y\) és \(X_i\)
közötti kapcsolat erősségét írja le, miközben a többi magyarázó változó
(\(X_1,X_2,\dots,X_r\), összesen \((r-1)\) db, \(X_i\) nincs köztük)
hatását kiküszöböljük a két változó korrelációjából.

A parciális korrelációs együtthatók és a többszörös lineáris regresszió
együtthatói között annyira közvetlen a kapcsolat, hogy azonos p-érték
tartozik hozzájuk, mint a lenti példában ez látható is lesz.

A lenti példában két modell szerepel, először az intelligencia és a
testmagasság függvényszerű kapcsolatát vizsgáljuk és azt a meglepő
dolgot tapasztaljuk, hogy minél magasabb valaki, annál intelligensebb
\((p=0,012)\), majd ha bevonjuk az életkor változót, akkor azt
tapasztalhatjuk, hogy eltűnik az intelligencia és a testmagasság közötti
kapcsolat \((p=0,537)\).

\begin{Shaded}
\begin{Highlighting}[]
\NormalTok{lm\_1 }\OtherTok{\textless{}{-}} \FunctionTok{lm}\NormalTok{(intelligencia }\SpecialCharTok{\textasciitilde{}}\NormalTok{ testmagassag, }\AttributeTok{data =}\NormalTok{ d)}
\FunctionTok{summary}\NormalTok{(lm\_1)}
\CommentTok{\#\textgreater{} }
\CommentTok{\#\textgreater{} Call:}
\CommentTok{\#\textgreater{} lm(formula = intelligencia \textasciitilde{} testmagassag, data = d)}
\CommentTok{\#\textgreater{} }
\CommentTok{\#\textgreater{} Residuals:}
\CommentTok{\#\textgreater{}       1       2       3       4       5 }
\CommentTok{\#\textgreater{}  1.9228  0.3114 {-}5.0779 {-}1.6892  4.5328 }
\CommentTok{\#\textgreater{} }
\CommentTok{\#\textgreater{} Coefficients:}
\CommentTok{\#\textgreater{}              Estimate Std. Error t value Pr(\textgreater{}|t|)  }
\CommentTok{\#\textgreater{} (Intercept)  {-}51.2613    26.6569  {-}1.923   0.1502  }
\CommentTok{\#\textgreater{} testmagassag   0.9445     0.1729   5.463   0.0121 *}
\CommentTok{\#\textgreater{} {-}{-}{-}}
\CommentTok{\#\textgreater{} Signif. codes:  }
\CommentTok{\#\textgreater{} 0 \textquotesingle{}***\textquotesingle{} 0.001 \textquotesingle{}**\textquotesingle{} 0.01 \textquotesingle{}*\textquotesingle{} 0.05 \textquotesingle{}.\textquotesingle{} 0.1 \textquotesingle{} \textquotesingle{} 1}
\CommentTok{\#\textgreater{} }
\CommentTok{\#\textgreater{} Residual standard error: 4.202 on 3 degrees of freedom}
\CommentTok{\#\textgreater{} Multiple R{-}squared:  0.9087, Adjusted R{-}squared:  0...}
\CommentTok{\#\textgreater{} F{-}statistic: 29.84 on 1 and 3 DF,  p{-}value: 0.01205}
\NormalTok{lm\_2 }\OtherTok{\textless{}{-}} \FunctionTok{lm}\NormalTok{(intelligencia }\SpecialCharTok{\textasciitilde{}}\NormalTok{ testmagassag }\SpecialCharTok{+}\NormalTok{ eletkor, }\AttributeTok{data =}\NormalTok{ d)}
\FunctionTok{summary}\NormalTok{(lm\_2)}
\CommentTok{\#\textgreater{} }
\CommentTok{\#\textgreater{} Call:}
\CommentTok{\#\textgreater{} lm(formula = intelligencia \textasciitilde{} testmagassag + eletkor...}
\CommentTok{\#\textgreater{} }
\CommentTok{\#\textgreater{} Residuals:}
\CommentTok{\#\textgreater{}        1        2        3        4        5 }
\CommentTok{\#\textgreater{}  0.89121 {-}1.16330 {-}0.09995  0.21170  0.16034 }
\CommentTok{\#\textgreater{} }
\CommentTok{\#\textgreater{} Coefficients:}
\CommentTok{\#\textgreater{}              Estimate Std. Error t value Pr(\textgreater{}|t|)  }
\CommentTok{\#\textgreater{} (Intercept)   73.1026    19.6042   3.729   0.0650 .}
\CommentTok{\#\textgreater{} testmagassag  {-}0.1210     0.1637  {-}0.739   0.5368  }
\CommentTok{\#\textgreater{} eletkor        2.6338     0.3902   6.750   0.0213 *}
\CommentTok{\#\textgreater{} {-}{-}{-}}
\CommentTok{\#\textgreater{} Signif. codes:  }
\CommentTok{\#\textgreater{} 0 \textquotesingle{}***\textquotesingle{} 0.001 \textquotesingle{}**\textquotesingle{} 0.01 \textquotesingle{}*\textquotesingle{} 0.05 \textquotesingle{}.\textquotesingle{} 0.1 \textquotesingle{} \textquotesingle{} 1}
\CommentTok{\#\textgreater{} }
\CommentTok{\#\textgreater{} Residual standard error: 1.055 on 2 degrees of freedom}
\CommentTok{\#\textgreater{} Multiple R{-}squared:  0.9962, Adjusted R{-}squared:  0...}
\CommentTok{\#\textgreater{} F{-}statistic: 259.3 on 2 and 2 DF,  p{-}value: 0.003841}
\end{Highlighting}
\end{Shaded}

\begin{figure}

{\centering \includegraphics{./images/lin_reg_intelligencia_testmagassag_eletkor_01_kep_02.jpg}

}

\caption{Az intelligencia és a testmagasság kapcsolata (N=5): két
modell, életkor nélkül és életkorral}

\end{figure}

\hypertarget{a-tuxf6bbszuxf6ruxf6s-lineuxe1ris-regressziuxf3-esetei}{%
\section{A többszörös lineáris regresszió
esetei}\label{a-tuxf6bbszuxf6ruxf6s-lineuxe1ris-regressziuxf3-esetei}}

\hypertarget{egyetlen-dichotuxf3m-magyaruxe1zuxf3-vuxe1ltozuxf3}{%
\subsection{Egyetlen dichotóm magyarázó
változó}\label{egyetlen-dichotuxf3m-magyaruxe1zuxf3-vuxe1ltozuxf3}}

A magyarázó változóink eddig kvantitatívak voltak, de kategorikus
változók is lehetnek. Ha a kategorikus változónk csupán 2 értékű, akkor
a becsült (\(b_0\), \(b_1\)) együtthatók értelmezése módosul. A
tengelymetszet (\(b_0\)) a kategorikus változó referencia szintjén a
függő változó átlagát tartalmazza, míg a (\(b_1\)) a kategorikus változó
másik szintjén számolt átlag eltérését a \(b_0\)-hoz képest.

\begin{Shaded}
\begin{Highlighting}[]
\NormalTok{d }\OtherTok{\textless{}{-}}\NormalTok{ rio}\SpecialCharTok{::}\FunctionTok{import}\NormalTok{(}\AttributeTok{file =} \StringTok{"adat/lin\_reg\_magassag\_hajhossz\_nem\_01.xlsx"}\NormalTok{)}
\NormalTok{d}\SpecialCharTok{$}\NormalTok{nem }\OtherTok{\textless{}{-}} \FunctionTok{factor}\NormalTok{(d}\SpecialCharTok{$}\NormalTok{nem, }\AttributeTok{levels =} \FunctionTok{c}\NormalTok{(}\StringTok{"nő"}\NormalTok{, }\StringTok{"férfi"}\NormalTok{))}
\FunctionTok{str}\NormalTok{(d)}
\CommentTok{\#\textgreater{} \textquotesingle{}data.frame\textquotesingle{}:    6 obs. of  3 variables:}
\CommentTok{\#\textgreater{}  $ magassag: num  158 159 162 170 182 179}
\CommentTok{\#\textgreater{}  $ hajhossz: num  28 25 20 1 1.5 3}
\CommentTok{\#\textgreater{}  $ nem     : Factor w/ 2 levels "nő","férfi": 1 1 1...}
\NormalTok{d}
\CommentTok{\#\textgreater{}   magassag hajhossz   nem}
\CommentTok{\#\textgreater{} 1      158     28.0    nő}
\CommentTok{\#\textgreater{} 2      159     25.0    nő}
\CommentTok{\#\textgreater{} 3      162     20.0    nő}
\CommentTok{\#\textgreater{} 4      170      1.0 férfi}
\CommentTok{\#\textgreater{} 5      182      1.5 férfi}
\CommentTok{\#\textgreater{} 6      179      3.0 férfi}
\end{Highlighting}
\end{Shaded}

\begin{Shaded}
\begin{Highlighting}[]
\NormalTok{lm\_1 }\OtherTok{\textless{}{-}} \FunctionTok{lm}\NormalTok{(magassag }\SpecialCharTok{\textasciitilde{}}\NormalTok{ nem, }\AttributeTok{data =}\NormalTok{ d)}
\FunctionTok{summary}\NormalTok{(lm\_1)}
\CommentTok{\#\textgreater{} }
\CommentTok{\#\textgreater{} Call:}
\CommentTok{\#\textgreater{} lm(formula = magassag \textasciitilde{} nem, data = d)}
\CommentTok{\#\textgreater{} }
\CommentTok{\#\textgreater{} Residuals:}
\CommentTok{\#\textgreater{}       1       2       3       4       5       6 }
\CommentTok{\#\textgreater{} {-}1.6667 {-}0.6667  2.3333 {-}7.0000  5.0000  2.0000 }
\CommentTok{\#\textgreater{} }
\CommentTok{\#\textgreater{} Coefficients:}
\CommentTok{\#\textgreater{}             Estimate Std. Error t value Pr(\textgreater{}|t|)    }
\CommentTok{\#\textgreater{} (Intercept)  159.667      2.687  59.413 4.81e{-}07 ***}
\CommentTok{\#\textgreater{} nemférfi      17.333      3.801   4.561   0.0103 *  }
\CommentTok{\#\textgreater{} {-}{-}{-}}
\CommentTok{\#\textgreater{} Signif. codes:  }
\CommentTok{\#\textgreater{} 0 \textquotesingle{}***\textquotesingle{} 0.001 \textquotesingle{}**\textquotesingle{} 0.01 \textquotesingle{}*\textquotesingle{} 0.05 \textquotesingle{}.\textquotesingle{} 0.1 \textquotesingle{} \textquotesingle{} 1}
\CommentTok{\#\textgreater{} }
\CommentTok{\#\textgreater{} Residual standard error: 4.655 on 4 degrees of freedom}
\CommentTok{\#\textgreater{} Multiple R{-}squared:  0.8387, Adjusted R{-}squared:  0...}
\CommentTok{\#\textgreater{} F{-}statistic:  20.8 on 1 and 4 DF,  p{-}value: 0.01033}
\end{Highlighting}
\end{Shaded}

\begin{figure}

{\centering \includegraphics{./images/lin_reg_magassag_hajhossz_nem_01_kep_01.jpg}

}

\caption{A magasság és a nem kapcsolata}

\end{figure}

A fenti példa a nem hatását vizsgálja testmagasságra. A p-érték alapján
ez a hatás szignifikáns, tehát a függés fennáll, a paraméterek pedig a
nők átlagáról \((b_0=159,67)\) és a férfiak és nők átlagának eltéréséről
tájékoztatnak \((b_1=17,33)\).

\hypertarget{modellvuxe1lasztuxe1s}{%
\section{Modellválasztás}\label{modellvuxe1lasztuxe1s}}

Előfordulhat, hogy egy jelenség vizsgálatakor több lineáris regressziós
modellt is meg tudunk fogalmazni, nem csak egyetlen modell létezik. Ez a
probléma leggyakrabban úgy jelenik meg, hogy rengeteg független
változónk van, és nem tudjuk eldönteni, hogy elég egy kisebb modell,
néhány változóval, vagy vegyük inkább a nagyobb modellt több változóval.

A megfelelő modell megtaláláshoz a modelleket összehasonlíthatjuk
F-próba segítségével, szignifikáns eredmény esetén a két modell
magyarázó ereje eltér egymástól. Ilyenkor a célunk a legszűkebb
(legkevesebb magyarázó változót tartalmazó), de a legbővebbtől
szignifikánsan nem különböző modell megtalálása.

A korrigált determinációs együttható \((R_{adj}^2)\) is alkalmas mód a
modellek összehasonlítására: az 1-hez legközelebbi értékkel bíró modell
rendelkezik a legnagyobb magyarázó erővel. Léteznek már kritériumok is:

\begin{itemize}
\tightlist
\item
  AIC (Akaike-kritérium): minél kisebb az AIC értéke, annál nagyobb a
  modell magyarázó ereje.
\item
  BIC (Bayes-kritérium): minél kisebb a BIC értéke, annál nagyobb a
  modell magyarázó ereje.
\item
  RMSE (négyzetes középérték, Root Mean Square Error) az a mennyiség,
  amennyivel a vizsgált értékek eltérnek az előre megbecsült értékektől.
  Minél kisebb ez az érték, annál jobban becsül a modell.
\end{itemize}

\begin{Shaded}
\begin{Highlighting}[]
\NormalTok{d }\OtherTok{\textless{}{-}}\NormalTok{ rio}\SpecialCharTok{::}\FunctionTok{import}\NormalTok{(}\AttributeTok{file =} \StringTok{"adat/lin\_reg\_fizetes\_eletkor\_elegedettseg\_01.xlsx"}\NormalTok{)}
\FunctionTok{str}\NormalTok{(d)}
\CommentTok{\#\textgreater{} \textquotesingle{}data.frame\textquotesingle{}:    5 obs. of  3 variables:}
\CommentTok{\#\textgreater{}  $ fizetes     : num  44 66 89 155 130}
\CommentTok{\#\textgreater{}  $ eletkor     : num  25 65 21 35 40}
\CommentTok{\#\textgreater{}  $ elegedettseg: num  37 36 61 92 76}
\NormalTok{d}
\CommentTok{\#\textgreater{}   fizetes eletkor elegedettseg}
\CommentTok{\#\textgreater{} 1      44      25           37}
\CommentTok{\#\textgreater{} 2      66      65           36}
\CommentTok{\#\textgreater{} 3      89      21           61}
\CommentTok{\#\textgreater{} 4     155      35           92}
\CommentTok{\#\textgreater{} 5     130      40           76}
\end{Highlighting}
\end{Shaded}

\begin{Shaded}
\begin{Highlighting}[]
\NormalTok{lm\_1 }\OtherTok{\textless{}{-}} \FunctionTok{lm}\NormalTok{(elegedettseg }\SpecialCharTok{\textasciitilde{}}\NormalTok{ fizetes, }\AttributeTok{data =}\NormalTok{ d)}
\FunctionTok{summary}\NormalTok{(lm\_1)}
\CommentTok{\#\textgreater{} }
\CommentTok{\#\textgreater{} Call:}
\CommentTok{\#\textgreater{} lm(formula = elegedettseg \textasciitilde{} fizetes, data = d)}
\CommentTok{\#\textgreater{} }
\CommentTok{\#\textgreater{} Residuals:}
\CommentTok{\#\textgreater{}      1      2      3      4      5 }
\CommentTok{\#\textgreater{}  4.249 {-}8.272  4.684  1.123 {-}1.785 }
\CommentTok{\#\textgreater{} }
\CommentTok{\#\textgreater{} Coefficients:}
\CommentTok{\#\textgreater{}             Estimate Std. Error t value Pr(\textgreater{}|t|)   }
\CommentTok{\#\textgreater{} (Intercept)  9.71048    7.07562   1.372  0.26355   }
\CommentTok{\#\textgreater{} fizetes      0.52365    0.06738   7.772  0.00443 **}
\CommentTok{\#\textgreater{} {-}{-}{-}}
\CommentTok{\#\textgreater{} Signif. codes:  }
\CommentTok{\#\textgreater{} 0 \textquotesingle{}***\textquotesingle{} 0.001 \textquotesingle{}**\textquotesingle{} 0.01 \textquotesingle{}*\textquotesingle{} 0.05 \textquotesingle{}.\textquotesingle{} 0.1 \textquotesingle{} \textquotesingle{} 1}
\CommentTok{\#\textgreater{} }
\CommentTok{\#\textgreater{} Residual standard error: 6.134 on 3 degrees of freedom}
\CommentTok{\#\textgreater{} Multiple R{-}squared:  0.9527, Adjusted R{-}squared:  0...}
\CommentTok{\#\textgreater{} F{-}statistic:  60.4 on 1 and 3 DF,  p{-}value: 0.004432}
\NormalTok{lm\_2 }\OtherTok{\textless{}{-}} \FunctionTok{lm}\NormalTok{(elegedettseg }\SpecialCharTok{\textasciitilde{}}\NormalTok{ fizetes }\SpecialCharTok{+}\NormalTok{ eletkor, }\AttributeTok{data =}\NormalTok{ d)}
\FunctionTok{summary}\NormalTok{(lm\_2)}
\CommentTok{\#\textgreater{} }
\CommentTok{\#\textgreater{} Call:}
\CommentTok{\#\textgreater{} lm(formula = elegedettseg \textasciitilde{} fizetes + eletkor, data...}
\CommentTok{\#\textgreater{} }
\CommentTok{\#\textgreater{} Residuals:}
\CommentTok{\#\textgreater{}        1        2        3        4        5 }
\CommentTok{\#\textgreater{}  0.28596  0.08556 {-}0.30015  0.71047 {-}0.78184 }
\CommentTok{\#\textgreater{} }
\CommentTok{\#\textgreater{} Coefficients:}
\CommentTok{\#\textgreater{}              Estimate Std. Error t value Pr(\textgreater{}|t|)    }
\CommentTok{\#\textgreater{} (Intercept) 21.508055   1.292166   16.64  0.00359 ** }
\CommentTok{\#\textgreater{} fizetes      0.519198   0.008847   58.69  0.00029 ***}
\CommentTok{\#\textgreater{} eletkor     {-}0.305549   0.023279  {-}13.12  0.00575 ** }
\CommentTok{\#\textgreater{} {-}{-}{-}}
\CommentTok{\#\textgreater{} Signif. codes:  }
\CommentTok{\#\textgreater{} 0 \textquotesingle{}***\textquotesingle{} 0.001 \textquotesingle{}**\textquotesingle{} 0.01 \textquotesingle{}*\textquotesingle{} 0.05 \textquotesingle{}.\textquotesingle{} 0.1 \textquotesingle{} \textquotesingle{} 1}
\CommentTok{\#\textgreater{} }
\CommentTok{\#\textgreater{} Residual standard error: 0.8047 on 2 degrees of fre...}
\CommentTok{\#\textgreater{} Multiple R{-}squared:  0.9995, Adjusted R{-}squared:  0...}
\CommentTok{\#\textgreater{} F{-}statistic:  1841 on 2 and 2 DF,  p{-}value: 0.000543}
\FunctionTok{anova}\NormalTok{(lm\_1, lm\_2)}
\CommentTok{\#\textgreater{} Analysis of Variance Table}
\CommentTok{\#\textgreater{} }
\CommentTok{\#\textgreater{} Model 1: elegedettseg \textasciitilde{} fizetes}
\CommentTok{\#\textgreater{} Model 2: elegedettseg \textasciitilde{} fizetes + eletkor}
\CommentTok{\#\textgreater{}   Res.Df     RSS Df Sum of Sq      F   Pr(\textgreater{}F)   }
\CommentTok{\#\textgreater{} 1      3 112.864                                }
\CommentTok{\#\textgreater{} 2      2   1.295  1    111.57 172.28 0.005754 **}
\CommentTok{\#\textgreater{} {-}{-}{-}}
\CommentTok{\#\textgreater{} Signif. codes:  }
\CommentTok{\#\textgreater{} 0 \textquotesingle{}***\textquotesingle{} 0.001 \textquotesingle{}**\textquotesingle{} 0.01 \textquotesingle{}*\textquotesingle{} 0.05 \textquotesingle{}.\textquotesingle{} 0.1 \textquotesingle{} \textquotesingle{} 1}
\NormalTok{performance}\SpecialCharTok{::}\FunctionTok{model\_performance}\NormalTok{(lm\_1)}
\CommentTok{\#\textgreater{} \# Indices of model performance}
\CommentTok{\#\textgreater{} }
\CommentTok{\#\textgreater{} AIC    |   AICc |    BIC |    R2 | R2 (adj.) |  RMS...}
\CommentTok{\#\textgreater{} {-}{-}{-}{-}{-}{-}{-}{-}{-}{-}{-}{-}{-}{-}{-}{-}{-}{-}{-}{-}{-}{-}{-}{-}{-}{-}{-}{-}{-}{-}{-}{-}{-}{-}{-}{-}{-}{-}{-}{-}{-}{-}{-}{-}{-}{-}{-}{-}{-}{-}{-}...}
\CommentTok{\#\textgreater{} 35.773 | 59.773 | 34.601 | 0.953 |     0.937 | 4.75...}
\NormalTok{performance}\SpecialCharTok{::}\FunctionTok{model\_performance}\NormalTok{(lm\_2)}
\CommentTok{\#\textgreater{} \# Indices of model performance}
\CommentTok{\#\textgreater{} }
\CommentTok{\#\textgreater{} AIC    | AICc |    BIC |    R2 | R2 (adj.) |  RMSE ...}
\CommentTok{\#\textgreater{} {-}{-}{-}{-}{-}{-}{-}{-}{-}{-}{-}{-}{-}{-}{-}{-}{-}{-}{-}{-}{-}{-}{-}{-}{-}{-}{-}{-}{-}{-}{-}{-}{-}{-}{-}{-}{-}{-}{-}{-}{-}{-}{-}{-}{-}{-}{-}{-}{-}{-}{-}...}
\CommentTok{\#\textgreater{} 15.436 |  Inf | 13.873 | 0.999 |     0.999 | 0.509 ...}
\end{Highlighting}
\end{Shaded}

\begin{figure}

{\centering \includegraphics{./images/lin_reg_fizetes_eletkor_elegedettseg_01_kep_05.jpg}

}

\caption{Elégedettség kapcsolata a fizetéssel és az élekorral (N=5):
modellek összehasonlítása}

\end{figure}

A fenti példán látható, hogy két modellt építettünk. Az 1. modell az
elégedettséget a fizetés segítségével próbálja jósolni. A 2. modell az
elégedettséget a fizetéssel és az életkorral. Láthatjuk a 2. modell
szignifikánsan eltér magyarázó erőben a az 1. modelltől, valamint a
modell ``jóságát'' leíró mutatók mindegyike kedvezőbb a 2. modell
esetén: \(R_{adj}^2\), \(AIC\), \(BIC\), \(RMSE\).

\hypertarget{alkalmazuxe1si-feltuxe9telek}{%
\section{Alkalmazási feltételek}\label{alkalmazuxe1si-feltuxe9telek}}

A regressziós modellt ne használjuk, ha az alkalmazási feltételek
valamelyike nem teljesül. Melyek ezek?

\begin{itemize}
\tightlist
\item
  \textbf{Kiugró értékek.} A kiugró értékek torzítják a regresszió
  eredményét, így lehetőség szerint az ilyen eseteket ki kell szűrnünk.
  Szűrésük történhet a Cook-féle távolság segítégével. A Cook-féle
  távolság egy eset általános hatását méri a modellre. A Cook-féle
  távolságnál a 4/N-nél nagyobb értékek jelenthetnek problémát. A
  megjelölt esetek így kiszűrésre kerülnek az adatbázisból.
\end{itemize}

\begin{figure}

{\centering \includegraphics{./images/lin_reg_fizetes_eletkor_elegedettseg_01_kep_06.jpg}

}

\caption{Elégedettség kapcsolata a fizetéssel és az élekorral (N=5):
kiugró értékek vizsgálata}

\end{figure}

\begin{itemize}
\tightlist
\item
  \textbf{Multikollinearitás.} A multikollinearitás a független változók
  közötti erős korrelációra utal. Multikollinearitás bizonytalanná teszi
  és korlátozza a modell magyarázó erejét, bizonyos esetekben a
  regressziós számítást el sem lehet végezni. Ki lehet szűrni a
  multikollinearitásban érintett változókat a variancianövelő tényezők
  (variance inflation factor, VIF) és a tolerancia értékek elemzésével.
  Ha legnagyobb VIF érték tíznél nagyobb, illetve ha az átlagos VIF
  érték jelentősen nagyobb, mint egy, akkor az problémát jelenthet. A
  tolerancia értékek gyakorlatilag a VIF értékek reciprok értékei
  (1/VIF). Az érintett változókat kihagyhatjuk a modellből, vagy
  származtatott adatokkal dolgozunk tovább (például főkomponens
  elemzéssel nyert adatokkal).
\end{itemize}

A VIF megmutatja a becsült regressziós együttható varianciája
„felfújódásának'' mértékét a hibatag varianciájához viszonyítva. A
mutató értéke bármilyen nagy lehet. A tolerancia mutató megmutatja, hogy
a magyarázóváltozó szórásnégyzetének mekkora része nem magyarázható
együttesen a többi magyarázó változóval. Ennek értéke nulla és egy közé
esik. Minél nagyobb a multikollinearitás mértéke annál közelebb van a
mutató értéke a nullához.

\begin{figure}

{\centering \includegraphics{./images/lin_reg_fizetes_eletkor_elegedettseg_01_kep_07.jpg}

}

\caption{Elégedettség kapcsolata a fizetéssel és az élekorral (N=5):
multikollinearitás vizsgálata}

\end{figure}

\begin{itemize}
\tightlist
\item
  \textbf{Homoszkedaszticitás.} A homoszkedaszticitás azt jelenti, hogy
  az eltérésváltozók varianciája állandó és független kell legyen, tehát
  a függő változó szórásának minden esetben ugyanannyinak kell lennie,
  függetlenül a független változóktól. Ha a Breusch-Pagan próba nem
  szignifikáns, akkor a homoszkedaszticitási előfeltétel teljesül.
\end{itemize}

\begin{figure}

{\centering \includegraphics{./images/lin_reg_fizetes_eletkor_elegedettseg_01_kep_08.jpg}

}

\caption{Elégedettség kapcsolata a fizetéssel és az élekorral (N=5):
homoszkedaszticitás vizsgálata}

\end{figure}

\begin{itemize}
\tightlist
\item
  \textbf{Autokorreláció.} A hibatagok szignifikáns együttmozgása az
  autokorreláció.A lineáris regresszió szempontjából fontos, hogy a
  hibatagok (reziduálisok) (vagyis a függő változó azon része, amit a
  független változók nem magyaráznak) ne korreláljanak egymással. Az
  autokorrelációt a Durbin-Watson próbával lehet ellenőrizni. A próba
  nullhipotézisének megtartása zat jelenti, hogy a hibatagokat nem
  tekintjük autokorreláltnak. (A Durbin-Watson próba esetében az egynél
  kisebb, illetve a háromnál nagyobb DW próbastatisztika értékek
  jelenthetnek problémát, a kettő közeli értékek kívánatosak.)
\end{itemize}

\begin{figure}

{\centering \includegraphics{./images/lin_reg_fizetes_eletkor_elegedettseg_01_kep_09.jpg}

}

\caption{Elégedettség kapcsolata a fizetéssel és az élekorral (N=5):
autokorreláció vizsgálata}

\end{figure}

\begin{itemize}
\tightlist
\item
  \textbf{Reziduálisok normális eloszlása.} A reziduális normális
  eloszlását a szokásos próbákkal és a QQ-ábrával is ellenőrizhetjük.
\end{itemize}

\begin{figure}

{\centering \includegraphics{./images/lin_reg_fizetes_eletkor_elegedettseg_01_kep_10.jpg}

}

\caption{Elégedettség kapcsolata a fizetéssel és az élekorral (N=5):
reziduális normális eloszlása}

\end{figure}

\hypertarget{puxe9lda-befolyuxe1solja-e-a-munkahellyel-valuxf3-eluxe9gedettsuxe9get-a-fizetuxe9s-nagysuxe1ga-uxe9s-az-uxe9letkor}{%
\section{Példa: Befolyásolja-e a munkahellyel való elégedettséget a
fizetés nagysága és az
életkor?}\label{puxe9lda-befolyuxe1solja-e-a-munkahellyel-valuxf3-eluxe9gedettsuxe9get-a-fizetuxe9s-nagysuxe1ga-uxe9s-az-uxe9letkor}}

\begin{itemize}
\tightlist
\item
  A példa forrása: Münnich és mtsai. (2006, o. 1.6.1 probléma)
\item
  Kapcsolódó jamovi állomány: \texttt{lin\_reg\_elegedettseg.omv}
\end{itemize}

\begin{Shaded}
\begin{Highlighting}[]
\NormalTok{d }\OtherTok{\textless{}{-}}\NormalTok{ rio}\SpecialCharTok{::}\FunctionTok{import}\NormalTok{(}\AttributeTok{file =} \StringTok{"adat/lin\_reg\_elegedettseg.xlsx"}\NormalTok{)}
\FunctionTok{str}\NormalTok{(d)}
\CommentTok{\#\textgreater{} \textquotesingle{}data.frame\textquotesingle{}:    30 obs. of  4 variables:}
\CommentTok{\#\textgreater{}  $ fizetes     : num  109 125 98 124 115 132 124 99...}
\CommentTok{\#\textgreater{}  $ elegedettseg: num  69.2 90.8 71 90.1 77.8 ...}
\CommentTok{\#\textgreater{}  $ kor         : num  20 46.3 36.2 46 31 ...}
\CommentTok{\#\textgreater{}  $ nem         : chr  "nő" "férfi" "nő" "férfi" ...}
\NormalTok{psych}\SpecialCharTok{::}\FunctionTok{headTail}\NormalTok{(d)}
\CommentTok{\#\textgreater{}     fizetes elegedettseg   kor   nem}
\CommentTok{\#\textgreater{} 1       109        69.15    20    nő}
\CommentTok{\#\textgreater{} 2       125        90.85 46.33 férfi}
\CommentTok{\#\textgreater{} 3        98        71.04  36.2    nő}
\CommentTok{\#\textgreater{} 4       124        90.12 45.95 férfi}
\CommentTok{\#\textgreater{} ...     ...          ...   ...  \textless{}NA\textgreater{}}
\CommentTok{\#\textgreater{} 27      129        96.32    53 férfi}
\CommentTok{\#\textgreater{} 28      135        97.86  49.4 férfi}
\CommentTok{\#\textgreater{} 29      145          100  46.9 férfi}
\CommentTok{\#\textgreater{} 30      120        81.08    32 férfi}
\end{Highlighting}
\end{Shaded}

\begin{Shaded}
\begin{Highlighting}[]
\NormalTok{lm\_1 }\OtherTok{\textless{}{-}} \FunctionTok{lm}\NormalTok{(elegedettseg }\SpecialCharTok{\textasciitilde{}}\NormalTok{ fizetes }\SpecialCharTok{+}\NormalTok{ kor, }\AttributeTok{data =}\NormalTok{ d)}
\FunctionTok{summary}\NormalTok{(lm\_1)}
\CommentTok{\#\textgreater{} }
\CommentTok{\#\textgreater{} Call:}
\CommentTok{\#\textgreater{} lm(formula = elegedettseg \textasciitilde{} fizetes + kor, data = d)}
\CommentTok{\#\textgreater{} }
\CommentTok{\#\textgreater{} Residuals:}
\CommentTok{\#\textgreater{}     Min      1Q  Median      3Q     Max }
\CommentTok{\#\textgreater{} {-}10.248  {-}1.543   1.188   2.437   4.333 }
\CommentTok{\#\textgreater{} }
\CommentTok{\#\textgreater{} Coefficients:}
\CommentTok{\#\textgreater{}             Estimate Std. Error t value Pr(\textgreater{}|t|)    }
\CommentTok{\#\textgreater{} (Intercept)  8.13712    3.26797   2.490   0.0192 *  }
\CommentTok{\#\textgreater{} fizetes      0.44404    0.02321  19.128  \textless{} 2e{-}16 ***}
\CommentTok{\#\textgreater{} kor          0.53361    0.07127   7.487 4.71e{-}08 ***}
\CommentTok{\#\textgreater{} {-}{-}{-}}
\CommentTok{\#\textgreater{} Signif. codes:  }
\CommentTok{\#\textgreater{} 0 \textquotesingle{}***\textquotesingle{} 0.001 \textquotesingle{}**\textquotesingle{} 0.01 \textquotesingle{}*\textquotesingle{} 0.05 \textquotesingle{}.\textquotesingle{} 0.1 \textquotesingle{} \textquotesingle{} 1}
\CommentTok{\#\textgreater{} }
\CommentTok{\#\textgreater{} Residual standard error: 3.59 on 27 degrees of freedom}
\CommentTok{\#\textgreater{} Multiple R{-}squared:  0.953,  Adjusted R{-}squared:  0....}
\CommentTok{\#\textgreater{} F{-}statistic: 273.9 on 2 and 27 DF,  p{-}value: \textless{} 2.2e{-}16}
\NormalTok{lsr}\SpecialCharTok{::}\FunctionTok{standardCoefs}\NormalTok{(lm\_1)}
\CommentTok{\#\textgreater{}                 b      beta}
\CommentTok{\#\textgreater{} fizetes 0.4440418 0.8322319}
\CommentTok{\#\textgreater{} kor     0.5336111 0.3257287}
\end{Highlighting}
\end{Shaded}

\begin{figure}

{\centering \includegraphics{./images/lin_reg_elegedettseg_kep_01.jpg}

}

\caption{Elégedettség kapcsolata a fizetéssel és lektkorra (N=30)}

\end{figure}

A fenti outputból láthatjuk, hogy a fizetés és a kor változó is
szignifikánsan befolyásolja az elégedettséget, hiszen a hozzájuk tartozó
szignifikanciaszint \(p<0,05\). A teljes modell vonatkozó F-próba is
szignifikáns. A fizetés változó együtthatója \((b_1)\) 0,44, a kor
változó együtthatója \((b_2)\) pedig 0,53, ami arra utal, hogy pozitív
kapcsolat van a változó között: minél magasabb a fizetés, és minél
idősebbek az emberek, annál elégedettebbek a munkahelyükkel.

A pontos becslés a regressziós egyenlet alapján a következőképpen fest:

\begin{Shaded}
\begin{Highlighting}[]
\NormalTok{elégedettség = 8,14 + 0,44 * fizetés + 0,53 * kor}
\end{Highlighting}
\end{Shaded}

Mivel a többszörös regresszió esetében a független változók hatása csak
a standardizált együtthatók mentén hasonlítható össze, így kiszámoltuk a
standardizált együtthatókat is. Az adatok jól példázzák, hogy miért
fontos a standardizált együtthatókat is vizsgálni, hiszen a nem
standardizált együtthatók esetén a kor változó együtthatójának értéke a
magasabb, míg a standardizált értékeknél fordítva. Vagyis, ha az egyes
változók relatív fontosságának vizsgálatakor nem nézzük a dimenziómentes
értékeket, akkor könnyen téves következtetésre juthatunk.

A négyzetes korrelációs együttható értéke 0,9, ami arra utal, hogy a
független változók igen jól magyarázzák a függő változót.

\hypertarget{puxe9lda-befolyuxe1solja-e-a-kalandvuxe1gy-a-hivatuxe1sos-katonai-szolguxe1latnuxe1l-eltuxf6ltuxf6tt-idux151t}{%
\section{Példa: Befolyásolja-e a kalandvágy a hivatásos katonai
szolgálatnál eltöltött
időt?}\label{puxe9lda-befolyuxe1solja-e-a-kalandvuxe1gy-a-hivatuxe1sos-katonai-szolguxe1latnuxe1l-eltuxf6ltuxf6tt-idux151t}}

\begin{itemize}
\tightlist
\item
  A példa forrása: Münnich és mtsai. (2006, o. 1.6.2 probléma)
\item
  Kapcsolódó jamovi állomány: \texttt{lin\_reg\_katonasag.omv}
\end{itemize}

\begin{Shaded}
\begin{Highlighting}[]
\NormalTok{d }\OtherTok{\textless{}{-}}\NormalTok{ rio}\SpecialCharTok{::}\FunctionTok{import}\NormalTok{(}\AttributeTok{file =} \StringTok{"adat/lin\_reg\_katonasag.xlsx"}\NormalTok{)}
\FunctionTok{str}\NormalTok{(d)}
\CommentTok{\#\textgreater{} \textquotesingle{}data.frame\textquotesingle{}:    156 obs. of  4 variables:}
\CommentTok{\#\textgreater{}  $ kaland  : num  3 3 5 1 3 4 2 2 3 5 ...}
\CommentTok{\#\textgreater{}  $ egyhangu: num  3 2 4 1 3 1 2 1 2 3 ...}
\CommentTok{\#\textgreater{}  $ sport   : num  1 1 2 1 2 2 2 1 2 2 ...}
\CommentTok{\#\textgreater{}  $ evek    : num  4 7 10 3 6 15 5 6 9 13 ...}
\NormalTok{psych}\SpecialCharTok{::}\FunctionTok{headTail}\NormalTok{(d)}
\CommentTok{\#\textgreater{}     kaland egyhangu sport evek}
\CommentTok{\#\textgreater{} 1        3        3     1    4}
\CommentTok{\#\textgreater{} 2        3        2     1    7}
\CommentTok{\#\textgreater{} 3        5        4     2   10}
\CommentTok{\#\textgreater{} 4        1        1     1    3}
\CommentTok{\#\textgreater{} ...    ...      ...   ...  ...}
\CommentTok{\#\textgreater{} 153      2        4     2    1}
\CommentTok{\#\textgreater{} 154      3        5     3    2}
\CommentTok{\#\textgreater{} 155      1        3     2    2}
\CommentTok{\#\textgreater{} 156      3        2     4   12}
\end{Highlighting}
\end{Shaded}

\begin{Shaded}
\begin{Highlighting}[]
\NormalTok{lm\_1 }\OtherTok{\textless{}{-}} \FunctionTok{lm}\NormalTok{(evek }\SpecialCharTok{\textasciitilde{}}\NormalTok{ egyhangu }\SpecialCharTok{+}\NormalTok{ sport }\SpecialCharTok{+}\NormalTok{ kaland, }\AttributeTok{data =}\NormalTok{ d)}
\FunctionTok{summary}\NormalTok{(lm\_1)}
\CommentTok{\#\textgreater{} }
\CommentTok{\#\textgreater{} Call:}
\CommentTok{\#\textgreater{} lm(formula = evek \textasciitilde{} egyhangu + sport + kaland, data...}
\CommentTok{\#\textgreater{} }
\CommentTok{\#\textgreater{} Residuals:}
\CommentTok{\#\textgreater{}     Min      1Q  Median      3Q     Max }
\CommentTok{\#\textgreater{} {-}1.6611 {-}0.5925 {-}0.0798  0.2726  9.7833 }
\CommentTok{\#\textgreater{} }
\CommentTok{\#\textgreater{} Coefficients:}
\CommentTok{\#\textgreater{}             Estimate Std. Error t value Pr(\textgreater{}|t|)    }
\CommentTok{\#\textgreater{} (Intercept)  0.63951    0.33263   1.923   0.0564 .  }
\CommentTok{\#\textgreater{} egyhangu    {-}2.28197    0.09160 {-}24.912   \textless{}2e{-}16 ***}
\CommentTok{\#\textgreater{} sport        1.52987    0.09447  16.194   \textless{}2e{-}16 ***}
\CommentTok{\#\textgreater{} kaland       3.17525    0.09884  32.125   \textless{}2e{-}16 ***}
\CommentTok{\#\textgreater{} {-}{-}{-}}
\CommentTok{\#\textgreater{} Signif. codes:  }
\CommentTok{\#\textgreater{} 0 \textquotesingle{}***\textquotesingle{} 0.001 \textquotesingle{}**\textquotesingle{} 0.01 \textquotesingle{}*\textquotesingle{} 0.05 \textquotesingle{}.\textquotesingle{} 0.1 \textquotesingle{} \textquotesingle{} 1}
\CommentTok{\#\textgreater{} }
\CommentTok{\#\textgreater{} Residual standard error: 1.213 on 152 degrees of fr...}
\CommentTok{\#\textgreater{} Multiple R{-}squared:  0.9195, Adjusted R{-}squared:  0...}
\CommentTok{\#\textgreater{} F{-}statistic: 578.4 on 3 and 152 DF,  p{-}value: \textless{} 2.2...}
\NormalTok{lsr}\SpecialCharTok{::}\FunctionTok{standardCoefs}\NormalTok{(lm\_1)}
\CommentTok{\#\textgreater{}                  b       beta}
\CommentTok{\#\textgreater{} egyhangu {-}2.281974 {-}0.5927751}
\CommentTok{\#\textgreater{} sport     1.529871  0.3821476}
\CommentTok{\#\textgreater{} kaland    3.175249  0.7676009}
\end{Highlighting}
\end{Shaded}

\begin{figure}

{\centering \includegraphics{./images/lin_reg_katonasag_kep_01.jpg}

}

\caption{Befolyásolja-e a kalandvágy a hivatásos katonai szolgálatnál
eltöltött időt? (N=156)}

\end{figure}

A fenti output a többszörös lineáris regresszió eredményét mutatja:

\begin{itemize}
\tightlist
\item
  A lineáris regressziós modellt megtarthatjuk, hiszen az F-statisztika
  értékét tekintve a modell szignifikáns, a változók együtthatóinak az
  értéke nem nulla.
\item
  A modell magyarázóértéke igen jó, hiszen a korrigált determinációs
  együttható értéke 0,92, vagyis a független változók a függő változó
  varianciájának kb. 92\%-át magyarázzák.
\item
  Minden egyes független változó hatással van a függő változóra, vagyis
  mind a kalandvágy, az extrém sportok szeretete és a nyugalom utáni
  vágy is befolyásolja azt, hogy mennyi időt tölt valaki a hivatásos
  katonai szolgálatban.
\item
  Ellenben a \(b_0\) vagyis a konstans értéke most nulla, hiszen a
  táblázatban szereplő érték nem szignifikáns.
\item
  Maga a regressziós egyenlet a pontos együtthatók ismeretében a
  következőképpen alakul:
\end{itemize}

\begin{Shaded}
\begin{Highlighting}[]
\InformationTok{    evek=3,18*kaland+1,53*sport{-}2,28*egyhangu}
\end{Highlighting}
\end{Shaded}

\begin{itemize}
\item
  Vagyis minél jobban kedveli valaki a kalandos életet és az extrém
  sportokat, és minél jobban irtózik a szürke hétköznapoktól, annál több
  időt tölt a katonaság kötelékében.
\item
  A standardizált változók alapján a kaland szeretetének a hatása a
  legerősebb (0,768), a második legerősebb hatás az egyhangúság
  kedvelése, ám hatásának iránya negatív (-0,593), leggyengébb hatása
  pedig az extrém sportok szeretetének van (0,382).
\end{itemize}

\bookmarksetup{startatroot}

\hypertarget{sec-fokomponens-elemzes}{%
\chapter{Főkomponens elemzés}\label{sec-fokomponens-elemzes}}

A főkomponens elemzés a legegyszerűbb többváltozós statisztikai
eljárások egyike. Olyan eljárás, amely az egy jelenségre vonatkozó
méréseket úgy „összegzi'', hogy közben az óhatatlanul fellépő információ
veszteséget a lehető legkisebb mértékűre csökkenti.

A módszer alapgondolata az, hogy vegyünk \(p\) változót:
\(X_1,X_2,\dots,X_p\), majd keressük meg ezek kombinációit, hogy ezáltal
\(Z_1,Z_2,..,Z_p\)-vel jelölt indexeket kaphassunk, melyek egymással nem
korrelálnak, továbbá
\(var(Z_1 )\geq var(Z_2 )\geq \dots \geq var(Z_p)\), ahol \(var(Z_i)\) a
\(Z_i\) komponens varianciáját jelöli.

A korreláció hiánya a \(Z_i\)-kben hasznos tulajdonság, ugyanis azt
jelenti, hogy az indexek az adatok különböző ``dimenzióit'' mérik.

A \(Z_i\)-t főkomponensnek nevezzük. Amikor főkomponens-analízist
végzünk, mindig abban bízunk, hogy a legtöbb index varianciája
elhanyagolhatóan kicsi. Ezáltal az adatok varianciája adekvátan leírható
néhány olyan \(Z_i\) változóval, melyek varianciája nem elhanyagolható.

A főkomponens analízis működéséhez szükséges, hogy az eredeti
\(X_1,X_2,\dots,X_p\) változók korreláljanak egymással (akár pozitív,
akár negatív irányban), ekkor elképzelhető az az eset, hogy 20-30
eredeti változót adekvátan reprezentálhat 2-3 főkomponens. Ha pedig ez
teljesül, akkor a fontosabb főkomponensek (melyek varianciája elég nagy)
lesznek csupán érdekesek, hiszen ezek fogják az adatok „dimenzióit''
mérni. Természetesen nagyon fontos azt is tudnunk, hogy rengeteg eredeti
változónk van, és legtöbbjük ugyanazt, vagy legalábbis hasonló dolgokat
mér.

\hypertarget{a-fux151komponens-elemzuxe9s-menete}{%
\section{A főkomponens elemzés
menete}\label{a-fux151komponens-elemzuxe9s-menete}}

Az eredeti \(X_1,X_2,\dots,X_p\) változókból a
\(Z_i=a_i1 X_1+ a_i2 X_2+\dots + a_ip X_p\) lineáris kombinációk
segítségével kapjuk meg a főkomponenseket, azzal feltétellel, hogy
\(a_{i1}^2+a_{i2}^2+\dots+a_{ip}^2=1\), és az egymás után létrejövő
\(Z_1,Z_2,…,Z_p\) főkomponensek nem korrelálnak egymással.

Gyakran az \(X_1,X_2,\dots,X_p\) változó standardizált értékeiből
indulunk ki, hogy a változók arányosan fejtsék ki hatásukat a
főkomponensekre. A jamovi is így végzi az elemzést. Ekkor a változok
átlaga nulla, szórása és variancia pedig 1 lesz.

A részletek ismertetése nélkül a keresett \(a_{i1},a_{i2},\dots,a_{ip}\)
együtthatók megtalálása egy sajátérték-sajátvektor keresési feladat az
eredeti \(X_1,X_2,\dots,X_p\) változók korrelációs mátrixában. A
megtalált \(p\) darab sajátérték
\$\lambda\_1\geq\lambda\_2\geq\dots\geq\lambda\_p\textgreater0
\$sorrendjét feltételezve, \(\lambda_i\) az \(i.\) főkomponens
varianciáját adja \((\lambda_i=var(Z_i))\), és a megtalált \(p\) darab
sajátvektorból az \(i.\) egyes elemei lesznek a
\(Z_i=a_{i1} X_1+ a_{i2} X_2+\dots+ a_{ip} X_p\) főkomponens
\(a_{i1},a_{i2},\dots,a_{ip}\) együtthatói.

Fontos összefüggés, hogy a főkomponensek (\(Z_i\)-k) varianciájának az
összege egyenlő az eredeti standardizált változók (\(X_i\)-k)
varianciájának összegével, azaz
\(\lambda_1+\lambda_2+\dots+\lambda_p=1+1+\dots+1=p.\)

\hypertarget{a-fux151komponens-elemzuxe9s-alkalmazuxe1si-feltuxe9telei}{%
\section{A főkomponens elemzés alkalmazási
feltételei}\label{a-fux151komponens-elemzuxe9s-alkalmazuxe1si-feltuxe9telei}}

\begin{itemize}
\tightlist
\item
  A főkomponens elemzés során általában 4-5-ször (egyes szerzőknél
  10-szer) nagyobb a mintaelemszám a vizsgált változók számánál.
\item
  A faktoranalízis feltétele, hogy egymással korreláló változókból
  induljunk ki. A Bartlett-féle szferikus próba nullhipotézise, hogy a
  változók korrelálatlanok (vagyis a korrelációs mátrixnak a főátlón
  kívüli elemei csak véletlenül térnek el a nullától). A szignifikáns
  p-érték a kedvező a főkomponens elemzés számára. (Megjegyezzük, hogy a
  túlságosan magas egyirányú korrelációk sem jók, ugyanis ez azt
  okozhatja, hogy a főkomponens elemzésnek nem lesz megoldása, ugyanis
  minden változó egy faktorba kerül.)
\item
  Az MSA (Measure of Sampling Adequecy) az egyes változók esetében
  mutatja meg, hogy mennyire van szoros kapcsolatban a többi változóval.
  Érdemes a 0,5 alatti MSA értékkel rendelkező változókat kizárni az
  elemzésből. Értéke 0 és 1 közötti lehet.
\item
  A Kaiser-Meyer-Olkin- (KMO) kritérium az MSA értékek átlaga. Míg az
  MSA érték az egyes változókra vonatkozik, a KMO az összes változóra
  egyidejűleg. A KMO mutatószám jelentését a következőképpen ítélhetjük
  meg:

  \begin{itemize}
  \tightlist
  \item
    KMO ≥ 0,9 kiváló
  \item
    KMO ≥ 0,8 nagyon jó
  \item
    KMO ≥ 0,7 megfelelő
  \item
    KMO ≥ 0,6 közepes
  \item
    KMO ≥ 0,5 gyenge
  \item
    KMO \textless{} 0,5 elfogadhatatlan.
  \end{itemize}
\end{itemize}

\hypertarget{a-fux151komponensek-forgatuxe1sa-rotuxe1ciuxf3}{%
\section{A főkomponensek forgatása
(rotáció)}\label{a-fux151komponensek-forgatuxe1sa-rotuxe1ciuxf3}}

A faktorkiválasztás (extrakció) során az elemzés elsődleges célja, hogy
maximalizálja a főkomponensek varianciáját, amely eredményeként
megkapjuk a rotálatlan faktorsúly-mátrixot. A faktorsúly az eredeti
változó és az adott faktor közötti korrelációt mutatja, amelynek értéke
a korrelációs együtthatókhoz hasonlóan -1 és 1 között változhat.

A faktorkiválasztás során azonban előfordulhat, hogy olyan változók
fognak korrelálni egy adott faktorral, amelyeknek semmi közük egymáshoz,
ezáltal lehetetlenné téve az értelmezést. Ezen a problémán segít a
forgatás, vagy más néven rotáció. A faktor-rotáció azt jelenti, hogy a
faktorok tengelyeit elforgatjuk úgy, hogy egyszerűbb és értelmezhetőbb
faktormegoldáshoz vezessen.

A rotáció (forgatás) során nem változnak sem a kommunalitás, sem pedig
az összes magyarázott variancia, csak a faktorok
sajátértékei/magyarázott varianciái módosulnak.

A rotáláson belül két típust különböztetünk meg: a derékszögű
(ortogonális) (Varimax, Equimax, Quartimax) és a hegyesszögű (nem
ortogonális) (Direct Oblimin, Promax) forgatási módszereket.

A derékszögű esetében a tengelyek merőlegesen állnak egymásra, ezáltal a
faktorok nem korrelálnak egymással, míg a hegyesszögű esetében ezek
tetszőleges szöget zárnak be egymással, vagyis a faktorok korrelálni
fognak egymással.

\hypertarget{puxe9lda-nuxe9gy-tantuxe1rgy-osztuxe1lyzata}{%
\section{Példa: Négy tantárgy
osztályzata}\label{puxe9lda-nuxe9gy-tantuxe1rgy-osztuxe1lyzata}}

\begin{itemize}
\tightlist
\item
  A példa forrása: Münnich és mtsai. (2006)
  \href{https://psycho.unideb.hu/statisztika/pages/p_2_4.html}{2.3.1
  fejezettől Probléma}
\item
  Kapcsolódó jamovi állomány: \texttt{fokomp\_elemzes\_tantargyak.omv}.
\end{itemize}

\textbf{1. Határozzuk meg a korrelációs mátrixot (jamovi-ban:
\texttt{Regression\ /\ Correlation\ matrix})}

Az adatok a \texttt{fokomp\_elemzes\_tantargyak.xlsx} állományban
találhatók.

\begin{Shaded}
\begin{Highlighting}[]
\NormalTok{d }\OtherTok{\textless{}{-}}\NormalTok{ rio}\SpecialCharTok{::}\FunctionTok{import}\NormalTok{(}\AttributeTok{file =} \StringTok{"adat/fokomp\_elemzes\_tantargyak.xlsx"}\NormalTok{)}
\FunctionTok{str}\NormalTok{(d)}
\CommentTok{\#\textgreater{} \textquotesingle{}data.frame\textquotesingle{}:    9 obs. of  4 variables:}
\CommentTok{\#\textgreater{}  $ matek      : num  5 4 3 2 5 1 5 2 5}
\CommentTok{\#\textgreater{}  $ fizika     : num  5 5 3 3 4 2 4 3 5}
\CommentTok{\#\textgreater{}  $ informatika: num  4 4 4 2 5 1 5 2 5}
\CommentTok{\#\textgreater{}  $ kemia      : num  5 5 3 3 5 1 5 3 5}
\NormalTok{d}
\CommentTok{\#\textgreater{}   matek fizika informatika kemia}
\CommentTok{\#\textgreater{} 1     5      5           4     5}
\CommentTok{\#\textgreater{} 2     4      5           4     5}
\CommentTok{\#\textgreater{} 3     3      3           4     3}
\CommentTok{\#\textgreater{} 4     2      3           2     3}
\CommentTok{\#\textgreater{} 5     5      4           5     5}
\CommentTok{\#\textgreater{} 6     1      2           1     1}
\CommentTok{\#\textgreater{} 7     5      4           5     5}
\CommentTok{\#\textgreater{} 8     2      3           2     3}
\CommentTok{\#\textgreater{} 9     5      5           5     5}
\end{Highlighting}
\end{Shaded}

\begin{Shaded}
\begin{Highlighting}[]
\FunctionTok{cor}\NormalTok{(d)}
\CommentTok{\#\textgreater{}                 matek    fizika informatika     kemia}
\CommentTok{\#\textgreater{} matek       1.0000000 0.8712476   0.9492623 0.9499475}
\CommentTok{\#\textgreater{} fizika      0.8712476 1.0000000   0.7662496 0.9271176}
\CommentTok{\#\textgreater{} informatika 0.9492623 0.7662496   1.0000000 0.8867155}
\CommentTok{\#\textgreater{} kemia       0.9499475 0.9271176   0.8867155 1.0000000}
\end{Highlighting}
\end{Shaded}

\begin{figure}

{\centering \includegraphics{./images/fokomponens_kep_01.jpg}

}

\caption{Korrelációs mátrix meghatározása}

\end{figure}

A korrelációs mátrix adatai arra utalnak, hogy szoros kapcsolat van a
változók között. A korrelációs értékek nullánál nagyobbak, ami azonos
irányú tendenciákra utal. E két mátrix is alátámasztja a
feltételezésünket, hogy a változók szorosan együtt változnak.

\textbf{2. Ellenőrizzük le az adatok alkalmasságát (jamovi-ban:
\texttt{Factor\ /\ Principal\ Component\ Analysis})}

A változóink eleget tesznek a Bartlett-féle szferikus próbának, a
korrelációs mátrix nem egységmátrix \((p<0,001)\), az MSA értékek is
nagyobban \(0,5\)-nél és a KMO érték is megfelelő.

\begin{figure}

{\centering \includegraphics{./images/fokomponens_kep_02.jpg}

}

\caption{Alkalmazási feltételek ellenőrzése}

\end{figure}

\textbf{3. Határozzuk meg a komponensek számát}

Elvileg annyi főkomponenst lehet kiszámolni, ahány változónk van, a
célunk azonban a komponensek számának minimalizálása.

Több eljárás létezik a főkomponensek számának meghatározására:

\begin{itemize}
\tightlist
\item
  Horn-féle párhuzamos analízis (jamovi-ban:
  \texttt{Based\ on\ parallel\ analysis}): modern eljárás, amely
  szimuláció segítségével állapítja meg a főkomponensek számát (Horn,
  1965).
\item
  A priori meghatározás (jamovi-ban: \texttt{Fixed\ number}): korábbi
  ismerete alapján megadjuk a főkomponensek számát.
\item
  Sajátértéken alapuló megoldás (jamovi-ban:
  \texttt{Based\ on\ eigenvalue}): tipikusan csak az 1-nél nagyobb
  sajátértékű faktorokat tartjuk bent a modellben. Az 1-nél kisebb
  varianciájú faktorok ugyanis nem jobbak mint az eredeti standardizált
  változók
\item
  Sajátértékábrán (scree-plot, kőtörmelék ábra) alapuló meghatározás
  (jamovi-ban: \texttt{Scree\ plot}): a sajátérték ábra a sajátértékek
  ábrázolása a főkomponensek sorrendjében. Az ábra formája alapján lehet
  következtetni a főkomponensek számára: ahol a görbe meredekségében van
  egy határozott törés, meredekebb rész után laposabb jön. Ahol tehát a
  görbe laposodása elkezdődik, az a figyelembe vett főkomponensek
  megfelelő száma.
\item
  Magyarázott varianciahányadon alapuló meghatározás (jamovi-ban:
  \texttt{Component\ summary}): ekkor az előállított főkomponensek
  számát úgy határozzuk meg, hogy a főkomponensek által magyarázott
  variancia kumulált százalékos értéke elérjen egy megfelelő szintet. A
  megfelelő szint (60\%-95\%-ig) a probléma jellegétől függ.
\end{itemize}

A Horn-féle párhuzamos elemzés 1 főkomponenst javasol.

\begin{figure}

{\centering \includegraphics{./images/fokomponens_kep_03.jpg}

}

\caption{Főkomponensek számának meghatározása}

\end{figure}

\textbf{4. Válasszunk forgatást (jamovi-ban: \texttt{Rotation})}

A jamovi alapértelmezés szerint a Varimax forgatást ajánlja, amely
derékszögű koordinátatengelyeket eredményez és a legtöbb esetben ez a
megfelelő választás. Lehetőségünk van ezen módosítani. Az összes
lehetőség:

\begin{itemize}
\tightlist
\item
  None -- rotálatlan elemzés
\item
  Varimax
\item
  Qartimax
\item
  Promax
\item
  Oblimin
\item
  Simplimax
\end{itemize}

Mivel egyetlen főkomponensünk van, így nem változtatunk az
alapértelmezett Varimax beállításon.

\textbf{5. A főkomponens elemzés eredménye}

\emph{Komponens mátrix (jamoviban: \texttt{Component\ loadings})}

A főkomponens elemzés eredménye a komponens mátrix (faktormátrix),
amelynek soraiban az eredeti változók, oszlopaiban a kinyert
főkomponensek vannak. A cellákban a komponens súlyok (faktorsúlyok)
szerepelnek, amelyek a főkomponens és a változó közötti korrelációt
jelentik. Ezek egyben a főkomponensek azon együtthatói, amelyekkel a
standardizált változó a főkomponensekkel kifejezhető.

A magas abszolút értékű faktorsúly azt jelzi, hogy komponens és a
változó szorosan összefügg.

A változókat tartalmazó sorok rendezhetők a faktorsúlyok csökkenő
sorrendjében (jamovi-ban: \texttt{Sort\ loading\ by\ size})

Az adott értéknél kisebb faktorsúlyok elrejthetők a táblázatban
(jamoviban: \texttt{Hide\ loadings\ below})

A \texttt{Uniqueness} oszlopban az egyes változók „egyediségét'' is
láthatjuk. Az egyediség a variancia azon aránya, amely „egyedi'' a
változóra nézve, és nem magyarázható a komponensekkel. Vegyük
figyelembe, hogy minél nagyobb az „egyediség'', annál kisebb a változó
relevanciája/hozzájárulása a modellben.

\emph{A kezdő sajátértékek (jamovi-ban: \texttt{Initial\ eigenvalues})}

A kezdő sajátértékek táblázat a sajátértékeket adja meg. A komponensek
sajátértékei csökkenő nagyságúak, ahogy az 1. komponenstől a 4.
komponensig haladunk. A komponens sajátértéke kifejezi a komponens által
magyarázott teljes varianciát. A 4 komponens összvarianciája pontosan 4.
A további két oszlopban ez alapján számoljuk a százalékos és a kummulált
százalékos varianciát.

\emph{A komponensek összegzése (jamovi-ban:
\texttt{Component\ summary})}

A komponensek összegzése táblázat tartalmazza a megtartott
komponenseket, a magyarázott varianciát, illetve utóbbit százalékosan is
kifejezve. Vegyük észre, hogy ez a sor teljesen megegyezik a kezdő
sajátértékek táblázat első sorával. Az \texttt{SS\ Loadings} felirat
magyarázata, hogy magyarázott variancia a komponenshez tartozó
faktorsúlyok négyzetösszege (sum of square).

\begin{figure}

{\centering \includegraphics{./images/fokomponens_kep_04.jpg}

}

\caption{Komponensek összegzése}

\end{figure}

\textbf{6. Főkomponens értékek kiszámítása}

A főkomponens elemzés célja az eredeti változók csökkentése. A
főkomponens(ek) az eredeti változók lineáris kombinációjával
kifejezhetők. Ez(ek) a főkomponens értékek (jamovi-ban:
\texttt{Component\ score}) az adatbázisban is rögzíthetők, és további
elemzések kiindulópontjai lehetnek.

\begin{figure}

{\centering \includegraphics{./images/fokomponens_kep_05.jpg}

}

\caption{Főkomponens értékek kiszámítása}

\end{figure}

Sikerült tehát az érdemjegyeket egyetlen mérőszámmal kifejezni, a fenti
főkomponens érték az, amely a lehető legjobban magában foglalja az egyes
tantárgyakból szerzett jegyeket és ezáltal a reál tantárgyak iránti
fogékonyság mérőszáma lehet. A legjobban a kilencedik személy teljesít a
reál tárgyakból, legrosszabbul pedig a hatodik. Ezek az értékek
standardizáltak, vagyis 0 átlagúak és 1 szórásúak.

\begin{figure}

{\centering \includegraphics{./images/fokomponens_kep_06.jpg}

}

\caption{Főkomponens értékek leíró statisztikája}

\end{figure}

R-ben több lehetőségünk van a főlomponenselemzés elvégzésére.

\begin{Shaded}
\begin{Highlighting}[]
\NormalTok{pca\_1 }\OtherTok{\textless{}{-}} \FunctionTok{prcomp}\NormalTok{(d, }\AttributeTok{scale. =} \ConstantTok{TRUE}\NormalTok{)}
\NormalTok{pca\_1}
\CommentTok{\#\textgreater{} Standard deviations (1, .., p=4):}
\CommentTok{\#\textgreater{} [1] 1.9177188 0.5017701 0.2045278 0.1695572}
\CommentTok{\#\textgreater{} }
\CommentTok{\#\textgreater{} Rotation (n x k) = (4 x 4):}
\CommentTok{\#\textgreater{}                    PC1        PC2         PC3}
\CommentTok{\#\textgreater{} matek       {-}0.5129614  0.2231620 {-}0.02972158}
\CommentTok{\#\textgreater{} fizika      {-}0.4843985 {-}0.7077528  0.50610146}
\CommentTok{\#\textgreater{} informatika {-}0.4900124  0.6474152  0.35260295}
\CommentTok{\#\textgreater{} kemia       {-}0.5119731 {-}0.1736040 {-}0.78654250}
\CommentTok{\#\textgreater{}                     PC4}
\CommentTok{\#\textgreater{} matek       {-}0.82836339}
\CommentTok{\#\textgreater{} fizika       0.09113404}
\CommentTok{\#\textgreater{} informatika  0.46520167}
\CommentTok{\#\textgreater{} kemia        0.29848968}
\end{Highlighting}
\end{Shaded}

\begin{Shaded}
\begin{Highlighting}[]
\NormalTok{pca\_1 }\OtherTok{\textless{}{-}} \FunctionTok{princomp}\NormalTok{(d, }\AttributeTok{cor =} \ConstantTok{TRUE}\NormalTok{)}
\NormalTok{pca\_1}
\CommentTok{\#\textgreater{} Call:}
\CommentTok{\#\textgreater{} princomp(x = d, cor = TRUE)}
\CommentTok{\#\textgreater{} }
\CommentTok{\#\textgreater{} Standard deviations:}
\CommentTok{\#\textgreater{}    Comp.1    Comp.2    Comp.3    Comp.4 }
\CommentTok{\#\textgreater{} 1.9177188 0.5017701 0.2045278 0.1695572 }
\CommentTok{\#\textgreater{} }
\CommentTok{\#\textgreater{}  4  variables and  9 observations.}
\end{Highlighting}
\end{Shaded}

\begin{Shaded}
\begin{Highlighting}[]
\NormalTok{psych}\SpecialCharTok{::}\FunctionTok{pca}\NormalTok{(d, }\AttributeTok{rotate =} \StringTok{"varimax"}\NormalTok{)}
\CommentTok{\#\textgreater{} Principal Components Analysis}
\CommentTok{\#\textgreater{} Call: principal(r = r, nfactors = nfactors, residua...}
\CommentTok{\#\textgreater{}     rotate = rotate, n.obs = n.obs, covar = covar, ...}
\CommentTok{\#\textgreater{}     missing = missing, impute = impute, oblique.sco...}
\CommentTok{\#\textgreater{}     method = method, use = use, cor = cor, correct ...}
\CommentTok{\#\textgreater{} Standardized loadings (pattern matrix) based upon c...}
\CommentTok{\#\textgreater{}              PC1   h2    u2 com}
\CommentTok{\#\textgreater{} matek       0.98 0.97 0.032   1}
\CommentTok{\#\textgreater{} fizika      0.93 0.86 0.137   1}
\CommentTok{\#\textgreater{} informatika 0.94 0.88 0.117   1}
\CommentTok{\#\textgreater{} kemia       0.98 0.96 0.036   1}
\CommentTok{\#\textgreater{} }
\CommentTok{\#\textgreater{}                 PC1}
\CommentTok{\#\textgreater{} SS loadings    3.68}
\CommentTok{\#\textgreater{} Proportion Var 0.92}
\CommentTok{\#\textgreater{} }
\CommentTok{\#\textgreater{} Mean item complexity =  1}
\CommentTok{\#\textgreater{} Test of the hypothesis that 1 component is sufficient.}
\CommentTok{\#\textgreater{} }
\CommentTok{\#\textgreater{} The root mean square of the residuals (RMSR) is  0.05 }
\CommentTok{\#\textgreater{}  with the empirical chi square  0.28  with prob \textless{}  ...}
\CommentTok{\#\textgreater{} }
\CommentTok{\#\textgreater{} Fit based upon off diagonal values = 1}
\end{Highlighting}
\end{Shaded}

\begin{Shaded}
\begin{Highlighting}[]
\NormalTok{pca\_1 }\OtherTok{\textless{}{-}}\NormalTok{ FactoMineR}\SpecialCharTok{::}\FunctionTok{PCA}\NormalTok{(d, }\AttributeTok{graph =} \ConstantTok{FALSE}\NormalTok{)}
\NormalTok{pca\_1}\SpecialCharTok{$}\NormalTok{eig}
\CommentTok{\#\textgreater{}        eigenvalue percentage of variance}
\CommentTok{\#\textgreater{} comp 1 3.67764553             91.9411383}
\CommentTok{\#\textgreater{} comp 2 0.25177321              6.2943303}
\CommentTok{\#\textgreater{} comp 3 0.04183160              1.0457901}
\CommentTok{\#\textgreater{} comp 4 0.02874965              0.7187414}
\CommentTok{\#\textgreater{}        cumulative percentage of variance}
\CommentTok{\#\textgreater{} comp 1                          91.94114}
\CommentTok{\#\textgreater{} comp 2                          98.23547}
\CommentTok{\#\textgreater{} comp 3                          99.28126}
\CommentTok{\#\textgreater{} comp 4                         100.00000}
\NormalTok{pca\_1}\SpecialCharTok{$}\NormalTok{var}
\CommentTok{\#\textgreater{} $coord}
\CommentTok{\#\textgreater{}                 Dim.1      Dim.2        Dim.3}
\CommentTok{\#\textgreater{} matek       0.9837158 {-}0.1119760 {-}0.006078888}
\CommentTok{\#\textgreater{} fizika      0.9289402  0.3551292  0.103511795}
\CommentTok{\#\textgreater{} informatika 0.9397061 {-}0.3248536  0.072117091}
\CommentTok{\#\textgreater{} kemia       0.9818205  0.0871093 {-}0.160869772}
\CommentTok{\#\textgreater{}                   Dim.4}
\CommentTok{\#\textgreater{} matek       {-}0.14045500}
\CommentTok{\#\textgreater{} fizika       0.01545243}
\CommentTok{\#\textgreater{} informatika  0.07887831}
\CommentTok{\#\textgreater{} kemia        0.05061108}
\CommentTok{\#\textgreater{} }
\CommentTok{\#\textgreater{} $cor}
\CommentTok{\#\textgreater{}                 Dim.1      Dim.2        Dim.3}
\CommentTok{\#\textgreater{} matek       0.9837158 {-}0.1119760 {-}0.006078888}
\CommentTok{\#\textgreater{} fizika      0.9289402  0.3551292  0.103511795}
\CommentTok{\#\textgreater{} informatika 0.9397061 {-}0.3248536  0.072117091}
\CommentTok{\#\textgreater{} kemia       0.9818205  0.0871093 {-}0.160869772}
\CommentTok{\#\textgreater{}                   Dim.4}
\CommentTok{\#\textgreater{} matek       {-}0.14045500}
\CommentTok{\#\textgreater{} fizika       0.01545243}
\CommentTok{\#\textgreater{} informatika  0.07887831}
\CommentTok{\#\textgreater{} kemia        0.05061108}
\CommentTok{\#\textgreater{} }
\CommentTok{\#\textgreater{} $cos2}
\CommentTok{\#\textgreater{}                 Dim.1       Dim.2        Dim.3}
\CommentTok{\#\textgreater{} matek       0.9676968 0.012538628 3.695288e{-}05}
\CommentTok{\#\textgreater{} fizika      0.8629298 0.126116721 1.071469e{-}02}
\CommentTok{\#\textgreater{} informatika 0.8830475 0.105529830 5.200875e{-}03}
\CommentTok{\#\textgreater{} kemia       0.9639714 0.007588031 2.587908e{-}02}
\CommentTok{\#\textgreater{}                    Dim.4}
\CommentTok{\#\textgreater{} matek       0.0197276074}
\CommentTok{\#\textgreater{} fizika      0.0002387777}
\CommentTok{\#\textgreater{} informatika 0.0062217871}
\CommentTok{\#\textgreater{} kemia       0.0025614817}
\CommentTok{\#\textgreater{} }
\CommentTok{\#\textgreater{} $contrib}
\CommentTok{\#\textgreater{}                Dim.1     Dim.2       Dim.3      Dim.4}
\CommentTok{\#\textgreater{} matek       26.31294  4.980128  0.08833723 68.6185908}
\CommentTok{\#\textgreater{} fizika      23.46419 50.091398 25.61386842  0.8305413}
\CommentTok{\#\textgreater{} informatika 24.01122 41.914638 12.43288421 21.6412590}
\CommentTok{\#\textgreater{} kemia       26.21165  3.013836 61.86491013  8.9096089}
\NormalTok{factoextra}\SpecialCharTok{::}\FunctionTok{fviz\_eig}\NormalTok{(pca\_1, }\AttributeTok{addlabels =} \ConstantTok{TRUE}\NormalTok{, }\AttributeTok{ylim =} \FunctionTok{c}\NormalTok{(}\DecValTok{0}\NormalTok{, }\DecValTok{110}\NormalTok{))}
\NormalTok{factoextra}\SpecialCharTok{::}\FunctionTok{get\_eigenvalue}\NormalTok{(pca\_1)}
\CommentTok{\#\textgreater{}       eigenvalue variance.percent}
\CommentTok{\#\textgreater{} Dim.1 3.67764553       91.9411383}
\CommentTok{\#\textgreater{} Dim.2 0.25177321        6.2943303}
\CommentTok{\#\textgreater{} Dim.3 0.04183160        1.0457901}
\CommentTok{\#\textgreater{} Dim.4 0.02874965        0.7187414}
\CommentTok{\#\textgreater{}       cumulative.variance.percent}
\CommentTok{\#\textgreater{} Dim.1                    91.94114}
\CommentTok{\#\textgreater{} Dim.2                    98.23547}
\CommentTok{\#\textgreater{} Dim.3                    99.28126}
\CommentTok{\#\textgreater{} Dim.4                   100.00000}
\NormalTok{factoextra}\SpecialCharTok{::}\FunctionTok{get\_pca\_ind}\NormalTok{(pca\_1)}
\CommentTok{\#\textgreater{} Principal Component Analysis Results for individuals}
\CommentTok{\#\textgreater{}  ===================================================}
\CommentTok{\#\textgreater{}   Name       Description                       }
\CommentTok{\#\textgreater{} 1 "$coord"   "Coordinates for the individuals" }
\CommentTok{\#\textgreater{} 2 "$cos2"    "Cos2 for the individuals"        }
\CommentTok{\#\textgreater{} 3 "$contrib" "contributions of the individuals"}
\NormalTok{factoextra}\SpecialCharTok{::}\FunctionTok{get\_pca\_var}\NormalTok{(pca\_1)}
\CommentTok{\#\textgreater{} Principal Component Analysis Results for variables}
\CommentTok{\#\textgreater{}  ===================================================}
\CommentTok{\#\textgreater{}   Name      }
\CommentTok{\#\textgreater{} 1 "$coord"  }
\CommentTok{\#\textgreater{} 2 "$cor"    }
\CommentTok{\#\textgreater{} 3 "$cos2"   }
\CommentTok{\#\textgreater{} 4 "$contrib"}
\CommentTok{\#\textgreater{}   Description                                    }
\CommentTok{\#\textgreater{} 1 "Coordinates for the variables"                }
\CommentTok{\#\textgreater{} 2 "Correlations between variables and dimensions"}
\CommentTok{\#\textgreater{} 3 "Cos2 for the variables"                       }
\CommentTok{\#\textgreater{} 4 "contributions of the variables"}
\NormalTok{factoextra}\SpecialCharTok{::}\FunctionTok{fviz\_pca\_ind}\NormalTok{(pca\_1)}
\NormalTok{factoextra}\SpecialCharTok{::}\FunctionTok{fviz\_pca\_var}\NormalTok{(pca\_1)}
\NormalTok{factoextra}\SpecialCharTok{::}\FunctionTok{fviz\_pca\_biplot}\NormalTok{(pca\_1)}
\NormalTok{corrplot}\SpecialCharTok{::}\FunctionTok{corrplot}\NormalTok{(pca\_1}\SpecialCharTok{$}\NormalTok{var}\SpecialCharTok{$}\NormalTok{cos2, }\AttributeTok{is.corr =} \ConstantTok{FALSE}\NormalTok{)}
\end{Highlighting}
\end{Shaded}

\begin{figure}[H]

{\centering \includegraphics{./sec_fokomponens_elemzes_files/figure-pdf/unnamed-chunk-7-1.pdf}

}

\end{figure}

\begin{figure}[H]

{\centering \includegraphics{./sec_fokomponens_elemzes_files/figure-pdf/unnamed-chunk-7-2.pdf}

}

\end{figure}

\begin{figure}[H]

{\centering \includegraphics{./sec_fokomponens_elemzes_files/figure-pdf/unnamed-chunk-7-3.pdf}

}

\end{figure}

\begin{figure}[H]

{\centering \includegraphics{./sec_fokomponens_elemzes_files/figure-pdf/unnamed-chunk-7-4.pdf}

}

\end{figure}

\begin{figure}[H]

{\centering \includegraphics{./sec_fokomponens_elemzes_files/figure-pdf/unnamed-chunk-7-5.pdf}

}

\end{figure}

\hypertarget{puxe9lda-luxe9tezik-a-reuxe1l-tuxe1rgyak-iruxe1nti-foguxe9konysuxe1g}{%
\section{Példa: Létezik a reál tárgyak iránti
fogékonyság?}\label{puxe9lda-luxe9tezik-a-reuxe1l-tuxe1rgyak-iruxe1nti-foguxe9konysuxe1g}}

\begin{itemize}
\tightlist
\item
  A példa forrása: Münnich és mtsai. (2006)
  \href{https://psycho.unideb.hu/statisztika/pages/p_2_11.html}{2.5.1
  Probléma}
\item
  Kapcsolódó jamovi állomány: \texttt{fokomp\_real\_targyak.omv}
\end{itemize}

Korábban már foglalkoztunk azzal a felvetéssel, hogy néhány tantárgy
eredményeit egyetlen mérőszámmal reprezentáljuk. Korábbi példánkban a
matematika, fizika, informatika és kémia jegyek közötti összefüggéseket
vizsgáltuk egy kisebb adatbázison, most egy sokkal nagyobb adatbázis
segítségével mutatjuk be, hogyan végezhetünk főkomponens-analízist. Az
adatok a \texttt{fokomp\_real\_targyak.xlsx} állományban találhatók.

\begin{Shaded}
\begin{Highlighting}[]
\NormalTok{d }\OtherTok{\textless{}{-}}\NormalTok{ rio}\SpecialCharTok{::}\FunctionTok{import}\NormalTok{(}\AttributeTok{file =} \StringTok{"adat/fokomp\_real\_targyak.xlsx"}\NormalTok{)}
\FunctionTok{str}\NormalTok{(d)}
\CommentTok{\#\textgreater{} \textquotesingle{}data.frame\textquotesingle{}:    30 obs. of  4 variables:}
\CommentTok{\#\textgreater{}  $ matek      : num  5 4 3 2 5 1 5 2 5 5 ...}
\CommentTok{\#\textgreater{}  $ fizika     : num  5 5 3 3 4 2 4 3 5 3 ...}
\CommentTok{\#\textgreater{}  $ informatika: num  4 4 4 2 5 1 5 2 5 4 ...}
\CommentTok{\#\textgreater{}  $ kemia      : num  5 5 3 3 5 1 5 3 5 5 ...}
\NormalTok{psych}\SpecialCharTok{::}\FunctionTok{headTail}\NormalTok{(d)}
\CommentTok{\#\textgreater{}     matek fizika informatika kemia}
\CommentTok{\#\textgreater{} 1       5      5           4     5}
\CommentTok{\#\textgreater{} 2       4      5           4     5}
\CommentTok{\#\textgreater{} 3       3      3           4     3}
\CommentTok{\#\textgreater{} 4       2      3           2     3}
\CommentTok{\#\textgreater{} ...   ...    ...         ...   ...}
\CommentTok{\#\textgreater{} 27      5      5           5     5}
\CommentTok{\#\textgreater{} 28      5      4           5     5}
\CommentTok{\#\textgreater{} 29      2      3           2     3}
\CommentTok{\#\textgreater{} 30      5      4           5     5}
\end{Highlighting}
\end{Shaded}

Hozzuk létre a korrelációs mátrixot!

\begin{Shaded}
\begin{Highlighting}[]
\FunctionTok{cor}\NormalTok{(d)}
\CommentTok{\#\textgreater{}                 matek    fizika informatika     kemia}
\CommentTok{\#\textgreater{} matek       1.0000000 0.7746503   0.9391649 0.9075545}
\CommentTok{\#\textgreater{} fizika      0.7746503 1.0000000   0.7038821 0.8344007}
\CommentTok{\#\textgreater{} informatika 0.9391649 0.7038821   1.0000000 0.8172819}
\CommentTok{\#\textgreater{} kemia       0.9075545 0.8344007   0.8172819 1.0000000}
\end{Highlighting}
\end{Shaded}

Látható, hogy a négy tantárgy jegyei viszonylag összhangban vannak
egymással abban az értelemben, hogy azok a diákok, akik az egyik
tárgyból jól teljesítenek, azok a másik három tárgyból is.

Ezek alapján van egy olyan sejtésünk, hogy egy úgynevezett reál tárgyak
iránti fogékonyság mutatóval reprezentálhatjuk a négy tantárgy
eredményeit. Vagyis főkomponens-analízis segítségével ellenőrizhetjük,
hogy az adatok valóban jól sűríthetőek-e egyetlen dimenzióba vagy
mérőszámba, és ha igen, akkor ezt a dimenziót elnevezhetjük reál tárgyak
iránti fogékonyságnak.

\begin{Shaded}
\begin{Highlighting}[]
\NormalTok{psych}\SpecialCharTok{::}\FunctionTok{pca}\NormalTok{(d, }\AttributeTok{rotate =} \StringTok{"varimax"}\NormalTok{)}
\CommentTok{\#\textgreater{} Principal Components Analysis}
\CommentTok{\#\textgreater{} Call: principal(r = r, nfactors = nfactors, residua...}
\CommentTok{\#\textgreater{}     rotate = rotate, n.obs = n.obs, covar = covar, ...}
\CommentTok{\#\textgreater{}     missing = missing, impute = impute, oblique.sco...}
\CommentTok{\#\textgreater{}     method = method, use = use, cor = cor, correct ...}
\CommentTok{\#\textgreater{} Standardized loadings (pattern matrix) based upon c...}
\CommentTok{\#\textgreater{}              PC1   h2    u2 com}
\CommentTok{\#\textgreater{} matek       0.97 0.94 0.057   1}
\CommentTok{\#\textgreater{} fizika      0.88 0.78 0.220   1}
\CommentTok{\#\textgreater{} informatika 0.93 0.86 0.139   1}
\CommentTok{\#\textgreater{} kemia       0.95 0.91 0.091   1}
\CommentTok{\#\textgreater{} }
\CommentTok{\#\textgreater{}                 PC1}
\CommentTok{\#\textgreater{} SS loadings    3.49}
\CommentTok{\#\textgreater{} Proportion Var 0.87}
\CommentTok{\#\textgreater{} }
\CommentTok{\#\textgreater{} Mean item complexity =  1}
\CommentTok{\#\textgreater{} Test of the hypothesis that 1 component is sufficient.}
\CommentTok{\#\textgreater{} }
\CommentTok{\#\textgreater{} The root mean square of the residuals (RMSR) is  0.07 }
\CommentTok{\#\textgreater{}  with the empirical chi square  1.59  with prob \textless{}  ...}
\CommentTok{\#\textgreater{} }
\CommentTok{\#\textgreater{} Fit based upon off diagonal values = 0.99}
\end{Highlighting}
\end{Shaded}

\begin{figure}

{\centering \includegraphics{./images/fokomponens_kep_07.jpg}

}

\caption{Főkomponens elemzés - Reál tantárgyak iránti fogékonyság}

\end{figure}

Összességében az adatok jól sűríthetők egyetlen mérőszámba, minimális
információveszteséggel, ezt a mutatót pedig hívhatjuk a reál tárgyak
iránti fogékonyság mutatójának.

\hypertarget{puxe9lda-egy-kuxe9rdux151uxedv-szerkesztuxe9suxe9nek-probluxe9muxe1i}{%
\section{Példa: Egy kérdőív szerkesztésének
problémái}\label{puxe9lda-egy-kuxe9rdux151uxedv-szerkesztuxe9suxe9nek-probluxe9muxe1i}}

\begin{itemize}
\tightlist
\item
  A példa forrása: Münnich és mtsai. (2006)
  \href{https://psycho.unideb.hu/statisztika/pages/p_2_12.html}{2.5.2
  Probléma}
\item
  Kapcsolódó jamovi állomány: \texttt{fokomp\_kerdoivtervezet.omv}
\end{itemize}

Kérdőívek kialakításkor gyakran előfordul az a probléma, hogy egy
dimenzió mérésére nem áll rendelkezésünkre valamilyen bevált mérőeszköz,
hanem magunknak kell egyet kialakítani. Egy jó kérdőív kialakítása
hosszas és nagyon alapos munkát igényel. Ez a folyamat nagyvonalakban
úgy néz ki, hogy az összeállított itemeket először egy kisebb mintán
teszteljük (elővizsgálat), majd megnézzük, hogy az itemek valóban úgy
„működnek-e'', ahogyan azt mi feltételeztük. Ez jelenti egyrészt a teszt
megbízhatóságának, másrészt érvényességének vizsgálatát.

A megbízhatóság vizsgálatának egyik módszere, hogy megnézzük, az itemek
valóban egy dimenzióra illeszkednek-e. A nem odaillő itemeket pedig
kivesszük a kérdőívből (itemszelekció). Ennek módszerei lehetnek a

\begin{itemize}
\tightlist
\item
  főkomponens-analízis
\item
  Cronbach-alfán lapuló
\end{itemize}

Az adatok a \texttt{fokomp\_kerdoivtervezet.xlsx} állományban
találhatók.

\begin{Shaded}
\begin{Highlighting}[]
\NormalTok{d }\OtherTok{\textless{}{-}}\NormalTok{ rio}\SpecialCharTok{::}\FunctionTok{import}\NormalTok{(}\AttributeTok{file =} \StringTok{"adat/fokomp\_kerdoivtervezet.xlsx"}\NormalTok{)}
\FunctionTok{str}\NormalTok{(d)}
\CommentTok{\#\textgreater{} \textquotesingle{}data.frame\textquotesingle{}:    125 obs. of  10 variables:}
\CommentTok{\#\textgreater{}  $ K\_1 : num  7 5 4 2 3 3 7 4 7 2 ...}
\CommentTok{\#\textgreater{}  $ K\_2 : num  7 5 4 2 2 3 7 3 7 2 ...}
\CommentTok{\#\textgreater{}  $ K\_3 : num  6 5 2 2 4 3 7 3 7 3 ...}
\CommentTok{\#\textgreater{}  $ K\_4 : num  7 5 5 4 2 3 7 3 7 3 ...}
\CommentTok{\#\textgreater{}  $ K\_5 : num  7 5 4 4 3 3 7 3 7 3 ...}
\CommentTok{\#\textgreater{}  $ K\_6 : num  5 3 4 2 1 3 5 3 4 1 ...}
\CommentTok{\#\textgreater{}  $ K\_7 : num  5 3 2 6 6 6 5 3 4 1 ...}
\CommentTok{\#\textgreater{}  $ K\_8 : num  7 5 4 2 3 3 7 3 7 3 ...}
\CommentTok{\#\textgreater{}  $ K\_9 : num  5 3 4 2 3 6 5 3 4 4 ...}
\CommentTok{\#\textgreater{}  $ K\_10: num  5 3 4 3 2 2 7 3 7 3 ...}
\NormalTok{psych}\SpecialCharTok{::}\FunctionTok{headTail}\NormalTok{(d)}
\CommentTok{\#\textgreater{}     K\_1 K\_2 K\_3 K\_4 K\_5 K\_6 K\_7 K\_8 K\_9 K\_10}
\CommentTok{\#\textgreater{} 1     7   7   6   7   7   5   5   7   5    5}
\CommentTok{\#\textgreater{} 2     5   5   5   5   5   3   3   5   3    3}
\CommentTok{\#\textgreater{} 3     4   4   2   5   4   4   2   4   4    4}
\CommentTok{\#\textgreater{} 4     2   2   2   4   4   2   6   2   2    3}
\CommentTok{\#\textgreater{} ... ... ... ... ... ... ... ... ... ...  ...}
\CommentTok{\#\textgreater{} 122   4   4   4   4   4   2   2   4   2    2}
\CommentTok{\#\textgreater{} 123   2   2   2   2   2   4   1   4   3    3}
\CommentTok{\#\textgreater{} 124   1   1   1   1   1   7   7   1   7    7}
\CommentTok{\#\textgreater{} 125   3   3   3   3   3   3   3   3   3    3}
\end{Highlighting}
\end{Shaded}

Először végezzünk főkomponens elemzést a változókra, hogy képet
kaphassunk az adatok egymáshoz való viszonyáról. Kezdjük a korrelációs
mátrixszal.

\begin{Shaded}
\begin{Highlighting}[]
\FunctionTok{print}\NormalTok{(}\FunctionTok{cor}\NormalTok{(d), }\AttributeTok{digits =} \DecValTok{2}\NormalTok{)}
\CommentTok{\#\textgreater{}        K\_1    K\_2    K\_3    K\_4    K\_5   K\_6   K\_7}
\CommentTok{\#\textgreater{} K\_1   1.00  0.935  0.824  0.819  0.889 {-}0.16 {-}0.30}
\CommentTok{\#\textgreater{} K\_2   0.93  1.000  0.824  0.807  0.884 {-}0.14 {-}0.26}
\CommentTok{\#\textgreater{} K\_3   0.82  0.824  1.000  0.774  0.822 {-}0.20 {-}0.26}
\CommentTok{\#\textgreater{} K\_4   0.82  0.807  0.774  1.000  0.826 {-}0.12 {-}0.26}
\CommentTok{\#\textgreater{} K\_5   0.89  0.884  0.822  0.826  1.000 {-}0.23 {-}0.31}
\CommentTok{\#\textgreater{} K\_6  {-}0.16 {-}0.140 {-}0.203 {-}0.117 {-}0.233  1.00  0.63}
\CommentTok{\#\textgreater{} K\_7  {-}0.30 {-}0.264 {-}0.257 {-}0.256 {-}0.315  0.63  1.00}
\CommentTok{\#\textgreater{} K\_8   0.75  0.753  0.702  0.695  0.786 {-}0.19 {-}0.31}
\CommentTok{\#\textgreater{} K\_9  {-}0.12 {-}0.099 {-}0.164 {-}0.109 {-}0.144  0.69  0.53}
\CommentTok{\#\textgreater{} K\_10  0.10  0.106  0.095 {-}0.031  0.076  0.42  0.25}
\CommentTok{\#\textgreater{}           K\_8    K\_9     K\_10}
\CommentTok{\#\textgreater{} K\_1   0.74648 {-}0.123  0.10038}
\CommentTok{\#\textgreater{} K\_2   0.75350 {-}0.099  0.10637}
\CommentTok{\#\textgreater{} K\_3   0.70244 {-}0.164  0.09462}
\CommentTok{\#\textgreater{} K\_4   0.69481 {-}0.109 {-}0.03121}
\CommentTok{\#\textgreater{} K\_5   0.78636 {-}0.144  0.07551}
\CommentTok{\#\textgreater{} K\_6  {-}0.18707  0.689  0.41988}
\CommentTok{\#\textgreater{} K\_7  {-}0.31395  0.533  0.24515}
\CommentTok{\#\textgreater{} K\_8   1.00000 {-}0.266 {-}0.00063}
\CommentTok{\#\textgreater{} K\_9  {-}0.26564  1.000  0.46786}
\CommentTok{\#\textgreater{} K\_10 {-}0.00063  0.468  1.00000}
\end{Highlighting}
\end{Shaded}

\begin{Shaded}
\begin{Highlighting}[]
\NormalTok{psych}\SpecialCharTok{::}\FunctionTok{pca}\NormalTok{(d, }\AttributeTok{rotate =} \StringTok{"varimax"}\NormalTok{)}
\CommentTok{\#\textgreater{} Principal Components Analysis}
\CommentTok{\#\textgreater{} Call: principal(r = r, nfactors = nfactors, residua...}
\CommentTok{\#\textgreater{}     rotate = rotate, n.obs = n.obs, covar = covar, ...}
\CommentTok{\#\textgreater{}     missing = missing, impute = impute, oblique.sco...}
\CommentTok{\#\textgreater{}     method = method, use = use, cor = cor, correct ...}
\CommentTok{\#\textgreater{} Standardized loadings (pattern matrix) based upon c...}
\CommentTok{\#\textgreater{}        PC1      h2   u2 com}
\CommentTok{\#\textgreater{} K\_1   0.93 0.87216 0.13   1}
\CommentTok{\#\textgreater{} K\_2   0.93 0.85839 0.14   1}
\CommentTok{\#\textgreater{} K\_3   0.89 0.78819 0.21   1}
\CommentTok{\#\textgreater{} K\_4   0.87 0.76526 0.23   1}
\CommentTok{\#\textgreater{} K\_5   0.94 0.88251 0.12   1}
\CommentTok{\#\textgreater{} K\_6  {-}0.34 0.11231 0.89   1}
\CommentTok{\#\textgreater{} K\_7  {-}0.45 0.19850 0.80   1}
\CommentTok{\#\textgreater{} K\_8   0.85 0.72287 0.28   1}
\CommentTok{\#\textgreater{} K\_9  {-}0.30 0.08965 0.91   1}
\CommentTok{\#\textgreater{} K\_10 {-}0.02 0.00027 1.00   1}
\CommentTok{\#\textgreater{} }
\CommentTok{\#\textgreater{}                 PC1}
\CommentTok{\#\textgreater{} SS loadings    5.29}
\CommentTok{\#\textgreater{} Proportion Var 0.53}
\CommentTok{\#\textgreater{} }
\CommentTok{\#\textgreater{} Mean item complexity =  1}
\CommentTok{\#\textgreater{} Test of the hypothesis that 1 component is sufficient.}
\CommentTok{\#\textgreater{} }
\CommentTok{\#\textgreater{} The root mean square of the residuals (RMSR) is  0.19 }
\CommentTok{\#\textgreater{}  with the empirical chi square  386.88  with prob \textless{}...}
\CommentTok{\#\textgreater{} }
\CommentTok{\#\textgreater{} Fit based upon off diagonal values = 0.87}
\end{Highlighting}
\end{Shaded}

Látható, hogy a 6, 7,9 és 10 -es itemek nem nagy súllyal vesznek részt
az első főkomponensben, az egyediségük (nem magyarázott varianciájuk)
nekik a legnagyobb. Az 1. főkomponens az összes variancia 53\%-át
magyarázza.

Ha elhagyjuk ezeket az itemeket, jelentősen javul a főkomponens elemzés
eredménye:

\begin{Shaded}
\begin{Highlighting}[]
\NormalTok{psych}\SpecialCharTok{::}\FunctionTok{pca}\NormalTok{(d[}\SpecialCharTok{{-}}\FunctionTok{c}\NormalTok{(}\DecValTok{6}\NormalTok{, }\DecValTok{7}\NormalTok{, }\DecValTok{9}\NormalTok{, }\DecValTok{10}\NormalTok{)], }\AttributeTok{rotate =} \StringTok{"varimax"}\NormalTok{)}
\CommentTok{\#\textgreater{} Principal Components Analysis}
\CommentTok{\#\textgreater{} Call: principal(r = r, nfactors = nfactors, residua...}
\CommentTok{\#\textgreater{}     rotate = rotate, n.obs = n.obs, covar = covar, ...}
\CommentTok{\#\textgreater{}     missing = missing, impute = impute, oblique.sco...}
\CommentTok{\#\textgreater{}     method = method, use = use, cor = cor, correct ...}
\CommentTok{\#\textgreater{} Standardized loadings (pattern matrix) based upon c...}
\CommentTok{\#\textgreater{}      PC1   h2    u2 com}
\CommentTok{\#\textgreater{} K\_1 0.95 0.90 0.096   1}
\CommentTok{\#\textgreater{} K\_2 0.95 0.90 0.099   1}
\CommentTok{\#\textgreater{} K\_3 0.90 0.81 0.190   1}
\CommentTok{\#\textgreater{} K\_4 0.90 0.80 0.198   1}
\CommentTok{\#\textgreater{} K\_5 0.95 0.90 0.100   1}
\CommentTok{\#\textgreater{} K\_8 0.85 0.72 0.280   1}
\CommentTok{\#\textgreater{} }
\CommentTok{\#\textgreater{}                 PC1}
\CommentTok{\#\textgreater{} SS loadings    5.04}
\CommentTok{\#\textgreater{} Proportion Var 0.84}
\CommentTok{\#\textgreater{} }
\CommentTok{\#\textgreater{} Mean item complexity =  1}
\CommentTok{\#\textgreater{} Test of the hypothesis that 1 component is sufficient.}
\CommentTok{\#\textgreater{} }
\CommentTok{\#\textgreater{} The root mean square of the residuals (RMSR) is  0.04 }
\CommentTok{\#\textgreater{}  with the empirical chi square  5.89  with prob \textless{}  ...}
\CommentTok{\#\textgreater{} }
\CommentTok{\#\textgreater{} Fit based upon off diagonal values = 1}
\end{Highlighting}
\end{Shaded}

A magyarázott variancia felmegy 84\%-ra, és mindegyik változónak szoros
a kapcsolata az 1. főkomponenssel. Az itemek igen jól illeszkednek
egyetlen dimenzióra.

A fenti vizsgálatot a Cronbach-alfa alapján is elvégezhetjük.

\begin{Shaded}
\begin{Highlighting}[]
\NormalTok{psych}\SpecialCharTok{::}\FunctionTok{alpha}\NormalTok{(d)}
\CommentTok{\#\textgreater{} Some items ( K\_6 K\_7 K\_9 K\_10 ) were negatively cor...}
\CommentTok{\#\textgreater{} probably should be reversed.  }
\CommentTok{\#\textgreater{} To do this, run the function again with the \textquotesingle{}check....}
\CommentTok{\#\textgreater{} }
\CommentTok{\#\textgreater{} Reliability analysis   }
\CommentTok{\#\textgreater{} Call: psych::alpha(x = d)}
\CommentTok{\#\textgreater{} }
\CommentTok{\#\textgreater{}   raw\_alpha std.alpha G6(smc) average\_r S/N   ase mean}
\CommentTok{\#\textgreater{}       0.77      0.78    0.91      0.26 3.5 0.032  3.5}
\CommentTok{\#\textgreater{}   sd median\_r}
\CommentTok{\#\textgreater{}  1.1      0.1}
\CommentTok{\#\textgreater{} }
\CommentTok{\#\textgreater{}     95\% confidence boundaries }
\CommentTok{\#\textgreater{}          lower alpha upper}
\CommentTok{\#\textgreater{} Feldt     0.71  0.77  0.83}
\CommentTok{\#\textgreater{} Duhachek  0.71  0.77  0.84}
\CommentTok{\#\textgreater{} }
\CommentTok{\#\textgreater{}  Reliability if an item is dropped:}
\CommentTok{\#\textgreater{}      raw\_alpha std.alpha G6(smc) average\_r S/N alph...}
\CommentTok{\#\textgreater{} K\_1       0.71      0.72    0.88      0.22 2.6    0...}
\CommentTok{\#\textgreater{} K\_2       0.71      0.72    0.88      0.22 2.6    0...}
\CommentTok{\#\textgreater{} K\_3       0.72      0.73    0.89      0.23 2.7    0...}
\CommentTok{\#\textgreater{} K\_4       0.72      0.73    0.88      0.23 2.7    0...}
\CommentTok{\#\textgreater{} K\_5       0.72      0.73    0.88      0.23 2.6    0...}
\CommentTok{\#\textgreater{} K\_6       0.79      0.80    0.90      0.31 4.0    0...}
\CommentTok{\#\textgreater{} K\_7       0.82      0.82    0.92      0.33 4.5    0...}
\CommentTok{\#\textgreater{} K\_8       0.74      0.75    0.89      0.25 2.9    0...}
\CommentTok{\#\textgreater{} K\_9       0.79      0.80    0.91      0.31 3.9    0...}
\CommentTok{\#\textgreater{} K\_10      0.77      0.78    0.91      0.29 3.6    0...}
\CommentTok{\#\textgreater{}      var.r med.r}
\CommentTok{\#\textgreater{} K\_1   0.19 0.085}
\CommentTok{\#\textgreater{} K\_2   0.20 0.085}
\CommentTok{\#\textgreater{} K\_3   0.20 0.088}
\CommentTok{\#\textgreater{} K\_4   0.20 0.097}
\CommentTok{\#\textgreater{} K\_5   0.19 0.097}
\CommentTok{\#\textgreater{} K\_6   0.22 0.176}
\CommentTok{\#\textgreater{} K\_7   0.20 0.263}
\CommentTok{\#\textgreater{} K\_8   0.21 0.097}
\CommentTok{\#\textgreater{} K\_9   0.22 0.176}
\CommentTok{\#\textgreater{} K\_10  0.25 0.217}
\CommentTok{\#\textgreater{} }
\CommentTok{\#\textgreater{}  Item statistics }
\CommentTok{\#\textgreater{}        n raw.r std.r r.cor r.drop mean  sd}
\CommentTok{\#\textgreater{} K\_1  125  0.81  0.82 0.839  0.729  3.2 1.9}
\CommentTok{\#\textgreater{} K\_2  125  0.82  0.83 0.853  0.748  3.2 1.9}
\CommentTok{\#\textgreater{} K\_3  125  0.76  0.76 0.756  0.660  3.3 1.9}
\CommentTok{\#\textgreater{} K\_4  125  0.75  0.76 0.756  0.659  3.4 1.8}
\CommentTok{\#\textgreater{} K\_5  125  0.78  0.79 0.809  0.701  3.2 1.7}
\CommentTok{\#\textgreater{} K\_6  125  0.31  0.29 0.241  0.140  3.7 1.8}
\CommentTok{\#\textgreater{} K\_7  125  0.15  0.12 0.031 {-}0.049  4.1 2.1}
\CommentTok{\#\textgreater{} K\_8  125  0.66  0.68 0.652  0.558  3.4 1.6}
\CommentTok{\#\textgreater{} K\_9  125  0.32  0.31 0.247  0.160  3.7 1.8}
\CommentTok{\#\textgreater{} K\_10 125  0.44  0.43 0.330  0.278  3.6 1.9}
\CommentTok{\#\textgreater{} }
\CommentTok{\#\textgreater{} Non missing response frequency for each item}
\CommentTok{\#\textgreater{}         1    2    3    4    5    6    7 miss}
\CommentTok{\#\textgreater{} K\_1  0.21 0.26 0.14 0.18 0.05 0.10 0.07    0}
\CommentTok{\#\textgreater{} K\_2  0.19 0.29 0.11 0.18 0.07 0.08 0.08    0}
\CommentTok{\#\textgreater{} K\_3  0.18 0.23 0.17 0.16 0.06 0.11 0.08    0}
\CommentTok{\#\textgreater{} K\_4  0.17 0.22 0.18 0.17 0.09 0.10 0.08    0}
\CommentTok{\#\textgreater{} K\_5  0.18 0.24 0.21 0.15 0.10 0.04 0.07    0}
\CommentTok{\#\textgreater{} K\_6  0.11 0.23 0.16 0.14 0.11 0.18 0.06    0}
\CommentTok{\#\textgreater{} K\_7  0.14 0.14 0.15 0.12 0.07 0.24 0.14    0}
\CommentTok{\#\textgreater{} K\_8  0.14 0.18 0.19 0.28 0.10 0.06 0.06    0}
\CommentTok{\#\textgreater{} K\_9  0.08 0.24 0.18 0.16 0.10 0.17 0.06    0}
\CommentTok{\#\textgreater{} K\_10 0.10 0.27 0.20 0.12 0.07 0.14 0.09    0}
\end{Highlighting}
\end{Shaded}

Látható, hogy a Cronbach-alfa értéke 0,77, de javítható a 7. item
eldobásával.

\begin{Shaded}
\begin{Highlighting}[]
\NormalTok{psych}\SpecialCharTok{::}\FunctionTok{alpha}\NormalTok{(d[}\FunctionTok{c}\NormalTok{(}\SpecialCharTok{{-}}\DecValTok{7}\NormalTok{)])}
\CommentTok{\#\textgreater{} Some items ( K\_6 K\_9 ) were negatively correlated w...}
\CommentTok{\#\textgreater{} probably should be reversed.  }
\CommentTok{\#\textgreater{} To do this, run the function again with the \textquotesingle{}check....}
\CommentTok{\#\textgreater{} }
\CommentTok{\#\textgreater{} Reliability analysis   }
\CommentTok{\#\textgreater{} Call: psych::alpha(x = d[c({-}7)])}
\CommentTok{\#\textgreater{} }
\CommentTok{\#\textgreater{}   raw\_alpha std.alpha G6(smc) average\_r S/N   ase mean}
\CommentTok{\#\textgreater{}       0.82      0.82    0.92      0.33 4.5 0.025  3.4}
\CommentTok{\#\textgreater{}   sd median\_r}
\CommentTok{\#\textgreater{}  1.2     0.26}
\CommentTok{\#\textgreater{} }
\CommentTok{\#\textgreater{}     95\% confidence boundaries }
\CommentTok{\#\textgreater{}          lower alpha upper}
\CommentTok{\#\textgreater{} Feldt     0.77  0.82  0.86}
\CommentTok{\#\textgreater{} Duhachek  0.77  0.82  0.87}
\CommentTok{\#\textgreater{} }
\CommentTok{\#\textgreater{}  Reliability if an item is dropped:}
\CommentTok{\#\textgreater{}      raw\_alpha std.alpha G6(smc) average\_r S/N alph...}
\CommentTok{\#\textgreater{} K\_1       0.76      0.76    0.89      0.29 3.2    0...}
\CommentTok{\#\textgreater{} K\_2       0.76      0.76    0.89      0.29 3.2    0...}
\CommentTok{\#\textgreater{} K\_3       0.77      0.77    0.90      0.30 3.4    0...}
\CommentTok{\#\textgreater{} K\_4       0.77      0.77    0.90      0.30 3.4    0...}
\CommentTok{\#\textgreater{} K\_5       0.77      0.77    0.89      0.29 3.3    0...}
\CommentTok{\#\textgreater{} K\_6       0.86      0.86    0.92      0.43 6.0    0...}
\CommentTok{\#\textgreater{} K\_8       0.79      0.79    0.90      0.32 3.7    0...}
\CommentTok{\#\textgreater{} K\_9       0.85      0.85    0.92      0.42 5.8    0...}
\CommentTok{\#\textgreater{} K\_10      0.84      0.83    0.93      0.39 5.0    0...}
\CommentTok{\#\textgreater{}      var.r med.r}
\CommentTok{\#\textgreater{} K\_1   0.19 0.100}
\CommentTok{\#\textgreater{} K\_2   0.19 0.097}
\CommentTok{\#\textgreater{} K\_3   0.20 0.103}
\CommentTok{\#\textgreater{} K\_4   0.20 0.103}
\CommentTok{\#\textgreater{} K\_5   0.19 0.103}
\CommentTok{\#\textgreater{} K\_6   0.19 0.699}
\CommentTok{\#\textgreater{} K\_8   0.20 0.103}
\CommentTok{\#\textgreater{} K\_9   0.19 0.699}
\CommentTok{\#\textgreater{} K\_10  0.24 0.699}
\CommentTok{\#\textgreater{} }
\CommentTok{\#\textgreater{}  Item statistics }
\CommentTok{\#\textgreater{}        n raw.r std.r r.cor r.drop mean  sd}
\CommentTok{\#\textgreater{} K\_1  125  0.87  0.87  0.89  0.818  3.2 1.9}
\CommentTok{\#\textgreater{} K\_2  125  0.88  0.88  0.90  0.829  3.2 1.9}
\CommentTok{\#\textgreater{} K\_3  125  0.81  0.81  0.80  0.735  3.3 1.9}
\CommentTok{\#\textgreater{} K\_4  125  0.81  0.81  0.80  0.733  3.4 1.8}
\CommentTok{\#\textgreater{} K\_5  125  0.85  0.85  0.87  0.792  3.2 1.7}
\CommentTok{\#\textgreater{} K\_6  125  0.19  0.18  0.11  0.012  3.7 1.8}
\CommentTok{\#\textgreater{} K\_8  125  0.73  0.74  0.71  0.640  3.4 1.6}
\CommentTok{\#\textgreater{} K\_9  125  0.22  0.22  0.15  0.051  3.7 1.8}
\CommentTok{\#\textgreater{} K\_10 125  0.39  0.39  0.29  0.227  3.6 1.9}
\CommentTok{\#\textgreater{} }
\CommentTok{\#\textgreater{} Non missing response frequency for each item}
\CommentTok{\#\textgreater{}         1    2    3    4    5    6    7 miss}
\CommentTok{\#\textgreater{} K\_1  0.21 0.26 0.14 0.18 0.05 0.10 0.07    0}
\CommentTok{\#\textgreater{} K\_2  0.19 0.29 0.11 0.18 0.07 0.08 0.08    0}
\CommentTok{\#\textgreater{} K\_3  0.18 0.23 0.17 0.16 0.06 0.11 0.08    0}
\CommentTok{\#\textgreater{} K\_4  0.17 0.22 0.18 0.17 0.09 0.10 0.08    0}
\CommentTok{\#\textgreater{} K\_5  0.18 0.24 0.21 0.15 0.10 0.04 0.07    0}
\CommentTok{\#\textgreater{} K\_6  0.11 0.23 0.16 0.14 0.11 0.18 0.06    0}
\CommentTok{\#\textgreater{} K\_8  0.14 0.18 0.19 0.28 0.10 0.06 0.06    0}
\CommentTok{\#\textgreater{} K\_9  0.08 0.24 0.18 0.16 0.10 0.17 0.06    0}
\CommentTok{\#\textgreater{} K\_10 0.10 0.27 0.20 0.12 0.07 0.14 0.09    0}
\end{Highlighting}
\end{Shaded}

Látható, hogy a Cronbach-alfa értéke 0,82, de javítható a 6. item
eldobásával.

\begin{Shaded}
\begin{Highlighting}[]
\NormalTok{psych}\SpecialCharTok{::}\FunctionTok{alpha}\NormalTok{(d[}\FunctionTok{c}\NormalTok{(}\SpecialCharTok{{-}}\DecValTok{7}\NormalTok{, }\SpecialCharTok{{-}}\DecValTok{6}\NormalTok{)])}
\CommentTok{\#\textgreater{} Some items ( K\_9 ) were negatively correlated with ...}
\CommentTok{\#\textgreater{} probably should be reversed.  }
\CommentTok{\#\textgreater{} To do this, run the function again with the \textquotesingle{}check....}
\CommentTok{\#\textgreater{} }
\CommentTok{\#\textgreater{} Reliability analysis   }
\CommentTok{\#\textgreater{} Call: psych::alpha(x = d[c({-}7, {-}6)])}
\CommentTok{\#\textgreater{} }
\CommentTok{\#\textgreater{}   raw\_alpha std.alpha G6(smc) average\_r S/N   ase mean}
\CommentTok{\#\textgreater{}       0.86      0.86    0.92      0.43   6 0.019  3.4}
\CommentTok{\#\textgreater{}   sd median\_r}
\CommentTok{\#\textgreater{}  1.3      0.7}
\CommentTok{\#\textgreater{} }
\CommentTok{\#\textgreater{}     95\% confidence boundaries }
\CommentTok{\#\textgreater{}          lower alpha upper}
\CommentTok{\#\textgreater{} Feldt     0.82  0.86  0.89}
\CommentTok{\#\textgreater{} Duhachek  0.82  0.86  0.89}
\CommentTok{\#\textgreater{} }
\CommentTok{\#\textgreater{}  Reliability if an item is dropped:}
\CommentTok{\#\textgreater{}      raw\_alpha std.alpha G6(smc) average\_r  S/N alp...}
\CommentTok{\#\textgreater{} K\_1       0.80      0.81    0.89      0.37  4.1    ...}
\CommentTok{\#\textgreater{} K\_2       0.80      0.80    0.89      0.37  4.1    ...}
\CommentTok{\#\textgreater{} K\_3       0.81      0.82    0.90      0.39  4.4    ...}
\CommentTok{\#\textgreater{} K\_4       0.82      0.82    0.90      0.39  4.5    ...}
\CommentTok{\#\textgreater{} K\_5       0.81      0.81    0.89      0.37  4.2    ...}
\CommentTok{\#\textgreater{} K\_8       0.83      0.83    0.91      0.41  4.8    ...}
\CommentTok{\#\textgreater{} K\_9       0.91      0.91    0.93      0.59 10.2    ...}
\CommentTok{\#\textgreater{} K\_10      0.89      0.89    0.93      0.53  8.0    ...}
\CommentTok{\#\textgreater{}      var.r med.r}
\CommentTok{\#\textgreater{} K\_1   0.19  0.47}
\CommentTok{\#\textgreater{} K\_2   0.19  0.47}
\CommentTok{\#\textgreater{} K\_3   0.20  0.47}
\CommentTok{\#\textgreater{} K\_4   0.20  0.47}
\CommentTok{\#\textgreater{} K\_5   0.18  0.47}
\CommentTok{\#\textgreater{} K\_8   0.20  0.47}
\CommentTok{\#\textgreater{} K\_9   0.12  0.77}
\CommentTok{\#\textgreater{} K\_10  0.20  0.77}
\CommentTok{\#\textgreater{} }
\CommentTok{\#\textgreater{}  Item statistics }
\CommentTok{\#\textgreater{}        n raw.r std.r r.cor r.drop mean  sd}
\CommentTok{\#\textgreater{} K\_1  125  0.92  0.92  0.94  0.880  3.2 1.9}
\CommentTok{\#\textgreater{} K\_2  125  0.92  0.92  0.94  0.886  3.2 1.9}
\CommentTok{\#\textgreater{} K\_3  125  0.86  0.86  0.85  0.803  3.3 1.9}
\CommentTok{\#\textgreater{} K\_4  125  0.84  0.85  0.84  0.779  3.4 1.8}
\CommentTok{\#\textgreater{} K\_5  125  0.91  0.91  0.92  0.868  3.2 1.7}
\CommentTok{\#\textgreater{} K\_8  125  0.77  0.78  0.75  0.697  3.4 1.6}
\CommentTok{\#\textgreater{} K\_9  125  0.10  0.10 {-}0.02 {-}0.072  3.7 1.8}
\CommentTok{\#\textgreater{} K\_10 125  0.33  0.32  0.21  0.150  3.6 1.9}
\CommentTok{\#\textgreater{} }
\CommentTok{\#\textgreater{} Non missing response frequency for each item}
\CommentTok{\#\textgreater{}         1    2    3    4    5    6    7 miss}
\CommentTok{\#\textgreater{} K\_1  0.21 0.26 0.14 0.18 0.05 0.10 0.07    0}
\CommentTok{\#\textgreater{} K\_2  0.19 0.29 0.11 0.18 0.07 0.08 0.08    0}
\CommentTok{\#\textgreater{} K\_3  0.18 0.23 0.17 0.16 0.06 0.11 0.08    0}
\CommentTok{\#\textgreater{} K\_4  0.17 0.22 0.18 0.17 0.09 0.10 0.08    0}
\CommentTok{\#\textgreater{} K\_5  0.18 0.24 0.21 0.15 0.10 0.04 0.07    0}
\CommentTok{\#\textgreater{} K\_8  0.14 0.18 0.19 0.28 0.10 0.06 0.06    0}
\CommentTok{\#\textgreater{} K\_9  0.08 0.24 0.18 0.16 0.10 0.17 0.06    0}
\CommentTok{\#\textgreater{} K\_10 0.10 0.27 0.20 0.12 0.07 0.14 0.09    0}
\end{Highlighting}
\end{Shaded}

Látható, hogy a Cronbach-alfa értéke 0,86, de javítható a 9. item
eldobásával.

\begin{Shaded}
\begin{Highlighting}[]
\NormalTok{psych}\SpecialCharTok{::}\FunctionTok{alpha}\NormalTok{(d[}\SpecialCharTok{{-}}\FunctionTok{c}\NormalTok{(}\DecValTok{7}\NormalTok{, }\DecValTok{6}\NormalTok{, }\DecValTok{9}\NormalTok{)])}
\CommentTok{\#\textgreater{} }
\CommentTok{\#\textgreater{} Reliability analysis   }
\CommentTok{\#\textgreater{} Call: psych::alpha(x = d[{-}c(7, 6, 9)])}
\CommentTok{\#\textgreater{} }
\CommentTok{\#\textgreater{}   raw\_alpha std.alpha G6(smc) average\_r S/N   ase mean}
\CommentTok{\#\textgreater{}       0.91      0.91    0.93      0.59  10 0.012  3.3}
\CommentTok{\#\textgreater{}   sd median\_r}
\CommentTok{\#\textgreater{}  1.5     0.77}
\CommentTok{\#\textgreater{} }
\CommentTok{\#\textgreater{}     95\% confidence boundaries }
\CommentTok{\#\textgreater{}          lower alpha upper}
\CommentTok{\#\textgreater{} Feldt     0.88  0.91  0.93}
\CommentTok{\#\textgreater{} Duhachek  0.89  0.91  0.93}
\CommentTok{\#\textgreater{} }
\CommentTok{\#\textgreater{}  Reliability if an item is dropped:}
\CommentTok{\#\textgreater{}      raw\_alpha std.alpha G6(smc) average\_r  S/N alp...}
\CommentTok{\#\textgreater{} K\_1       0.87      0.88    0.90      0.54  7.1   0...}
\CommentTok{\#\textgreater{} K\_2       0.87      0.88    0.90      0.54  7.1   0...}
\CommentTok{\#\textgreater{} K\_3       0.88      0.88    0.92      0.56  7.6   0...}
\CommentTok{\#\textgreater{} K\_4       0.89      0.89    0.91      0.57  7.9   0...}
\CommentTok{\#\textgreater{} K\_5       0.88      0.88    0.91      0.54  7.1   0...}
\CommentTok{\#\textgreater{} K\_8       0.89      0.89    0.92      0.58  8.4   0...}
\CommentTok{\#\textgreater{} K\_10      0.96      0.96    0.96      0.81 24.9   0...}
\CommentTok{\#\textgreater{}       var.r med.r}
\CommentTok{\#\textgreater{} K\_1  0.1331  0.75}
\CommentTok{\#\textgreater{} K\_2  0.1340  0.75}
\CommentTok{\#\textgreater{} K\_3  0.1436  0.75}
\CommentTok{\#\textgreater{} K\_4  0.1348  0.75}
\CommentTok{\#\textgreater{} K\_5  0.1327  0.75}
\CommentTok{\#\textgreater{} K\_8  0.1440  0.82}
\CommentTok{\#\textgreater{} K\_10 0.0044  0.82}
\CommentTok{\#\textgreater{} }
\CommentTok{\#\textgreater{}  Item statistics }
\CommentTok{\#\textgreater{}        n raw.r std.r r.cor r.drop mean  sd}
\CommentTok{\#\textgreater{} K\_1  125  0.94  0.94 0.956  0.914  3.2 1.9}
\CommentTok{\#\textgreater{} K\_2  125  0.94  0.94 0.955  0.914  3.2 1.9}
\CommentTok{\#\textgreater{} K\_3  125  0.89  0.89 0.877  0.846  3.3 1.9}
\CommentTok{\#\textgreater{} K\_4  125  0.87  0.87 0.850  0.808  3.4 1.8}
\CommentTok{\#\textgreater{} K\_5  125  0.93  0.94 0.943  0.906  3.2 1.7}
\CommentTok{\#\textgreater{} K\_8  125  0.82  0.83 0.794  0.759  3.4 1.6}
\CommentTok{\#\textgreater{} K\_10 125  0.24  0.24 0.077  0.064  3.6 1.9}
\CommentTok{\#\textgreater{} }
\CommentTok{\#\textgreater{} Non missing response frequency for each item}
\CommentTok{\#\textgreater{}         1    2    3    4    5    6    7 miss}
\CommentTok{\#\textgreater{} K\_1  0.21 0.26 0.14 0.18 0.05 0.10 0.07    0}
\CommentTok{\#\textgreater{} K\_2  0.19 0.29 0.11 0.18 0.07 0.08 0.08    0}
\CommentTok{\#\textgreater{} K\_3  0.18 0.23 0.17 0.16 0.06 0.11 0.08    0}
\CommentTok{\#\textgreater{} K\_4  0.17 0.22 0.18 0.17 0.09 0.10 0.08    0}
\CommentTok{\#\textgreater{} K\_5  0.18 0.24 0.21 0.15 0.10 0.04 0.07    0}
\CommentTok{\#\textgreater{} K\_8  0.14 0.18 0.19 0.28 0.10 0.06 0.06    0}
\CommentTok{\#\textgreater{} K\_10 0.10 0.27 0.20 0.12 0.07 0.14 0.09    0}
\end{Highlighting}
\end{Shaded}

Látható, hogy a Cronbach-alfa értéke 0,91, de javítható a 10. item
eldobásával.

\begin{Shaded}
\begin{Highlighting}[]
\NormalTok{psych}\SpecialCharTok{::}\FunctionTok{alpha}\NormalTok{(d[}\SpecialCharTok{{-}}\FunctionTok{c}\NormalTok{(}\DecValTok{7}\NormalTok{, }\DecValTok{6}\NormalTok{, }\DecValTok{9}\NormalTok{, }\DecValTok{10}\NormalTok{)])}
\CommentTok{\#\textgreater{} }
\CommentTok{\#\textgreater{} Reliability analysis   }
\CommentTok{\#\textgreater{} Call: psych::alpha(x = d[{-}c(7, 6, 9, 10)])}
\CommentTok{\#\textgreater{} }
\CommentTok{\#\textgreater{}   raw\_alpha std.alpha G6(smc) average\_r S/N    ase ...}
\CommentTok{\#\textgreater{}       0.96      0.96    0.96      0.81  25 0.0053  3.3}
\CommentTok{\#\textgreater{}   sd median\_r}
\CommentTok{\#\textgreater{}  1.7     0.82}
\CommentTok{\#\textgreater{} }
\CommentTok{\#\textgreater{}     95\% confidence boundaries }
\CommentTok{\#\textgreater{}          lower alpha upper}
\CommentTok{\#\textgreater{} Feldt     0.95  0.96  0.97}
\CommentTok{\#\textgreater{} Duhachek  0.95  0.96  0.97}
\CommentTok{\#\textgreater{} }
\CommentTok{\#\textgreater{}  Reliability if an item is dropped:}
\CommentTok{\#\textgreater{}     raw\_alpha std.alpha G6(smc) average\_r S/N alpha se}
\CommentTok{\#\textgreater{} K\_1      0.95      0.95    0.94      0.79  19   0.0072}
\CommentTok{\#\textgreater{} K\_2      0.95      0.95    0.94      0.79  19   0.0072}
\CommentTok{\#\textgreater{} K\_3      0.96      0.96    0.95      0.81  22   0.0061}
\CommentTok{\#\textgreater{} K\_4      0.96      0.96    0.95      0.82  22   0.0060}
\CommentTok{\#\textgreater{} K\_5      0.95      0.95    0.94      0.79  19   0.0071}
\CommentTok{\#\textgreater{} K\_8      0.96      0.96    0.96      0.84  26   0.0053}
\CommentTok{\#\textgreater{}      var.r med.r}
\CommentTok{\#\textgreater{} K\_1 0.0034  0.80}
\CommentTok{\#\textgreater{} K\_2 0.0037  0.80}
\CommentTok{\#\textgreater{} K\_3 0.0054  0.81}
\CommentTok{\#\textgreater{} K\_4 0.0052  0.82}
\CommentTok{\#\textgreater{} K\_5 0.0050  0.79}
\CommentTok{\#\textgreater{} K\_8 0.0022  0.82}
\CommentTok{\#\textgreater{} }
\CommentTok{\#\textgreater{}  Item statistics }
\CommentTok{\#\textgreater{}       n raw.r std.r r.cor r.drop mean  sd}
\CommentTok{\#\textgreater{} K\_1 125  0.95  0.95  0.95   0.93  3.2 1.9}
\CommentTok{\#\textgreater{} K\_2 125  0.95  0.95  0.95   0.92  3.2 1.9}
\CommentTok{\#\textgreater{} K\_3 125  0.90  0.90  0.87   0.86  3.3 1.9}
\CommentTok{\#\textgreater{} K\_4 125  0.90  0.90  0.86   0.85  3.4 1.8}
\CommentTok{\#\textgreater{} K\_5 125  0.95  0.95  0.94   0.92  3.2 1.7}
\CommentTok{\#\textgreater{} K\_8 125  0.85  0.85  0.80   0.79  3.4 1.6}
\CommentTok{\#\textgreater{} }
\CommentTok{\#\textgreater{} Non missing response frequency for each item}
\CommentTok{\#\textgreater{}        1    2    3    4    5    6    7 miss}
\CommentTok{\#\textgreater{} K\_1 0.21 0.26 0.14 0.18 0.05 0.10 0.07    0}
\CommentTok{\#\textgreater{} K\_2 0.19 0.29 0.11 0.18 0.07 0.08 0.08    0}
\CommentTok{\#\textgreater{} K\_3 0.18 0.23 0.17 0.16 0.06 0.11 0.08    0}
\CommentTok{\#\textgreater{} K\_4 0.17 0.22 0.18 0.17 0.09 0.10 0.08    0}
\CommentTok{\#\textgreater{} K\_5 0.18 0.24 0.21 0.15 0.10 0.04 0.07    0}
\CommentTok{\#\textgreater{} K\_8 0.14 0.18 0.19 0.28 0.10 0.06 0.06    0}
\end{Highlighting}
\end{Shaded}

Látható, hogy a Cronbach-alfa értéke 0,96.

A kapott eredmények alapján az itemszelekciót ennél a lépésnél
befejezhetjük. Az így kapott hat itemünk a statisztikai eredmények
alapján egészen jól lefednek egy dimenziót, ezáltal használhatóak egy
jelenség kérdőíves vizsgálatára.

\begin{figure}

{\centering \includegraphics{./images/fokomp_kerdoivtervezet_kep_01.jpg}

}

\caption{Egy kérdőív szerkesztésének problémái: főkomponens elemzés}

\end{figure}

\hypertarget{puxe9lda-mi-is-az-a-munkahelyi-tolerancia}{%
\section{Példa: Mi is az a munkahelyi
tolerancia?}\label{puxe9lda-mi-is-az-a-munkahelyi-tolerancia}}

\begin{itemize}
\tightlist
\item
  A példa forrása: Münnich és mtsai. (2006)
  \href{https://psycho.unideb.hu/statisztika/pages/p_2_13.html}{2.5.3
  Probléma}
\item
  Kapcsolódó jamovi állomány: \texttt{fokomp\_kerdoivtervezet.omv}
\end{itemize}

Ebben a példában azt vizsgáljuk meg, hogy ha a toleranciát munkahelyen
vizsgáljuk, akkor mely jelenségeket, viselkedéseket kell figyelembe
vennünk. Az adatok a \texttt{fokomp\_munkahelyi\_tolarencia.xlsx}
állományban találhatók.

\begin{Shaded}
\begin{Highlighting}[]
\NormalTok{d }\OtherTok{\textless{}{-}}\NormalTok{ rio}\SpecialCharTok{::}\FunctionTok{import}\NormalTok{(}\StringTok{"adat/fokomp\_munkahelyi\_tolarencia.xlsx"}\NormalTok{)}
\FunctionTok{str}\NormalTok{(d)}
\CommentTok{\#\textgreater{} \textquotesingle{}data.frame\textquotesingle{}:    155 obs. of  18 variables:}
\CommentTok{\#\textgreater{}  $ alkohol    : num  1 1 1 1 1 2 1 1 1 1 ...}
\CommentTok{\#\textgreater{}  $ kabitoszer : num  1 1 1 1 1 1 1 1 1 1 ...}
\CommentTok{\#\textgreater{}  $ hianyzik   : num  3 1 1 2 1 1 5 1 1 3 ...}
\CommentTok{\#\textgreater{}  $ dohanyzas  : num  4 5 1 3 1 4 5 1 5 5 ...}
\CommentTok{\#\textgreater{}  $ udvariatlan: num  3 1 5 2 2 1 1 1 1 1 ...}
\CommentTok{\#\textgreater{}  $ rendetlen  : num  3 1 5 2 2 1 1 1 1 3 ...}
\CommentTok{\#\textgreater{}  $ pontatlan  : num  3 1 5 3 2 1 1 1 1 3 ...}
\CommentTok{\#\textgreater{}  $ pletykas   : num  1 1 5 2 1 2 1 1 1 3 ...}
\CommentTok{\#\textgreater{}  $ harsany    : num  4 3 5 2 2 4 1 1 2 3 ...}
\CommentTok{\#\textgreater{}  $ tudalekos  : num  3 2 4 3 2 2 1 1 2 1 ...}
\CommentTok{\#\textgreater{}  $ csamcsog   : num  3 1 5 3 3 1 1 1 1 3 ...}
\CommentTok{\#\textgreater{}  $ lusta      : num  3 1 5 2 3 4 1 1 1 5 ...}
\CommentTok{\#\textgreater{}  $ szemtelen  : num  3 1 5 2 2 1 1 1 1 1 ...}
\CommentTok{\#\textgreater{}  $ bufog      : num  3 1 5 2 2 5 5 1 1 1 ...}
\CommentTok{\#\textgreater{}  $ felelotlen : num  3 1 5 2 2 1 1 1 1 1 ...}
\CommentTok{\#\textgreater{}  $ bosszuallo : num  2 2 3 2 1 1 1 1 1 1 ...}
\CommentTok{\#\textgreater{}  $ durva      : num  2 1 5 2 2 1 1 1 1 1 ...}
\CommentTok{\#\textgreater{}  $ agressziv  : num  2 1 5 1 2 1 1 1 1 1 ...}
\NormalTok{psych}\SpecialCharTok{::}\FunctionTok{headTail}\NormalTok{(d)}
\CommentTok{\#\textgreater{}     alkohol kabitoszer hianyzik dohanyzas udvariatlan}
\CommentTok{\#\textgreater{} 1         1          1        3         4           3}
\CommentTok{\#\textgreater{} 2         1          1        1         5           1}
\CommentTok{\#\textgreater{} 3         1          1        1         1           5}
\CommentTok{\#\textgreater{} 4         1          1        2         3           2}
\CommentTok{\#\textgreater{} ...     ...        ...      ...       ...         ...}
\CommentTok{\#\textgreater{} 152       3          1        2         5           3}
\CommentTok{\#\textgreater{} 153       3          1        2         2           2}
\CommentTok{\#\textgreater{} 154       4          4        2         5           3}
\CommentTok{\#\textgreater{} 155       3          3        4         5           3}
\CommentTok{\#\textgreater{}     rendetlen pontatlan pletykas harsany tudalekos}
\CommentTok{\#\textgreater{} 1           3         3        1       4         3}
\CommentTok{\#\textgreater{} 2           1         1        1       3         2}
\CommentTok{\#\textgreater{} 3           5         5        5       5         4}
\CommentTok{\#\textgreater{} 4           2         3        2       2         3}
\CommentTok{\#\textgreater{} ...       ...       ...      ...     ...       ...}
\CommentTok{\#\textgreater{} 152         4         2        2       5         3}
\CommentTok{\#\textgreater{} 153         1         1        1       1         2}
\CommentTok{\#\textgreater{} 154         4         5        5       3         2}
\CommentTok{\#\textgreater{} 155         4         4        2       2         1}
\CommentTok{\#\textgreater{}     csamcsog lusta szemtelen bufog felelotlen bossz...}
\CommentTok{\#\textgreater{} 1          3     3         3     3          3      ...}
\CommentTok{\#\textgreater{} 2          1     1         1     1          1      ...}
\CommentTok{\#\textgreater{} 3          5     5         5     5          5      ...}
\CommentTok{\#\textgreater{} 4          3     2         2     2          2      ...}
\CommentTok{\#\textgreater{} ...      ...   ...       ...   ...        ...      ...}
\CommentTok{\#\textgreater{} 152        4     5         4     4          2      ...}
\CommentTok{\#\textgreater{} 153        1     1         1     1          1      ...}
\CommentTok{\#\textgreater{} 154        2     3         3     4          5      ...}
\CommentTok{\#\textgreater{} 155        1     3         2     1          2      ...}
\CommentTok{\#\textgreater{}     durva agressziv}
\CommentTok{\#\textgreater{} 1       2         2}
\CommentTok{\#\textgreater{} 2       1         1}
\CommentTok{\#\textgreater{} 3       5         5}
\CommentTok{\#\textgreater{} 4       2         1}
\CommentTok{\#\textgreater{} ...   ...       ...}
\CommentTok{\#\textgreater{} 152     3         1}
\CommentTok{\#\textgreater{} 153     1         1}
\CommentTok{\#\textgreater{} 154     5         5}
\CommentTok{\#\textgreater{} 155     2         2}
\end{Highlighting}
\end{Shaded}

A fenti outputban látható, hogy adatokat találunk arról, hogy egyes
viselkedéseket (pl. agresszivitás, dohányzás, durva beszéd stb.)
mennyire tartanak zavarónak az emberek. Az adatokból a főkomponens
analízis és a Cronbach-alfa segítségével pedig megnézhetjük, hogy az
adatok összegezhetőek-e egy általános munkahelyi tolerancia
főkomponensbe.

\begin{Shaded}
\begin{Highlighting}[]
\NormalTok{psych}\SpecialCharTok{::}\FunctionTok{pca}\NormalTok{(d, }\AttributeTok{rotate =} \StringTok{"varimax"}\NormalTok{)}
\CommentTok{\#\textgreater{} Principal Components Analysis}
\CommentTok{\#\textgreater{} Call: principal(r = r, nfactors = nfactors, residua...}
\CommentTok{\#\textgreater{}     rotate = rotate, n.obs = n.obs, covar = covar, ...}
\CommentTok{\#\textgreater{}     missing = missing, impute = impute, oblique.sco...}
\CommentTok{\#\textgreater{}     method = method, use = use, cor = cor, correct ...}
\CommentTok{\#\textgreater{} Standardized loadings (pattern matrix) based upon c...}
\CommentTok{\#\textgreater{}              PC1    h2   u2 com}
\CommentTok{\#\textgreater{} alkohol     0.54 0.295 0.71   1}
\CommentTok{\#\textgreater{} kabitoszer  0.62 0.386 0.61   1}
\CommentTok{\#\textgreater{} hianyzik    0.62 0.384 0.62   1}
\CommentTok{\#\textgreater{} dohanyzas   0.18 0.033 0.97   1}
\CommentTok{\#\textgreater{} udvariatlan 0.78 0.607 0.39   1}
\CommentTok{\#\textgreater{} rendetlen   0.80 0.633 0.37   1}
\CommentTok{\#\textgreater{} pontatlan   0.74 0.548 0.45   1}
\CommentTok{\#\textgreater{} pletykas    0.47 0.224 0.78   1}
\CommentTok{\#\textgreater{} harsany     0.41 0.170 0.83   1}
\CommentTok{\#\textgreater{} tudalekos   0.53 0.283 0.72   1}
\CommentTok{\#\textgreater{} csamcsog    0.70 0.492 0.51   1}
\CommentTok{\#\textgreater{} lusta       0.73 0.538 0.46   1}
\CommentTok{\#\textgreater{} szemtelen   0.83 0.692 0.31   1}
\CommentTok{\#\textgreater{} bufog       0.62 0.386 0.61   1}
\CommentTok{\#\textgreater{} felelotlen  0.77 0.587 0.41   1}
\CommentTok{\#\textgreater{} bosszuallo  0.73 0.532 0.47   1}
\CommentTok{\#\textgreater{} durva       0.73 0.535 0.46   1}
\CommentTok{\#\textgreater{} agressziv   0.75 0.559 0.44   1}
\CommentTok{\#\textgreater{} }
\CommentTok{\#\textgreater{}                 PC1}
\CommentTok{\#\textgreater{} SS loadings    7.88}
\CommentTok{\#\textgreater{} Proportion Var 0.44}
\CommentTok{\#\textgreater{} }
\CommentTok{\#\textgreater{} Mean item complexity =  1}
\CommentTok{\#\textgreater{} Test of the hypothesis that 1 component is sufficient.}
\CommentTok{\#\textgreater{} }
\CommentTok{\#\textgreater{} The root mean square of the residuals (RMSR) is  0.11 }
\CommentTok{\#\textgreater{}  with the empirical chi square  578.76  with prob \textless{}...}
\CommentTok{\#\textgreater{} }
\CommentTok{\#\textgreater{} Fit based upon off diagonal values = 0.93}
\end{Highlighting}
\end{Shaded}

Az első főkomponens csupán az összvariancia 43\%-át magyarázza. A
fentiek alapján főleg a ``dohányzás'', a ``harsány'' és a ``pletykás''
változó az, amely valamennyire „kilóg'' a modellből, hiszen a hozzájuk
tartozó súlyok a legkisebbek a fenti outputban.

\begin{Shaded}
\begin{Highlighting}[]
\FunctionTok{library}\NormalTok{(tidyverse)}
\NormalTok{psych}\SpecialCharTok{::}\FunctionTok{pca}\NormalTok{(d }\SpecialCharTok{\%\textgreater{}\%}
    \FunctionTok{select}\NormalTok{(}\SpecialCharTok{{-}}\NormalTok{dohanyzas, }\SpecialCharTok{{-}}\NormalTok{harsany, }\SpecialCharTok{{-}}\NormalTok{pletykas), }\AttributeTok{rotate =} \StringTok{"varimax"}\NormalTok{)}
\CommentTok{\#\textgreater{} Principal Components Analysis}
\CommentTok{\#\textgreater{} Call: principal(r = r, nfactors = nfactors, residua...}
\CommentTok{\#\textgreater{}     rotate = rotate, n.obs = n.obs, covar = covar, ...}
\CommentTok{\#\textgreater{}     missing = missing, impute = impute, oblique.sco...}
\CommentTok{\#\textgreater{}     method = method, use = use, cor = cor, correct ...}
\CommentTok{\#\textgreater{} Standardized loadings (pattern matrix) based upon c...}
\CommentTok{\#\textgreater{}              PC1   h2   u2 com}
\CommentTok{\#\textgreater{} alkohol     0.56 0.31 0.69   1}
\CommentTok{\#\textgreater{} kabitoszer  0.64 0.41 0.59   1}
\CommentTok{\#\textgreater{} hianyzik    0.63 0.40 0.60   1}
\CommentTok{\#\textgreater{} udvariatlan 0.77 0.60 0.40   1}
\CommentTok{\#\textgreater{} rendetlen   0.78 0.61 0.39   1}
\CommentTok{\#\textgreater{} pontatlan   0.73 0.54 0.46   1}
\CommentTok{\#\textgreater{} tudalekos   0.50 0.25 0.75   1}
\CommentTok{\#\textgreater{} csamcsog    0.69 0.48 0.52   1}
\CommentTok{\#\textgreater{} lusta       0.72 0.52 0.48   1}
\CommentTok{\#\textgreater{} szemtelen   0.84 0.71 0.29   1}
\CommentTok{\#\textgreater{} bufog       0.62 0.38 0.62   1}
\CommentTok{\#\textgreater{} felelotlen  0.78 0.61 0.39   1}
\CommentTok{\#\textgreater{} bosszuallo  0.75 0.56 0.44   1}
\CommentTok{\#\textgreater{} durva       0.75 0.56 0.44   1}
\CommentTok{\#\textgreater{} agressziv   0.77 0.59 0.41   1}
\CommentTok{\#\textgreater{} }
\CommentTok{\#\textgreater{}                 PC1}
\CommentTok{\#\textgreater{} SS loadings    7.53}
\CommentTok{\#\textgreater{} Proportion Var 0.50}
\CommentTok{\#\textgreater{} }
\CommentTok{\#\textgreater{} Mean item complexity =  1}
\CommentTok{\#\textgreater{} Test of the hypothesis that 1 component is sufficient.}
\CommentTok{\#\textgreater{} }
\CommentTok{\#\textgreater{} The root mean square of the residuals (RMSR) is  0.1 }
\CommentTok{\#\textgreater{}  with the empirical chi square  335.97  with prob \textless{}...}
\CommentTok{\#\textgreater{} }
\CommentTok{\#\textgreater{} Fit based upon off diagonal values = 0.95}
\end{Highlighting}
\end{Shaded}

Így az első főkomponens által magyarázott összvariancia már elérte az
50\%-ot.

Vizsgáljuk meg a Cronbach-alfa értékét is.

\begin{Shaded}
\begin{Highlighting}[]
\NormalTok{RcmdrMisc}\SpecialCharTok{::}\FunctionTok{reliability}\NormalTok{(}\FunctionTok{cov}\NormalTok{(d))}
\CommentTok{\#\textgreater{} Alpha reliability =  0.9155 }
\CommentTok{\#\textgreater{} Standardized alpha =  0.9175 }
\CommentTok{\#\textgreater{} }
\CommentTok{\#\textgreater{} Reliability deleting each item in turn:}
\CommentTok{\#\textgreater{}              Alpha Std.Alpha r(item, total)}
\CommentTok{\#\textgreater{} alkohol     0.9130    0.9154         0.5086}
\CommentTok{\#\textgreater{} kabitoszer  0.9115    0.9138         0.5657}
\CommentTok{\#\textgreater{} hianyzik    0.9114    0.9136         0.5711}
\CommentTok{\#\textgreater{} dohanyzas   0.9233    0.9236         0.1696}
\CommentTok{\#\textgreater{} udvariatlan 0.9075    0.9093         0.7314}
\CommentTok{\#\textgreater{} rendetlen   0.9066    0.9087         0.7537}
\CommentTok{\#\textgreater{} pontatlan   0.9085    0.9106         0.6837}
\CommentTok{\#\textgreater{} pletykas    0.9149    0.9170         0.4295}
\CommentTok{\#\textgreater{} harsany     0.9160    0.9183         0.3788}
\CommentTok{\#\textgreater{} tudalekos   0.9136    0.9158         0.4791}
\CommentTok{\#\textgreater{} csamcsog    0.9090    0.9113         0.6587}
\CommentTok{\#\textgreater{} lusta       0.9086    0.9105         0.6862}
\CommentTok{\#\textgreater{} szemtelen   0.9065    0.9083         0.7730}
\CommentTok{\#\textgreater{} bufog       0.9115    0.9136         0.5714}
\CommentTok{\#\textgreater{} felelotlen  0.9084    0.9105         0.6847}
\CommentTok{\#\textgreater{} bosszuallo  0.9090    0.9113         0.6599}
\CommentTok{\#\textgreater{} durva       0.9087    0.9111         0.6699}
\CommentTok{\#\textgreater{} agressziv   0.9084    0.9109         0.6764}
\end{Highlighting}
\end{Shaded}

A fenti output alapján már viszonylag magas a Cronbach-alfa értéke
(0,915), de látható, hogy a ``dohányzás'' eltávolításával tovább
növelhető.

\begin{Shaded}
\begin{Highlighting}[]
\NormalTok{RcmdrMisc}\SpecialCharTok{::}\FunctionTok{reliability}\NormalTok{(}\FunctionTok{cov}\NormalTok{(d }\SpecialCharTok{\%\textgreater{}\%}
    \FunctionTok{select}\NormalTok{(}\SpecialCharTok{{-}}\NormalTok{dohanyzas)))}
\CommentTok{\#\textgreater{} Alpha reliability =  0.9233 }
\CommentTok{\#\textgreater{} Standardized alpha =  0.9236 }
\CommentTok{\#\textgreater{} }
\CommentTok{\#\textgreater{} Reliability deleting each item in turn:}
\CommentTok{\#\textgreater{}              Alpha Std.Alpha r(item, total)}
\CommentTok{\#\textgreater{} alkohol     0.9222    0.9226         0.4892}
\CommentTok{\#\textgreater{} kabitoszer  0.9202    0.9206         0.5673}
\CommentTok{\#\textgreater{} hianyzik    0.9205    0.9208         0.5573}
\CommentTok{\#\textgreater{} udvariatlan 0.9161    0.9161         0.7331}
\CommentTok{\#\textgreater{} rendetlen   0.9155    0.9156         0.7493}
\CommentTok{\#\textgreater{} pontatlan   0.9170    0.9172         0.6900}
\CommentTok{\#\textgreater{} pletykas    0.9234    0.9238         0.4320}
\CommentTok{\#\textgreater{} harsany     0.9249    0.9253         0.3684}
\CommentTok{\#\textgreater{} tudalekos   0.9220    0.9224         0.4883}
\CommentTok{\#\textgreater{} csamcsog    0.9177    0.9180         0.6629}
\CommentTok{\#\textgreater{} lusta       0.9173    0.9174         0.6856}
\CommentTok{\#\textgreater{} szemtelen   0.9149    0.9148         0.7856}
\CommentTok{\#\textgreater{} bufog       0.9205    0.9205         0.5674}
\CommentTok{\#\textgreater{} felelotlen  0.9164    0.9168         0.7100}
\CommentTok{\#\textgreater{} bosszuallo  0.9173    0.9178         0.6777}
\CommentTok{\#\textgreater{} durva       0.9175    0.9179         0.6699}
\CommentTok{\#\textgreater{} agressziv   0.9169    0.9174         0.6909}
\end{Highlighting}
\end{Shaded}

A fenti output alapján a Cronbach-alfa értéke (0,923), de látható, hogy
a ``harsany'' eltávolításával tovább növelhető.

\begin{Shaded}
\begin{Highlighting}[]
\NormalTok{RcmdrMisc}\SpecialCharTok{::}\FunctionTok{reliability}\NormalTok{(}\FunctionTok{cov}\NormalTok{(d }\SpecialCharTok{\%\textgreater{}\%}
    \FunctionTok{select}\NormalTok{(}\SpecialCharTok{{-}}\NormalTok{dohanyzas, }\SpecialCharTok{{-}}\NormalTok{harsany)))}
\CommentTok{\#\textgreater{} Alpha reliability =  0.9249 }
\CommentTok{\#\textgreater{} Standardized alpha =  0.9253 }
\CommentTok{\#\textgreater{} }
\CommentTok{\#\textgreater{} Reliability deleting each item in turn:}
\CommentTok{\#\textgreater{}              Alpha Std.Alpha r(item, total)}
\CommentTok{\#\textgreater{} alkohol     0.9238    0.9245         0.5019}
\CommentTok{\#\textgreater{} kabitoszer  0.9215    0.9220         0.5903}
\CommentTok{\#\textgreater{} hianyzik    0.9221    0.9225         0.5694}
\CommentTok{\#\textgreater{} udvariatlan 0.9178    0.9180         0.7306}
\CommentTok{\#\textgreater{} rendetlen   0.9173    0.9176         0.7419}
\CommentTok{\#\textgreater{} pontatlan   0.9188    0.9192         0.6890}
\CommentTok{\#\textgreater{} pletykas    0.9262    0.9269         0.4018}
\CommentTok{\#\textgreater{} tudalekos   0.9246    0.9254         0.4601}
\CommentTok{\#\textgreater{} csamcsog    0.9199    0.9204         0.6470}
\CommentTok{\#\textgreater{} lusta       0.9195    0.9198         0.6660}
\CommentTok{\#\textgreater{} szemtelen   0.9164    0.9165         0.7874}
\CommentTok{\#\textgreater{} bufog       0.9226    0.9228         0.5616}
\CommentTok{\#\textgreater{} felelotlen  0.9177    0.9182         0.7236}
\CommentTok{\#\textgreater{} bosszuallo  0.9184    0.9191         0.6989}
\CommentTok{\#\textgreater{} durva       0.9188    0.9195         0.6832}
\CommentTok{\#\textgreater{} agressziv   0.9182    0.9188         0.7053}
\end{Highlighting}
\end{Shaded}

A fenti output alapján a Cronbach-alfa értéke (0,925), de látható, hogy
a ``pletykas'' eltávolításával tovább növelhető.

\begin{Shaded}
\begin{Highlighting}[]
\NormalTok{RcmdrMisc}\SpecialCharTok{::}\FunctionTok{reliability}\NormalTok{(}\FunctionTok{cov}\NormalTok{(d }\SpecialCharTok{\%\textgreater{}\%}
    \FunctionTok{select}\NormalTok{(}\SpecialCharTok{{-}}\NormalTok{dohanyzas, }\SpecialCharTok{{-}}\NormalTok{harsany, }\SpecialCharTok{{-}}\NormalTok{pletykas)))}
\CommentTok{\#\textgreater{} Alpha reliability =  0.9262 }
\CommentTok{\#\textgreater{} Standardized alpha =  0.9269 }
\CommentTok{\#\textgreater{} }
\CommentTok{\#\textgreater{} Reliability deleting each item in turn:}
\CommentTok{\#\textgreater{}              Alpha Std.Alpha r(item, total)}
\CommentTok{\#\textgreater{} alkohol     0.9254    0.9262         0.5101}
\CommentTok{\#\textgreater{} kabitoszer  0.9231    0.9238         0.5931}
\CommentTok{\#\textgreater{} hianyzik    0.9236    0.9242         0.5752}
\CommentTok{\#\textgreater{} udvariatlan 0.9193    0.9198         0.7250}
\CommentTok{\#\textgreater{} rendetlen   0.9189    0.9195         0.7319}
\CommentTok{\#\textgreater{} pontatlan   0.9206    0.9212         0.6769}
\CommentTok{\#\textgreater{} tudalekos   0.9269    0.9280         0.4410}
\CommentTok{\#\textgreater{} csamcsog    0.9215    0.9222         0.6453}
\CommentTok{\#\textgreater{} lusta       0.9211    0.9217         0.6617}
\CommentTok{\#\textgreater{} szemtelen   0.9174    0.9177         0.7958}
\CommentTok{\#\textgreater{} bufog       0.9243    0.9246         0.5652}
\CommentTok{\#\textgreater{} felelotlen  0.9190    0.9197         0.7274}
\CommentTok{\#\textgreater{} bosszuallo  0.9198    0.9207         0.7007}
\CommentTok{\#\textgreater{} durva       0.9200    0.9208         0.6947}
\CommentTok{\#\textgreater{} agressziv   0.9194    0.9202         0.7133}
\end{Highlighting}
\end{Shaded}

Az eredmények alapján az adatredukciót ezzel a lépéssel be is
fejezhetjük. A modellben maradt változókat tekinthetjük az általános
munkahelyi toleranciát lefedő viselkedéseknek.

\begin{figure}

{\centering \includegraphics{./images/fokomp_munkahelyi_tolarencia_kep_01.jpg}

}

\caption{Mi is az a munkahelyi tolerancia: főkomponens elemzés}

\end{figure}

\hypertarget{puxe9lda-egy-eluxe9gedettsuxe9gvizsguxe1lat-tanulsuxe1gai}{%
\section{Példa: Egy elégedettségvizsgálat
tanulságai}\label{puxe9lda-egy-eluxe9gedettsuxe9gvizsguxe1lat-tanulsuxe1gai}}

\begin{itemize}
\tightlist
\item
  A példa forrása: Münnich és mtsai. (2006)
  \href{https://psycho.unideb.hu/statisztika/pages/p_2_14.html}{2.5.4
  Probléma}
\item
  Kapcsolódó jamovi állomány:
  \texttt{fokomp\_munkahelyi\_elegedettseg.omv}
\end{itemize}

Ebben a példánkban azt a kérdést járjuk körbe, hogy mely tényezők
befolyásolják azt, hogy elégedett-e valaki az egyetemi oktatással, mely
tényezők kerülhetnének be egy tolerancia kérdőív itemei közé.

A \texttt{fokomp\_munkahelyi\_elegedettseg.xlsx} adatbázis a következő
kérdésekre adott válaszokat tartalmazza:

Mennyire vagy elégedett\ldots{}

\begin{itemize}
\tightlist
\item
  az egyetemen szerzett ismeretek felhasználhatóságával?
  (\texttt{DK210})
\item
  az egyetem ösztönző, fejlesztő tevékenységével? (\texttt{DK212})
\item
  az egyetemen az információ áramlással? (\texttt{DK214})
\item
  a szakodon tanított tárgyakkal? (\texttt{DK215})
\item
  a tanárok előadókészségével? (\texttt{DK219})
\item
  a tanárok szakmai felkészültségével? (\texttt{DK220})
\item
  az oktatóid tanítási módszereivel? (\texttt{DK221})
\item
  a kutatási lehetőségekkel? (\texttt{DK217})\\
\item
  a szakod által adott elhelyezkedési lehetőségekkel? (\texttt{DK218})
\end{itemize}

\begin{Shaded}
\begin{Highlighting}[]
\NormalTok{d }\OtherTok{\textless{}{-}}\NormalTok{ rio}\SpecialCharTok{::}\FunctionTok{import}\NormalTok{(}\AttributeTok{file =} \StringTok{"adat/fokomp\_munkahelyi\_elegedettseg.xlsx"}\NormalTok{)}
\FunctionTok{str}\NormalTok{(d)}
\CommentTok{\#\textgreater{} \textquotesingle{}data.frame\textquotesingle{}:    622 obs. of  9 variables:}
\CommentTok{\#\textgreater{}  $ DK210: num  6 3 20 13 5 5 10 16 14 5 ...}
\CommentTok{\#\textgreater{}  $ DK212: num  6 7 12 10 7 5 10 13 15 10 ...}
\CommentTok{\#\textgreater{}  $ DK214: num  11 1 16 14 10 7 10 17 14 13 ...}
\CommentTok{\#\textgreater{}  $ DK215: num  13 16 18 14 12 8 10 18 11 14 ...}
\CommentTok{\#\textgreater{}  $ DK217: num  4 10 19 10 5 15 10 15 9 18 ...}
\CommentTok{\#\textgreater{}  $ DK218: num  11 15 17 10 3 15 10 18 10 16 ...}
\CommentTok{\#\textgreater{}  $ DK219: num  10 8 17 11 9 2 10 18 13 19 ...}
\CommentTok{\#\textgreater{}  $ DK220: num  10 18 16 12 13 11 15 20 15 19 ...}
\CommentTok{\#\textgreater{}  $ DK221: num  10 10 17 13 7 5 10 15 15 17 ...}
\NormalTok{psych}\SpecialCharTok{::}\FunctionTok{headTail}\NormalTok{(d)}
\CommentTok{\#\textgreater{}     DK210 DK212 DK214 DK215 DK217 DK218 DK219 DK220}
\CommentTok{\#\textgreater{} 1       6     6    11    13     4    11    10    10}
\CommentTok{\#\textgreater{} 2       3     7     1    16    10    15     8    18}
\CommentTok{\#\textgreater{} 3      20    12    16    18    19    17    17    16}
\CommentTok{\#\textgreater{} 4      13    10    14    14    10    10    11    12}
\CommentTok{\#\textgreater{} ...   ...   ...   ...   ...   ...   ...   ...   ...}
\CommentTok{\#\textgreater{} 619     2    13    20     5    10     5    14    16}
\CommentTok{\#\textgreater{} 620    16    17    11    18    20    18    15    16}
\CommentTok{\#\textgreater{} 621    19    17     8     9    11    15    16    17}
\CommentTok{\#\textgreater{} 622    13    14     7    11    11     4     9    15}
\CommentTok{\#\textgreater{}     DK221}
\CommentTok{\#\textgreater{} 1      10}
\CommentTok{\#\textgreater{} 2      10}
\CommentTok{\#\textgreater{} 3      17}
\CommentTok{\#\textgreater{} 4      13}
\CommentTok{\#\textgreater{} ...   ...}
\CommentTok{\#\textgreater{} 619    17}
\CommentTok{\#\textgreater{} 620    19}
\CommentTok{\#\textgreater{} 621    16}
\CommentTok{\#\textgreater{} 622    10}
\end{Highlighting}
\end{Shaded}

\begin{Shaded}
\begin{Highlighting}[]
\NormalTok{psych}\SpecialCharTok{::}\FunctionTok{pca}\NormalTok{(d, }\AttributeTok{rotate =} \StringTok{"varimax"}\NormalTok{)}
\CommentTok{\#\textgreater{} Principal Components Analysis}
\CommentTok{\#\textgreater{} Call: principal(r = r, nfactors = nfactors, residua...}
\CommentTok{\#\textgreater{}     rotate = rotate, n.obs = n.obs, covar = covar, ...}
\CommentTok{\#\textgreater{}     missing = missing, impute = impute, oblique.sco...}
\CommentTok{\#\textgreater{}     method = method, use = use, cor = cor, correct ...}
\CommentTok{\#\textgreater{} Standardized loadings (pattern matrix) based upon c...}
\CommentTok{\#\textgreater{}        PC1   h2   u2 com}
\CommentTok{\#\textgreater{} DK210 0.71 0.50 0.50   1}
\CommentTok{\#\textgreater{} DK212 0.77 0.59 0.41   1}
\CommentTok{\#\textgreater{} DK214 0.55 0.30 0.70   1}
\CommentTok{\#\textgreater{} DK215 0.74 0.54 0.46   1}
\CommentTok{\#\textgreater{} DK217 0.59 0.35 0.65   1}
\CommentTok{\#\textgreater{} DK218 0.40 0.16 0.84   1}
\CommentTok{\#\textgreater{} DK219 0.79 0.62 0.38   1}
\CommentTok{\#\textgreater{} DK220 0.70 0.49 0.51   1}
\CommentTok{\#\textgreater{} DK221 0.78 0.61 0.39   1}
\CommentTok{\#\textgreater{} }
\CommentTok{\#\textgreater{}                 PC1}
\CommentTok{\#\textgreater{} SS loadings    4.17}
\CommentTok{\#\textgreater{} Proportion Var 0.46}
\CommentTok{\#\textgreater{} }
\CommentTok{\#\textgreater{} Mean item complexity =  1}
\CommentTok{\#\textgreater{} Test of the hypothesis that 1 component is sufficient.}
\CommentTok{\#\textgreater{} }
\CommentTok{\#\textgreater{} The root mean square of the residuals (RMSR) is  0.09 }
\CommentTok{\#\textgreater{}  with the empirical chi square  342.93  with prob \textless{}...}
\CommentTok{\#\textgreater{} }
\CommentTok{\#\textgreater{} Fit based upon off diagonal values = 0.95}
\end{Highlighting}
\end{Shaded}

Az első főkomponens által magyarázott variancia az összvariancia 46\%-át
teszi ki. Vizsgáljuk meg mely változók járulnak kevésbé hozzá az első
főkomponens kialakításához. A \texttt{DK214}, \texttt{DK217} és a
\texttt{DK218}-as kérdés „lóg ki'' a sorból, hiszen a hozzájuk tartozó
főkomponenssúlyok rendre alacsonyak.

\begin{Shaded}
\begin{Highlighting}[]
\NormalTok{psych}\SpecialCharTok{::}\FunctionTok{pca}\NormalTok{(d }\SpecialCharTok{\%\textgreater{}\%}
    \FunctionTok{select}\NormalTok{(}\SpecialCharTok{{-}}\NormalTok{DK214, }\SpecialCharTok{{-}}\NormalTok{DK217, }\SpecialCharTok{{-}}\NormalTok{DK218), }\AttributeTok{rotate =} \StringTok{"varimax"}\NormalTok{)}
\CommentTok{\#\textgreater{} Principal Components Analysis}
\CommentTok{\#\textgreater{} Call: principal(r = r, nfactors = nfactors, residua...}
\CommentTok{\#\textgreater{}     rotate = rotate, n.obs = n.obs, covar = covar, ...}
\CommentTok{\#\textgreater{}     missing = missing, impute = impute, oblique.sco...}
\CommentTok{\#\textgreater{}     method = method, use = use, cor = cor, correct ...}
\CommentTok{\#\textgreater{} Standardized loadings (pattern matrix) based upon c...}
\CommentTok{\#\textgreater{}        PC1   h2   u2 com}
\CommentTok{\#\textgreater{} DK210 0.71 0.51 0.49   1}
\CommentTok{\#\textgreater{} DK212 0.74 0.55 0.45   1}
\CommentTok{\#\textgreater{} DK215 0.74 0.55 0.45   1}
\CommentTok{\#\textgreater{} DK219 0.83 0.68 0.32   1}
\CommentTok{\#\textgreater{} DK220 0.74 0.54 0.46   1}
\CommentTok{\#\textgreater{} DK221 0.82 0.67 0.33   1}
\CommentTok{\#\textgreater{} }
\CommentTok{\#\textgreater{}                 PC1}
\CommentTok{\#\textgreater{} SS loadings    3.51}
\CommentTok{\#\textgreater{} Proportion Var 0.59}
\CommentTok{\#\textgreater{} }
\CommentTok{\#\textgreater{} Mean item complexity =  1}
\CommentTok{\#\textgreater{} Test of the hypothesis that 1 component is sufficient.}
\CommentTok{\#\textgreater{} }
\CommentTok{\#\textgreater{} The root mean square of the residuals (RMSR) is  0.11 }
\CommentTok{\#\textgreater{}  with the empirical chi square  208.36  with prob \textless{}...}
\CommentTok{\#\textgreater{} }
\CommentTok{\#\textgreater{} Fit based upon off diagonal values = 0.96}
\end{Highlighting}
\end{Shaded}

az első főkomponens által magyarázott variancia immár elérte az 50\%-ot
(pontosan 59\%), vagyis magyarázóértéke ezen mutató alapján elégséges. A
komponens mátrixban szereplő korrelációs értékek megfelelőek.

Vizsgáljuk meg a Cronbach-alfa értékét is.

\begin{Shaded}
\begin{Highlighting}[]
\NormalTok{RcmdrMisc}\SpecialCharTok{::}\FunctionTok{reliability}\NormalTok{(}\FunctionTok{cov}\NormalTok{(d))}
\CommentTok{\#\textgreater{} Alpha reliability =  0.8405 }
\CommentTok{\#\textgreater{} Standardized alpha =  0.8478 }
\CommentTok{\#\textgreater{} }
\CommentTok{\#\textgreater{} Reliability deleting each item in turn:}
\CommentTok{\#\textgreater{}        Alpha Std.Alpha r(item, total)}
\CommentTok{\#\textgreater{} DK210 0.8188    0.8280         0.6014}
\CommentTok{\#\textgreater{} DK212 0.8087    0.8194         0.6872}
\CommentTok{\#\textgreater{} DK214 0.8374    0.8444         0.4424}
\CommentTok{\#\textgreater{} DK215 0.8164    0.8247         0.6302}
\CommentTok{\#\textgreater{} DK217 0.8315    0.8396         0.4911}
\CommentTok{\#\textgreater{} DK218 0.8512    0.8565         0.3223}
\CommentTok{\#\textgreater{} DK219 0.8126    0.8202         0.6595}
\CommentTok{\#\textgreater{} DK220 0.8232    0.8301         0.5743}
\CommentTok{\#\textgreater{} DK221 0.8138    0.8208         0.6582}
\end{Highlighting}
\end{Shaded}

A fenti output alapján a Cronbach-alfa értéke (0,841), de látható, hogy
a ``DK218'' eltávolításával tovább növelhető.

\begin{Shaded}
\begin{Highlighting}[]
\NormalTok{RcmdrMisc}\SpecialCharTok{::}\FunctionTok{reliability}\NormalTok{(}\FunctionTok{cov}\NormalTok{(d }\SpecialCharTok{\%\textgreater{}\%}
    \FunctionTok{select}\NormalTok{(}\SpecialCharTok{{-}}\NormalTok{DK218)))}
\CommentTok{\#\textgreater{} Alpha reliability =  0.8512 }
\CommentTok{\#\textgreater{} Standardized alpha =  0.8565 }
\CommentTok{\#\textgreater{} }
\CommentTok{\#\textgreater{} Reliability deleting each item in turn:}
\CommentTok{\#\textgreater{}        Alpha Std.Alpha r(item, total)}
\CommentTok{\#\textgreater{} DK210 0.8336    0.8404         0.5886}
\CommentTok{\#\textgreater{} DK212 0.8220    0.8302         0.6807}
\CommentTok{\#\textgreater{} DK214 0.8533    0.8570         0.4454}
\CommentTok{\#\textgreater{} DK215 0.8282    0.8345         0.6380}
\CommentTok{\#\textgreater{} DK217 0.8481    0.8528         0.4828}
\CommentTok{\#\textgreater{} DK219 0.8231    0.8285         0.6758}
\CommentTok{\#\textgreater{} DK220 0.8346    0.8395         0.5905}
\CommentTok{\#\textgreater{} DK221 0.8236    0.8286         0.6797}
\end{Highlighting}
\end{Shaded}

A fenti output alapján a Cronbach-alfa értéke (0,851), de látható, hogy
a ``DK214'' eltávolításával tovább növelhető.

\begin{Shaded}
\begin{Highlighting}[]
\NormalTok{RcmdrMisc}\SpecialCharTok{::}\FunctionTok{reliability}\NormalTok{(}\FunctionTok{cov}\NormalTok{(d }\SpecialCharTok{\%\textgreater{}\%}
    \FunctionTok{select}\NormalTok{(}\SpecialCharTok{{-}}\NormalTok{DK218, }\SpecialCharTok{{-}}\NormalTok{DK214)))}
\CommentTok{\#\textgreater{} Alpha reliability =  0.8533 }
\CommentTok{\#\textgreater{} Standardized alpha =  0.857 }
\CommentTok{\#\textgreater{} }
\CommentTok{\#\textgreater{} Reliability deleting each item in turn:}
\CommentTok{\#\textgreater{}        Alpha Std.Alpha r(item, total)}
\CommentTok{\#\textgreater{} DK210 0.8359    0.8410         0.5953}
\CommentTok{\#\textgreater{} DK212 0.8274    0.8332         0.6520}
\CommentTok{\#\textgreater{} DK215 0.8303    0.8350         0.6350}
\CommentTok{\#\textgreater{} DK217 0.8564    0.8576         0.4763}
\CommentTok{\#\textgreater{} DK219 0.8206    0.8244         0.6975}
\CommentTok{\#\textgreater{} DK220 0.8367    0.8402         0.5945}
\CommentTok{\#\textgreater{} DK221 0.8221    0.8256         0.6943}
\end{Highlighting}
\end{Shaded}

A fenti output alapján a Cronbach-alfa értéke (0,853), de látható, hogy
a ``DK217'' eltávolításával tovább növelhető.

\begin{Shaded}
\begin{Highlighting}[]
\NormalTok{RcmdrMisc}\SpecialCharTok{::}\FunctionTok{reliability}\NormalTok{(}\FunctionTok{cov}\NormalTok{(d }\SpecialCharTok{\%\textgreater{}\%}
    \FunctionTok{select}\NormalTok{(}\SpecialCharTok{{-}}\NormalTok{DK218, }\SpecialCharTok{{-}}\NormalTok{DK214, }\SpecialCharTok{{-}}\NormalTok{DK217)))}
\CommentTok{\#\textgreater{} Alpha reliability =  0.8564 }
\CommentTok{\#\textgreater{} Standardized alpha =  0.8576 }
\CommentTok{\#\textgreater{} }
\CommentTok{\#\textgreater{} Reliability deleting each item in turn:}
\CommentTok{\#\textgreater{}        Alpha Std.Alpha r(item, total)}
\CommentTok{\#\textgreater{} DK210 0.8419    0.8440         0.5954}
\CommentTok{\#\textgreater{} DK212 0.8363    0.8377         0.6286}
\CommentTok{\#\textgreater{} DK215 0.8354    0.8374         0.6280}
\CommentTok{\#\textgreater{} DK219 0.8194    0.8202         0.7121}
\CommentTok{\#\textgreater{} DK220 0.8398    0.8406         0.6064}
\CommentTok{\#\textgreater{} DK221 0.8208    0.8217         0.7090}
\end{Highlighting}
\end{Shaded}

A Cronbach-alfa értékét már nem tudjuk tovább növelni a változók
eltávolításával.

Összegezve, az eredmények alapján csupán a szak által adott
elhelyezkedési lehetőségek, az információáramlás és a kutatási
lehetőségek nem kerülnek be az egyetemi oktatással való elégedettség
mérőszámába, míg a többi változó eredményei igen.

\begin{figure}

{\centering \includegraphics{./images/fokomponens_munkahelyi_elegedettseg_kep_01.jpg}

}

\caption{Egy elégedettségvizsgálat tanulságai: főkomponens elemzés}

\end{figure}

\bookmarksetup{startatroot}

\hypertarget{sec-megbizhatosag-elemzes}{%
\chapter{Megbízhatóság elemzés}\label{sec-megbizhatosag-elemzes}}

A pszichológiai tesztelés során használt mérőeszközök legfontosabb
tulajdonsága a megbízhatóság (reliabilitás) és a validitás (Carver és
Scheier, 2006; Nagy, 2006). A reliabilitás azt mutatja meg, hogy az
eszköz mennyire mér megbízhatóan, pontosan, mennyire bízhatunk abban,
hogy a mérés második és harmadik alkalommal is ugyanazt az eredményt
adja, amit az első esetben. A validitás vagy érvényesség azt jelenti,
hogy a mérőeszköz azt méri, amit mérni szeretnénk. A klasszikus
ábrázolás szerint mérőeszközünk a megbízhatóság és az érvényesség
alapján a lenti négy csoportok egyikébe is eshet:

\begin{figure}

{\centering \includegraphics{./images/megbizhatosag_kep_01.png}

}

\caption{Megbízhatóság és validitás esetei}

\end{figure}

Egy mérőeszköz megbízhatóságát mindig úgy vizsgáljuk, hogy a mérés
eredményét egy vagy több más eszköz eredményével hasonlítjuk össze. Az
összehasonlítás mindig korrelációt típusú vizsgálatot jelent, és a
magasabb korreláció egyben magasabb megbízhatóságot jelöl. Ennek
megfelelően a megbízhatósági mutatók értéktartománya megegyezik a
korrelációs együttható értéktartományával (feltehetően azonban 0 és 1
közötti lesz az értéke, negatív értéket ritkán kapunk és el is
szeretnénk kerülni).

A megbízhatósággal kapcsolatban három aspektust érdemes vizsgálni:

\begin{itemize}
\tightlist
\item
  \textbf{belső konzisztencia} -- az önjellemző skálák sok tételből
  állnak (melyik mindegyike külön mérőeszköznek tekinthető), ezek
  kitöltésével egyidejűleg végzünk egymással ekvivalens, párhuzamos
  méréseket.
\item
  \textbf{időbeli stabiltás (teszt-reteszt reliabilitás)} -- időben
  eltolva, ugyanazon mérést egy későbbi időpontban megismételve jutunk
  két mérési eredményhez, például, ha ugyanazt az önjellemző skálát
  mondjuk, egy nap eltéréssel felvesszük ugyanazon személyekkel.
\item
  \textbf{értékelő megbízhatóság (inter-rater reliabilitás)} -- amikor
  megfigyelő pontoz, akkor a megfigyelő személy a mérőeszköz, így a
  megfigyelői ítéleteket az értékelő megbízhatóságának meghatározásával
  ellenőrizzük.
\end{itemize}

\hypertarget{cronbach-alfa-belsux151-konzisztencia-muxe9ruxe9se}{%
\section{Cronbach-alfa -- belső konzisztencia
mérése}\label{cronbach-alfa-belsux151-konzisztencia-muxe9ruxe9se}}

Főkomponens elemzés segítségével könnyen tudunk több változót -
viszonylag csekély veszteséggel - egyetlen változóba tömöríteni, ezért
gyakran használják kérdőívek itemeinek szelekciójára, valamint
megbízhatóság (reliabilitás) vizsgálatra. A klasszikus tesztelmélet
keretein belül azonban a tesztek megbízhatóságának (reliabilitásának)
több lehetséges mutatója is létezik.

Cronbach 1951-es munkájában publikálta azon nézetét, hogy a korábbi
egyszerű tesztfelezéses eljárás helyett egy annál tökéletesebb mutatót
kellene használni a tesztek megbízhatóságának indikátoraként. Ha az
itemek száma alacsony vagy az itemek közötti átlagos korreláció
alacsony, akkor csökkenni fog a Cronbach-féle alfa értéke is. Az is
egyértelmű, hogy az itemek közötti alacsony korreláció arra enged
következtetni, hogy a teszt itemjei nem egy és ugyanazon dolog
vizsgálatára szolgálnak, a belőlük képzendő tesztérték nem alkalmas sem
elméleti, sem pedig gyakorlati felhasználásra.

Az ómega (McDonald \(\omega\)) korrigálja a Cronbach-alfa torzítását,
érdemes elvégezni az elemzést ezzel a mutatóval is (Kárász és mtsai.,
2022; Malkewitz és mtsai., 2023).

\hypertarget{puxe9lda-real-tuxe1rgyak-iruxe1nti-foguxe9konysuxe1g}{%
\section{Példa: Real tárgyak iránti
fogékonyság}\label{puxe9lda-real-tuxe1rgyak-iruxe1nti-foguxe9konysuxe1g}}

Egy fiktív adatbázis 9 tanuló iskolai jegyeit tartalmazza reál
tantárgyakból (matematika, fizika, kémia, informatika)
(\texttt{megbizhatosag\_tantargyak.xlsx}). Vizsgáljuk meg, ha a reál
tantárgyak iránti fogékonyságot ezzel a 4 érdemjeggyel mérnénk, akkor ez
megbízhatóság szempontjából alkalmas mérőeszköz lenne.

\begin{Shaded}
\begin{Highlighting}[]
\NormalTok{real }\OtherTok{\textless{}{-}}\NormalTok{ rio}\SpecialCharTok{::}\FunctionTok{import}\NormalTok{(}\AttributeTok{file =} \StringTok{"adat/megbizhatosag\_tantargyak.xlsx"}\NormalTok{)}
\FunctionTok{str}\NormalTok{(real)}
\CommentTok{\#\textgreater{} \textquotesingle{}data.frame\textquotesingle{}:    9 obs. of  4 variables:}
\CommentTok{\#\textgreater{}  $ matek      : num  5 4 3 2 5 1 5 2 5}
\CommentTok{\#\textgreater{}  $ fizika     : num  5 5 3 3 4 2 4 3 5}
\CommentTok{\#\textgreater{}  $ informatika: num  4 4 4 2 5 1 5 2 5}
\CommentTok{\#\textgreater{}  $ kemia      : num  5 5 3 3 5 1 5 3 5}
\end{Highlighting}
\end{Shaded}

A Cronbach alfa meghatározását végezhetjük a \texttt{\{psych\}} csomag
\texttt{alpha()} függvényével.

\begin{Shaded}
\begin{Highlighting}[]
\NormalTok{psych}\SpecialCharTok{::}\FunctionTok{alpha}\NormalTok{(real)  }\CommentTok{\# Cronbach{-}alfa}
\CommentTok{\#\textgreater{} }
\CommentTok{\#\textgreater{} Reliability analysis   }
\CommentTok{\#\textgreater{} Call: psych::alpha(x = real)}
\CommentTok{\#\textgreater{} }
\CommentTok{\#\textgreater{}   raw\_alpha std.alpha G6(smc) average\_r S/N   ase mean}
\CommentTok{\#\textgreater{}       0.97      0.97    0.98      0.89  33 0.017  3.7}
\CommentTok{\#\textgreater{}   sd median\_r}
\CommentTok{\#\textgreater{}  1.4     0.91}
\CommentTok{\#\textgreater{} }
\CommentTok{\#\textgreater{}     95\% confidence boundaries }
\CommentTok{\#\textgreater{}          lower alpha upper}
\CommentTok{\#\textgreater{} Feldt     0.91  0.97  0.99}
\CommentTok{\#\textgreater{} Duhachek  0.93  0.97  1.00}
\CommentTok{\#\textgreater{} }
\CommentTok{\#\textgreater{}  Reliability if an item is dropped:}
\CommentTok{\#\textgreater{}             raw\_alpha std.alpha G6(smc) average\_r S/N}
\CommentTok{\#\textgreater{} matek            0.94      0.95    0.95      0.86  18}
\CommentTok{\#\textgreater{} fizika           0.97      0.98    0.97      0.93  39}
\CommentTok{\#\textgreater{} informatika      0.96      0.97    0.97      0.92  33}
\CommentTok{\#\textgreater{} kemia            0.94      0.95    0.96      0.86  19}
\CommentTok{\#\textgreater{}             alpha se  var.r med.r}
\CommentTok{\#\textgreater{} matek          0.032 0.0070  0.89}
\CommentTok{\#\textgreater{} fizika         0.015 0.0013  0.95}
\CommentTok{\#\textgreater{} informatika    0.019 0.0016  0.93}
\CommentTok{\#\textgreater{} kemia          0.029 0.0084  0.87}
\CommentTok{\#\textgreater{} }
\CommentTok{\#\textgreater{}  Item statistics }
\CommentTok{\#\textgreater{}             n raw.r std.r r.cor r.drop mean  sd}
\CommentTok{\#\textgreater{} matek       9  0.99  0.98  0.98   0.97  3.6 1.6}
\CommentTok{\#\textgreater{} fizika      9  0.92  0.93  0.91   0.88  3.8 1.1}
\CommentTok{\#\textgreater{} informatika 9  0.95  0.94  0.93   0.90  3.6 1.5}
\CommentTok{\#\textgreater{} kemia       9  0.98  0.98  0.98   0.96  3.9 1.5}
\CommentTok{\#\textgreater{} }
\CommentTok{\#\textgreater{} Non missing response frequency for each item}
\CommentTok{\#\textgreater{}                1    2    3    4    5 miss}
\CommentTok{\#\textgreater{} matek       0.11 0.22 0.11 0.11 0.44    0}
\CommentTok{\#\textgreater{} fizika      0.00 0.11 0.33 0.22 0.33    0}
\CommentTok{\#\textgreater{} informatika 0.11 0.22 0.00 0.33 0.33    0}
\CommentTok{\#\textgreater{} kemia       0.11 0.00 0.33 0.00 0.56    0}
\end{Highlighting}
\end{Shaded}

A McDonald \(\omega\) értékét kiszámolhatjuk a \texttt{\{psych\}} csomag
\texttt{omega()} függvényével.

\begin{Shaded}
\begin{Highlighting}[]
\NormalTok{psych}\SpecialCharTok{::}\FunctionTok{omega}\NormalTok{(real, }\AttributeTok{plot =}\NormalTok{ F)  }\CommentTok{\# McDonald{-}ómega}
\CommentTok{\#\textgreater{} Omega }
\CommentTok{\#\textgreater{} Call: omegah(m = m, nfactors = nfactors, fm = fm, k...}
\CommentTok{\#\textgreater{}     digits = digits, title = title, sl = sl, labels...}
\CommentTok{\#\textgreater{}     plot = plot, n.obs = n.obs, rotate = rotate, Ph...}
\CommentTok{\#\textgreater{}     covar = covar)}
\CommentTok{\#\textgreater{} Alpha:                 0.97 }
\CommentTok{\#\textgreater{} G.6:                   0.98 }
\CommentTok{\#\textgreater{} Omega Hierarchical:    0.95 }
\CommentTok{\#\textgreater{} Omega H asymptotic:    0.96 }
\CommentTok{\#\textgreater{} Omega Total            0.99 }
\CommentTok{\#\textgreater{} }
\CommentTok{\#\textgreater{} Schmid Leiman Factor loadings greater than  0.2 }
\CommentTok{\#\textgreater{}                g   F1*   F2*   F3*   h2   u2   p2}
\CommentTok{\#\textgreater{} matek       0.97        0.28       0.99 0.01 0.94}
\CommentTok{\#\textgreater{} fizika      0.89  0.29             0.92 0.08 0.86}
\CommentTok{\#\textgreater{} informatika 0.91        0.28       0.95 0.05 0.87}
\CommentTok{\#\textgreater{} kemia       0.96  0.29             0.99 0.01 0.94}
\CommentTok{\#\textgreater{} }
\CommentTok{\#\textgreater{} With Sums of squares  of:}
\CommentTok{\#\textgreater{}    g  F1*  F2*  F3* }
\CommentTok{\#\textgreater{} 3.48 0.17 0.16 0.04 }
\CommentTok{\#\textgreater{} }
\CommentTok{\#\textgreater{} general/max  21.07   max/min =   4.03}
\CommentTok{\#\textgreater{} mean percent general =  0.9    with sd =  0.04 and ...}
\CommentTok{\#\textgreater{} Explained Common Variance of the general factor =  ...}
\CommentTok{\#\textgreater{} }
\CommentTok{\#\textgreater{} The degrees of freedom are {-}3  and the fit is  0 }
\CommentTok{\#\textgreater{} The number of observations was  9  with Chi Square ...}
\CommentTok{\#\textgreater{} The root mean square of the residuals is  0 }
\CommentTok{\#\textgreater{} The df corrected root mean square of the residuals ...}
\CommentTok{\#\textgreater{} }
\CommentTok{\#\textgreater{} Compare this with the adequacy of just a general fa...}
\CommentTok{\#\textgreater{} The degrees of freedom for just the general factor ...}
\CommentTok{\#\textgreater{} The number of observations was  9  with Chi Square ...}
\CommentTok{\#\textgreater{} The root mean square of the residuals is  0.05 }
\CommentTok{\#\textgreater{} The df corrected root mean square of the residuals ...}
\CommentTok{\#\textgreater{} }
\CommentTok{\#\textgreater{} RMSEA index =  0.401  and the 10 \% confidence inter...}
\CommentTok{\#\textgreater{} BIC =  0.75 }
\CommentTok{\#\textgreater{} }
\CommentTok{\#\textgreater{} Measures of factor score adequacy             }
\CommentTok{\#\textgreater{}                                                  g ...}
\CommentTok{\#\textgreater{} Correlation of scores with factors            0.98 ...}
\CommentTok{\#\textgreater{} Multiple R square of scores with factors      0.95 ...}
\CommentTok{\#\textgreater{} Minimum correlation of factor score estimates 0.90 ...}
\CommentTok{\#\textgreater{}                                                F2* ...}
\CommentTok{\#\textgreater{} Correlation of scores with factors            0.86 ...}
\CommentTok{\#\textgreater{} Multiple R square of scores with factors      0.74 ...}
\CommentTok{\#\textgreater{} Minimum correlation of factor score estimates 0.48 ...}
\CommentTok{\#\textgreater{} }
\CommentTok{\#\textgreater{}  Total, General and Subset omega for each subset}
\CommentTok{\#\textgreater{}                                                  g ...}
\CommentTok{\#\textgreater{} Omega total for total scores and subscales    0.99 ...}
\CommentTok{\#\textgreater{} Omega general for total scores and subscales  0.95 ...}
\CommentTok{\#\textgreater{} Omega group for total scores and subscales    0.04 ...}
\CommentTok{\#\textgreater{}                                                F2* F3*}
\CommentTok{\#\textgreater{} Omega total for total scores and subscales    0.99  NA}
\CommentTok{\#\textgreater{} Omega general for total scores and subscales  0.91  NA}
\CommentTok{\#\textgreater{} Omega group for total scores and subscales    0.08  NA}
\end{Highlighting}
\end{Shaded}

A fenti elemzéseket jamovi-ban a
\texttt{Factor\ /\ Reliability\ Analysis} menüpont segítségével
végezhetjük el.

\begin{figure}

{\centering \includegraphics{./images/megbizhatosag_jamovi_kep_02.jpg}

}

\caption{Megbízhatóság elemzés jamovi-ban}

\end{figure}

A fenti megbízhatósági elemzések azt mutatják, hogy a négy tantárgy alfa
értéke 0,966, ami egy igen jó érték, hiszen közel van 1-hez (jamovi-ban:
Scale Reliability Statistics). Az Item Reliability Statistics táblázat
oszlopában szereplő értékek azt mutatják, mi történik, ha egy változót
kiveszünk a modellből. Láthatjuk, hogy egyedül a fizika változó értéke
növelné az alfát, de a növekedés mértéke elenyésző lenne, tehát nem éri
meg eltávolítani a változót, hiszen minél több információnk van egy
személyről, annál jobb.

\bookmarksetup{startatroot}

\hypertarget{sec-feltaro-faktorelemzes}{%
\chapter{Feltáró faktorelemzés}\label{sec-feltaro-faktorelemzes}}

A feltáró faktorelemzést új faktorok létrehozására használjuk, a
megerősítő faktorelemzést egy meglévő modell tesztelésére (lásd
következő fejezet). Ebben a fejezetben faktorelemzés alatt a feltáró
faktorelemzést értjük (Rózsa és mtsai., 2019; Watkins, 2018).

A lineáris regresszióelemzéstől eltérően a főkomponens- és a
faktorelemzés nagy számú változó kölcsönös összefüggésén alapuló
módszer, így nincsenek függő vagy független változóink. Elemzéskor a
változók korrelációs mátrixából indulunk ki.

\hypertarget{a-fux151komponens-elemzuxe9s-uxe9s-a-faktorelemzuxe9s-uxf6sszehasonluxedtuxe1sa}{%
\section{A főkomponens elemzés és a faktorelemzés
összehasonlítása}\label{a-fux151komponens-elemzuxe9s-uxe9s-a-faktorelemzuxe9s-uxf6sszehasonluxedtuxe1sa}}

\begin{itemize}
\tightlist
\item
  A főkomponens elemzés során az adatok teljes varianciáját vesszük
  figyelembe, míg faktorelemzés során a faktorokat csak a közös
  variancia alapján becsüljük. A két eljárás egyébként nagyon hasonló
  elvekre épül.

  \begin{itemize}
  \tightlist
  \item
    Főkomponens elemzés során a korrelációs mátrix átlójában lévő 1-esek
    összege adja teljes varianciát, ami teljes egészében bekerül a
    faktormodellbe. Ezért ez az eljárás akkor javasolt, ha a fő cél,
    hogy meghatározzuk azon főkomponensek (faktorok) legkisebb számát,
    amelyek a legtöbb varianciát magyarázzák. Ezek a faktorok később jól
    alkalmazhatók többváltozós elemzésekben. Összegezve: a főkomponens
    elemzésnél részinformációkat próbálunk összegezni a lehető legkisebb
    információveszteséggel (vagyis a variancia maximalizálásával).
  \item
    Az ún. közös faktorelemzésnél a faktorokat csak a közös variancia
    alapján becsüljük, vagyis a kommunalitások kerülnek a korrelációs
    mátrix átlójába (ezek 1-nél kisebb számok). Összegezve a
    faktorelemzés általános célja egy látens, lineáris struktúra
    feltárása manifeszt változók segítségével.
  \end{itemize}
\end{itemize}

\hypertarget{fogalmak}{%
\section{Fogalmak}\label{fogalmak}}

A főkomponens- és fakorelemzésben a következő fogalmak fordulnak elő (a
komponens és a faktor szavak felcserélhetők, attól függően, hogy
főkomponens- vagy fakorelemzésről van szó).

\begin{itemize}
\tightlist
\item
  \textbf{Kommuninalitás:} a variancia azon hányada, amelyen egy változó
  osztozik a többi elemzésbe vont változóval. Ez egyben a közös faktorok
  által magyarázott variancia aránya.
\item
  \textbf{Sajátérték:} Az egyes faktorok által magyarázott teljes
  varianciát fejezi ki.
\item
  \textbf{Faktorsúly:} A változók és a faktorok közötti közönséges
  korreláció.
\item
  \textbf{Faktormátrix:} Valamennyi változónak az összes előállított
  faktorra vonatkozó faktorsúlyát tartalmazza.
\item
  \textbf{Faktorértékek:} Az előállított faktoroknak az egyes
  megkérdezettekre vonatkozóan becsült értékei.
\item
  \textbf{Sajátértékábra (scree-plot, kőtörmelék ábra):} A sajátértékek
  ábrázolása az előállított faktorok sorszámának függvényében.
\item
  \textbf{Varianciahányad:} A teljes variancia egy adott faktornak
  tulajdonított része százalékban kifejezve.
\end{itemize}

\hypertarget{a-faktormodell}{%
\section{A faktormodell}\label{a-faktormodell}}

A főkomponens és faktorelemzés annyiban hasonlít a többszörös
regressziószámításhoz, hogy minden változót kifejezhető a háttérben
meghúzódó faktorok lineáris kombinációjaként. Minden egyes változó
kifejezhető kisszámú közös faktor és egy egyedi faktor segítségével.
Ezek a faktorok nem figyelhetők meg közvetlenül. Standardizált kiinduló
változók esetén a faktormodell így írható fel:

\begin{itemize}
\item
  \(X_i=A_{i1} F_1+A_{i2} F_2+\dots+ A_{im} F_m+V_i U_i\), ahol

  \begin{itemize}
  \tightlist
  \item
    \(X_i\) az \(i.\) standardizált változó
  \item
    \(A_{i1}\) az \(i.\) változó \(j.\) közös faktorra vonatkozó
    többszörös standardizált parciális regressziós együtthatója
  \item
    \(F_j\) a \(j.\) közös faktor
  \item
    \(V_i\) az \(i.\) változó \(j.\) egyedi faktorra vonatkozó
    többszörös standardizált parciális regressziós együtthatója
  \item
    \(U_i\) az \(i.\) változó egyedi faktora
  \item
    \(m\) a közös faktorok száma.
  \end{itemize}
\end{itemize}

Az egyedi faktorok egymással és a közös faktorokkal is korrelálatlanok.
A közös faktorok kifejezhetők a megfigyelt változók lineáris
kombinációiként:

\begin{itemize}
\tightlist
\item
  \(F_i=W_{i1} X_1+W_i2 X_2+\dots+ W_im X_m+\dots+ W_ik X_k\), ahol

  \begin{itemize}
  \tightlist
  \item
    \(F_i\) az \(i.\) faktor becslése
  \item
    \(W_i\) súly vagy a faktorérték együtthatója
  \item
    \(k\) a változók száma
  \end{itemize}
\end{itemize}

A súlyokat vagy faktorérték együtthatókat úgy is meg lehet választani,
hogy az első faktor magyarázza a teljes variancia legnagyobb részét, a
második faktor a második legnagyobb részét és így tovább, valamint, hogy
a faktorok korrelálatlanok legyenek egymással. Ez történik főkomponens
elemzés esetén.

\hypertarget{a-faktorelmzuxe9s-menete}{%
\section{A faktorelmzés menete}\label{a-faktorelmzuxe9s-menete}}

\textbf{1. A probléma megfogalmazása}

A kutató arra volt kíváncsi, milyen előnyöket keresnek a fogyasztók a
fogrémvásárlásnál. Egy 30 fős mintán a válaszadókat arra kérték, hogy
jelezzék, mennyire értenek egyet a következő állításokkal (1 =
egyáltalán nem ért egyet; 7 = teljes mértékben egyetért)

\begin{itemize}
\tightlist
\item
  v1: Fontos, hogy olyan fogkrémet vásároljak, amellyel megelőzhető a
  fogszuvasodás.
\item
  v2: Az olyan fogkrémeket szeretem, amely fényessé teszi a fogaimat.
\item
  v3: Egy fogkrémnek erősítenie kell a fogínyt.
\item
  v4: Az olyan fogkrémeket szeretem, amely friss leheletet biztosít.
\item
  v5: A fog romlásának megelőzése számomra nem fontos elvárás.
\item
  v6: A legfontosabb szempont a fogkrém vásárlásánál a szép fog.
\end{itemize}

\textbf{2. A korrelációs mátrix előállítása}

A korrelációs mátrix előállítása során arra számítunk, hogy azok a
változók, amelyek között magas a korreláció, ugyanazzal a faktorral
fognak korrelálni.

\begin{figure}

{\centering \includegraphics{./images/efa_fogkrem_01.jpg}

}

\caption{Korrelációs mátrix}

\end{figure}

A fogkrémvásárlás során keresett előnyök korrelációs mátrix
tanulmányozásával látható:

\begin{itemize}
\tightlist
\item
  viszonylag magas a korreláció a v1 (fogszuvasodás megelőzése), v3
  (erős fogíny) és v5 (a fog romlásának megelőzése) között. Arra
  számítunk, hogy ezek a változók ugyanazokkal a faktorokkal fognak
  korrelálni.
\item
  viszonylag magas a korreláció a v2 (fényes fogak), v4 (friss lehelet)
  és v6 (szép fogak) változók között, ezek is feltehetőleg ugyanazokkal
  a faktorokkal fognak korrelálni.
\end{itemize}

\textbf{3. Az alkalmazási feltételek ellenőrzése}

Ahhoz, hogy a faktorelemzés alkalmazható legyen, a változóknak
korrelálniuk kell egymással. Erről meggyőződhetünk kétféle objektív
módszerrel is:

\begin{itemize}
\tightlist
\item
  Bartlett-féle szferikus próba: nullhipotézise szerint a korrelációs
  mátrix egységmátrix (a változók korrelálatlanok), azaz az átlón kívül
  minden elem nulla. Amennyiben a nullhipotézis nem vethető el, a
  faktorelemzés alkalmazhatósága megkérdőjelezhető.
\item
  Kaiser-Meyer-Olkin-féle megfelelőségi mutató: a megfigyelt korrelációs
  együtthatók nagyságát viszonyítja a parciális korrelációs együtthatók
  nagyságához. Az alacsony KMO-mutató azt jelzi, hogy a változópárok
  közötti korreláció nem magyarázható más változókkal, így a
  faktorelemzés nem megfelelő módszer. Általában 0,5 fölött érték
  kívánatos.
\end{itemize}

Jamovi-ban a \texttt{Factor\ /\ Exploratory\ Factor\ Analysis}
menüpontban tudjuk a fenti vizsgálatokat elvégezni.

\includegraphics{./images/efa_fogkrem_02.jpg} A Bartlett-féle szferikus
próba szerint a pupulációban a korrelációs mátrix nem egységmátrix (ez
számunkra kedvező), valamint a KMO-érték 0,66, ameéy elég magas
(\textgreater0,5), így megállapíthatjuk, hogy a faktorelemzés alkalmas
módszer a korrelációs mátrix elemzésére.

\textbf{4. A faktorelemzés módszerének meghatározása}

A faktorelemzés egyes módszerei abban különböznek egymástól, hogy milyen
módon határozzák meg a súlyokat vagy faktorérték együtthatókat.

A jamovi 3 módszert ismer a közös faktorok becslésére:

\begin{itemize}
\tightlist
\item
  Principal axis -- főtengelyelemzés
\item
  Minimum residuals
\item
  Maximum likelihood
\end{itemize}

\textbf{5. A faktorok számának meghatározása}

A faktorelemzés akkor ér célt, ha a változók számánál kevesebb számú
közös faktort hozunk létre. De mi legyen ez a szám. Több eljárás
létezik. Ezeket részletesen a főkomponens elemzés során bemutattuk. Itt
csak felsoroljuk őket:

\begin{itemize}
\tightlist
\item
  Horn-féle párhuzamos analízis (jamovi-ban:
  \texttt{Based\ on\ parallel\ analysis})
\item
  A priori meghatározás (jamovi-ban: \texttt{Fixed\ number})
\item
  Sajátértéken alapuló megoldás (jamovi-ban:
  \texttt{Based\ on\ eigenvalue})
\item
  Sajátértékábrán (scree-plot, kőtörmelék ábra) alapuló meghatározás
  (jamovi-ban: \texttt{Scree\ plot})
\item
  Magyarázott varianciahányadon alapuló meghatározás (jamovi-ban:
  \texttt{Component\ summary})
\end{itemize}

\textbf{6. A faktorok forgatása}

A faktorelemzés fontos eredménye a faktormátrix (jamovi-ban:
\texttt{Factor\ Loadings}).

\begin{figure}

{\centering \includegraphics{./images/efa_fogkrem_03.jpg}

}

\caption{A faktormátrix}

\end{figure}

A faktormátrix tartalmazza azokat az együtthatókat, amelyekkel a
standardizált változókat ki lehet fejezni a faktorokkal. Ezeket az
együtthatókat faktorsúlyoknak nevezzük, a faktorok és a faktorsúlyok
közötti korrelációt mutatják. A magas abszolút értékű együttható azt
jelzi, hogy a faktor és a változó szorosan összefügg. A faktormátrix
együttható alapján lehet a faktorokat értelmezni.

A kiinduló vagy rotálatlan faktormátrix jelzi ugyan az egyes változók és
a faktorok kapcsolatát, de ritkán eredményez könnyen értelmezhető
faktorokat. Ennek főképp az az oka, hogy a faktorok túl sok változóval
korrelálnak. (A rotálatlan faktormátrix beállításához jamovi-ban a
\texttt{Rotation:\ None} beállítást használjuk.)

A faktorok forgatásával a faktormátrix egyszerűbbé, könnyebben
értelmezhetővé válik. A faktorok forgatásával szeretnénk elérni:

\begin{itemize}
\tightlist
\item
  minden faktor csak néhány vátozóra rendelkezzen szignifikánsan nem
  nulla súllyal
\item
  minden változónak lehetőleg egy faktorral legyen nem nulla, azaz
  szignifikáns faktorsúlya.
\end{itemize}

A forgatás nem érinti a kommunalitásokat és a magyarázott
varianciahányadot, azonban az egy faktor által magyarázott
varianciahányad változik (és természetesen a faktorsúlyok is).

A forgatási módszereket érdemes jól megválasztani, mert más-más faktorok
azonosításához vezetnek.

Az ortogonális (derékszögű) forgatási eljárások egymással nem korreláló
faktorokat eredményeznek.

\begin{itemize}
\item
  Ezek közül az egyik legnépszerűbb a Varimax eljárás, amely
  minimalizálja a nagy faktorsúllyal rendelkező változók számát, így
  segíti a faktorok értelmezését. A magyarázott variancia egyenletesen
  próbálja elosztani a faktorok között.
\item
  A Quartimax eljárás első faktorként egy általános faktort faktort hoz
  létre, amellyel szinte mindegyik változó magasan korrelál.
\end{itemize}

A ferdeszögű forgatási eljárások során a tengelyek hegyeszöget zárnak be
egymással, és a kapott faktorok korrelálni fognak egymással. Ferdeszögű
forgatást akkor kell használni, ha feltételezhető, hogy a sokaságban a
faktorok erősen összefüggenek.

\begin{itemize}
\item
  A Promax eljárás gyorsan lefuttatható, amely főképp nagy
  adatbázisoknál jelent előnyt.
\item
  A Simplimax a Promax egy módosított formája.
\item
  Az Oblimin eljárás a tengelyek egymással bezárt szögét fokozatosan
  változtatja, ami egyben a faktorok korreláltságát is meghatározza.
\end{itemize}

\textbf{7. A faktorok értelmezése}

Az értelmezést megkönnyíti, ha meghatározzuk azokat a változókat,
amelyeknek ugyanazon a faktorra nagy a súlyuk. A faktort a magas
faktorsúlyú változók alapján lehet értelmezni.

Az 1. faktornak magasabb az együtthatói a v1 (fugszuvasodás megelőzése),
v3 (erős fogíny) változókkal, negatív az együttható a v5 (a fog
romlásának megeleőzése nem fontos) változó esetében. Ezt a faktort az
``egészséggel kapcsolatos előnyöknek'' nevezhetjük. A 2. faktor a v2
(fényes fogak), a v4 (friss lehelet), v6 (szép fogak) változókkal függ
össze. Ezt a 2. faktort ``társadalmi előnyök''-nek nevezhetjük.

Összegezve, a fogyasztók feltehetőleg két fő előnyt keresnek a
fogkrémvásárlás során: egészséggel kapcsolatos és társadalmi előnyöket.

\textbf{8. A faktorértékek kiszámítása}

A faktorelemzésnek önmagában is van értelme, hiszen látens változók
azonosításához vezet, azonban hasznos lehet a későbbi elemzések számára
a faktorértékek kiszámítása minden egyes megkérdezettre. A faktor az
eredeti változók lineáris kombinációja. A standardizált változó
értékeinek és a megfelelő faktorérték-együtthatónak a szorzata adja a
faktorétéket, amely jelen példában minden válaszadóra két faktorértéket
jelent. A faktorérték csak főkomponens elemzés esetében lehet pontosan
kiszámítani, egyébként csak közelítő értékeket kapunk.

\begin{figure}

{\centering \includegraphics{./images/efa_fogkrem_04.jpg}

}

\caption{Faktorértékek kiszámítása}

\end{figure}

\hypertarget{illeszkeduxe9si-mutatuxf3k}{%
\section{Illeszkedési mutatók}\label{illeszkeduxe9si-mutatuxf3k}}

\begin{itemize}
\item
  \textbf{CFI} - összehasonlító illeszkedési mutató (Comparative Fit
  Index) - A CFI azt méri fel, hogy egy feltételezett hipotetikus modell
  milyen mértékben reprodukálja a valós adatokon nyugvó
  kovarianciamátrixot egy független modellhez képest.
\item
  \textbf{TLI} - Tucker--Lewis-féle Illeszkedési mutató - A TLI a
  CFI-hez hasonló módon méri az illeszkedést, annyi különbséggel, hogy
  ez a mutató a modellben használt szabadságfokot is figyelembe veszi,
  így kiküszöböli a vizsgálati minta méretének befolyásoló szerepét
\end{itemize}

A CFI és TLI mutatók értéke 0 és 1 közötti tartományba eshet, ahol az
1-hez közeli érték jelzi a szoros illeszkedést. Kezdetben a mutatók
elfogadhatósági kritériumának 0,90-et adtak meg, de az utóbbi időkben
inkább a 0,95-ot tekintik az elfogadhatóság alsó határának.

\begin{itemize}
\tightlist
\item
  \textbf{RMSEA} - a becslési hiba négyzetes átlagának gyöke
  (Root-Mean-Square Error of Approximation) - A Steiger-féle RMSEA
  mutatót a modell populációs kovariancia mátrixhoz viszonyított
  illeszkedésének becsléséhez használjuk. Az RMSEA az elemszámtól
  függetlenül hasonlítja össze, hogy a valós és az optimális
  paraméterekkel rendelkező hipotetikus modell kovarianciamátrixa milyen
  mértékben illeszkedik. Az RMSEA a modell takarékosságának megbízható
  jelzője, a komplex modellek hibás specifikálásának hatékony mutatója.
  Az RMSEA értéke is 0 és 1 közé eshet, itt azonban a kisebb, 0-hoz
  közel eső érték jelzi a jobb illeszkedést. Az RMSEA értékei 0,05-ig
  szoros illeszkedést jeleznek; 0,08-os értékig pedig megfelelő
  illeszkedést.
\end{itemize}

\textbf{Model Test}

Az adatok és a teoretikus modell egybeesésének vizsgálata. Az egyik
leggyakrabban használt illeszkedési mutató a \(\chi^2\)-próba mértéke,
amelyet általában akkor tekinthetünk elfogadhatónak, ha a
szabadságfokhoz viszonyított értéke alacsony (pl. kisebb, mint a
szabadságfok kétszerese) és nem szignifikáns (p \textgreater{} 0,05).
Ennek a mutatónak azonban több korlátja létezik. A legjellemzőbbek a
többváltozós normalitás sérülésére és a mintanagyságra való érzékenység.
Számos empirikus eredmény és szimulációs vizsgálat támasztja alá, hogy a
normalitás sérülésekor vagy nagy elemszámú minta esetében a
\(\chi^2\)-próba kevésbé informatív, és a legtöbb esetben a modell
elvetését jelzi. A mintanagyságból fakadó korlátot gyakran a
\(\chi^2\)-próba szabadságfokhoz mért arányával próbálják kompenzálni
(\(\chi^2\)/szabadságfok), amelynek ugyan nincs pontos kritériuma, de az
ajánlások általában 2-től 5-ig terjednek, és a határérték alatti érték
jelez megfelelő illeszkedést.

\begin{figure}

{\centering \includegraphics{./images/efa_fogkrem_05.jpg}

}

\caption{Illeszkedési mutatók}

\end{figure}

\hypertarget{puxe9lda-vonuxe1s--uxe9s-uxe1llapot-szoronguxe1s}{%
\section{Példa: Vonás- és
állapot-szorongás}\label{puxe9lda-vonuxe1s--uxe9s-uxe1llapot-szoronguxe1s}}

\begin{itemize}
\tightlist
\item
  A példa forrása: Münnich és mtsai. (2006)
  \href{https://psycho.unideb.hu/statisztika/pages/p_3_2.html}{3.2
  fejezet}
\item
  Kapcsolódó jamovi állomány: \texttt{faktor\_szorongas.omv}
\end{itemize}

Az adatbázisunkban ún. állapot- és vonás-szorongásra vonatkozó
(hipotetikus) adatokat találunk. Az első 3 item az állapot-szorongásra,
míg a többi 3 item a vonás-szorongásra irányul. A következő kérdésekre
vártunk választ a felmérésben:

\begin{itemize}
\item
  v1: Nyugtalan vagyok.
\item
  v2: Aggódom.
\item
  v3: Félek, hogy valami baj fog történni.
\item
  v4: Hajlamos vagyok mindent a szívemre venni.
\item
  v5: Szerintem csupa nehézségből áll az életem.
\item
  v6: Ennél könnyebb életem nem is lehetne.
\item
  v1-v3: állapot-szorongás
\item
  v4-v6: vonás-szorongás
\end{itemize}

Az adatok a \texttt{faktor\_szorongas.xlsx} állományban találhatók.

\begin{Shaded}
\begin{Highlighting}[]
\NormalTok{d }\OtherTok{\textless{}{-}}\NormalTok{ rio}\SpecialCharTok{::}\FunctionTok{import}\NormalTok{(}\StringTok{"adat/faktor\_szorongas.xlsx"}\NormalTok{)}
\FunctionTok{str}\NormalTok{(d)}
\CommentTok{\#\textgreater{} \textquotesingle{}data.frame\textquotesingle{}:    10 obs. of  6 variables:}
\CommentTok{\#\textgreater{}  $ v1: num  2 5 4 7 3 5 4 7 2 3}
\CommentTok{\#\textgreater{}  $ v2: num  3 5 4 7 4 5 6 4 2 3}
\CommentTok{\#\textgreater{}  $ v3: num  2 4 3 7 4 5 6 4 2 3}
\CommentTok{\#\textgreater{}  $ v4: num  5 2 4 7 2 3 5 7 4 1}
\CommentTok{\#\textgreater{}  $ v5: num  4 1 4 7 2 3 5 6 4 1}
\CommentTok{\#\textgreater{}  $ v6: num  4 7 4 1 7 5 3 1 4 7}
\NormalTok{d}
\CommentTok{\#\textgreater{}    v1 v2 v3 v4 v5 v6}
\CommentTok{\#\textgreater{} 1   2  3  2  5  4  4}
\CommentTok{\#\textgreater{} 2   5  5  4  2  1  7}
\CommentTok{\#\textgreater{} 3   4  4  3  4  4  4}
\CommentTok{\#\textgreater{} 4   7  7  7  7  7  1}
\CommentTok{\#\textgreater{} 5   3  4  4  2  2  7}
\CommentTok{\#\textgreater{} 6   5  5  5  3  3  5}
\CommentTok{\#\textgreater{} 7   4  6  6  5  5  3}
\CommentTok{\#\textgreater{} 8   7  4  4  7  6  1}
\CommentTok{\#\textgreater{} 9   2  2  2  4  4  4}
\CommentTok{\#\textgreater{} 10  3  3  3  1  1  7}
\end{Highlighting}
\end{Shaded}

Mivel a faktoranalízis is a korrelációs mátrixból indul ki - a
főkomponens-analízishez hasonlóan -, így elsőként az adatok korrelációs
mátrixát érdemes megvizsgálni

\begin{Shaded}
\begin{Highlighting}[]
\FunctionTok{print}\NormalTok{(}\FunctionTok{cor}\NormalTok{(d), }\AttributeTok{digits =} \DecValTok{2}\NormalTok{)}
\CommentTok{\#\textgreater{}       v1    v2    v3    v4    v5    v6}
\CommentTok{\#\textgreater{} v1  1.00  0.71  0.71  0.54  0.51 {-}0.56}
\CommentTok{\#\textgreater{} v2  0.71  1.00  0.96  0.36  0.40 {-}0.36}
\CommentTok{\#\textgreater{} v3  0.71  0.96  1.00  0.36  0.44 {-}0.39}
\CommentTok{\#\textgreater{} v4  0.54  0.36  0.36  1.00  0.97 {-}0.98}
\CommentTok{\#\textgreater{} v5  0.51  0.40  0.44  0.97  1.00 {-}0.98}
\CommentTok{\#\textgreater{} v6 {-}0.56 {-}0.36 {-}0.39 {-}0.98 {-}0.98  1.00}
\end{Highlighting}
\end{Shaded}

A korrelációs mátrix értékei azt sugallják, hogy két faktort
azonosíthatunk.

Először a forgatás előtti faktorokat vizsgáljuk meg.

\begin{Shaded}
\begin{Highlighting}[]
\NormalTok{fa\_1 }\OtherTok{\textless{}{-}} \FunctionTok{factanal}\NormalTok{(d, }\AttributeTok{factors =} \DecValTok{2}\NormalTok{, }\AttributeTok{rotation =} \StringTok{"none"}\NormalTok{)}
\NormalTok{fa\_1}
\CommentTok{\#\textgreater{} }
\CommentTok{\#\textgreater{} Call:}
\CommentTok{\#\textgreater{} factanal(x = d, factors = 2, rotation = "none")}
\CommentTok{\#\textgreater{} }
\CommentTok{\#\textgreater{} Uniquenesses:}
\CommentTok{\#\textgreater{}    v1    v2    v3    v4    v5    v6 }
\CommentTok{\#\textgreater{} 0.408 0.081 0.005 0.030 0.022 0.009 }
\CommentTok{\#\textgreater{} }
\CommentTok{\#\textgreater{} Loadings:}
\CommentTok{\#\textgreater{}    Factor1 Factor2}
\CommentTok{\#\textgreater{} v1  0.763   0.102 }
\CommentTok{\#\textgreater{} v2  0.830   0.479 }
\CommentTok{\#\textgreater{} v3  0.871   0.486 }
\CommentTok{\#\textgreater{} v4  0.765  {-}0.620 }
\CommentTok{\#\textgreater{} v5  0.817  {-}0.558 }
\CommentTok{\#\textgreater{} v6 {-}0.788   0.609 }
\CommentTok{\#\textgreater{} }
\CommentTok{\#\textgreater{}                Factor1 Factor2}
\CommentTok{\#\textgreater{} SS loadings      3.904   1.541}
\CommentTok{\#\textgreater{} Proportion Var   0.651   0.257}
\CommentTok{\#\textgreater{} Cumulative Var   0.651   0.908}
\CommentTok{\#\textgreater{} }
\CommentTok{\#\textgreater{} Test of the hypothesis that 2 factors are sufficient.}
\CommentTok{\#\textgreater{} The chi square statistic is 5.27 on 4 degrees of fr...}
\CommentTok{\#\textgreater{} The p{-}value is 0.261}
\end{Highlighting}
\end{Shaded}

A fenti outputban láthatjuk a forgatás előtti faktorok adatait. Elsőként
az egyes változók egyedi hatását, az egyedi fakorokat láthatjuk a
„uniquenesses'' címszó alatt. A „loadings'' címszóval a faktorsúlyokat
jelölik. A forgatás nélküli faktorok esetében több olyan változó van,
amely mindkét faktorral erős kapcsolatban van. Ilyen például a v6
változó, amely faktorsúlya az első faktornál -0,788, a második faktornál
pedig 0,609. Ezáltal a vizsgált látens struktúra kevésbé áttekinthető.

A faktoranalízis modelljének végtelen számú alternatív megoldása van, és
ez vezet a faktoranalízis második lépéséhez, amelyet faktor-rotációnak,
vagy faktorforgatásnak hívnak.

A lenti kódban kétfaktoros megoldást kértünk, a forgatásnál a
``varimax'' módszert, míg az egyes személyek faktorértékeinek
kiszámításánál a Bartlett-módszert alkalmazzuk.

\begin{Shaded}
\begin{Highlighting}[]
\NormalTok{fa\_2 }\OtherTok{\textless{}{-}} \FunctionTok{factanal}\NormalTok{(d, }\AttributeTok{factors =} \DecValTok{2}\NormalTok{, }\AttributeTok{rotation =} \StringTok{"varimax"}\NormalTok{, }\AttributeTok{scores =} \StringTok{"Bartlett"}\NormalTok{)}
\NormalTok{fa\_2}
\CommentTok{\#\textgreater{} }
\CommentTok{\#\textgreater{} Call:}
\CommentTok{\#\textgreater{} factanal(x = d, factors = 2, scores = "Bartlett", r...}
\CommentTok{\#\textgreater{} }
\CommentTok{\#\textgreater{} Uniquenesses:}
\CommentTok{\#\textgreater{}    v1    v2    v3    v4    v5    v6 }
\CommentTok{\#\textgreater{} 0.408 0.081 0.005 0.030 0.022 0.009 }
\CommentTok{\#\textgreater{} }
\CommentTok{\#\textgreater{} Loadings:}
\CommentTok{\#\textgreater{}    Factor1 Factor2}
\CommentTok{\#\textgreater{} v1  0.404   0.655 }
\CommentTok{\#\textgreater{} v2  0.155   0.946 }
\CommentTok{\#\textgreater{} v3  0.175   0.982 }
\CommentTok{\#\textgreater{} v4  0.964   0.200 }
\CommentTok{\#\textgreater{} v5  0.949   0.280 }
\CommentTok{\#\textgreater{} v6 {-}0.970  {-}0.225 }
\CommentTok{\#\textgreater{} }
\CommentTok{\#\textgreater{}                Factor1 Factor2}
\CommentTok{\#\textgreater{} SS loadings      2.988   2.457}
\CommentTok{\#\textgreater{} Proportion Var   0.498   0.410}
\CommentTok{\#\textgreater{} Cumulative Var   0.498   0.908}
\CommentTok{\#\textgreater{} }
\CommentTok{\#\textgreater{} Test of the hypothesis that 2 factors are sufficient.}
\CommentTok{\#\textgreater{} The chi square statistic is 5.27 on 4 degrees of fr...}
\CommentTok{\#\textgreater{} The p{-}value is 0.261}
\end{Highlighting}
\end{Shaded}

A fenti outputban elsőként az egyes változók egyedi hatását, az egyedi
fakorokat láthatjuk a „uniquenesses'' címszó alatt. A egyedi faktorok és
a kommunalitások kapcsolatban vannak egymással, összegük 1. Minél
nagyobb egy változó egyedi faktorbeli értéke, annál kisebb lesz a
kommunalitása, és minél nagyobb a kommunalitás értéke, annál nagyobb
mértékben őrzi meg a faktor az eredeti változók szórását.

\begin{Shaded}
\begin{Highlighting}[]
\NormalTok{kommunalitas }\OtherTok{\textless{}{-}} \DecValTok{1} \SpecialCharTok{{-}}\NormalTok{ fa\_2}\SpecialCharTok{$}\NormalTok{uniquenesses}
\FunctionTok{print}\NormalTok{(kommunalitas, }\AttributeTok{digits =} \DecValTok{2}\NormalTok{)}
\CommentTok{\#\textgreater{}   v1   v2   v3   v4   v5   v6 }
\CommentTok{\#\textgreater{} 0.59 0.92 0.99 0.97 0.98 0.99}
\end{Highlighting}
\end{Shaded}

Az egyes változók kommunalitását a fenti output tartalmazza. Láthatjuk,
hogy a faktorok a legjobban a v6-os változó szórását őrizték meg,
legkevésbé pedig az első (v1) itemét, hiszen ezek kommunalitása a
legnagyobb, illetve a legkisebb. Mindez arra utal, hogy a faktorok az
utolsó itemből származó információkat őrizték meg a leginkább, és az
első itemből származókat a legkevésbé.

Az egyedi faktorok után a ``loadings'' címszóval a faktorsúlyokat
láthatjuk. A faktorsúlyok mutatják az egyes változók faktorokhoz való
relatív hozzájárulását, a változók és a faktor közötti korrelációt. Ezek
az értékek a már rotált faktorsúlyok. A faktorsúlyok megerősítik a
korrelációs mátrix alapján felállított hipotézisünket, mely szerint két
faktoros modell illeszkedik az adatokra. A v1-v3 faktor a második
faktornál, míg a v4-v6 az első faktornál szerepel nagyobb súllyal. A v6
item faktorsúlya negatív előjelű, ennek oka, hogy fordított itemről van
szó.

Ezután a főkomponens-analízisből már ismert varianciák és magyarázott
varianciahányadok szerepelnek. A táblázatban látható, hogy az első
faktor varianciája majdnem 3, míg a második faktoré 2,5 (``SS
loadings''). Az első faktor a varianciahányad közel 50\%-át magyarázza,
míg a második a 41\%-át („Proportion var''). A „Cumulative var'' sor
mutatja, hogy a két faktor összesen az összvariancia 91\%-át magyarázza.

Az eredmény utolsó soraiban egy khi-négyzet próbát látunk, amely azt
teszteli, hogy illeszkedik-e az adatokra az általunk választott
kétfaktoros modell. Ha a tesztstatisztika értéke túl nagy, akkor nem
illeszkedik a modell, egy másik megoldást kell választanunk. A
khi-négyzet statisztika értéke a mintára 5,27, 4 szabadsági fokkal, a
hozzá tartozó valószínűség p=0,261. Mivel jelen esetben
p\textgreater0,05, így megtartjuk a null-hipotézist, vagyis a
kétfaktoros modell valóban jól illeszkedik az adatokra. A két faktort
pedig a faktorsúlyoknál vizsgált szerkezet alapján a következőképpen
nevezhetjük el: mivel az első három változó a második faktorral mutat
szorosabb kapcsolatot, így azt elnevezhetjük az állapot-szorongás
faktorának, míg az első faktort - amely a második három változóval mutat
szorosabb kapcsolatot - a vonás-szorongásnak.

Összegezve, statisztikai mutatók megerősítették az elméletben leírt
kétfaktoros szorongás-modellt, melynek egyik faktora a vonás-, másik
faktora az állapot-szorongás.

\begin{figure}

{\centering \includegraphics{./images/faktor_szorongas_kep_01.jpg}

}

\caption{Vonás- és állapot-szorongás: Feltáró faktorelemzés}

\end{figure}

\hypertarget{puxe9lda-valuxf3ban-szuxe9tvuxe1laszthatuxf3-a-reuxe1l-uxe9s-a-humuxe1n-tuxe1rgyakhoz-szuxfcksuxe9ges-tuduxe1s}{%
\section{Példa: Valóban szétválasztható a reál és a humán tárgyakhoz
szükséges
tudás?}\label{puxe9lda-valuxf3ban-szuxe9tvuxe1laszthatuxf3-a-reuxe1l-uxe9s-a-humuxe1n-tuxe1rgyakhoz-szuxfcksuxe9ges-tuduxe1s}}

\begin{itemize}
\tightlist
\item
  A példa forrása: Münnich és mtsai. (2006)
  \href{https://psycho.unideb.hu/statisztika/pages/p_3_11.html}{3.7.1
  Probléma}
\item
  Kapcsolódó jamovi állomány: \texttt{faktor\_real\_human\_targyak.omv}
\end{itemize}

Ebben a példában azzal foglalkozunk, hogy a diákok teljesítménye alapján
a tantárgyak ``szétválnak-e'' reál és humán tárgyakra, avagy
illeszthetünk-e egy kétfaktoros modellt az adatokra.

Az adatok a \texttt{faktor\_real\_human\_targyak.xlsx} állományban
találhatók.

\begin{Shaded}
\begin{Highlighting}[]
\NormalTok{d }\OtherTok{\textless{}{-}}\NormalTok{ rio}\SpecialCharTok{::}\FunctionTok{import}\NormalTok{(}\AttributeTok{file =} \StringTok{"adat/faktor\_real\_human\_targyak.xlsx"}\NormalTok{)}
\FunctionTok{str}\NormalTok{(d)}
\CommentTok{\#\textgreater{} \textquotesingle{}data.frame\textquotesingle{}:    30 obs. of  6 variables:}
\CommentTok{\#\textgreater{}  $ matek      : num  5 4 3 2 5 1 5 2 5 5 ...}
\CommentTok{\#\textgreater{}  $ informatika: num  4 4 4 2 5 1 5 2 5 4 ...}
\CommentTok{\#\textgreater{}  $ kemia      : num  5 5 3 3 5 1 5 3 5 5 ...}
\CommentTok{\#\textgreater{}  $ irodalom   : num  5 4 2 5 3 5 3 5 4 2 ...}
\CommentTok{\#\textgreater{}  $ nyelvtan   : num  4 4 2 5 3 5 3 5 5 2 ...}
\CommentTok{\#\textgreater{}  $ angol      : num  5 5 3 5 3 5 3 5 5 2 ...}
\NormalTok{psych}\SpecialCharTok{::}\FunctionTok{headTail}\NormalTok{(d)}
\CommentTok{\#\textgreater{}     matek informatika kemia irodalom nyelvtan angol}
\CommentTok{\#\textgreater{} 1       5           4     5        5        4     5}
\CommentTok{\#\textgreater{} 2       4           4     5        4        4     5}
\CommentTok{\#\textgreater{} 3       3           4     3        2        2     3}
\CommentTok{\#\textgreater{} 4       2           2     3        5        5     5}
\CommentTok{\#\textgreater{} ...   ...         ...   ...      ...      ...   ...}
\CommentTok{\#\textgreater{} 27      5           5     5        2        2     3}
\CommentTok{\#\textgreater{} 28      5           5     5        4        4     4}
\CommentTok{\#\textgreater{} 29      2           2     3        4        5     5}
\CommentTok{\#\textgreater{} 30      5           5     5        4        5     5}
\end{Highlighting}
\end{Shaded}

Összesen hat változónk van. Az első hármat ``hétköznapi'' tudásunk
alapján a reál tárgyak csoportjába, míg a második hármat a humán tárgyak
csoportjába sorolnánk.

\begin{Shaded}
\begin{Highlighting}[]
\FunctionTok{print}\NormalTok{(}\FunctionTok{cor}\NormalTok{(d), }\AttributeTok{digits =} \DecValTok{2}\NormalTok{)}
\CommentTok{\#\textgreater{}             matek informatika kemia irodalom nyelvtan}
\CommentTok{\#\textgreater{} matek        1.00        0.94  0.91    {-}0.15    {-}0.21}
\CommentTok{\#\textgreater{} informatika  0.94        1.00  0.82    {-}0.20    {-}0.23}
\CommentTok{\#\textgreater{} kemia        0.91        0.82  1.00    {-}0.19    {-}0.23}
\CommentTok{\#\textgreater{} irodalom    {-}0.15       {-}0.20 {-}0.19     1.00     0.93}
\CommentTok{\#\textgreater{} nyelvtan    {-}0.21       {-}0.23 {-}0.23     0.93     1.00}
\CommentTok{\#\textgreater{} angol       {-}0.21       {-}0.23 {-}0.21     0.89     0.95}
\CommentTok{\#\textgreater{}             angol}
\CommentTok{\#\textgreater{} matek       {-}0.21}
\CommentTok{\#\textgreater{} informatika {-}0.23}
\CommentTok{\#\textgreater{} kemia       {-}0.21}
\CommentTok{\#\textgreater{} irodalom     0.89}
\CommentTok{\#\textgreater{} nyelvtan     0.95}
\CommentTok{\#\textgreater{} angol        1.00}
\end{Highlighting}
\end{Shaded}

A korrelációs mátrix értékei azt sugallják, hogy két faktort
azonosíthatunk. Az első faktort az első három változó (vagyis a reál
tárgyak) alkotják, míg a második faktort a második három változó, azaz a
humán tárgyak adják. A következő lépésben faktoranalízis segítségével
teszteljük, hogy helyes-e a megérzésünk.

\begin{Shaded}
\begin{Highlighting}[]
\NormalTok{fa\_1 }\OtherTok{\textless{}{-}} \FunctionTok{factanal}\NormalTok{(d, }\AttributeTok{factors =} \DecValTok{2}\NormalTok{, }\AttributeTok{rotation =} \StringTok{"varimax"}\NormalTok{, }\AttributeTok{scores =} \StringTok{"Bartlett"}\NormalTok{)}
\NormalTok{fa\_1}
\CommentTok{\#\textgreater{} }
\CommentTok{\#\textgreater{} Call:}
\CommentTok{\#\textgreater{} factanal(x = d, factors = 2, scores = "Bartlett", r...}
\CommentTok{\#\textgreater{} }
\CommentTok{\#\textgreater{} Uniquenesses:}
\CommentTok{\#\textgreater{}       matek informatika       kemia    irodalom }
\CommentTok{\#\textgreater{}       0.005       0.114       0.173       0.122 }
\CommentTok{\#\textgreater{}    nyelvtan       angol }
\CommentTok{\#\textgreater{}       0.006       0.094 }
\CommentTok{\#\textgreater{} }
\CommentTok{\#\textgreater{} Loadings:}
\CommentTok{\#\textgreater{}             Factor1 Factor2}
\CommentTok{\#\textgreater{} matek       {-}0.107   0.992 }
\CommentTok{\#\textgreater{} informatika {-}0.136   0.931 }
\CommentTok{\#\textgreater{} kemia       {-}0.139   0.898 }
\CommentTok{\#\textgreater{} irodalom     0.936         }
\CommentTok{\#\textgreater{} nyelvtan     0.991  {-}0.106 }
\CommentTok{\#\textgreater{} angol        0.946  {-}0.105 }
\CommentTok{\#\textgreater{} }
\CommentTok{\#\textgreater{}                Factor1 Factor2}
\CommentTok{\#\textgreater{} SS loadings      2.802   2.683}
\CommentTok{\#\textgreater{} Proportion Var   0.467   0.447}
\CommentTok{\#\textgreater{} Cumulative Var   0.467   0.914}
\CommentTok{\#\textgreater{} }
\CommentTok{\#\textgreater{} Test of the hypothesis that 2 factors are sufficient.}
\CommentTok{\#\textgreater{} The chi square statistic is 5.24 on 4 degrees of fr...}
\CommentTok{\#\textgreater{} The p{-}value is 0.264}
\end{Highlighting}
\end{Shaded}

Láthatjuk, hogy a khi-négyzet statisztika alapján a kétfaktoros modell
illeszkedik az adatokra, hiszen a statisztikához tartozó
szignifikancia-szint p=0,264.

A ``Proportion Var'' sor mutatja, hogy egyes faktorok az összvariancia
hány százalékát magyarázzák. Láthatjuk, hogy az első faktor 47\%-át
magyarázza az összvarianciának, míg a második 45\%-át - kerekített
értékben. A két faktor összesen kb. 91\%-át magyarázza az
összvarianciának.

A ``Loadings'' résznél láthatjuk a faktorsúlyokat. A faktorsúlyok
értékei megerősítik azt, amit a korrelációs mátrix és az előzetes
tudásunk alapján véltünk: a matek, informatika és a kémia tárgyak
alkotják az egyik faktort (a másodikat), a faktorsúlyok a második
faktornál 0,9-es érték körül mozognak. Az irodalom, nyelvtan és angol
tárgyak alkotják a másik faktort (az elsőt), az ide tartozó faktorsúlyok
is 0,9 felett vannak.

\begin{Shaded}
\begin{Highlighting}[]
\NormalTok{kommunalitas }\OtherTok{\textless{}{-}} \DecValTok{1} \SpecialCharTok{{-}}\NormalTok{ fa\_1}\SpecialCharTok{$}\NormalTok{uniquenesses}
\FunctionTok{print}\NormalTok{(kommunalitas, }\AttributeTok{digits =} \DecValTok{3}\NormalTok{)}
\CommentTok{\#\textgreater{}       matek informatika       kemia    irodalom }
\CommentTok{\#\textgreater{}       0.995       0.886       0.827       0.878 }
\CommentTok{\#\textgreater{}    nyelvtan       angol }
\CommentTok{\#\textgreater{}       0.994       0.906}
\end{Highlighting}
\end{Shaded}

A kommunalitások alapján látható, hogy az eredeti változók a szórásuk
nagy részét megőrizték a faktorba kerüléskor. A magas, 0,9 körüli
értékek arra utalnak, hogy a kétfaktoros modellnél az
információveszteség elenyészően kicsi.

\begin{Shaded}
\begin{Highlighting}[]
\FunctionTok{print}\NormalTok{(fa\_1}\SpecialCharTok{$}\NormalTok{scores, }\AttributeTok{digits =} \DecValTok{3}\NormalTok{)}
\CommentTok{\#\textgreater{}    Factor1 Factor2}
\CommentTok{\#\textgreater{} 1    0.400   0.950}
\CommentTok{\#\textgreater{} 2    0.299   0.314}
\CommentTok{\#\textgreater{} 3   {-}1.481  {-}0.542}
\CommentTok{\#\textgreater{} 4    0.965  {-}0.952}
\CommentTok{\#\textgreater{} 5   {-}0.534   0.875}
\CommentTok{\#\textgreater{} 6    0.893  {-}1.648}
\CommentTok{\#\textgreater{} 7   {-}0.534   0.875}
\CommentTok{\#\textgreater{} 8    0.965  {-}0.952}
\CommentTok{\#\textgreater{} 9    1.143   1.057}
\CommentTok{\#\textgreater{} 10  {-}1.391   0.752}
\CommentTok{\#\textgreater{} 11   1.107   0.403}
\CommentTok{\#\textgreater{} 12   0.181  {-}0.359}
\CommentTok{\#\textgreater{} 13  {-}1.601  {-}1.234}
\CommentTok{\#\textgreater{} 14   1.175   1.034}
\CommentTok{\#\textgreater{} 15  {-}0.818  {-}1.836}
\CommentTok{\#\textgreater{} 16   1.175   1.026}
\CommentTok{\#\textgreater{} 17   0.109  {-}1.046}
\CommentTok{\#\textgreater{} 18  {-}1.389   0.781}
\CommentTok{\#\textgreater{} 19   0.353   0.945}
\CommentTok{\#\textgreater{} 20  {-}0.604   0.215}
\CommentTok{\#\textgreater{} 21   0.181  {-}0.359}
\CommentTok{\#\textgreater{} 22   0.965  {-}0.952}
\CommentTok{\#\textgreater{} 23  {-}1.389   0.781}
\CommentTok{\#\textgreater{} 24   0.894  {-}1.620}
\CommentTok{\#\textgreater{} 25  {-}0.534   0.875}
\CommentTok{\#\textgreater{} 26  {-}1.586  {-}1.235}
\CommentTok{\#\textgreater{} 27  {-}1.341   0.786}
\CommentTok{\#\textgreater{} 28   0.321   0.969}
\CommentTok{\#\textgreater{} 29   0.931  {-}0.958}
\CommentTok{\#\textgreater{} 30   1.143   1.057}
\end{Highlighting}
\end{Shaded}

Végül, nézzük meg az egyes személyek faktorértékeit. A faktorértékeknél
azt láthatjuk, hogy akik reál tárgyakból értek el jobb eredményt, azok a
második faktorban kaptak magasabb pontszámot, míg akik a humán
tárgyakból kaptak jobb jegyeket, azok az első faktorban kaptak magasabb
pontszámokat.

Összefoglalva, az adatokra jól illeszkedik a kétfaktoros modell, vagyis
azonosíthatjuk a humán és a reál tárgyakat az egyes tantárgyakból
nyújtott eredmények alapján. Az egyes tárgyak faktorba történő
besorolása összhangban van ``hétköznapi'', előzetes tudásunkkal: a
matek, informatika és a kémia sorolható a reál, míg az irodalom,
nyelvtan és angol tárgyak a humán tárgyakhoz.

\begin{figure}

{\centering \includegraphics{./images/faktor_real_human_targyak_kep_01.jpg}

}

\caption{Valóban szétválasztható a reál és a humán tárgyakhoz szükséges
tudás: Feltáró faktorelemzés}

\end{figure}

\hypertarget{puxe9lda-toleranciavizsguxe1lat-egy-muxe1sik-aspektusbuxf3l}{%
\section{Példa: Toleranciavizsgálat egy másik
aspektusból}\label{puxe9lda-toleranciavizsguxe1lat-egy-muxe1sik-aspektusbuxf3l}}

\begin{itemize}
\tightlist
\item
  A példa forrása: Münnich és mtsai. (2006)
  \href{https://psycho.unideb.hu/statisztika/pages/p_3_12.html}{3.7.2
  Probléma}
\item
  Kapcsolódó jamovi állomány:
  \texttt{faktor\_munkahelyi\_tolarencia.omv}
\end{itemize}

A főkomponenselemzés kapcsán már volt szó a toleranciáról.
Megvizsgáltuk, hogy milyen jelenségek, milyen változók tartoznak a
tolerancia körébe. Most azt próbáljuk megállapítani, hogyan épül fel a
tolerancia, milyen a szerkezete, vannak-e látens dimenziói, ha igen,
akkor melyek ezek.

Az adatok a \texttt{faktor\_munkahelyi\_tolarencia.xlsx} állományban
találhatók.

\begin{Shaded}
\begin{Highlighting}[]
\NormalTok{d }\OtherTok{\textless{}{-}}\NormalTok{ rio}\SpecialCharTok{::}\FunctionTok{import}\NormalTok{(}\StringTok{"adat/faktor\_munkahelyi\_tolarencia.xlsx"}\NormalTok{)}
\FunctionTok{str}\NormalTok{(d)}
\CommentTok{\#\textgreater{} \textquotesingle{}data.frame\textquotesingle{}:    155 obs. of  18 variables:}
\CommentTok{\#\textgreater{}  $ alkohol    : num  1 1 1 1 1 2 1 1 1 1 ...}
\CommentTok{\#\textgreater{}  $ kabitoszer : num  1 1 1 1 1 1 1 1 1 1 ...}
\CommentTok{\#\textgreater{}  $ hianyzik   : num  3 1 1 2 1 1 5 1 1 3 ...}
\CommentTok{\#\textgreater{}  $ dohanyzas  : num  4 5 1 3 1 4 5 1 5 5 ...}
\CommentTok{\#\textgreater{}  $ udvariatlan: num  3 1 5 2 2 1 1 1 1 1 ...}
\CommentTok{\#\textgreater{}  $ rendetlen  : num  3 1 5 2 2 1 1 1 1 3 ...}
\CommentTok{\#\textgreater{}  $ pontatlan  : num  3 1 5 3 2 1 1 1 1 3 ...}
\CommentTok{\#\textgreater{}  $ pletykas   : num  1 1 5 2 1 2 1 1 1 3 ...}
\CommentTok{\#\textgreater{}  $ harsany    : num  4 3 5 2 2 4 1 1 2 3 ...}
\CommentTok{\#\textgreater{}  $ tudalekos  : num  3 2 4 3 2 2 1 1 2 1 ...}
\CommentTok{\#\textgreater{}  $ csamcsog   : num  3 1 5 3 3 1 1 1 1 3 ...}
\CommentTok{\#\textgreater{}  $ lusta      : num  3 1 5 2 3 4 1 1 1 5 ...}
\CommentTok{\#\textgreater{}  $ szemtelen  : num  3 1 5 2 2 1 1 1 1 1 ...}
\CommentTok{\#\textgreater{}  $ bufog      : num  3 1 5 2 2 5 5 1 1 1 ...}
\CommentTok{\#\textgreater{}  $ felelotlen : num  3 1 5 2 2 1 1 1 1 1 ...}
\CommentTok{\#\textgreater{}  $ bosszuallo : num  2 2 3 2 1 1 1 1 1 1 ...}
\CommentTok{\#\textgreater{}  $ durva      : num  2 1 5 2 2 1 1 1 1 1 ...}
\CommentTok{\#\textgreater{}  $ agressziv  : num  2 1 5 1 2 1 1 1 1 1 ...}
\NormalTok{psych}\SpecialCharTok{::}\FunctionTok{headTail}\NormalTok{(d)}
\CommentTok{\#\textgreater{}     alkohol kabitoszer hianyzik dohanyzas udvariatlan}
\CommentTok{\#\textgreater{} 1         1          1        3         4           3}
\CommentTok{\#\textgreater{} 2         1          1        1         5           1}
\CommentTok{\#\textgreater{} 3         1          1        1         1           5}
\CommentTok{\#\textgreater{} 4         1          1        2         3           2}
\CommentTok{\#\textgreater{} ...     ...        ...      ...       ...         ...}
\CommentTok{\#\textgreater{} 152       3          1        2         5           3}
\CommentTok{\#\textgreater{} 153       3          1        2         2           2}
\CommentTok{\#\textgreater{} 154       4          4        2         5           3}
\CommentTok{\#\textgreater{} 155       3          3        4         5           3}
\CommentTok{\#\textgreater{}     rendetlen pontatlan pletykas harsany tudalekos}
\CommentTok{\#\textgreater{} 1           3         3        1       4         3}
\CommentTok{\#\textgreater{} 2           1         1        1       3         2}
\CommentTok{\#\textgreater{} 3           5         5        5       5         4}
\CommentTok{\#\textgreater{} 4           2         3        2       2         3}
\CommentTok{\#\textgreater{} ...       ...       ...      ...     ...       ...}
\CommentTok{\#\textgreater{} 152         4         2        2       5         3}
\CommentTok{\#\textgreater{} 153         1         1        1       1         2}
\CommentTok{\#\textgreater{} 154         4         5        5       3         2}
\CommentTok{\#\textgreater{} 155         4         4        2       2         1}
\CommentTok{\#\textgreater{}     csamcsog lusta szemtelen bufog felelotlen bossz...}
\CommentTok{\#\textgreater{} 1          3     3         3     3          3      ...}
\CommentTok{\#\textgreater{} 2          1     1         1     1          1      ...}
\CommentTok{\#\textgreater{} 3          5     5         5     5          5      ...}
\CommentTok{\#\textgreater{} 4          3     2         2     2          2      ...}
\CommentTok{\#\textgreater{} ...      ...   ...       ...   ...        ...      ...}
\CommentTok{\#\textgreater{} 152        4     5         4     4          2      ...}
\CommentTok{\#\textgreater{} 153        1     1         1     1          1      ...}
\CommentTok{\#\textgreater{} 154        2     3         3     4          5      ...}
\CommentTok{\#\textgreater{} 155        1     3         2     1          2      ...}
\CommentTok{\#\textgreater{}     durva agressziv}
\CommentTok{\#\textgreater{} 1       2         2}
\CommentTok{\#\textgreater{} 2       1         1}
\CommentTok{\#\textgreater{} 3       5         5}
\CommentTok{\#\textgreater{} 4       2         1}
\CommentTok{\#\textgreater{} ...   ...       ...}
\CommentTok{\#\textgreater{} 152     3         1}
\CommentTok{\#\textgreater{} 153     1         1}
\CommentTok{\#\textgreater{} 154     5         5}
\CommentTok{\#\textgreater{} 155     2         2}
\end{Highlighting}
\end{Shaded}

\begin{Shaded}
\begin{Highlighting}[]
\FunctionTok{print}\NormalTok{(}\FunctionTok{cor}\NormalTok{(d), }\AttributeTok{digits =} \DecValTok{2}\NormalTok{)}
\CommentTok{\#\textgreater{}             alkohol kabitoszer hianyzik dohanyzas}
\CommentTok{\#\textgreater{} alkohol       1.000     0.7293     0.50     0.285}
\CommentTok{\#\textgreater{} kabitoszer    0.729     1.0000     0.49     0.110}
\CommentTok{\#\textgreater{} hianyzik      0.498     0.4901     1.00     0.246}
\CommentTok{\#\textgreater{} dohanyzas     0.285     0.1095     0.25     1.000}
\CommentTok{\#\textgreater{} udvariatlan   0.404     0.4497     0.53     0.145}
\CommentTok{\#\textgreater{} rendetlen     0.372     0.4119     0.60     0.202}
\CommentTok{\#\textgreater{} pontatlan     0.340     0.4265     0.52     0.095}
\CommentTok{\#\textgreater{} pletykas      0.138     0.2321     0.19     0.071}
\CommentTok{\#\textgreater{} harsany       0.064    {-}0.0092     0.10     0.178}
\CommentTok{\#\textgreater{} tudalekos     0.129     0.1349     0.21     0.021}
\CommentTok{\#\textgreater{} csamcsog      0.324     0.3034     0.31     0.108}
\CommentTok{\#\textgreater{} lusta         0.274     0.2901     0.34     0.156}
\CommentTok{\#\textgreater{} szemtelen     0.304     0.4449     0.45     0.060}
\CommentTok{\#\textgreater{} bufog         0.283     0.3011     0.34     0.160}
\CommentTok{\#\textgreater{} felelotlen    0.304     0.4867     0.38    {-}0.068}
\CommentTok{\#\textgreater{} bosszuallo    0.372     0.5641     0.33    {-}0.010}
\CommentTok{\#\textgreater{} durva         0.389     0.4186     0.38     0.146}
\CommentTok{\#\textgreater{} agressziv     0.344     0.4207     0.36     0.024}
\CommentTok{\#\textgreater{}             udvariatlan rendetlen pontatlan pletykas}
\CommentTok{\#\textgreater{} alkohol            0.40      0.37     0.340    0.138}
\CommentTok{\#\textgreater{} kabitoszer         0.45      0.41     0.427    0.232}
\CommentTok{\#\textgreater{} hianyzik           0.53      0.60     0.519    0.193}
\CommentTok{\#\textgreater{} dohanyzas          0.15      0.20     0.095    0.071}
\CommentTok{\#\textgreater{} udvariatlan        1.00      0.70     0.577    0.375}
\CommentTok{\#\textgreater{} rendetlen          0.70      1.00     0.795    0.421}
\CommentTok{\#\textgreater{} pontatlan          0.58      0.80     1.000    0.420}
\CommentTok{\#\textgreater{} pletykas           0.38      0.42     0.420    1.000}
\CommentTok{\#\textgreater{} harsany            0.32      0.38     0.290    0.502}
\CommentTok{\#\textgreater{} tudalekos          0.42      0.38     0.361    0.397}
\CommentTok{\#\textgreater{} csamcsog           0.46      0.49     0.447    0.300}
\CommentTok{\#\textgreater{} lusta              0.47      0.55     0.469    0.335}
\CommentTok{\#\textgreater{} szemtelen          0.70      0.62     0.532    0.264}
\CommentTok{\#\textgreater{} bufog              0.45      0.39     0.336    0.212}
\CommentTok{\#\textgreater{} felelotlen         0.53      0.54     0.582    0.280}
\CommentTok{\#\textgreater{} bosszuallo         0.46      0.41     0.416    0.290}
\CommentTok{\#\textgreater{} durva              0.50      0.48     0.419    0.190}
\CommentTok{\#\textgreater{} agressziv          0.48      0.46     0.467    0.234}
\CommentTok{\#\textgreater{}             harsany tudalekos csamcsog lusta szemtelen}
\CommentTok{\#\textgreater{} alkohol      0.0636     0.129     0.32  0.27      0.30}
\CommentTok{\#\textgreater{} kabitoszer  {-}0.0092     0.135     0.30  0.29      0.44}
\CommentTok{\#\textgreater{} hianyzik     0.0995     0.206     0.31  0.34      0.45}
\CommentTok{\#\textgreater{} dohanyzas    0.1781     0.021     0.11  0.16      0.06}
\CommentTok{\#\textgreater{} udvariatlan  0.3230     0.416     0.46  0.47      0.70}
\CommentTok{\#\textgreater{} rendetlen    0.3791     0.378     0.49  0.55      0.62}
\CommentTok{\#\textgreater{} pontatlan    0.2896     0.361     0.45  0.47      0.53}
\CommentTok{\#\textgreater{} pletykas     0.5020     0.397     0.30  0.34      0.26}
\CommentTok{\#\textgreater{} harsany      1.0000     0.501     0.43  0.48      0.30}
\CommentTok{\#\textgreater{} tudalekos    0.5012     1.000     0.51  0.43      0.43}
\CommentTok{\#\textgreater{} csamcsog     0.4316     0.514     1.00  0.54      0.55}
\CommentTok{\#\textgreater{} lusta        0.4796     0.429     0.54  1.00      0.66}
\CommentTok{\#\textgreater{} szemtelen    0.2995     0.434     0.55  0.66      1.00}
\CommentTok{\#\textgreater{} bufog        0.2902     0.272     0.67  0.41      0.52}
\CommentTok{\#\textgreater{} felelotlen   0.1495     0.319     0.40  0.58      0.69}
\CommentTok{\#\textgreater{} bosszuallo   0.0583     0.333     0.46  0.47      0.60}
\CommentTok{\#\textgreater{} durva        0.1358     0.229     0.44  0.54      0.66}
\CommentTok{\#\textgreater{} agressziv    0.1348     0.309     0.45  0.53      0.60}
\CommentTok{\#\textgreater{}             bufog felelotlen bosszuallo durva agres...}
\CommentTok{\#\textgreater{} alkohol      0.28      0.304      0.372  0.39     0...}
\CommentTok{\#\textgreater{} kabitoszer   0.30      0.487      0.564  0.42     0...}
\CommentTok{\#\textgreater{} hianyzik     0.34      0.380      0.331  0.38     0...}
\CommentTok{\#\textgreater{} dohanyzas    0.16     {-}0.068     {-}0.010  0.15     0...}
\CommentTok{\#\textgreater{} udvariatlan  0.45      0.532      0.458  0.50     0...}
\CommentTok{\#\textgreater{} rendetlen    0.39      0.542      0.410  0.48     0...}
\CommentTok{\#\textgreater{} pontatlan    0.34      0.582      0.416  0.42     0...}
\CommentTok{\#\textgreater{} pletykas     0.21      0.280      0.290  0.19     0...}
\CommentTok{\#\textgreater{} harsany      0.29      0.149      0.058  0.14     0...}
\CommentTok{\#\textgreater{} tudalekos    0.27      0.319      0.333  0.23     0...}
\CommentTok{\#\textgreater{} csamcsog     0.67      0.400      0.461  0.44     0...}
\CommentTok{\#\textgreater{} lusta        0.41      0.579      0.472  0.54     0...}
\CommentTok{\#\textgreater{} szemtelen    0.52      0.695      0.601  0.66     0...}
\CommentTok{\#\textgreater{} bufog        1.00      0.389      0.432  0.44     0...}
\CommentTok{\#\textgreater{} felelotlen   0.39      1.000      0.712  0.57     0...}
\CommentTok{\#\textgreater{} bosszuallo   0.43      0.712      1.000  0.56     0...}
\CommentTok{\#\textgreater{} durva        0.44      0.571      0.556  1.00     0...}
\CommentTok{\#\textgreater{} agressziv    0.39      0.643      0.731  0.79     1...}
\end{Highlighting}
\end{Shaded}

Vannak változók, melyek között szinte nincs is kapcsolat, olyan gyenge a
korreláció (ilyen például az ``alkohol'' és a ``harsány'' változó
közötti korreláció, melynek értéke 0,06), és vannak olyan változók is,
melyek között szorosabb kapcsolat figyelhető meg (ilyen például az
``alkohol'' és a ``kábítószer'' változó, melyek közötti korreláció
mértéke 0,73).

Végezzünk faktorelemzést.

\begin{Shaded}
\begin{Highlighting}[]
\NormalTok{fa\_1 }\OtherTok{\textless{}{-}} \FunctionTok{factanal}\NormalTok{(d, }\AttributeTok{factors =} \DecValTok{6}\NormalTok{, }\AttributeTok{rotation =} \StringTok{"varimax"}\NormalTok{, }\AttributeTok{scores =} \StringTok{"Bartlett"}\NormalTok{)}
\NormalTok{fa\_1}
\CommentTok{\#\textgreater{} }
\CommentTok{\#\textgreater{} Call:}
\CommentTok{\#\textgreater{} factanal(x = d, factors = 6, scores = "Bartlett", r...}
\CommentTok{\#\textgreater{} }
\CommentTok{\#\textgreater{} Uniquenesses:}
\CommentTok{\#\textgreater{}     alkohol  kabitoszer    hianyzik   dohanyzas }
\CommentTok{\#\textgreater{}       0.161       0.262       0.485       0.805 }
\CommentTok{\#\textgreater{} udvariatlan   rendetlen   pontatlan    pletykas }
\CommentTok{\#\textgreater{}       0.400       0.101       0.296       0.641 }
\CommentTok{\#\textgreater{}     harsany   tudalekos    csamcsog       lusta }
\CommentTok{\#\textgreater{}       0.005       0.584       0.005       0.436 }
\CommentTok{\#\textgreater{}   szemtelen       bufog  felelotlen  bosszuallo }
\CommentTok{\#\textgreater{}       0.320       0.513       0.274       0.190 }
\CommentTok{\#\textgreater{}       durva   agressziv }
\CommentTok{\#\textgreater{}       0.005       0.247 }
\CommentTok{\#\textgreater{} }
\CommentTok{\#\textgreater{} Loadings:}
\CommentTok{\#\textgreater{}             Factor1 Factor2 Factor3 Factor4 Factor5}
\CommentTok{\#\textgreater{} alkohol      0.157   0.182           0.827   0.146 }
\CommentTok{\#\textgreater{} kabitoszer   0.275   0.271           0.761         }
\CommentTok{\#\textgreater{} hianyzik     0.174   0.552           0.383   0.104 }
\CommentTok{\#\textgreater{} dohanyzas            0.108           0.178         }
\CommentTok{\#\textgreater{} udvariatlan  0.333   0.569   0.263   0.244   0.184 }
\CommentTok{\#\textgreater{} rendetlen    0.244   0.833   0.275   0.139   0.190 }
\CommentTok{\#\textgreater{} pontatlan    0.233   0.733   0.228   0.171   0.178 }
\CommentTok{\#\textgreater{} pletykas     0.114   0.272   0.506                 }
\CommentTok{\#\textgreater{} harsany                      0.945           0.180 }
\CommentTok{\#\textgreater{} tudalekos    0.150   0.200   0.473           0.346 }
\CommentTok{\#\textgreater{} csamcsog     0.220   0.209   0.266   0.127   0.903 }
\CommentTok{\#\textgreater{} lusta        0.441   0.304   0.432   0.105   0.283 }
\CommentTok{\#\textgreater{} szemtelen    0.566   0.433   0.255   0.156   0.273 }
\CommentTok{\#\textgreater{} bufog        0.293   0.174   0.187   0.158   0.557 }
\CommentTok{\#\textgreater{} felelotlen   0.564   0.412   0.198   0.250   0.129 }
\CommentTok{\#\textgreater{} bosszuallo   0.576   0.211   0.133   0.402   0.240 }
\CommentTok{\#\textgreater{} durva        0.914   0.203           0.135   0.183 }
\CommentTok{\#\textgreater{} agressziv    0.769   0.224   0.117   0.209   0.202 }
\CommentTok{\#\textgreater{}             Factor6}
\CommentTok{\#\textgreater{} alkohol      0.275 }
\CommentTok{\#\textgreater{} kabitoszer         }
\CommentTok{\#\textgreater{} hianyzik     0.146 }
\CommentTok{\#\textgreater{} dohanyzas    0.377 }
\CommentTok{\#\textgreater{} udvariatlan        }
\CommentTok{\#\textgreater{} rendetlen    0.121 }
\CommentTok{\#\textgreater{} pontatlan          }
\CommentTok{\#\textgreater{} pletykas           }
\CommentTok{\#\textgreater{} harsany      0.241 }
\CommentTok{\#\textgreater{} tudalekos          }
\CommentTok{\#\textgreater{} csamcsog           }
\CommentTok{\#\textgreater{} lusta              }
\CommentTok{\#\textgreater{} szemtelen          }
\CommentTok{\#\textgreater{} bufog              }
\CommentTok{\#\textgreater{} felelotlen  {-}0.346 }
\CommentTok{\#\textgreater{} bosszuallo  {-}0.443 }
\CommentTok{\#\textgreater{} durva        0.256 }
\CommentTok{\#\textgreater{} agressziv   {-}0.118 }
\CommentTok{\#\textgreater{} }
\CommentTok{\#\textgreater{}                Factor1 Factor2 Factor3 Factor4 Factor5}
\CommentTok{\#\textgreater{} SS loadings      3.116   2.756   2.009   1.927   1.731}
\CommentTok{\#\textgreater{} Proportion Var   0.173   0.153   0.112   0.107   0.096}
\CommentTok{\#\textgreater{} Cumulative Var   0.173   0.326   0.438   0.545   0.641}
\CommentTok{\#\textgreater{}                Factor6}
\CommentTok{\#\textgreater{} SS loadings      0.731}
\CommentTok{\#\textgreater{} Proportion Var   0.041}
\CommentTok{\#\textgreater{} Cumulative Var   0.682}
\CommentTok{\#\textgreater{} }
\CommentTok{\#\textgreater{} Test of the hypothesis that 6 factors are sufficient.}
\CommentTok{\#\textgreater{} The chi square statistic is 112.68 on 60 degrees of...}
\CommentTok{\#\textgreater{} The p{-}value is 4.55e{-}05}
\end{Highlighting}
\end{Shaded}

A fenti hatfaktoros megoldás eredményen látható, hogy a khi-négyzet
statisztika szignifikancia-szintje szerint nem jól illeszkedik az
adatokra. A ``Cumulative Var'' sorban azt is láthatjuk, hogy a hat
faktor összesen az összvariancia 68\%-át magyarázza. A faktorsúlyok
alapján (3.17. R-eredmény) az egyes faktorok a következőképpen
alakulnak. Az első faktorban olyan változók szerepelnek, mint a
„lusta'', „szemtelen'', „felelőtlen'', „bosszúálló'', „durva'',
„agresszív''. A második faktorban szerepel a „hiányzik'',
„udvariatlan'', „rendetlen'' és „pontatlan''. A harmadikban szerepel a
„pletykás'', „harsány'' és „tudálékos''. A negyedik faktorban következik
az „alkohol'' és a „kábítószer'', ötödikben a „csámcsog'' és a „büfög'',
míg az utolsóban a „dohányzás''.

\begin{figure}

{\centering \includegraphics{./images/faktor_munkahelyi_tolarencia_kep_01.jpg}

}

\caption{Toleranciavizsgálat egy másik aspektusból: Feltáró
faktorelemzés}

\end{figure}

\hypertarget{puxe9lda-a-big-five-szemuxe9lyisuxe9gvizsguxe1luxf3-eljuxe1ruxe1s-faktoranaluxedzise}{%
\section{Példa: A Big Five személyiségvizsgáló eljárás
faktoranalízise}\label{puxe9lda-a-big-five-szemuxe9lyisuxe9gvizsguxe1luxf3-eljuxe1ruxe1s-faktoranaluxedzise}}

\begin{itemize}
\tightlist
\item
  A példa forrása: Münnich és mtsai. (2006)
  \href{https://psycho.unideb.hu/statisztika/pages/p_3_13.html}{3.7.3
  Probléma}
\item
  Kapcsolódó jamovi állomány: \texttt{faktor\_bigfive.omv}
\end{itemize}

Szinte minden pszichológus számára ismert a Big Five személyiségvizsgáló
eljárás. A Big Five - ahogyan a neve is mutatja - egy olyan
személyiségmodell, és arra épülő személyiségvizsgáló eljárás, amely azt
feltételezi, hogy a személyiséget öt, egymástól független dimenzió, öt
faktor alkotja. Az egyes dimenzióknak, faktoroknak több elnevezése is
ismert, ebben a vizsgálatban a következő elnevezéseket fogjuk használni:

\begin{itemize}
\tightlist
\item
  Extroverzió - introverzió
\item
  Együttműködés
\item
  Lelkiismeretesség
\item
  Stabilitás - neurocitás
\item
  Élményekre való nyitottság
\end{itemize}

Az adatok a \texttt{faktor\_bigfive.xlsx} állományban találhatók.

\begin{Shaded}
\begin{Highlighting}[]
\NormalTok{d }\OtherTok{\textless{}{-}}\NormalTok{ rio}\SpecialCharTok{::}\FunctionTok{import}\NormalTok{(}\AttributeTok{file =} \StringTok{"adat/faktor\_bigfive.xlsx"}\NormalTok{)}
\FunctionTok{str}\NormalTok{(d)}
\CommentTok{\#\textgreater{} \textquotesingle{}data.frame\textquotesingle{}:    20 obs. of  10 variables:}
\CommentTok{\#\textgreater{}  $ V1 : num  2 3 4 5 7 5 1 4 5 6 ...}
\CommentTok{\#\textgreater{}  $ V2 : num  7 5 4 3 1 3 7 4 3 1 ...}
\CommentTok{\#\textgreater{}  $ V3 : num  2 4 5 7 6 3 5 6 1 4 ...}
\CommentTok{\#\textgreater{}  $ V4 : num  6 5 3 1 2 5 3 2 7 4 ...}
\CommentTok{\#\textgreater{}  $ V5 : num  4 7 5 7 1 2 5 4 5 3 ...}
\CommentTok{\#\textgreater{}  $ V6 : num  4 2 3 1 7 6 3 4 3 5 ...}
\CommentTok{\#\textgreater{}  $ V7 : num  1 2 5 7 5 4 3 5 4 2 ...}
\CommentTok{\#\textgreater{}  $ V8 : num  6 6 3 1 3 4 5 3 5 6 ...}
\CommentTok{\#\textgreater{}  $ V9 : num  2 5 7 4 5 3 5 4 6 6 ...}
\CommentTok{\#\textgreater{}  $ V10: num  7 3 2 4 3 5 3 4 2 2 ...}
\NormalTok{psych}\SpecialCharTok{::}\FunctionTok{headTail}\NormalTok{(d)}
\CommentTok{\#\textgreater{}      V1  V2  V3  V4  V5  V6  V7  V8  V9 V10}
\CommentTok{\#\textgreater{} 1     2   7   2   6   4   4   1   6   2   7}
\CommentTok{\#\textgreater{} 2     3   5   4   5   7   2   2   6   5   3}
\CommentTok{\#\textgreater{} 3     4   4   5   3   5   3   5   3   7   2}
\CommentTok{\#\textgreater{} 4     5   3   7   1   7   1   7   1   4   4}
\CommentTok{\#\textgreater{} ... ... ... ... ... ... ... ... ... ... ...}
\CommentTok{\#\textgreater{} 17    5   3   2   6   1   7   5   3   5   3}
\CommentTok{\#\textgreater{} 18    7   1   5   3   2   6   5   3   6   2}
\CommentTok{\#\textgreater{} 19    4   3   2   5   7   2   2   6   3   5}
\CommentTok{\#\textgreater{} 20    7   1   4   4   5   3   7   1   5   3}
\end{Highlighting}
\end{Shaded}

A fenti adatok egy Big Five eljárásra épülő hipotetikus vizsgálat
adatait tartalmazza. Az egyes változókhoz tartozó itemeket egy 1-7
skálán jelölték meg a vizsgálati személyek attól függően, hogy mennyire
illik vagy nem illik rájuk az adott állítás. A 7 jelenteti azt, hogy
teljes mértékben illik, és az 1, hogy egyáltalán nem. Az egyes
változókhoz tartozó itemek a következők:

\begin{itemize}
\tightlist
\item
  v1 (extroverzió): Általában beszédes, aktív és társaságkedvelő vagyok.
\item
  v2 (introverzió): Jobban szeretek csendesen visszahúzódni egy sarokba,
  semmint a középpontban lenni.
\item
  v3 (együttműködés): Szívesen segítek másoknak, vagy dolgozok másokkal
  együtt valamilyen közös feladaton.
\item
  v4 (együttműködés): Gyakran viselkedem ellenségesen és kötözködően
  másokkal.
\item
  v5 (lelkiismeretesség): Általában tudom, hogy mit akarok, és
  céltudatosan igyekszem elérni azt.
\item
  v6 (lelkiismeretesség):Sokak szerint megbízhatatlan vagyok.
\item
  v7 (stabilitás): Érzelmileg kiegyensúlyozottnak, higgadtnak tartom
  magam.
\item
  v8 (neurocitás): Gyakran vagyok érzelmileg csapongó.
\item
  v9 (élményekre való nyitottság): Kíváncsi vagyok.
\item
  v10 (élményekre való nyitottság): Ragaszkodom a szokásaimhoz.
\end{itemize}

\begin{Shaded}
\begin{Highlighting}[]
\FunctionTok{print}\NormalTok{(}\FunctionTok{cor}\NormalTok{(d), }\AttributeTok{digits =} \DecValTok{3}\NormalTok{)}
\CommentTok{\#\textgreater{}           V1       V2       V3      V4      V5      V6}
\CommentTok{\#\textgreater{} V1   1.00000 {-}0.98085  0.00408 {-}0.0645 {-}0.2771  0.2393}
\CommentTok{\#\textgreater{} V2  {-}0.98085  1.00000  0.00000  0.0726  0.2307 {-}0.2069}
\CommentTok{\#\textgreater{} V3   0.00408  0.00000  1.00000 {-}0.9751  0.0661 {-}0.0680}
\CommentTok{\#\textgreater{} V4  {-}0.06446  0.07256 {-}0.97510  1.0000 {-}0.0895  0.0905}
\CommentTok{\#\textgreater{} V5  {-}0.27711  0.23072  0.06615 {-}0.0895  1.0000 {-}0.9814}
\CommentTok{\#\textgreater{} V6   0.23926 {-}0.20689 {-}0.06797  0.0905 {-}0.9814  1.0000}
\CommentTok{\#\textgreater{} V7   0.37135 {-}0.32942  0.26315 {-}0.2257 {-}0.1954  0.0881}
\CommentTok{\#\textgreater{} V8  {-}0.33508  0.28028 {-}0.28799  0.2500  0.2179 {-}0.1087}
\CommentTok{\#\textgreater{} V9  {-}0.11472  0.06565  0.08152 {-}0.0377 {-}0.1790  0.1864}
\CommentTok{\#\textgreater{} V10  0.05897  0.00323 {-}0.10682  0.0630  0.1865 {-}0.2005}
\CommentTok{\#\textgreater{}          V7      V8      V9      V10}
\CommentTok{\#\textgreater{} V1   0.3714 {-}0.3351 {-}0.1147  0.05897}
\CommentTok{\#\textgreater{} V2  {-}0.3294  0.2803  0.0656  0.00323}
\CommentTok{\#\textgreater{} V3   0.2631 {-}0.2880  0.0815 {-}0.10682}
\CommentTok{\#\textgreater{} V4  {-}0.2257  0.2500 {-}0.0377  0.06298}
\CommentTok{\#\textgreater{} V5  {-}0.1954  0.2179 {-}0.1790  0.18647}
\CommentTok{\#\textgreater{} V6   0.0881 {-}0.1087  0.1864 {-}0.20051}
\CommentTok{\#\textgreater{} V7   1.0000 {-}0.9848  0.1554 {-}0.19002}
\CommentTok{\#\textgreater{} V8  {-}0.9848  1.0000 {-}0.0891  0.10854}
\CommentTok{\#\textgreater{} V9   0.1554 {-}0.0891  1.0000 {-}0.98281}
\CommentTok{\#\textgreater{} V10 {-}0.1900  0.1085 {-}0.9828  1.00000}
\end{Highlighting}
\end{Shaded}

Láthatjuk, hogy a Big Five modellje szerint összekapcsolódó itemek
nagyon szoros, ám negatív korrelációban vannak egymással (tehát a v1 a
v2-vel, v3 a v4-gyel stb.) a korreláció értékek -0,98 körül mozognak. A
negatív előjelű kapcsolat utal arra, hogy az összekapcsolódó itemek egy
dimenzió két végpontját ragadják meg. Hogy mennyire helytálló a
korrelációs mátrix által felállított elképzelésünk, arra a
faktoranalízis adhat választ.

\begin{Shaded}
\begin{Highlighting}[]
\NormalTok{fa\_1 }\OtherTok{\textless{}{-}} \FunctionTok{factanal}\NormalTok{(d, }\AttributeTok{factors =} \DecValTok{5}\NormalTok{, }\AttributeTok{rotation =} \StringTok{"varimax"}\NormalTok{, }\AttributeTok{scores =} \StringTok{"Bartlett"}\NormalTok{)}
\NormalTok{fa\_1}
\CommentTok{\#\textgreater{} }
\CommentTok{\#\textgreater{} Call:}
\CommentTok{\#\textgreater{} factanal(x = d, factors = 5, scores = "Bartlett", r...}
\CommentTok{\#\textgreater{} }
\CommentTok{\#\textgreater{} Uniquenesses:}
\CommentTok{\#\textgreater{}    V1    V2    V3    V4    V5    V6    V7    V8    V9 }
\CommentTok{\#\textgreater{} 0.022 0.005 0.005 0.034 0.005 0.019 0.012 0.005 0.025 }
\CommentTok{\#\textgreater{}   V10 }
\CommentTok{\#\textgreater{} 0.005 }
\CommentTok{\#\textgreater{} }
\CommentTok{\#\textgreater{} Loadings:}
\CommentTok{\#\textgreater{}     Factor1 Factor2 Factor3 Factor4 Factor5}
\CommentTok{\#\textgreater{} V1          {-}0.194  {-}0.955   0.146         }
\CommentTok{\#\textgreater{} V2           0.136   0.983                 }
\CommentTok{\#\textgreater{} V3          {-}0.151                  {-}0.984 }
\CommentTok{\#\textgreater{} V4           0.104                   0.974 }
\CommentTok{\#\textgreater{} V5           0.124   0.123  {-}0.977         }
\CommentTok{\#\textgreater{} V6   0.109          {-}0.113   0.977         }
\CommentTok{\#\textgreater{} V7   0.114  {-}0.959  {-}0.197          {-}0.122 }
\CommentTok{\#\textgreater{} V8           0.972   0.140           0.149 }
\CommentTok{\#\textgreater{} V9   0.978                                 }
\CommentTok{\#\textgreater{} V10 {-}0.989                                 }
\CommentTok{\#\textgreater{} }
\CommentTok{\#\textgreater{}                Factor1 Factor2 Factor3 Factor4 Factor5}
\CommentTok{\#\textgreater{} SS loadings      1.978   1.977   1.975   1.972   1.961}
\CommentTok{\#\textgreater{} Proportion Var   0.198   0.198   0.197   0.197   0.196}
\CommentTok{\#\textgreater{} Cumulative Var   0.198   0.395   0.593   0.790   0.986}
\CommentTok{\#\textgreater{} }
\CommentTok{\#\textgreater{} Test of the hypothesis that 5 factors are sufficient.}
\CommentTok{\#\textgreater{} The chi square statistic is 9.02 on 5 degrees of fr...}
\CommentTok{\#\textgreater{} The p{-}value is 0.108}
\end{Highlighting}
\end{Shaded}

A Big Five jellegéből adódik, hogy egy ötfaktoros modellt teszteltünk,
amely a khi-négyzet statisztika szerint illeszkedik is az adatokra. A
``Cumulative Var'' sorban azt is láthatjuk, hogy a modell
magyarázóértéke igen jó, hiszen az öt faktor az összvarianciának majdnem
a 99\%-át magyarázza. A ``Loadings''-szal jelölt faktorsúlyoknál
megnézhetjük, hogyan alakulnak az egyes faktorok. A faktorok szerkezete
teljes mértékben összecseng előzetes várakozásunkkal: minden egyes
faktorba két változó tartozik, az összetartozó változók pedig úgy
kapcsolódnak össze, ahogyan azt az elmélet alapján is vártuk (vagyis a
v1 a v2-vel, v3 a v4-gyel stb.). A faktorsúlyok alapján az öt faktor a
következőképpen alakul:

\begin{enumerate}
\def\labelenumi{\arabic{enumi}.}
\tightlist
\item
  faktor: élményekre való nyitottság (v9, v10)
\item
  faktor: stabilitás-neurocitás (v7, v8)
\item
  faktor: extroverzió-introverzió (v1, v2)
\item
  faktor: lelkiismeretesség (v5, v6)
\item
  faktor: együttműködés (v3, v4)
\end{enumerate}

\begin{Shaded}
\begin{Highlighting}[]
\FunctionTok{print}\NormalTok{(fa\_1}\SpecialCharTok{$}\NormalTok{scores, }\AttributeTok{digits =} \DecValTok{3}\NormalTok{)}
\CommentTok{\#\textgreater{}    Factor1 Factor2 Factor3 Factor4 Factor5}
\CommentTok{\#\textgreater{} 1   {-}1.971  0.8369  1.7713  0.5375  1.0879}
\CommentTok{\#\textgreater{} 2    0.559  1.0027  0.5314 {-}1.4064  0.1571}
\CommentTok{\#\textgreater{} 3    1.066 {-}0.5120  0.3125 {-}0.5840 {-}0.2786}
\CommentTok{\#\textgreater{} 4   {-}0.250 {-}1.5838 {-}0.1453 {-}1.7374 {-}1.3075}
\CommentTok{\#\textgreater{} 5    0.155 {-}0.0311 {-}1.0713  1.7501 {-}1.0874}
\CommentTok{\#\textgreater{} 6   {-}0.956  0.0323 {-}0.0697  1.3117  0.6154}
\CommentTok{\#\textgreater{} 7    0.355  0.3940  1.8085 {-}0.2350 {-}0.4319}
\CommentTok{\#\textgreater{} 8   {-}0.350 {-}0.4407  0.4149  0.1847 {-}0.9641}
\CommentTok{\#\textgreater{} 9    1.148  0.2481 {-}0.3858 {-}0.7208  1.9149}
\CommentTok{\#\textgreater{} 10   1.035  1.5221 {-}1.3600  0.5723 {-}0.0805}
\CommentTok{\#\textgreater{} 11   0.403  1.9639  0.5569 {-}0.0237 {-}1.8494}
\CommentTok{\#\textgreater{} 12  {-}0.236 {-}0.7084 {-}0.1276 {-}0.0423  1.3856}
\CommentTok{\#\textgreater{} 13  {-}1.397  0.7453 {-}0.8566 {-}0.3500 {-}0.6176}
\CommentTok{\#\textgreater{} 14  {-}2.060 {-}0.3921 {-}1.2143 {-}0.4555 {-}0.4865}
\CommentTok{\#\textgreater{} 15   0.109 {-}1.7587  1.7177  0.7326 {-}0.6666}
\CommentTok{\#\textgreater{} 16   1.560 {-}0.0238  0.8113  0.0349  0.2353}
\CommentTok{\#\textgreater{} 17   0.210 {-}0.5591  0.0271  1.6983  1.3235}
\CommentTok{\#\textgreater{} 18   0.852 {-}0.1328 {-}1.1529  1.0491 {-}0.4370}
\CommentTok{\#\textgreater{} 19  {-}0.593  0.9647 {-}0.4275 {-}1.4764  1.0903}
\CommentTok{\#\textgreater{} 20   0.362 {-}1.5674 {-}1.1406 {-}0.8398  0.3971}
\end{Highlighting}
\end{Shaded}

Előhívhatjuk a személyek egyes faktorbeli értékeit is.

\begin{figure}

{\centering \includegraphics{./images/faktor_bigfive_kep_01.jpg}

}

\caption{A Big Five személyiségvizsgáló eljárás faktoranalízise: Feltáró
faktorelemzés}

\end{figure}

\hypertarget{puxe9lda-milyen-dimenziuxf3i-vannak-a-kockuxe1zatvuxe1llaluxe1snak-uxe9s-vuxe1ltozik-e-a-korral-a-kockuxe1zatvuxe1llaluxe1s}{%
\section{Példa: Milyen dimenziói vannak a kockázatvállalásnak és
változik-e a korral a
kockázatvállalás?}\label{puxe9lda-milyen-dimenziuxf3i-vannak-a-kockuxe1zatvuxe1llaluxe1snak-uxe9s-vuxe1ltozik-e-a-korral-a-kockuxe1zatvuxe1llaluxe1s}}

\begin{itemize}
\tightlist
\item
  A példa forrása: Münnich és mtsai. (2006)
  \href{https://psycho.unideb.hu/statisztika/pages/p_3_14.html}{3.7.4
  Probléma}
\item
  Kapcsolódó jamovi állomány: \texttt{faktor\_kockazat.omv}
\end{itemize}

A példák is mutatják, hogy a kockázatvállalásokat csoportosíthatjuk a
kockázatot jelentő tényezők alapján, ahol az egyik csoportban az emberek
saját testi épségüket teszik kockára (mint az autóversenyzés esetében),
de kockáztathatnak pénzt vagy valamilyen becsületbeli dolgot is (mint a
kártyázás és a blöffölés esetében). Példánkban megnézzük, hogy a
faktoranalízis alátámasztja-e feltevésünket, majd a faktoranalízis
eredményeit felhasználva megnézzük, hogy a kockázatvállaló viselkedésre
hatással van-e a kor.

Az adatbázisban szereplő adatokat úgy kaptuk, hogy a vizsgálati
személyeknek különböző foglalkozású, illetve különböző tevékenységet
végző embereket kellett megítélniük, hogy mennyire tartják őket
szimpatikusnak egy 1-7 skálán, ahol a 7 jelentette azt, hogy nagyon
szimpatikus. Ily módon megkaptuk a személyek kockázat iránti attitűdjét.
A változók között olyan személyek szerepelnek, mint kártyajátékosok,
autóversenyzők, üzletemberek (akik sok pénzt kockáztatnak), veszélyes
sportot űző emberek, nagy pénzekben fogadó emberek, blöffölők és
hivatásos katonák.

Ezen kívül két adat szerepel az adatbázisban: a nem és a kor.

Az adatok a \texttt{faktor\_kockazat.xlsx} állományban találhatók.

\begin{Shaded}
\begin{Highlighting}[]
\NormalTok{d }\OtherTok{\textless{}{-}}\NormalTok{ rio}\SpecialCharTok{::}\FunctionTok{import}\NormalTok{(}\AttributeTok{file =} \StringTok{"adat/faktor\_kockazat.xlsx"}\NormalTok{)}
\FunctionTok{str}\NormalTok{(d)}
\CommentTok{\#\textgreater{} \textquotesingle{}data.frame\textquotesingle{}:    156 obs. of  9 variables:}
\CommentTok{\#\textgreater{}  $ kartya        : num  5 4 2 1 4 5 4 4 3 5 ...}
\CommentTok{\#\textgreater{}  $ autoversenyzo : num  3 3 5 1 3 4 2 2 3 5 ...}
\CommentTok{\#\textgreater{}  $ uzletember    : num  3 3 2 1 4 3 2 3 2 3 ...}
\CommentTok{\#\textgreater{}  $ veszelyessport: num  3 2 4 1 3 1 2 1 2 3 ...}
\CommentTok{\#\textgreater{}  $ fogadas       : num  3 3 4 2 3 4 4 4 3 3 ...}
\CommentTok{\#\textgreater{}  $ bloff         : num  3 1 1 1 3 3 3 5 4 2 ...}
\CommentTok{\#\textgreater{}  $ katona        : num  1 1 2 1 2 2 2 1 2 2 ...}
\CommentTok{\#\textgreater{}  $ kor           : num  25 19 18 18 24 28 25 39 19 ...}
\CommentTok{\#\textgreater{}  $ nem           : num  1 0 1 0 1 1 0 1 0 1 ...}
\NormalTok{psych}\SpecialCharTok{::}\FunctionTok{headTail}\NormalTok{(d)}
\CommentTok{\#\textgreater{}     kartya autoversenyzo uzletember veszelyessport}
\CommentTok{\#\textgreater{} 1        5             3          3              3}
\CommentTok{\#\textgreater{} 2        4             3          3              2}
\CommentTok{\#\textgreater{} 3        2             5          2              4}
\CommentTok{\#\textgreater{} 4        1             1          1              1}
\CommentTok{\#\textgreater{} ...    ...           ...        ...            ...}
\CommentTok{\#\textgreater{} 153      4             2          3              4}
\CommentTok{\#\textgreater{} 154      5             3          4              5}
\CommentTok{\#\textgreater{} 155      4             1          1              3}
\CommentTok{\#\textgreater{} 156      4             3          3              2}
\CommentTok{\#\textgreater{}     fogadas bloff katona kor nem}
\CommentTok{\#\textgreater{} 1         3     3      1  25   1}
\CommentTok{\#\textgreater{} 2         3     1      1  19   0}
\CommentTok{\#\textgreater{} 3         4     1      2  18   1}
\CommentTok{\#\textgreater{} 4         2     1      1  18   0}
\CommentTok{\#\textgreater{} ...     ...   ...    ... ... ...}
\CommentTok{\#\textgreater{} 153       4     4      2  26   1}
\CommentTok{\#\textgreater{} 154       5     3      3  24   1}
\CommentTok{\#\textgreater{} 155       5     3      2  25   0}
\CommentTok{\#\textgreater{} 156       4     3      4  21   1}
\end{Highlighting}
\end{Shaded}

\begin{Shaded}
\begin{Highlighting}[]
\FunctionTok{print}\NormalTok{(}\FunctionTok{cor}\NormalTok{(d[}\DecValTok{1}\SpecialCharTok{:}\DecValTok{7}\NormalTok{]), }\AttributeTok{digits =} \DecValTok{2}\NormalTok{)}
\CommentTok{\#\textgreater{}                kartya autoversenyzo uzletember}
\CommentTok{\#\textgreater{} kartya           1.00        0.1702      0.316}
\CommentTok{\#\textgreater{} autoversenyzo    0.17        1.0000      0.113}
\CommentTok{\#\textgreater{} uzletember       0.32        0.1132      1.000}
\CommentTok{\#\textgreater{} veszelyessport   0.08        0.2224      0.177}
\CommentTok{\#\textgreater{} fogadas          0.56       {-}0.0062      0.148}
\CommentTok{\#\textgreater{} bloff            0.27        0.0023      0.253}
\CommentTok{\#\textgreater{} katona          {-}0.23        0.1833     {-}0.092}
\CommentTok{\#\textgreater{}                veszelyessport fogadas   bloff katona}
\CommentTok{\#\textgreater{} kartya                  0.080  0.5579  0.2671 {-}0.233}
\CommentTok{\#\textgreater{} autoversenyzo           0.222 {-}0.0062  0.0023  0.183}
\CommentTok{\#\textgreater{} uzletember              0.177  0.1482  0.2535 {-}0.092}
\CommentTok{\#\textgreater{} veszelyessport          1.000  0.0553  0.1424  0.160}
\CommentTok{\#\textgreater{} fogadas                 0.055  1.0000  0.1986 {-}0.096}
\CommentTok{\#\textgreater{} bloff                   0.142  0.1986  1.0000 {-}0.120}
\CommentTok{\#\textgreater{} katona                  0.160 {-}0.0961 {-}0.1197  1.000}
\end{Highlighting}
\end{Shaded}

A korrelációs mátrixon nem látunk kiemelkedően magas értékeket, kissé
nehéz egyértelmű következtetéseket levonni a faktorokra vonatkozóan,
ezért teszteljünk egy kétfaktoros faktoranalízist az adatokra.

\begin{Shaded}
\begin{Highlighting}[]
\NormalTok{fa\_1 }\OtherTok{\textless{}{-}} \FunctionTok{factanal}\NormalTok{(d[}\DecValTok{1}\SpecialCharTok{:}\DecValTok{7}\NormalTok{], }\AttributeTok{factors =} \DecValTok{2}\NormalTok{, }\AttributeTok{rotation =} \StringTok{"varimax"}\NormalTok{, }\AttributeTok{scores =} \StringTok{"Bartlett"}\NormalTok{)}
\NormalTok{fa\_1}
\CommentTok{\#\textgreater{} }
\CommentTok{\#\textgreater{} Call:}
\CommentTok{\#\textgreater{} factanal(x = d[1:7], factors = 2, scores = "Bartlet...}
\CommentTok{\#\textgreater{} }
\CommentTok{\#\textgreater{} Uniquenesses:}
\CommentTok{\#\textgreater{}         kartya  autoversenyzo     uzletember }
\CommentTok{\#\textgreater{}          0.073          0.724          0.863 }
\CommentTok{\#\textgreater{} veszelyessport        fogadas          bloff }
\CommentTok{\#\textgreater{}          0.775          0.663          0.916 }
\CommentTok{\#\textgreater{}         katona }
\CommentTok{\#\textgreater{}          0.796 }
\CommentTok{\#\textgreater{} }
\CommentTok{\#\textgreater{} Loadings:}
\CommentTok{\#\textgreater{}                Factor1 Factor2}
\CommentTok{\#\textgreater{} kartya          0.963         }
\CommentTok{\#\textgreater{} autoversenyzo   0.164   0.499 }
\CommentTok{\#\textgreater{} uzletember      0.330   0.169 }
\CommentTok{\#\textgreater{} veszelyessport          0.467 }
\CommentTok{\#\textgreater{} fogadas         0.577         }
\CommentTok{\#\textgreater{} bloff           0.285         }
\CommentTok{\#\textgreater{} katona         {-}0.244   0.380 }
\CommentTok{\#\textgreater{} }
\CommentTok{\#\textgreater{}                Factor1 Factor2}
\CommentTok{\#\textgreater{} SS loadings      1.544   0.646}
\CommentTok{\#\textgreater{} Proportion Var   0.221   0.092}
\CommentTok{\#\textgreater{} Cumulative Var   0.221   0.313}
\CommentTok{\#\textgreater{} }
\CommentTok{\#\textgreater{} Test of the hypothesis that 2 factors are sufficient.}
\CommentTok{\#\textgreater{} The chi square statistic is 14.81 on 8 degrees of f...}
\CommentTok{\#\textgreater{} The p{-}value is 0.063}
\end{Highlighting}
\end{Shaded}

Láthatjuk, hogy a khi-négyzet statisztika szerint a modell illeszkedik
az adatokra, ellenben a modell magyarázóértéke egy picit csekély: a két
faktor az összvariancia 31\%-át magyarázza (``Cumulative Var'').

A faktorsúlyok alapján (``Loadings'') pedig az egyes faktorok a
következőképpen alakulnak: első faktorba tartoznak a kártyajátékosok,
üzletemberek, akik fogadnak, illetve a blöffölők, míg a második faktorba
az autóversenyzők, veszélyes sportot űzők és a hivatásos katonák
tartoznak. Az egyes faktorok szerkezete teljes mértékben összhangban van
az előzetes elvárásunkkal.

\begin{Shaded}
\begin{Highlighting}[]
\FunctionTok{print}\NormalTok{(fa\_1}\SpecialCharTok{$}\NormalTok{scores, }\AttributeTok{digits =} \DecValTok{3}\NormalTok{)}
\CommentTok{\#\textgreater{}     Factor1 Factor2}
\CommentTok{\#\textgreater{} 1     1.338  0.2060}
\CommentTok{\#\textgreater{} 2     0.191 {-}0.4557}
\CommentTok{\#\textgreater{} 3    {-}2.007  2.7851}
\CommentTok{\#\textgreater{} 4    {-}3.271 {-}2.7221}
\CommentTok{\#\textgreater{} 5     0.221  1.0535}
\CommentTok{\#\textgreater{} 6     1.390  0.1120}
\CommentTok{\#\textgreater{} 7     0.244 {-}0.9563}
\CommentTok{\#\textgreater{} 8     0.337 {-}1.8187}
\CommentTok{\#\textgreater{} 9    {-}0.918  0.1116}
\CommentTok{\#\textgreater{} 10    1.298  2.2779}
\CommentTok{\#\textgreater{} 11    0.127 {-}0.3024}
\CommentTok{\#\textgreater{} 12   {-}1.103 {-}1.0278}
\CommentTok{\#\textgreater{} 13    0.116  1.4488}
\CommentTok{\#\textgreater{} 14   {-}0.942  0.8179}
\CommentTok{\#\textgreater{} 15    0.153  0.0308}
\CommentTok{\#\textgreater{} 16    0.260 {-}0.2581}
\CommentTok{\#\textgreater{} 17    0.317 {-}1.0907}
\CommentTok{\#\textgreater{} 18   {-}0.952 {-}0.2752}
\CommentTok{\#\textgreater{} 19   {-}1.006  0.0461}
\CommentTok{\#\textgreater{} 20    0.248 {-}0.1641}
\CommentTok{\#\textgreater{} 21    1.263  0.9515}
\CommentTok{\#\textgreater{} 22   {-}0.872 {-}0.8891}
\CommentTok{\#\textgreater{} 23    0.252 {-}2.9866}
\CommentTok{\#\textgreater{} 24    1.410 {-}0.1078}
\CommentTok{\#\textgreater{} 25   {-}0.940  0.7251}
\CommentTok{\#\textgreater{} 26   {-}0.956 {-}0.0078}
\CommentTok{\#\textgreater{} 27   {-}0.980  1.3390}
\CommentTok{\#\textgreater{} 28    0.186  1.4272}
\CommentTok{\#\textgreater{} 29    1.302  0.3356}
\CommentTok{\#\textgreater{} 30    0.279  0.2009}
\CommentTok{\#\textgreater{} 31    1.291  0.0477}
\CommentTok{\#\textgreater{} 32   {-}0.958 {-}0.8053}
\CommentTok{\#\textgreater{} 33   {-}2.177  0.8512}
\CommentTok{\#\textgreater{} 34    1.502 {-}1.4686}
\CommentTok{\#\textgreater{} 35   {-}1.060 {-}0.7568}
\CommentTok{\#\textgreater{} 36    0.309 {-}1.9046}
\CommentTok{\#\textgreater{} 37   {-}0.989 {-}0.2453}
\CommentTok{\#\textgreater{} 38    0.268 {-}0.7376}
\CommentTok{\#\textgreater{} 39   {-}2.066  0.6141}
\CommentTok{\#\textgreater{} 40    0.187  1.5617}
\CommentTok{\#\textgreater{} 41    1.442 {-}0.3686}
\CommentTok{\#\textgreater{} 42    0.136 {-}3.1233}
\CommentTok{\#\textgreater{} 43    0.167 {-}0.7090}
\CommentTok{\#\textgreater{} 44    0.284 {-}0.0394}
\CommentTok{\#\textgreater{} 45   {-}0.938 {-}2.9296}
\CommentTok{\#\textgreater{} 46    0.239  1.8862}
\CommentTok{\#\textgreater{} 47   {-}0.918  0.1116}
\CommentTok{\#\textgreater{} 48    0.148 {-}0.8959}
\CommentTok{\#\textgreater{} 49   {-}0.917  1.5696}
\CommentTok{\#\textgreater{} 50    0.287 {-}2.2361}
\CommentTok{\#\textgreater{} 51    0.362  1.5434}
\CommentTok{\#\textgreater{} 52   {-}0.902  0.8298}
\CommentTok{\#\textgreater{} 53    0.177 {-}0.5210}
\CommentTok{\#\textgreater{} 54    0.233 {-}3.0390}
\CommentTok{\#\textgreater{} 55    0.215 {-}1.0421}
\CommentTok{\#\textgreater{} 56    1.338  0.2060}
\CommentTok{\#\textgreater{} 57   {-}0.917 {-}0.3597}
\CommentTok{\#\textgreater{} 58    0.290  2.2836}
\CommentTok{\#\textgreater{} 59    0.380  2.2216}
\CommentTok{\#\textgreater{} 60   {-}1.009  1.2532}
\CommentTok{\#\textgreater{} 61    0.167 {-}0.7090}
\CommentTok{\#\textgreater{} 62    0.237  0.2937}
\CommentTok{\#\textgreater{} 63    0.321 {-}0.4330}
\CommentTok{\#\textgreater{} 64   {-}0.865 {-}0.5882}
\CommentTok{\#\textgreater{} 65    1.472  1.5473}
\CommentTok{\#\textgreater{} 66    1.382  1.3820}
\CommentTok{\#\textgreater{} 67    0.216  2.3004}
\CommentTok{\#\textgreater{} 68   {-}0.901 {-}0.1799}
\CommentTok{\#\textgreater{} 69    0.136  0.4914}
\CommentTok{\#\textgreater{} 70    1.352 {-}0.1272}
\CommentTok{\#\textgreater{} 71    0.213 {-}0.3963}
\CommentTok{\#\textgreater{} 72    0.316 {-}1.2252}
\CommentTok{\#\textgreater{} 73    0.253 {-}0.5389}
\CommentTok{\#\textgreater{} 74    1.498 {-}1.3357}
\CommentTok{\#\textgreater{} 75    0.313 {-}1.1124}
\CommentTok{\#\textgreater{} 76    0.236 {-}2.5607}
\CommentTok{\#\textgreater{} 77    1.412 {-}1.5140}
\CommentTok{\#\textgreater{} 78   {-}0.806 {-}0.5440}
\CommentTok{\#\textgreater{} 79    0.357  1.3770}
\CommentTok{\#\textgreater{} 80    1.526 {-}1.8557}
\CommentTok{\#\textgreater{} 81   {-}0.956 {-}0.9328}
\CommentTok{\#\textgreater{} 82    0.213 {-}0.5508}
\CommentTok{\#\textgreater{} 83   {-}2.165 {-}2.0523}
\CommentTok{\#\textgreater{} 84    0.361 {-}2.3705}
\CommentTok{\#\textgreater{} 85    0.313 {-}1.1124}
\CommentTok{\#\textgreater{} 86   {-}0.953  0.2860}
\CommentTok{\#\textgreater{} 87    1.506 {-}1.2822}
\CommentTok{\#\textgreater{} 88    0.221  1.0535}
\CommentTok{\#\textgreater{} 89    0.287 {-}1.4456}
\CommentTok{\#\textgreater{} 90    0.163  0.5826}
\CommentTok{\#\textgreater{} 91   {-}0.919 {-}1.9523}
\CommentTok{\#\textgreater{} 92   {-}1.011 {-}1.8702}
\CommentTok{\#\textgreater{} 93    1.365 {-}0.0867}
\CommentTok{\#\textgreater{} 94    1.366  0.4117}
\CommentTok{\#\textgreater{} 95    0.133  1.7484}
\CommentTok{\#\textgreater{} 96    0.298  2.3371}
\CommentTok{\#\textgreater{} 97    1.484 {-}2.1467}
\CommentTok{\#\textgreater{} 98    0.226 {-}1.6345}
\CommentTok{\#\textgreater{} 99    1.443 {-}0.9944}
\CommentTok{\#\textgreater{} 100   1.447 {-}0.2023}
\CommentTok{\#\textgreater{} 101   0.124  0.5708}
\CommentTok{\#\textgreater{} 102   0.229  1.3344}
\CommentTok{\#\textgreater{} 103  {-}1.073  3.4903}
\CommentTok{\#\textgreater{} 104  {-}0.958  2.2036}
\CommentTok{\#\textgreater{} 105  {-}0.996  1.0046}
\CommentTok{\#\textgreater{} 106   0.219 {-}0.3845}
\CommentTok{\#\textgreater{} 107  {-}0.933 {-}1.2125}
\CommentTok{\#\textgreater{} 108  {-}0.944  1.4637}
\CommentTok{\#\textgreater{} 109  {-}0.822 {-}1.2422}
\CommentTok{\#\textgreater{} 110   0.215 {-}0.6436}
\CommentTok{\#\textgreater{} 111  {-}3.362  1.4583}
\CommentTok{\#\textgreater{} 112   0.216  0.8872}
\CommentTok{\#\textgreater{} 113   0.282 {-}1.4774}
\CommentTok{\#\textgreater{} 114   1.365  5.5972}
\CommentTok{\#\textgreater{} 115  {-}0.981  1.8921}
\CommentTok{\#\textgreater{} 116   0.276 {-}2.7679}
\CommentTok{\#\textgreater{} 117   0.256 {-}1.5686}
\CommentTok{\#\textgreater{} 118   0.167  1.3748}
\CommentTok{\#\textgreater{} 119   0.188 {-}1.2679}
\CommentTok{\#\textgreater{} 120  {-}0.878  0.6500}
\CommentTok{\#\textgreater{} 121   1.549 {-}1.0112}
\CommentTok{\#\textgreater{} 122  {-}2.162  1.4349}
\CommentTok{\#\textgreater{} 123   0.202 {-}0.3223}
\CommentTok{\#\textgreater{} 124   0.181 {-}0.5340}
\CommentTok{\#\textgreater{} 125   0.138  1.9148}
\CommentTok{\#\textgreater{} 126   1.421  2.1542}
\CommentTok{\#\textgreater{} 127   0.283 {-}1.4974}
\CommentTok{\#\textgreater{} 128   1.509  1.6139}
\CommentTok{\#\textgreater{} 129  {-}0.988  2.7435}
\CommentTok{\#\textgreater{} 130   0.432  0.4623}
\CommentTok{\#\textgreater{} 131   0.290  0.1269}
\CommentTok{\#\textgreater{} 132  {-}3.307  0.8877}
\CommentTok{\#\textgreater{} 133  {-}0.874 {-}0.2432}
\CommentTok{\#\textgreater{} 134  {-}2.065 {-}1.4897}
\CommentTok{\#\textgreater{} 135   1.471 {-}0.9086}
\CommentTok{\#\textgreater{} 136   1.485 {-}0.0976}
\CommentTok{\#\textgreater{} 137   0.277  0.8468}
\CommentTok{\#\textgreater{} 138   0.430 {-}0.4427}
\CommentTok{\#\textgreater{} 139   1.373  0.5580}
\CommentTok{\#\textgreater{} 140   0.252  1.0119}
\CommentTok{\#\textgreater{} 141   0.267  1.3663}
\CommentTok{\#\textgreater{} 142  {-}0.928 {-}0.6924}
\CommentTok{\#\textgreater{} 143   0.334  2.4618}
\CommentTok{\#\textgreater{} 144  {-}0.732 {-}1.4387}
\CommentTok{\#\textgreater{} 145  {-}0.811 {-}2.4603}
\CommentTok{\#\textgreater{} 146   0.178 {-}1.3014}
\CommentTok{\#\textgreater{} 147  {-}1.001  1.8978}
\CommentTok{\#\textgreater{} 148   1.424  0.8826}
\CommentTok{\#\textgreater{} 149  {-}0.895  2.4341}
\CommentTok{\#\textgreater{} 150   0.260  0.6669}
\CommentTok{\#\textgreater{} 151  {-}2.101 {-}0.2909}
\CommentTok{\#\textgreater{} 152   0.250 {-}1.5703}
\CommentTok{\#\textgreater{} 153   0.282  0.6064}
\CommentTok{\#\textgreater{} 154   1.445  2.5275}
\CommentTok{\#\textgreater{} 155   0.286 {-}1.4557}
\CommentTok{\#\textgreater{} 156   0.213  1.1345}
\end{Highlighting}
\end{Shaded}

Végül kérjük le az egyes személyek faktorértékeit. Tehát az első faktor
jelenti az anyagi/erkölcsi kockázatvállalás iránti attitűdöt, míg a
második faktor a testi épséget veszélyeztető kockázatvállalás iránti
attitűdöt.

Ezek után nézzük meg, hatással van-e az életkor a kockázatvállalás
iránti attitűdre. A további munkát megkönnyítendő, bővítsük ki az
adatbázisunkat a két faktor értékeivel.

\begin{Shaded}
\begin{Highlighting}[]
\NormalTok{d }\OtherTok{\textless{}{-}} \FunctionTok{cbind}\NormalTok{(d, fa\_1}\SpecialCharTok{$}\NormalTok{scores)}
\end{Highlighting}
\end{Shaded}

Első lépésben azt nézzük meg, hogy van-e kapcsolat az életkor és az
anyagi/erkölcsi téren vállalt kockázat iránti attitűd között. Eddigi
ismereteink alapján ez azt jelenti, hogy lineáris regresszió-analízissel
megnézzük, hogy van-e kapcsolat a kor és az első faktor faktorértékei
között.

\begin{Shaded}
\begin{Highlighting}[]
\FunctionTok{summary}\NormalTok{(}\FunctionTok{lm}\NormalTok{(Factor1 }\SpecialCharTok{\textasciitilde{}}\NormalTok{ kor, }\AttributeTok{data =}\NormalTok{ d))}
\CommentTok{\#\textgreater{} }
\CommentTok{\#\textgreater{} Call:}
\CommentTok{\#\textgreater{} lm(formula = Factor1 \textasciitilde{} kor, data = d)}
\CommentTok{\#\textgreater{} }
\CommentTok{\#\textgreater{} Residuals:}
\CommentTok{\#\textgreater{}      Min       1Q   Median       3Q      Max }
\CommentTok{\#\textgreater{} {-}2.76946 {-}0.43248  0.06417  0.61206  1.95774 }
\CommentTok{\#\textgreater{} }
\CommentTok{\#\textgreater{} Coefficients:}
\CommentTok{\#\textgreater{}              Estimate Std. Error t value Pr(\textgreater{}|t|)    }
\CommentTok{\#\textgreater{} (Intercept) {-}2.046470   0.254725  {-}8.034 2.29e{-}13 ***}
\CommentTok{\#\textgreater{} kor          0.080761   0.009676   8.347 3.75e{-}14 ***}
\CommentTok{\#\textgreater{} {-}{-}{-}}
\CommentTok{\#\textgreater{} Signif. codes:  }
\CommentTok{\#\textgreater{} 0 \textquotesingle{}***\textquotesingle{} 0.001 \textquotesingle{}**\textquotesingle{} 0.01 \textquotesingle{}*\textquotesingle{} 0.05 \textquotesingle{}.\textquotesingle{} 0.1 \textquotesingle{} \textquotesingle{} 1}
\CommentTok{\#\textgreater{} }
\CommentTok{\#\textgreater{} Residual standard error: 0.8628 on 154 degrees of f...}
\CommentTok{\#\textgreater{} Multiple R{-}squared:  0.3115, Adjusted R{-}squared:  0...}
\CommentTok{\#\textgreater{} F{-}statistic: 69.67 on 1 and 154 DF,  p{-}value: 3.751...}
\end{Highlighting}
\end{Shaded}

A fenti output kimutatja ugyan a kapcsolatot (a t- és az F-statisztika
is szignifikáns), ám az R-négyzet („Multiple R-Squared'') értéke kissé
gyenge magyarázóerőre utal (a független változó a függő változó
varianciájának csupán 30\%-át magyarázza). A kapcsolat irányáról azt
állapíthatjuk meg, hogy minél idősebb valaki, annál inkább pozitívabb az
anyagi/erkölcsi téren vállalt kockázat iránti attitűdje (kor változó
együtthatója 0,08).

Ezt követően nézzük meg, hogy van-e kapcsolat az életkor és testi épség
terén vállalt kockázat iránti attitűd között. Most is lineáris
regresszió-analízissel nézzük meg, hogy van-e kapcsolat a kor és a
második faktor faktorértékei között.

\begin{Shaded}
\begin{Highlighting}[]
\FunctionTok{summary}\NormalTok{(}\FunctionTok{lm}\NormalTok{(Factor2 }\SpecialCharTok{\textasciitilde{}}\NormalTok{ kor, }\AttributeTok{data =}\NormalTok{ d))}
\CommentTok{\#\textgreater{} }
\CommentTok{\#\textgreater{} Call:}
\CommentTok{\#\textgreater{} lm(formula = Factor2 \textasciitilde{} kor, data = d)}
\CommentTok{\#\textgreater{} }
\CommentTok{\#\textgreater{} Residuals:}
\CommentTok{\#\textgreater{}     Min      1Q  Median      3Q     Max }
\CommentTok{\#\textgreater{} {-}3.6017 {-}0.7960 {-}0.0183  0.7935  4.9395 }
\CommentTok{\#\textgreater{} }
\CommentTok{\#\textgreater{} Coefficients:}
\CommentTok{\#\textgreater{}             Estimate Std. Error t value Pr(\textgreater{}|t|)    }
\CommentTok{\#\textgreater{} (Intercept)  2.27054    0.39416   5.760 4.42e{-}08 ***}
\CommentTok{\#\textgreater{} kor         {-}0.08960    0.01497  {-}5.985 1.46e{-}08 ***}
\CommentTok{\#\textgreater{} {-}{-}{-}}
\CommentTok{\#\textgreater{} Signif. codes:  }
\CommentTok{\#\textgreater{} 0 \textquotesingle{}***\textquotesingle{} 0.001 \textquotesingle{}**\textquotesingle{} 0.01 \textquotesingle{}*\textquotesingle{} 0.05 \textquotesingle{}.\textquotesingle{} 0.1 \textquotesingle{} \textquotesingle{} 1}
\CommentTok{\#\textgreater{} }
\CommentTok{\#\textgreater{} Residual standard error: 1.335 on 154 degrees of fr...}
\CommentTok{\#\textgreater{} Multiple R{-}squared:  0.1887, Adjusted R{-}squared:  0...}
\CommentTok{\#\textgreater{} F{-}statistic: 35.82 on 1 and 154 DF,  p{-}value: 1.465...}
\end{Highlighting}
\end{Shaded}

Az eredmény kimutatja ugyan a kapcsolatot ( a t- és az F-statisztika is
szignifikáns), ám az R-négyzet („Multiple R-Squared'') értéke kissé
gyenge magyarázóerőre utal (a független változó a függő változó
varianciájának csupán 20\%-át magyarázza. A kapcsolat irányáról azt
állapíthatjuk meg, hogy minél idősebb valaki, annál inkább
kedvezőtlenebb a testi épség terén vállalt kockázat iránti attitűdje
(kor változó együtthatója -0,08).

Összefoglalva, sikerült a faktoranalízissel alátámasztanunk a
kockázatvállalás két faktorát. Azt is megállapítottuk, hogy mindkét
faktor függ a kortól: az anyagi/erkölcsi téren vállalt kockázat iránti
attitűd az évek múlásával egyre kedvezőbbé válik, míg a testi épség
terén vállalt kockázat iránti attitűd idővel egyre elutasítóbbá válik.

\begin{figure}

{\centering \includegraphics{./images/faktor_kockazat_kep_01.jpg}

}

\caption{Milyen dimenziói vannak a kockázatvállalásnak és változik-e a
korral a kockázatvállalás: Feltáró faktorelemzés}

\end{figure}

\hypertarget{sec-bf2}{%
\section{Példa: Még egyszer Big Five}\label{sec-bf2}}

\begin{itemize}
\tightlist
\item
  A példa forrása: NavarroFoxcroft2022
  \href{https://davidfoxcroft.github.io/lsj-book/15-Factor-Analysis.html\#exploratory-factor-analysis}{15.1
  Exploratory Factor Analysis}
\item
  Kapcsolódó jamovi állomány: \texttt{faktor\_bfi\_sample.omv}
\end{itemize}

A feltáró faktorlemzés (Exploratory Factor Analysis, EFA) feltár minden
olyan rejtett, látens tényezőt, amelyre a megfigyelt adatainkból
következtethetünk. A pszichológiában a látens tényezők olyan
pszichológiai jelenségeket vagy konstruktumokat képviselnek, amelyeket
nehéz közvetlenül megfigyelni vagy mérni.

Ebben a példában 25 személyiségpszichológai item elemzését végezzük,
amely része a Synthetic Aperture Personality Assessment
\href{http://sapa-project.org}{SAPA} webalapú rendszernek.

Az itemek a következők (az R-rel jelölt itemek fordított pontozásúak):

\begin{itemize}
\tightlist
\item
  BARATSA\_1 - (R) Közömbös vagyok mások érzései iránt.
\item
  BARATSA\_2 - Érdeklődöm mások jólétéről.
\item
  BARATSA\_3 - Tudom, hogyan vigasztaljak meg másokat.
\item
  BARATSA\_4 - Szeretem a gyerekeket.
\item
  BARATSA\_5 - Megnyugtatom az embereket.
\item
  LELKIIS\_1 - Igényes vagyok a munkában.
\item
  LELKIIS\_2 - Addig dolgozom, amíg minden tökéletes nem lesz.
\item
  LELKIIS\_3 - A dolgokat terv szerint csinálom.
\item
  LELKIIS\_4 - (R) Félgőzzel csinálom a dolgaimat..
\item
  LELKIIS\_5 - (R) Vesztegetem az időmet.
\item
  EXTRAVE\_1 - (R) Nem beszélek sokat.
\item
  EXTRAVE\_2 - (R) Nehezemre esik másokhoz közeledni.
\item
  EXTRAVE\_3 - Tudom, hogyan nyűgözzem le az embereket.
\item
  EXTRAVE\_4 - Könnyen szerzek barátokat.
\item
  EXTRAVE\_5 - Szeretek irányítani.
\item
  NEUROTI\_1 - Hamar dühbe gurulok.
\item
  NEUROTI\_2 - Könnyen felbosszantanak.
\item
  NEUROTI\_3 - Gyakran vannak hangulat-ingadozásaim.
\item
  NEUROTI\_4 - Gyakran vagyok szomorú.
\item
  NEUROTI\_5 - Könnyen pánikba esem.
\item
  NYITOTT\_1 - Tele vagyok ötletekkel.
\item
  NYITOTT\_2 - (R) Kerülöm a nehéz olvasmányokat.
\item
  NYITOTT\_3 - A beszélgetéseket magasabb szintre viszem.
\item
  NYITOTT\_4 - Fordítok időt arra, hogy visszatekintve elmélkedjek a
  dolgokon.
\item
  NYITOTT\_5 - (R) Nem szoktam elmélyülni egy adott témában.
\end{itemize}

A fenti táblázat összeállításához felhasználtam:

\begin{itemize}
\tightlist
\item
  \href{https://ipip.ori.org/HungarianIPIP-NEODomains.htm}{Hungarian
  Translation of the IPIP NEO Domains}
\item
  \href{https://ipip.ori.org/HungarianIntelligence.htm}{Hungarian
  Translation of IPIP Scales Related to Intelligence and Creativity}
\end{itemize}

Az itemekre adott válaszok 1-6 pontos válaszskálával rendelkeztek, ahol

\begin{itemize}
\tightlist
\item
  1 - Nagyon nem értek egyet
\item
  2 - Közepesen nem értek egyet
\item
  3 - Kissé nem értek egyet
\item
  4 - Kissé egyetértek
\item
  5 - Közepesen egyetértek
\item
  6 - Nagyon egyetértek.
\end{itemize}

A válaszokat a \texttt{bfi\_sample.xlsx} adatbázis tartalmazza.
Kutatóként szeretnénk feltárni az adatokat, hogy megtudjuk, vannak-e
olyan mögöttes látens tényezők, amelyeket ésszerűen jól mérnek a 25
megfigyelt változóval kapcsolatban.

\begin{Shaded}
\begin{Highlighting}[]
\NormalTok{d }\OtherTok{\textless{}{-}}\NormalTok{ rio}\SpecialCharTok{::}\FunctionTok{import}\NormalTok{(}\AttributeTok{file =} \StringTok{"adat/faktor\_bfi\_sample.xlsx"}\NormalTok{)}
\FunctionTok{str}\NormalTok{(d)}
\CommentTok{\#\textgreater{} \textquotesingle{}data.frame\textquotesingle{}:    250 obs. of  28 variables:}
\CommentTok{\#\textgreater{}  $ ID       : num  64432 66278 66391 62920 64835 ...}
\CommentTok{\#\textgreater{}  $ BARATSA\_1: num  2 1 1 2 1 4 2 3 2 1 ...}
\CommentTok{\#\textgreater{}  $ BARATSA\_2: num  3 6 6 6 5 2 5 6 5 6 ...}
\CommentTok{\#\textgreater{}  $ BARATSA\_3: num  3 5 5 6 6 1 4 3 4 5 ...}
\CommentTok{\#\textgreater{}  $ BARATSA\_4: num  5 1 1 6 5 4 4 5 4 6 ...}
\CommentTok{\#\textgreater{}  $ BARATSA\_5: num  5 5 3 6 6 1 6 4 5 5 ...}
\CommentTok{\#\textgreater{}  $ LELKIIS\_1: num  4 3 6 5 1 3 4 2 3 5 ...}
\CommentTok{\#\textgreater{}  $ LELKIIS\_2: num  2 2 6 5 1 2 3 5 3 5 ...}
\CommentTok{\#\textgreater{}  $ LELKIIS\_3: num  4 2 5 5 1 1 3 5 5 3 ...}
\CommentTok{\#\textgreater{}  $ LELKIIS\_4: num  2 4 1 2 6 2 2 2 5 5 ...}
\CommentTok{\#\textgreater{}  $ LELKIIS\_5: num  3 6 4 3 6 1 4 5 5 4 ...}
\CommentTok{\#\textgreater{}  $ EXTRAVE\_1: num  4 5 1 5 6 6 3 2 5 4 ...}
\CommentTok{\#\textgreater{}  $ EXTRAVE\_2: num  5 5 6 5 6 6 2 2 5 1 ...}
\CommentTok{\#\textgreater{}  $ EXTRAVE\_3: num  4 3 4 5 6 2 4 4 3 4 ...}
\CommentTok{\#\textgreater{}  $ EXTRAVE\_4: num  3 4 5 5 5 1 5 5 5 5 ...}
\CommentTok{\#\textgreater{}  $ EXTRAVE\_5: num  2 5 6 4 2 5 3 5 4 6 ...}
\CommentTok{\#\textgreater{}  $ NEUROTI\_1: num  5 2 1 1 6 1 1 5 3 4 ...}
\CommentTok{\#\textgreater{}  $ NEUROTI\_2: num  4 4 5 5 5 1 1 6 2 4 ...}
\CommentTok{\#\textgreater{}  $ NEUROTI\_3: num  3 4 1 4 6 1 2 6 2 3 ...}
\CommentTok{\#\textgreater{}  $ NEUROTI\_4: num  2 5 6 6 6 1 4 5 6 5 ...}
\CommentTok{\#\textgreater{}  $ NEUROTI\_5: num  4 2 5 6 6 1 2 5 5 4 ...}
\CommentTok{\#\textgreater{}  $ NYITOTT\_1: num  4 4 6 6 6 5 3 6 3 4 ...}
\CommentTok{\#\textgreater{}  $ NYITOTT\_2: num  2 4 3 2 1 1 4 2 5 5 ...}
\CommentTok{\#\textgreater{}  $ NYITOTT\_3: num  4 3 5 6 6 6 5 4 3 5 ...}
\CommentTok{\#\textgreater{}  $ NYITOTT\_4: num  6 6 6 6 6 4 5 5 5 6 ...}
\CommentTok{\#\textgreater{}  $ NYITOTT\_5: num  3 1 2 2 1 1 3 5 4 1 ...}
\CommentTok{\#\textgreater{}  $ nem      : chr  "Females" "Females" "Females" "F...}
\CommentTok{\#\textgreater{}  $ kor      : num  27 24 19 22 32 24 29 14 23 51 ...}
\NormalTok{psych}\SpecialCharTok{::}\FunctionTok{headTail}\NormalTok{(d)}
\CommentTok{\#\textgreater{}        ID BARATSA\_1 BARATSA\_2 BARATSA\_3 BARATSA\_4}
\CommentTok{\#\textgreater{} 1   64432         2         3         3         5}
\CommentTok{\#\textgreater{} 2   66278         1         6         5         1}
\CommentTok{\#\textgreater{} 3   66391         1         6         5         1}
\CommentTok{\#\textgreater{} 4   62920         2         6         6         6}
\CommentTok{\#\textgreater{} ...   ...       ...       ...       ...       ...}
\CommentTok{\#\textgreater{} 247 67401         1         5         5         6}
\CommentTok{\#\textgreater{} 248 61661         1         5         6         5}
\CommentTok{\#\textgreater{} 249 65674         2         6         5         6}
\CommentTok{\#\textgreater{} 250 63479         1         2         5         6}
\CommentTok{\#\textgreater{}     BARATSA\_5 LELKIIS\_1 LELKIIS\_2 LELKIIS\_3 LELKIIS\_4}
\CommentTok{\#\textgreater{} 1           5         4         2         4         2}
\CommentTok{\#\textgreater{} 2           5         3         2         2         4}
\CommentTok{\#\textgreater{} 3           3         6         6         5         1}
\CommentTok{\#\textgreater{} 4           6         5         5         5         2}
\CommentTok{\#\textgreater{} ...       ...       ...       ...       ...       ...}
\CommentTok{\#\textgreater{} 247         4         6         5         5         1}
\CommentTok{\#\textgreater{} 248         6         4         3         2         4}
\CommentTok{\#\textgreater{} 249         5         4         3         5         2}
\CommentTok{\#\textgreater{} 250         5         3         4         5         1}
\CommentTok{\#\textgreater{}     LELKIIS\_5 EXTRAVE\_1 EXTRAVE\_2 EXTRAVE\_3 EXTRAVE\_4}
\CommentTok{\#\textgreater{} 1           3         4         5         4         3}
\CommentTok{\#\textgreater{} 2           6         5         5         3         4}
\CommentTok{\#\textgreater{} 3           4         1         6         4         5}
\CommentTok{\#\textgreater{} 4           3         5         5         5         5}
\CommentTok{\#\textgreater{} ...       ...       ...       ...       ...       ...}
\CommentTok{\#\textgreater{} 247         1         3         2         5         5}
\CommentTok{\#\textgreater{} 248         5         2         1         2         5}
\CommentTok{\#\textgreater{} 249         3         1         2         5         2}
\CommentTok{\#\textgreater{} 250         1         2         2         5         5}
\CommentTok{\#\textgreater{}     EXTRAVE\_5 NEUROTI\_1 NEUROTI\_2 NEUROTI\_3 NEUROTI\_4}
\CommentTok{\#\textgreater{} 1           2         5         4         3         2}
\CommentTok{\#\textgreater{} 2           5         2         4         4         5}
\CommentTok{\#\textgreater{} 3           6         1         5         1         6}
\CommentTok{\#\textgreater{} 4           4         1         5         4         6}
\CommentTok{\#\textgreater{} ...       ...       ...       ...       ...       ...}
\CommentTok{\#\textgreater{} 247         6         2         4         3         4}
\CommentTok{\#\textgreater{} 248         2         2         2         2         2}
\CommentTok{\#\textgreater{} 249         6         4         2         4         5}
\CommentTok{\#\textgreater{} 250         5         1         1         2         4}
\CommentTok{\#\textgreater{}     NEUROTI\_5 NYITOTT\_1 NYITOTT\_2 NYITOTT\_3 NYITOTT\_4}
\CommentTok{\#\textgreater{} 1           4         4         2         4         6}
\CommentTok{\#\textgreater{} 2           2         4         4         3         6}
\CommentTok{\#\textgreater{} 3           5         6         3         5         6}
\CommentTok{\#\textgreater{} 4           6         6         2         6         6}
\CommentTok{\#\textgreater{} ...       ...       ...       ...       ...       ...}
\CommentTok{\#\textgreater{} 247         2         5         1         5         4}
\CommentTok{\#\textgreater{} 248         2         6         1         5         5}
\CommentTok{\#\textgreater{} 249         4         4         2         5         6}
\CommentTok{\#\textgreater{} 250         2         5         1         5         5}
\CommentTok{\#\textgreater{}     NYITOTT\_5     nem kor}
\CommentTok{\#\textgreater{} 1           3 Females  27}
\CommentTok{\#\textgreater{} 2           1 Females  24}
\CommentTok{\#\textgreater{} 3           2 Females  19}
\CommentTok{\#\textgreater{} 4           2 Females  22}
\CommentTok{\#\textgreater{} ...       ...    \textless{}NA\textgreater{} ...}
\CommentTok{\#\textgreater{} 247         2 Females  40}
\CommentTok{\#\textgreater{} 248         2   Males  68}
\CommentTok{\#\textgreater{} 249         2 Females  45}
\CommentTok{\#\textgreater{} 250         2   Males  34}
\end{Highlighting}
\end{Shaded}

\textbf{Alkalmazási feltételek.} Először ellenőrizzük az alkalmazási
feltételeket:

\begin{itemize}
\tightlist
\item
  a Bartlett-féle szférikus teszt szignifikáns, tehát ez a feltétel
  teljesül
\item
  mintavétel megfelelőségének KMO-mértéke (MSA), összességében jó
  mintavételi megfelelőségre utal.
\end{itemize}

\textbf{Faktorok száma.} Most a párhuzamos elemzési technikával kapott
faktorszámot őrizzük meg, ez jelen esetben 5. A Horn-féle szimulációs
módszer lényege, hogy az adatokból kapott sajátértékeket
összehasonlítjuk azokkal, amelyeket véletlenszerű adatokból kapnánk. A
kinyert faktorok száma az a szám, amelynek a sajátértéke nagyobb, mint
amit véletlenszerű adatokkal kapnánk.

\textbf{Forgatás moódja.} A forgatásnak két fő megközelítése van: az
ortogonális (például ``varimax'') forgatás, amikor a kapott faktorok nem
fognak korrelálni egymással, míg a hegyesszögű (ferde) (például
``Oblimin'') forgatás lehetővé teszi a kiválasztott tényezők
korrelációját. A pszichológusok tipikusan olyan dimenziókat vizsgálnak,
amelyekről nem azt feltételezzük, hogy ortogonálisak egymásra, így a
ferde megoldások vitathatatlanul ésszerűbbek!

Ha a ferde forgatás során a faktorok kimutatható korrelációt mutatnak
(pozitív vagy negatív, és \textgreater0,3) -- mint esetünkben -- ez
megerősítené megérzésünket, hogy a ferde forgatást részesítsük előnyben.
Ha a tényezők valójában korrelálnak, akkor a ferde elforgatás jobb
becslést ad a valódi tényezőkről és jobb egyszerű struktúrát, mint az
ortogonális elforgatás. És ha a ferde elforgatás azt jelzi, hogy a
tényezők közel nulla korrelációt mutatnak egymás között, akkor a kutató
továbbléphet és végrehajthat egy ortogonális elforgatást (ami ekkor
körülbelül ugyanazt a megoldást adja, mint a ferde elforgatás). A
kinyert faktorok közötti korreláció ellenőrzésekor legalább egy
korreláció nagyobb volt, mint 0,3, ezért az öt kinyert faktor ferde
(``oblimin'') elforgatása előnyös.

\textbf{Magyarázott varianciahányad.} Az adatok összesített
varianciájának aránya, amelyet az öt tényező magyaráz, 46\%. Az első
faktor a variancia körülbelül 10\%-át, a 2-4 faktor egyenként körülbelül
9\%-át, az ötös faktor pedig valamivel több mint 7\%-át teszi ki. Ez nem
öröm, jobb lett volna ha ez az arány nagyobb.

\textbf{Faktorsúlyok - Faktorok értelmezése.} A faktormátrix tartalmazza
a faktorsúlyokat, vagyis, hogy a 25 különböző személyiségitem, hogyan
töltődik be az öt kiválasztott faktor mindegyikére.Az 1-4 faktor az
előzetes elvárásoknak megfelelően tartalmazza az itemeket. Az 5. faktor
is majdnem rendben van, mindössze a \texttt{NYITOTT\_4} item nem az 5.
faktorra, hanem a 4. faktorra illeszkedik.

Vegyük észre, hogy a fordított pontozású itemek negatív
faktorterhelésűek. Például a \texttt{BARATSA\_1} (``Közömbös vagyok
mások érzései iránt.'') és a \texttt{BARATSA\_2} ('' Érdeklődöm mások
jólétéről.'') itemek esetében láthatjuk, hogy a \texttt{BARATSA\_1}-on a
magas pontszám alacsony barátságosságot jelent, míg
\texttt{BARATSA\_2}-n a magas pontszám magas barátságosságot jelez.
Emiatt \texttt{BARATSA\_1} negatívan korrelál a többi ``barátságosság''
változóval, és ezért van negatív faktorterhelése.

A faktormátrix mellett az ``egyediségét'' is láthatjuk. Az egyediség a
variancia azon aránya, amely ``egyedi'' a változóra nézve, és nem
magyarázható a faktorokkal. Például \texttt{BARATSA\_1} varianciájának
72\%-a nem magyarázható az ötfaktoros megoldásban szereplő tényezőkkel.
Ezzel szemben a \texttt{NEUROTI\_1}-nek viszonylag alacsony az a
varianciája, amelyet a faktormegoldás nem vesz figyelembe (35\%). Minél
nagyobb az „egyediség'', annál kisebb a változó relevanciája vagy
hozzájárulása a faktormodellben.

\textbf{Faktorértékek.} Úgy tűnik, hogy van egy elég jó öttényezős
megoldásunk, bár a megfigyelt varianciahányad összességében viszonylag
alacsony. Tételezzük fel, hogy elégedettek vagyunk ezzel a megoldással,
és szeretnénk felhasználni tényezőinket a további elemzésekhez. Az
egyszerű lehetőség az, hogy minden egyes tényezőre kiszámoljuk az
összesített (átlagos) pontszámot úgy, hogy összeadjuk minden olyan
változó pontszámát, amely érdemben töltődik a faktoron, majd elosztjuk a
változók számával (más szóval létrehozunk egy ``átlagpontszámot'' minden
egyes személy számára az egyes skálák elemei között). Az adatbázisunkban
szereplő minden egyes személy esetében számoljuk ki a Barátságosság
pontszámát.

Másik megoldás a faktor pontszám meghatározása. Ehhez mentenünk kell a
faktorértékeket, és öt új oszlop keletkezik az adatbázisunkban. A 4.
faktor jelenti a Barátságosság pontszámát.

\begin{figure}

{\centering \includegraphics{./images/faktor_bfi_sample_kep_01.jpg}

}

\caption{Még egyszer Big Five: Feltáró faktorelemzés}

\end{figure}

\bookmarksetup{startatroot}

\hypertarget{sec-megerosito-faktorelemzes}{%
\chapter{Megerősítő faktorelemzés}\label{sec-megerosito-faktorelemzes}}

\begin{itemize}
\tightlist
\item
  A fejezet forrása: NavarroFoxcroft2022
  \href{https://davidfoxcroft.github.io/lsj-book/15-Factor-Analysis.html\#confirmatory-factor-analysis}{15.3
  Confirmatory Factor Analysis}
\item
  Kapcsolódó jamovi állomány: \texttt{faktor\_bfi\_sample2.omv}
\end{itemize}

A feltáró faktorelemzés segítségével könnyen azonosíthatók a mögöttes,
látens faktorok. Sok esetben azonban azt szeretnénk látni, hogy az épp
rendelkezésre álló faktorok helytállóak-e, tudjuk-e igazolni adatainkkal
a létezésüket. Ezt a szigorúbb ellenőrzést megerősítő faktorelemzésnek
(Confirmatory Factor Analysis, CFA) nevezzük. Célja egy előre
meghatározott látens faktorstruktúra megerősítése: megnézzük, hogy az
adatok mennyire illeszkednek az előre megadott faktorstruktúrára. Abban
az értelemben megerősítő az elemzés, hogy megnézzük, mennyire erősítik
meg a megfigyelt adatok az előre meghatározott modellt.

A \ref{sec-bf2} fejezetben már foglalkoztunk a személyiség 5 faktoros
modelljével. Ott feltáró elemzéssel sikerült azonosítani azt az 5
faktort, amelyre a 25 személyiségitem épp a megfelelő módon töltött
(egyetlen item kivételével).

Most a célünk az 5 faktoros modell megerősítése lesz. A
\texttt{bfi\_sample2.xlsx} állományt fogjuk használni. Rendelkezésre áll
250 vizsgálati személy adata, akik a 25 személyiségitemre adtak választ.
Minden item csoportosítható valamely személyiségfaktor egyikébe,
méghozzá ötös csoportokban. Az adatok visszaigazolják az 5 faktoros
struktúrát? A következő struktúráról van szó:

\begin{figure}

{\centering \includegraphics{./images/faktor_bfi_sample2_cfa_modell.png}

}

\caption{Megerősítő faktorelemzés faktorstruktúrája}

\end{figure}

\begin{itemize}
\item
  Barátságosság faktor itemei:

  \begin{itemize}
  \tightlist
  \item
    BARATSA\_1
  \item
    BARATSA\_2
  \item
    BARATSA\_3
  \item
    BARATSA\_4
  \item
    BARATSA\_5
  \end{itemize}
\item
  Lelkiismeretesség

  \begin{itemize}
  \tightlist
  \item
    LELKIIS\_1
  \item
    LELKIIS\_2
  \item
    LELKIIS\_3
  \item
    LELKIIS\_4
  \item
    LELKIIS\_5
  \end{itemize}
\item
  Extraverzió faktor itemei:

  \begin{itemize}
  \tightlist
  \item
    EXTRAVE\_1
  \item
    EXTRAVE\_2
  \item
    EXTRAVE\_3
  \item
    EXTRAVE\_4
  \item
    EXTRAVE\_5
  \end{itemize}
\item
  Neuroticitás faktor itemei:

  \begin{itemize}
  \tightlist
  \item
    NEUROTI\_1
  \item
    NEUROTI\_2
  \item
    NEUROTI\_3
  \item
    NEUROTI\_4
  \item
    NEUROTI\_5
  \end{itemize}
\item
  Nyitottság faktor itemei:

  \begin{itemize}
  \tightlist
  \item
    NYITOTT\_1
  \item
    NYITOTT\_2
  \item
    NYITOTT\_3
  \item
    NYITOTT\_4
  \item
    NYITOTT\_5
  \end{itemize}
\end{itemize}

A modellünk felépítése előtt néhány tényezőt érdemes figyelembe venni:

\begin{itemize}
\item
  \textbf{A látens tényezők korrelációja.} Ahogy korábban említettük, a
  pszichológiában a konstruktumok legtöbbször összefüggenek egymással,
  azaz esetünkben a személyiségfaktorok korrelálhatnak egymással.
  Modellünkben meg kell engednünk, hogy ezek a látens tényezők együtt
  változzanak.
\item
  \textbf{Hibatagok korrelációja.} A itemek értékeinek meghatározása
  bizonyos hibával történik a modellben. Ezek a hibák korrelálhatnak
  egymással? Meg kell vizsgálni, hogy van-e szisztematikus oka annak,
  hogy egyes hibák korrelálnak egymással. Egyelőre nincs olyan
  egyértelmű ok, amelyek igazolnák egyes hibák összefüggenek egymással.
\end{itemize}

\hypertarget{faktorstruktuxfara-megaduxe1sa}{%
\section{Faktorstruktúra
megadása}\label{faktorstruktuxfara-megaduxe1sa}}

A jamovi-ban a \texttt{Factor\ /\ Confirmatory\ Factor\ Analysis}
menüpontban kell felvinni a fenti struktúrát.

\begin{figure}

{\centering \includegraphics{./images/cfa_kep_01.jpg}

}

\caption{Big Five: Faktorstruktúra felépítése}

\end{figure}

\hypertarget{eredmuxe9nyek-uxe9rtuxe9keluxe9se}{%
\section{Eredmények
értékelése}\label{eredmuxe9nyek-uxe9rtuxe9keluxe9se}}

Az elemzés elvégzése után nézzük meg az eredményeket.

\hypertarget{illeszkeduxe9si-mutatuxf3k-vizsguxe1lata}{%
\subsection{Illeszkedési mutatók
vizsgálata}\label{illeszkeduxe9si-mutatuxf3k-vizsguxe1lata}}

Az első dolog, amit meg kell nézni, a modellillesztés, mivel ez
megmutatja, hogy a modellünk mennyire illeszkedik a megfigyelt adatokra.
Számos módja van a modell illeszkedésének értékelésére.

\begin{itemize}
\item
  \textbf{Khí-négyzet próba.} A khí-négyzet próbastatisztika kis értéke
  azt jelzi, hogy a modell jól illeszkedik az adatokhoz. Ebben az
  esetben a próba nem szignifikáns \((p>0,05)\). A modell
  illeszkedésének értékelésére használt khí-négyzet statisztika azonban
  meglehetősen érzékeny a minta méretére, azaz nagy minta esetén a
  modell és az adatok elég jó illeszkedése is szinte mindig szignifikáns
  próbastatisztikát eredményez.
\item
  \textbf{CFI} - összehasonlító illeszkedési mutató (Comparative Fit
  Index) - A CFI azt méri fel, hogy faktorstruktúra milyen mértékben
  reprodukálja a valós adatokon nyugvó kovarianciamátrixot egy független
  modellhez képest. A CFI mutatók értéke 0 és 1 közötti tartományba
  eshet, ahol az 1-hez közeli érték jelzi a szoros illeszkedést.
  Kezdetben a mutatók elfogadhatósági kritériumának 0,90-et adtak meg,
  de az utóbbi időkben inkább a 0,95-ot tekintik az elfogadhatóság alsó
  határának.
\item
  \textbf{TLI} - Tucker--Lewis-féle Illeszkedési mutató - A TLI a
  CFI-hez hasonló módon méri az illeszkedést, annyi különbséggel, hogy
  ez a mutató a modellben használt szabadságfokot is figyelembe veszi,
  így kiküszöböli a vizsgálati minta méretének befolyásoló szerepét. A
  TLI mutatók értéke 0 és 1 közötti tartományba eshet, ahol az 1-hez
  közeli érték jelzi a szoros illeszkedést. Kezdetben a mutatók
  elfogadhatósági kritériumának 0,90-et adtak meg, de az utóbbi időkben
  inkább a 0,95-ot tekintik az elfogadhatóság alsó határának.
\item
  \textbf{RMSEA} - a becslési hiba négyzetes átlagának gyöke
  (Root-Mean-Square Error of Approximation) - A Steiger-féle RMSEA
  mutatót a modell populációs kovariancia mátrixhoz viszonyított
  illeszkedésének becsléséhez használjuk. Az RMSEA az elemszámtól
  függetlenül hasonlítja össze, hogy a valós és az optimális
  paraméterekkel rendelkező hipotetikus modell kovarianciamátrixa milyen
  mértékben illeszkedik. Az RMSEA a modell takarékosságának megbízható
  jelzője, a komplex modellek hibás specifikálásának hatékony mutatója.
  Az RMSEA értéke is 0 és 1 közé eshet, itt azonban a kisebb, 0-hoz
  közel eső érték jelzi a jobb illeszkedést. Az RMSEA értékei 0,05-ig
  szoros illeszkedést jeleznek; 0,08-os értékig pedig megfelelő
  illeszkedést.
\end{itemize}

A saját eredményeinket szemlélve azt láthatjuk, hogy a khi-négyzet
értéke nagy és nagyon szignifikáns. A mintánk mérete nem túl nagy, így
ez valószínűleg rossz illeszkedést jelez. A CFI 0,762 a TLI pedig 0,731,
ami rossz illeszkedést jelez a modell és az adatok között. Az RMSEA
0,085 90\%-kos konfidencia intervallum 0,077 és 0,092, és ez megint nem
jó illeszkedést jelez.

\begin{figure}

{\centering \includegraphics{./images/cfa_kep_02.jpg}

}

\caption{Big Five: Illeszkedési mutatók}

\end{figure}

\hypertarget{faktorterheluxe9sek-uxe9s-faktorkovariancia-becsluxe9se}{%
\subsection{Faktorterhelések és faktorkovariancia
becslése}\label{faktorterheluxe9sek-uxe9s-faktorkovariancia-becsluxe9se}}

Nézzük tovább a faktorterheléseket és a faktorkovariancia becsléseket. A
táblázatokban látható a Z-statisztika és a p-érték mindegyik paraméterre
azt jelzi, hogy ésszerűen járulnak hozzá a modellhez (azaz nem nullák),
így úgy tűnik, nincs ok a megadott változó-faktor útvonalak, vagy
faktor-faktor korrelációk eltávolítására a modellből. A standardizált
becslések gyakran könnyebben értelmezhetők. Ezeket a
\texttt{Estimates\ /\ Standardized\ estimate} opciónál lehet megadni
jamovi-ban. Ezek a táblázatok könnyen beépíthetők a tudományos írásokba.

\begin{figure}

{\centering \includegraphics{./images/cfa_kep_03.jpg}

}

\caption{Big Five: Faktorterhelés becslése}

\end{figure}

\begin{figure}

{\centering \includegraphics{./images/cfa_kep_04.jpg}

}

\caption{Big Five: Faktorkovariancia becslése}

\end{figure}

\hypertarget{modell-javuxedtuxe1sa}{%
\subsection{Modell javítása}\label{modell-javuxedtuxe1sa}}

Hogyan javíthatnánk a modellt? Az egyik lehetőség az, hogy átírjuk az
általunk kifejlesztett itemeket. Egy másik lehetőség az, hogy néhány
utólagos (post-hoc) módosítást végzünk a modellen az illeszkedés
javítása érdekében. Ennek egyik módja a \texttt{Modification\ indices}
használata, amely a jamovi-ban az \texttt{Additional\ Output} részben
jelölhető be.

\begin{figure}

{\centering \includegraphics{./images/cfa_kep_05.jpg}

}

\caption{Big Five: Faktorterhelések MI értékei}

\end{figure}

Az első táblázatban
(\texttt{Factor\ Loadings\ -\ Modification\ Indices}) a legmagasabb
módosítási index (MI) értéket keressük. Eldöntjük, hogy van-e értelme
ezt az itemet a modellbe bevinni. Például a táblázatban láthatjuk, hogy
a modellben még nem szereplő faktorterhelések legnagyobb MI értéke
28,786, a \texttt{NEUROTI\_4} (``Gyakran vagyok szomorú.'') item töltése
a látens ``Extraverzió'' faktorra. Ez azt jelzi, hogy ha ezt az utat
hozzáadjuk a modellhez, akkor a khí-négyzet értéke körülbelül
ugyanennyivel csökken.

De a mi modellünkben ennek az itemnek a hozzáadása sem elméleti sem
módszertani szempontból nem támasztható alá, ezért nem jó ötlet (hacsak
nem tud olyan meggyőző érvvel előállni, hogy a ``Gyakran vagyok
szomorú.'' a neuroticizmust és az extraverziót is méri). A példa
kedvéért tegyünk úgy, mintha lenne valami értelme, és adjuk hozzá ezt az
utat a modellhez. Menjünk vissza a CFA felülethez, és adjuk hozzá
\texttt{NEUROTI\_4}-t az ``Extraverzió'' faktorhoz. A CFA eredményei
most megváltoznak; a khi-négyzet 728 környékére zuhant (10 körüli esés,
nagyjából az MI méretéhez hasonló, de azért annál kisebb), és a többi
illeszkedési index is javult, bár csak egy kicsit. Ez nem elég: ez még
mindig nem egy jól illeszkedő modell.

Ha új paramétereket adunk a modellhez, akkor mindig ellenőrizzük le újra
az MI-táblázatokat, mivel az MI-k minden alkalommal frissülnek.

Van egy másik jamovi táblázat is
(\texttt{Residual\ Covariance\ -\ Modification\ Indices}), amely a
maradék kovariancia-módosítási indexeket tartalmazza. Más szavakkal, ha
egymással korreláló hibatagokat adnánk a modellhez, ilyen mértékben nőne
a modell illeszkedése. Esetünkben a legnagyobb MI érték a
\texttt{NEUROTI\_1} és \texttt{NEUROTI\_2} itemek által meghatározott
cellában olvasható (45), vagyis e két item kovarianciájának modellhez
adása esetén nő leginkább a modell illeszkedése. Adjuk a modellhez ezt
az együttjárást.

\begin{figure}

{\centering \includegraphics{./images/cfa_kep_06.jpg}

}

\caption{Big Five: Reziduálisok kovarianciája}

\end{figure}

Célszerű egyszerre vizsgálni mindkét MI táblát, és úgy meghatározni a
legnagyobb MI-t, majd átgondolni, hogy a javasolt paraméter hozzáadása
ésszerűen-e, és ha lehet, akkor hozzáadjuk a modellhez. Ezután újra
megkeressük a legnagyobb MI-t a már újraszámolt táblázatokban.

\bookmarksetup{startatroot}

\hypertarget{sec-tobbszempontos-varianciaelemzes}{%
\chapter{Többszempontos
varianciaelemzés}\label{sec-tobbszempontos-varianciaelemzes}}

JÖN.

\bookmarksetup{startatroot}

\hypertarget{sec-klaszterelemzes}{%
\chapter{Klaszterelemzés}\label{sec-klaszterelemzes}}

A klaszteranalízis célja, hogy értelmes és használható csoportokba
(klaszterekbe) sorolja az adatokat. Úgy alakíthatunk ki csoportokat az
\(n\) elemű mintánkban, hogy a ``hasonlóak'' egy csoportba kerülnek.
Minden klaszter elemei viszonylag hasonlóak legyenek egymáshoz, de
különbözzenek más klaszterek elemeitől.

Fogalmak:

\begin{itemize}
\tightlist
\item
  \textbf{Összevonási séma:} Megmutatja, hogy a hierarchikus
  klaszterelemzés egyes lépéseinél mely megfigyelési egységek vagy
  esetek kerültek összevonásra.
\item
  \textbf{Klaszterközép:} Az összes megfigyelési egységet alapul véve a
  változók átlaga egy adott klaszterben.
\item
  \textbf{Klaszterközéppont:} A nem hierarchikus klaszterelemzés
  kiindulópontjai. A klasztereket ezen középpontok vagy magok köré építi
  az eljárás.
\item
  \textbf{Dendrogram:} Fadiagramnak is szokás nevezni, amely a
  klaszterelemzés eredményeinek grafikus ábrázolása. A függőleges
  vonalak az összetartozó klasztereket jelölik. A vonalak skálán való
  elhelyezkedése megmutatja az összevont klaszterek közötti távolságot.
\item
  \textbf{A klaszterközéppontok közötti távolságok.} Megmutatják,
  mennyire különböznek a klaszterek egymástól. Kívánatos, hogy a
  klaszterek egymástól jól elkülönüljenek és jól jellemezhetőek
  legyenek.
\item
  \textbf{Hasonlósági/távolsági együtthatók mátrixa:} A
  hasonlósági/távolsági együtthatók mátrixa egy alsó/felső
  háromszögmátrix, amely a megfigyelési egységek vagy esetek közötti
  páronkénti távolságot tartalmazza.
\end{itemize}

A klaszterelemzés több lépésből áll:

\begin{enumerate}
\def\labelenumi{\arabic{enumi}.}
\tightlist
\item
  A probléma megfogalmazása
\item
  Távolságmérték kiválasztása
\item
  Klasztermódszer kiválasztása
\item
  Döntés klaszterek számáról
\item
  A klaszterek értelmezése és jellemzése
\item
  A klaszterelemzés érvényességének ellenőrzése
\end{enumerate}

A fenti folyamatot egy példán mutatjuk be: Malhotra és Simon (2008, o.
642).

\hypertarget{puxe9lda-fogyasztuxf3k-vuxe1suxe1rluxe1ssal-kapcsolatos-attitux171djei}{%
\section{Példa: Fogyasztók vásárlással kapcsolatos
attitűdjei}\label{puxe9lda-fogyasztuxf3k-vuxe1suxe1rluxe1ssal-kapcsolatos-attitux171djei}}

\hypertarget{a-probluxe9ma-megfogalmazuxe1sa}{%
\subsection{1. A probléma
megfogalmazása}\label{a-probluxe9ma-megfogalmazuxe1sa}}

Ebben a példában fogyasztókat vásárlással kapcsolatos attitűdjeik
alapján szeretnénk csoportosítani. Összesen hat, attitűdváltozót vettek
figyelembe: megkérték őket, hogy fejezzék ki a következő állításokkal
kapcsolatban egy hétfokozatú skálán az egyetértésüket:

\begin{itemize}
\tightlist
\item
  v1: A vásárlás szórakozás.
\item
  v2: A vásárlás nem tesz jót a pénztárcának.
\item
  v3: A vásárlást gyakran összekötöm étteremlátogatással.
\item
  v4: Vásárláskor megpróbálom a legjobb vételt csinálni.
\item
  v5: Nem érdekel a vásárlás.
\item
  v6: Az árak összehasonlításával rengeteg pénzt lehet megtakarítani.
\end{itemize}

Az adatok a \texttt{klaszter\_fogyaszto.xlsx} Excel állományban
találhatók.

\begin{Shaded}
\begin{Highlighting}[]
\CommentTok{\# adatok beolvasása R{-}ben}
\NormalTok{fogyaszto }\OtherTok{\textless{}{-}}\NormalTok{ rio}\SpecialCharTok{::}\FunctionTok{import}\NormalTok{(}\AttributeTok{file =} \StringTok{"adat/klaszter\_fogyaszto.xlsx"}\NormalTok{)}
\FunctionTok{str}\NormalTok{(fogyaszto)}
\CommentTok{\#\textgreater{} \textquotesingle{}data.frame\textquotesingle{}:    20 obs. of  6 variables:}
\CommentTok{\#\textgreater{}  $ v1: num  6 2 7 4 1 6 5 7 2 3 ...}
\CommentTok{\#\textgreater{}  $ v2: num  4 3 2 6 3 4 3 3 4 5 ...}
\CommentTok{\#\textgreater{}  $ v3: num  7 1 6 4 2 6 6 7 3 3 ...}
\CommentTok{\#\textgreater{}  $ v4: num  3 4 4 5 2 3 3 4 3 6 ...}
\CommentTok{\#\textgreater{}  $ v5: num  2 5 1 3 6 3 3 1 6 4 ...}
\CommentTok{\#\textgreater{}  $ v6: num  3 4 3 6 4 4 4 4 3 6 ...}
\NormalTok{psych}\SpecialCharTok{::}\FunctionTok{headTail}\NormalTok{(fogyaszto)}
\CommentTok{\#\textgreater{}      v1  v2  v3  v4  v5  v6}
\CommentTok{\#\textgreater{} 1     6   4   7   3   2   3}
\CommentTok{\#\textgreater{} 2     2   3   1   4   5   4}
\CommentTok{\#\textgreater{} 3     7   2   6   4   1   3}
\CommentTok{\#\textgreater{} 4     4   6   4   5   3   6}
\CommentTok{\#\textgreater{} ... ... ... ... ... ... ...}
\CommentTok{\#\textgreater{} 17    4   4   7   2   2   5}
\CommentTok{\#\textgreater{} 18    3   7   2   6   4   3}
\CommentTok{\#\textgreater{} 19    4   6   3   7   2   7}
\CommentTok{\#\textgreater{} 20    2   3   2   4   7   2}
\end{Highlighting}
\end{Shaded}

\hypertarget{tuxe1volsuxe1gi-vagy-hasonluxf3suxe1gi-muxe9rtuxe9k-kivuxe1lasztuxe1sa}{%
\subsection{2. Távolsági vagy hasonlósági mérték
kiválasztása}\label{tuxe1volsuxe1gi-vagy-hasonluxf3suxe1gi-muxe9rtuxe9k-kivuxe1lasztuxe1sa}}

Mivel a klaszterelemzés célja, hogy a hasonló megfigyelési egységek egy
csoportba kerüljenek, szükségünk van egy mérőszámra, azaz a hasonlóság
vagy különbség számszerűsítésére.

A klaszteranalízis kiindulópontja tehát az elemek közötti hasonlóság
vagy távolság. Ezzel kapcsolatban általában rendelkezünk előzetes
információkkal, amelyek alapján kiszámítjuk ezeket a hasonlóságokat vagy
távolságokat. Más esetekben csak a hasonlóságok vagy távolságok
mértékéről rendelkezünk információkkal.

A legelterjedtebb módszer a hasonlóság mérésére a megfigyelési egységek
páronkénti távolsága. Azok a megfigyelési egységek, amelyek között
kisebb a távolság hasonlóbbak egymáshoz, mint azok, amelyek között
nagyobb. Megjegyezzük, hogy a hasonlóság és a távolság egymással
ellentétes fogalmak. Ebből a kapcsolatból adódik, hogy a hasonlóság és a
távolság mérőszáma egymásba átalakítható. Ennek képlete a következő:

\[h_{ij}=100\frac{d_{max}−d_{ij}}{d_{max}}\] A képletben a \(h_{ij}\)
jelöli az \(i\)-edik és a \(j\)-edik objektum közötti hasonlóságot, míg
a \(d_{ij}\) a távolságot, a \(d_{max}\) pedig a távolságmátrix
legnagyobb elemét jelöli.

A legelterjedtebb távolsági mérték az euklideszi távolság: az egyes
változók értékei közötti különbség négyzetösszegének a négyzetgyöke.
\href{https://psycho.unideb.hu/statisztika/pages/p_5_10.html}{További}
távolságmértékek is léteznek.

Ebben a példában az Euklideszi távolságot használjuk.

\begin{Shaded}
\begin{Highlighting}[]
\CommentTok{\# Euklideszi távolság kiszámítása a}
\NormalTok{tavolsagmatrix }\OtherTok{\textless{}{-}} \FunctionTok{dist}\NormalTok{(fogyaszto, }\AttributeTok{method =} \StringTok{"euclidean"}\NormalTok{)}
\NormalTok{tavolsagmatrix}
\CommentTok{\#\textgreater{}           1        2        3        4        5    ...}
\CommentTok{\#\textgreater{} 2  8.000000                                        ...}
\CommentTok{\#\textgreater{} 3  2.828427 8.246211                               ...}
\CommentTok{\#\textgreater{} 4  5.567764 5.567764 6.557439                      ...}
\CommentTok{\#\textgreater{} 5  8.306624 2.645751 9.110434 6.633250             ...}
\CommentTok{\#\textgreater{} 6  1.732051 6.855655 3.316625 4.472136 7.211103    ...}
\CommentTok{\#\textgreater{} 7  2.236068 6.244998 3.316625 4.690416 6.480741 1.4...}
\CommentTok{\#\textgreater{} 8  2.236068 8.774964 1.732051 6.000000 9.486833 2.8...}
\CommentTok{\#\textgreater{} 9  6.928203 2.828427 8.000000 5.567764 2.236068 5.9...}
\CommentTok{\#\textgreater{} 10 6.928203 4.242641 7.483315 2.236068 5.744563 5.7...}
\CommentTok{\#\textgreater{} 11 7.745967 2.000000 8.366600 6.244998 1.732051 6.8...}
\CommentTok{\#\textgreater{} 12 2.645751 5.916080 3.316625 3.464102 6.782330 2.0...}
\CommentTok{\#\textgreater{} 13 8.062258 1.732051 7.810250 5.830952 4.000000 7.0...}
\CommentTok{\#\textgreater{} 14 6.782330 6.000000 7.615773 1.732051 7.141428 5.5...}
\CommentTok{\#\textgreater{} 15 3.605551 7.000000 4.358899 4.690416 7.615773 3.1...}
\CommentTok{\#\textgreater{} 16 6.928203 5.291503 7.615773 2.236068 6.403124 5.7...}
\CommentTok{\#\textgreater{} 17 3.000000 7.416198 4.795832 4.898979 7.211103 2.8...}
\CommentTok{\#\textgreater{} 18 7.483315 4.898979 8.366600 4.123106 6.403124 6.7...}
\CommentTok{\#\textgreater{} 19 7.483315 6.633250 7.745967 2.645751 8.306624 6.5...}
\CommentTok{\#\textgreater{} 20 8.306624 3.000000 8.888194 7.071068 3.162278 7.3...}
\CommentTok{\#\textgreater{}           7        8        9       10       11    ...}
\CommentTok{\#\textgreater{} 2                                                  ...}
\CommentTok{\#\textgreater{} 3                                                  ...}
\CommentTok{\#\textgreater{} 4                                                  ...}
\CommentTok{\#\textgreater{} 5                                                  ...}
\CommentTok{\#\textgreater{} 6                                                  ...}
\CommentTok{\#\textgreater{} 7                                                  ...}
\CommentTok{\#\textgreater{} 8  3.162278                                        ...}
\CommentTok{\#\textgreater{} 9  5.385165 8.306624                               ...}
\CommentTok{\#\textgreater{} 10 5.567764 7.280110 4.898979                      ...}
\CommentTok{\#\textgreater{} 11 6.082763 8.888194 2.000000 5.291503             ...}
\CommentTok{\#\textgreater{} 12 2.000000 3.162278 5.567764 4.582576 6.082763    ...}
\CommentTok{\#\textgreater{} 13 6.324555 8.485281 4.123106 4.358899 3.000000 5.8...}
\CommentTok{\#\textgreater{} 14 5.744563 7.000000 6.164414 2.000000 6.928203 4.7...}
\CommentTok{\#\textgreater{} 15 3.741657 4.242641 6.708204 6.244998 7.141428 2.8...}
\CommentTok{\#\textgreater{} 16 5.567764 7.141428 5.656854 1.414214 6.164414 4.7...}
\CommentTok{\#\textgreater{} 17 2.449490 4.000000 6.403124 6.244998 7.000000 3.1...}
\CommentTok{\#\textgreater{} 18 6.855655 8.426150 4.898979 3.741657 5.477226 5.5...}
\CommentTok{\#\textgreater{} 19 6.708204 7.280110 7.483315 2.828427 7.745967 5.1...}
\CommentTok{\#\textgreater{} 20 6.782330 9.486833 2.236068 5.916080 2.645751 6.9...}
\CommentTok{\#\textgreater{}          13       14       15       16       17    ...}
\CommentTok{\#\textgreater{} 2                                                  ...}
\CommentTok{\#\textgreater{} 3                                                  ...}
\CommentTok{\#\textgreater{} 4                                                  ...}
\CommentTok{\#\textgreater{} 5                                                  ...}
\CommentTok{\#\textgreater{} 6                                                  ...}
\CommentTok{\#\textgreater{} 7                                                  ...}
\CommentTok{\#\textgreater{} 8                                                  ...}
\CommentTok{\#\textgreater{} 9                                                  ...}
\CommentTok{\#\textgreater{} 10                                                 ...}
\CommentTok{\#\textgreater{} 11                                                 ...}
\CommentTok{\#\textgreater{} 12                                                 ...}
\CommentTok{\#\textgreater{} 13                                                 ...}
\CommentTok{\#\textgreater{} 14 6.244998                                        ...}
\CommentTok{\#\textgreater{} 15 7.211103 6.244998                               ...}
\CommentTok{\#\textgreater{} 16 5.385165 1.414214 6.557439                      ...}
\CommentTok{\#\textgreater{} 17 7.615773 6.082763 4.000000 5.916080             ...}
\CommentTok{\#\textgreater{} 18 5.385165 4.690416 6.557439 4.898979 7.681146    ...}
\CommentTok{\#\textgreater{} 19 6.403124 2.449490 6.403124 2.828427 7.000000 4.8...}
\CommentTok{\#\textgreater{} 20 4.000000 7.416198 8.246211 6.855655 8.246211 5.5...}
\CommentTok{\#\textgreater{}          19}
\CommentTok{\#\textgreater{} 2          }
\CommentTok{\#\textgreater{} 3          }
\CommentTok{\#\textgreater{} 4          }
\CommentTok{\#\textgreater{} 5          }
\CommentTok{\#\textgreater{} 6          }
\CommentTok{\#\textgreater{} 7          }
\CommentTok{\#\textgreater{} 8          }
\CommentTok{\#\textgreater{} 9          }
\CommentTok{\#\textgreater{} 10         }
\CommentTok{\#\textgreater{} 11         }
\CommentTok{\#\textgreater{} 12         }
\CommentTok{\#\textgreater{} 13         }
\CommentTok{\#\textgreater{} 14         }
\CommentTok{\#\textgreater{} 15         }
\CommentTok{\#\textgreater{} 16         }
\CommentTok{\#\textgreater{} 17         }
\CommentTok{\#\textgreater{} 18         }
\CommentTok{\#\textgreater{} 19         }
\CommentTok{\#\textgreater{} 20 8.544004}
\end{Highlighting}
\end{Shaded}

\hypertarget{a-klasztermuxf3dszer-kivuxe1lasztuxe1sa}{%
\subsection{3. A klasztermódszer
kiválasztása}\label{a-klasztermuxf3dszer-kivuxe1lasztuxe1sa}}

Számos eljárás született a klaszteranalízis módszerén belül. Ebben a
könyvben két eljárással foglalkozunk részletesen:

\begin{itemize}
\tightlist
\item
  a hierarchikus eljárások: hierarchikus, faszerű felépítéssel
  jellemezhetők

  \begin{itemize}
  \tightlist
  \item
    összevonó: kiinduláskor minden elem külön klasztert alkot. A
    klaszterek képzése úgy történik, hogy a klasztereket egyre nagyobb
    klaszterekbe vonják össze.

    \begin{itemize}
    \tightlist
    \item
      láncmódszerek: az elemeket a köztük lévő távolság kiszámíása
      alapján csoportosítjuk

      \begin{itemize}
      \tightlist
      \item
        egyszerű lánc: a minimális távolság, vagyis a legközelebbi
        szomszéd elvén alapul
      \item
        teljes lánc: a maximális távolság, vagyis a legtávolabb szomszéd
        elvén alapul
      \item
        átlagos lánc: két klaszter távolságát az összes elem páronkénti
        távolságának átlagából számítja ki.
      \end{itemize}
    \item
      variancia-módszer: ahol a klasztereket oly módon képzik, hogy a
      klasztereken belül a szórásnégyzetet minimalizálják

      \begin{itemize}
      \tightlist
      \item
        Ward-féle eljárás: a klaszterátlagoktól való négyzetes
        euklideszi távolságot minimalizálják
      \item
        centroidmódszer: a klaszeterek közötti távolságot az összes
        változó átlagaként számított centroidok közötti távolságként
        határozzák meg.
      \end{itemize}
    \end{itemize}
  \item
    felosztó: kiinduláskor az összes elem egyetlen egy klasztert alkot.
    A klaszterek képzése úgy történik, hogy a klasztereket egyre kisebb
    klaszterekre osztják fel.
  \end{itemize}
\item
  K-középpontú klaszteranalízis: olyan eljárás, amely előre
  meghatározott klaszterközéppontból indul ki, és úgy csoportosítja az
  elemeket, hogy a középponttól számított küszöbértéken belül essenek.

  \begin{itemize}
  \tightlist
  \item
    szekvenciális küszöbértékek: kiválasztanak egy klaszterközéppontot,
    és minden megfigyelési egység, amely a középponttól az előre
    meghatározott küszöbértéken belül esik, azonos csoportba kerül.
    Ezután új küszöbértéket választanak és a folyamatot megismétlik a
    még nem csoportosított pontokra. Egy korábban már csoportosított
    megfigyelési egységet nem fognak újra csoportosítani.
  \item
    párhuzamos küszöbértékek: az előző módszertől annyiban tér el, hogy
    a klaszterközéppontokat egyidejűleg választják ki és a
    küszöbértékeken belüli megfigyelési egységeket a legközelebb eső
    középponthoz rendelik.
  \item
    az optimális felosztás módszere: abban különbözik a két fenti
    módszertől, hogy a megfigyelési egységeket újra hozzárendelik a
    klaszterhez, hogy egy általános kritériumot (például adott számú
    klaszterre a klaszteren belüli távolságok átlagát) optimalizáljanak.
  \end{itemize}
\end{itemize}

A K-középpontú klaszterelemzés hátránya, hogy a klaszterek számát előre
meg kell adni, és a klaszterközéppontok kiválasztása esetleges. Azonban
ez az eljárás gyorsabb, mint a hierarchikus eljárás és főképp nagy
mintaelemszám esetén javasolt a használata.

Célszerű a hierarchikus és nem hierarchikus módszereket egymásra építve
alkalmazni. Először a hierarchikus klaszterelemzéssel, az átlagos lánc
vagy Ward-féle módszert felhasználva egy kiinduló klasztermegoldáshoz
jutunk. A kapott klaszterszámot és a klaszterközéppontokat inputként
felhasználhatjuk az optimális felosztás módszeréhez.

Most a hierarchikus klaszterelemzést szemléltetjük Ward-féle eljárással.

\begin{Shaded}
\begin{Highlighting}[]
\CommentTok{\# hierarchikus klaszterelemzés R{-}ben}
\NormalTok{klaszter }\OtherTok{\textless{}{-}} \FunctionTok{hclust}\NormalTok{(tavolsagmatrix, }\AttributeTok{method =} \StringTok{"ward.D2"}\NormalTok{)}
\FunctionTok{plot}\NormalTok{(klaszter)}
\end{Highlighting}
\end{Shaded}

\begin{figure}[H]

{\centering \includegraphics{./sec_klaszter_files/figure-pdf/unnamed-chunk-4-1.pdf}

}

\end{figure}

A klaszterelemzés eredményének értékes ábrája a dendrogram. A fenti
ábrán a vízszintes vonalak az összevont klasztereket ábrázolják. A vonal
skálán való elhelyezkedése azt a távolságot mutatja meg, ahol a
klasztereket összevonták. Mivel a kezdeti lépésekben a távolságok
hasonló méretűek, nehéz megmondani, milyen sorrendben alakultak ki a
klaszterek. Erről a pontosabb információt a következő parancsokkal
kaphatunk.

\begin{Shaded}
\begin{Highlighting}[]
\NormalTok{klaszter}\SpecialCharTok{$}\NormalTok{merge  }\CommentTok{\# az összevonások lépései: az egyes lépésekben miket vont össze: negatív szám elem, pozitív klaszter}
\CommentTok{\#\textgreater{}       [,1] [,2]}
\CommentTok{\#\textgreater{}  [1,]   {-}6   {-}7}
\CommentTok{\#\textgreater{}  [2,]  {-}10  {-}16}
\CommentTok{\#\textgreater{}  [3,]   {-}2  {-}13}
\CommentTok{\#\textgreater{}  [4,]   {-}3   {-}8}
\CommentTok{\#\textgreater{}  [5,]   {-}4  {-}14}
\CommentTok{\#\textgreater{}  [6,]   {-}5  {-}11}
\CommentTok{\#\textgreater{}  [7,]   {-}1    1}
\CommentTok{\#\textgreater{}  [8,]   {-}9    6}
\CommentTok{\#\textgreater{}  [9,]    2    5}
\CommentTok{\#\textgreater{} [10,]  {-}12    7}
\CommentTok{\#\textgreater{} [11,]  {-}20    8}
\CommentTok{\#\textgreater{} [12,]  {-}19    9}
\CommentTok{\#\textgreater{} [13,]  {-}17   10}
\CommentTok{\#\textgreater{} [14,]  {-}15   13}
\CommentTok{\#\textgreater{} [15,]    3   11}
\CommentTok{\#\textgreater{} [16,]    4   14}
\CommentTok{\#\textgreater{} [17,]  {-}18   12}
\CommentTok{\#\textgreater{} [18,]   15   17}
\CommentTok{\#\textgreater{} [19,]   16   18}
\NormalTok{klaszter}\SpecialCharTok{$}\NormalTok{height  }\CommentTok{\# az egyes összevonások milyen távolság esetén történtek meg}
\CommentTok{\#\textgreater{}  [1]  1.414214  1.414214  1.732051  1.732051  1.732051}
\CommentTok{\#\textgreater{}  [6]  1.732051  2.160247  2.236068  2.345208  2.415229}
\CommentTok{\#\textgreater{} [11]  3.000000  3.082207  3.271085  4.057914  4.582576}
\CommentTok{\#\textgreater{} [16]  5.033223  5.507571 13.638182 17.659747}
\FunctionTok{cbind}\NormalTok{(klaszter}\SpecialCharTok{$}\NormalTok{merge, klaszter}\SpecialCharTok{$}\NormalTok{height)  }\CommentTok{\# együtt a két fenti információ}
\CommentTok{\#\textgreater{}       [,1] [,2]      [,3]}
\CommentTok{\#\textgreater{}  [1,]   {-}6   {-}7  1.414214}
\CommentTok{\#\textgreater{}  [2,]  {-}10  {-}16  1.414214}
\CommentTok{\#\textgreater{}  [3,]   {-}2  {-}13  1.732051}
\CommentTok{\#\textgreater{}  [4,]   {-}3   {-}8  1.732051}
\CommentTok{\#\textgreater{}  [5,]   {-}4  {-}14  1.732051}
\CommentTok{\#\textgreater{}  [6,]   {-}5  {-}11  1.732051}
\CommentTok{\#\textgreater{}  [7,]   {-}1    1  2.160247}
\CommentTok{\#\textgreater{}  [8,]   {-}9    6  2.236068}
\CommentTok{\#\textgreater{}  [9,]    2    5  2.345208}
\CommentTok{\#\textgreater{} [10,]  {-}12    7  2.415229}
\CommentTok{\#\textgreater{} [11,]  {-}20    8  3.000000}
\CommentTok{\#\textgreater{} [12,]  {-}19    9  3.082207}
\CommentTok{\#\textgreater{} [13,]  {-}17   10  3.271085}
\CommentTok{\#\textgreater{} [14,]  {-}15   13  4.057914}
\CommentTok{\#\textgreater{} [15,]    3   11  4.582576}
\CommentTok{\#\textgreater{} [16,]    4   14  5.033223}
\CommentTok{\#\textgreater{} [17,]  {-}18   12  5.507571}
\CommentTok{\#\textgreater{} [18,]   15   17 13.638182}
\CommentTok{\#\textgreater{} [19,]   16   18 17.659747}
\end{Highlighting}
\end{Shaded}

A fenti outputból kiolvasható, hogy 19 lépésben jutottunk el az 1
klaszteres struktúrához. Az első két oszlopban az összevont elemek vagy
más kialakított klaszterek azonosítója szerepel. Negatív azonosító az
objektum adatbázisban elfoglalt helyét mutatja, a pozitív azonosító
pedig azt a klasztert, amelyet a hivatkozott lépésben alakítottunk ki. A
3.oszlopban azt a távolságot láthatjuk, amelyen az összevonás történt.

Világos, hogy az utolsó két lépésben az összevont klaszterek közötti
távolság nagy. Ez az információ hasznos lehet a klaszterek számának
eldöntésénél.

A fenti elemzés jamovi-ban a
\texttt{snowCluster\ /\ Hierarchical\ Clustering} vagy
\texttt{snowCluster\ /\ Clustering\ Dendrogram} menüpontjaival is
elvégezhető.

\begin{figure}

{\centering \includegraphics{./images/klaszter_fogyaszto_01.jpg}

}

\caption{Fogyasztók hierarchikus klaszterelemzése: snowCluster /
Hierarchical Clustering}

\end{figure}

\begin{figure}

{\centering \includegraphics{./images/klaszter_fogyaszto_02.jpg}

}

\caption{Fogyasztók hierarchikus klaszterelemzése: snowCluster /
Clustering Dendrogram}

\end{figure}

Amennyiben K-középpontú klaszterelemzést szeretnénk végrehajtani, akkor
ismerettel kell rendelkezünk a klaszterek számáról. A korábbi
hierarchikus klaszterelemzés eredménye alapján a 3 klaszteres megoldás
mellett döntünk. A kiinduló klaszterközéppontokat az első három
véletlenszerűen választott eset értéke adja. A csoportosítás
középpontjai ideiglenes középpontok, amelyekhez eseteket rendel hozzá az
algoritmus. Mindegyik esetet a legközelebbi középponthoz rendeli. A
klasszifikációs középpontokat mindig módosítják, amíg egy határértéket
el nem érnek. A végső klaszterközéppontok a változók átlagait tükrözik a
végleges megoldásban.

\begin{Shaded}
\begin{Highlighting}[]
\NormalTok{kkozep }\OtherTok{\textless{}{-}} \FunctionTok{kmeans}\NormalTok{(}\AttributeTok{x =}\NormalTok{ tavolsagmatrix, }\AttributeTok{centers =} \DecValTok{3}\NormalTok{)}
\NormalTok{kkozep}
\CommentTok{\#\textgreater{} K{-}means clustering with 3 clusters of sizes 6, 8, 6}
\CommentTok{\#\textgreater{} }
\CommentTok{\#\textgreater{} Cluster means:}
\CommentTok{\#\textgreater{}          1        2        3        4        5     ...}
\CommentTok{\#\textgreater{} 1 6.862188 5.439023 7.564144 2.162174 6.772019 5.79...}
\CommentTok{\#\textgreater{} 2 2.285490 7.306763 2.958135 5.042656 7.775617 2.16...}
\CommentTok{\#\textgreater{} 3 7.891613 2.034372 8.403615 6.152633 2.296025 6.87...}
\CommentTok{\#\textgreater{}          7        8        9       10       11     ...}
\CommentTok{\#\textgreater{} 1 5.855728 7.187966 5.778384 2.036728 6.308718 4.73...}
\CommentTok{\#\textgreater{} 2 2.290041 2.670468 6.651896 6.259566 7.270421 2.38...}
\CommentTok{\#\textgreater{} 3 6.216759 8.904788 2.237278 5.075444 1.896300 6.18...}
\CommentTok{\#\textgreater{}         13       14       15       16       17     ...}
\CommentTok{\#\textgreater{} 1 5.601384 2.047695 6.116402 2.131984 6.303994 3.72...}
\CommentTok{\#\textgreater{} 2 7.301405 6.229253 3.242432 6.283385 3.029503 7.20...}
\CommentTok{\#\textgreater{} 3 2.809193 6.649207 7.320453 5.959452 7.315402 5.43...}
\CommentTok{\#\textgreater{}         19       20}
\CommentTok{\#\textgreater{} 1 2.608512 6.895128}
\CommentTok{\#\textgreater{} 2 6.796789 8.029135}
\CommentTok{\#\textgreater{} 3 7.519380 2.507349}
\CommentTok{\#\textgreater{} }
\CommentTok{\#\textgreater{} Clustering vector:}
\CommentTok{\#\textgreater{}  1  2  3  4  5  6  7  8  9 10 11 12 13 14 15 16 17 ...}
\CommentTok{\#\textgreater{}  2  3  2  1  3  2  2  2  3  1  3  2  3  1  2  1  2 ...}
\CommentTok{\#\textgreater{} 20 }
\CommentTok{\#\textgreater{}  3 }
\CommentTok{\#\textgreater{} }
\CommentTok{\#\textgreater{} Within cluster sum of squares by cluster:}
\CommentTok{\#\textgreater{} [1] 120.51554 170.68012  77.21363}
\CommentTok{\#\textgreater{}  (between\_SS / total\_SS =  82.1 \%)}
\CommentTok{\#\textgreater{} }
\CommentTok{\#\textgreater{} Available components:}
\CommentTok{\#\textgreater{} }
\CommentTok{\#\textgreater{} [1] "cluster"      "centers"      "totss"       }
\CommentTok{\#\textgreater{} [4] "withinss"     "tot.withinss" "betweenss"   }
\CommentTok{\#\textgreater{} [7] "size"         "iter"         "ifault"}
\NormalTok{kkozep}\SpecialCharTok{$}\NormalTok{totss}
\CommentTok{\#\textgreater{} [1] 2063.469}
\NormalTok{fogyaszto}\SpecialCharTok{$}\NormalTok{group\_2 }\OtherTok{\textless{}{-}}\NormalTok{ kkozep}\SpecialCharTok{$}\NormalTok{cluster  }\CommentTok{\# a kapott csoportváltozó beszúrása}
\FunctionTok{table}\NormalTok{(fogyaszto}\SpecialCharTok{$}\NormalTok{group\_2)}
\CommentTok{\#\textgreater{} }
\CommentTok{\#\textgreater{} 1 2 3 }
\CommentTok{\#\textgreater{} 6 8 6}
\end{Highlighting}
\end{Shaded}

A fenti eredmény a klaszteranalízis eredményét mutatja. Az első sor
(\texttt{K-means\ clustering\ with\ 3\ clusters\ of\ sizes\ 8,\ 6,\ 6})
arról ad információt, hogy háromklaszteres megoldásunk van, melyek
mérete 8, illetve 6 és 6 elemszám. Hogy az egyes elemek melyik
klaszterbe esnek, arról a \texttt{Clustering\ vector} ad információt. Az
első sor az egyes elemeket, a második pedig a csoporttagságot mutatja.

A \texttt{Cluster\ means} az egyes klaszterek átlagos tagjának,
centroidjának a jellemzőit mutatják. A
\texttt{Within\ cluster\ sum\ of\ squares\ by\ cluster} értékei a
klaszteren belüli eltérések négyzetösszegét mutatja.

A fenti K-közép klaszterelemzés jamovi-ban is elvégezhető a
\texttt{snowCluster\ /\ K-means\ Clustering} menüponttal.

\includegraphics{./images/klaszter_fogyaszto_03.jpg} \#\#\# 4. Döntés a
klaszterek számáról

A klaszterelemzés egyik legfontosabb kérdése a klaszterek számának
eldöntése. Milyen általános szabályok alapján dönthetünk:

\begin{itemize}
\tightlist
\item
  Elméleti vagy gyakorlati megfontolások alapján dönthetünk a klaszterek
  számáról.
\item
  Hierarchikus klaszterelemzés során a klaszterek összevonására
  alkalmazott távolságok felhasználhatók kritériumként. A dendrogramból
  kiolvasható ez az információ. Az utolsó lépésnél az összevont
  klaszterek között nagy a távolság. Ebből következően a háromklaszteres
  megoldás tűnik megfelelőnek.
\item
  A nem hierarchikus klaszterelemzésénél a belső és a külső variancia
  hányadosát ábrázolják a klaszterek számának függvényében. Az a pont,
  ahol egy könyök vagy éles törés látható, a megfelelő klaszterek
  számára utal. E ponton túl nem érdemes a klaszterek számát növelni.
\item
  A klaszterek relatív méretét is érdemes figyelembe venni. Az egyelemű
  vagy túl kicsi gyakoriságú csoportoknak nincs értelme.
\end{itemize}

Amennyiben a klaszterek összevonásánál használt távolságokat használjuk,
akkor egy 10-es távolságot beállítva 3 csoportot képezhetünk:

\begin{Shaded}
\begin{Highlighting}[]
\NormalTok{fogyaszto}\SpecialCharTok{$}\NormalTok{group\_1 }\OtherTok{\textless{}{-}} \FunctionTok{cutree}\NormalTok{(}\AttributeTok{tree =}\NormalTok{ klaszter, }\AttributeTok{h =} \DecValTok{10}\NormalTok{)}
\FunctionTok{table}\NormalTok{(fogyaszto}\SpecialCharTok{$}\NormalTok{group\_1)}
\CommentTok{\#\textgreater{} }
\CommentTok{\#\textgreater{} 1 2 3 }
\CommentTok{\#\textgreater{} 8 6 6}
\end{Highlighting}
\end{Shaded}

Látható, hogy az egyes csoportok 8, 6 és 6 elemet tartalmaznak.

A fenti 3 csoportnak megfelelő dendrogramot a \texttt{\{factoextra\}}
csomaggal is megjeleníthetjük.

\begin{Shaded}
\begin{Highlighting}[]
\NormalTok{hc }\OtherTok{\textless{}{-}}\NormalTok{ factoextra}\SpecialCharTok{::}\FunctionTok{hcut}\NormalTok{(fogyaszto, }\DecValTok{3}\NormalTok{, }\AttributeTok{stand =} \ConstantTok{FALSE}\NormalTok{, }\AttributeTok{hc\_method =} \StringTok{"ward.D2"}\NormalTok{,}
    \AttributeTok{hc\_metric =} \StringTok{"euclidean"}\NormalTok{)}
\NormalTok{factoextra}\SpecialCharTok{::}\FunctionTok{fviz\_dend}\NormalTok{(hc, }\AttributeTok{repel =} \ConstantTok{TRUE}\NormalTok{, }\AttributeTok{lwd =} \DecValTok{1}\NormalTok{, }\AttributeTok{horiz =} \ConstantTok{TRUE}\NormalTok{,}
    \AttributeTok{cex =} \FloatTok{0.9}\NormalTok{, }\AttributeTok{color\_labels\_by\_k =} \ConstantTok{TRUE}\NormalTok{)}
\end{Highlighting}
\end{Shaded}

\begin{figure}[H]

{\centering \includegraphics{./sec_klaszter_files/figure-pdf/unnamed-chunk-8-1.pdf}

}

\end{figure}

Kassambara összefoglalja az optimális klaszterszám meghatározásának 3
leggyakoribb módszerét
(\href{https://www.datanovia.com/en/lessons/determining-the-optimal-number-of-clusters-3-must-know-methods/}{Determining
The Optimal Number Of Clusters: 3 Must Know Methods}). A jamovi
\texttt{snowCluster\ /\ K-means\ Clustering} menüpontja alatt az egyik
eljárás, a gap-módszer elérhető.

A három módszer a

\begin{itemize}
\tightlist
\item
  könyök módszer (elbow method),
\item
  sziluett módszer (silhouette method) és a
\item
  gap-statisztika módszer (gap statistic method).
\end{itemize}

\hypertarget{kuxf6nyuxf6k-muxf3dszer}{%
\subsubsection{Könyök módszer}\label{kuxf6nyuxf6k-muxf3dszer}}

Ismert, hogy a K-közép klaszterezési eljárás mögötti alapötlet az, hogy
a klasztereket úgy határozzuk meg, hogy a teljes klaszteren belüli
variabilitás (a teljes klaszteren belüli négyzetösszeg, WSS) minimális
legyen. A teljes WSS a klaszterezés tömörségét méri, és azt szeretnénk,
hogy az a lehető legkisebb legyen.

A könyök módszer a teljes WSS-t a klaszterek számának függvényében
vizsgálja: az az optimális klaszterszám, amikor már egy újabb klaszter
hozzáadása nem javítja számottevően a teljes WSS-t.

A klaszterek optimális száma a következőképpen határozható meg:

\begin{enumerate}
\def\labelenumi{\arabic{enumi}.}
\tightlist
\item
  Valamely klaszterezési algoritmust (például K-közép klaszterezés)
  futtatása \(k\) különböző értékeire. Például \(k\) értéke 1-től 10-ig
  fut.
\item
  Minden \(k\) esetében kiszámítjuk a teljes WSS értéket.
\item
  Ábrázoljuk a WSS görbéjét a \(k\) klaszterek számának megfelelően.
\item
  A könyök elhelyezkedését az ábrán általában a klaszterek megfelelő
  számát jelenti.
\end{enumerate}

\hypertarget{uxe1tlagos-sziluett-muxf3dszer}{%
\subsubsection{Átlagos sziluett
módszer}\label{uxe1tlagos-sziluett-muxf3dszer}}

Az átlagos sziluett megközelítés a klaszterezés minőségét méri. Minden
egyes \(i\) megfigyelési egységre kiszámítható a sziluett szélessége,
amely az \(S_i=(b_i−a_i)/max(a_i,b_i)\) képlettel számolható. A fenti
képletben

\begin{itemize}
\tightlist
\item
  \(a_i\) az \(i\)-t tartalmazó klaszteren belül az átlagos klaszteren
  belüli távolság, azaz a klaszteren belüli egyes pontok közötti átlagos
  távolság
\item
  \(b_i\) az \(i\) és a tőle minimális távolságra lévő (szomszédos)
  klaszter távolsága (\(i\) és egy másik klaszter távolsága: \(i\) és
  másik klaszterben lévő pontok átlagos távolsága).
\end{itemize}

Az \(S_i\) értéke -1 és +1 közötti lehet:

\begin{itemize}
\tightlist
\item
  A nagy \(S_i\)-értékkel (majdnem 1) végzett megfigyelések nagyon jól
  klaszterezettek.
\item
  A kis \(S_i\) (körülbelül 0) azt jelenti, hogy a megfigyelés két
  klaszter között van.
\item
  A negatív \(S_i\) értékkel rendelkező megfigyelések valószínűleg rossz
  klaszterbe kerültek.
\end{itemize}

Az átlagos sziluett módszer a megfigyelések átlagos sziluettjét számítja
ki különböző \(k\) értékeihez. A \(k\) klaszterek optimális száma az,
amely maximalizálja az átlagos sziluettet a \(k\) lehetséges értékeinek
tartományában.

Az algoritmus hasonló a könyök módszerhez, és a következőképpen
számítható ki:

\begin{enumerate}
\def\labelenumi{\arabic{enumi}.}
\tightlist
\item
  Valamely klaszterezési algoritmust (például K-közép klaszterezés)
  futtatása \(k\) különböző értékeire. Például \(k\) értéke 1-től 10-ig
  fut.
\item
  Minden \(k\) esetében kiszámítjuk a megfigyelések átlagos sziluettjét
  (\texttt{avg.sil}).
\item
  Ábrázoljuk az \texttt{avg.sil} görbéjét a klaszterek száma szerint
  (\(k\)).
\item
  A görbe maximum helyét tekintjük megfelelő számú klaszternek.
\end{enumerate}

\hypertarget{a-gap-statisztika-muxf3dszer}{%
\subsubsection{A gap-statisztika
módszer}\label{a-gap-statisztika-muxf3dszer}}

A gap-statisztikát R. Tibshirani, G. Walther és T. Hastie tette közzé
2001-ben. A megközelítés bármely klaszterezési módszerre alkalmazható.

A gap-statisztika összehasonlítja a klaszteren belüli összesített
variabilitást az adatok null referenciaeloszlása mellett várt értékévek,
különböző \(k\)-értékeknél. Az optimális klaszterek becslése olyan érték
lesz, amely maximalizálja a gap-statisztikát (azaz a legnagyobb
gap-statisztikát eredményezi). Ilyenkor a klaszterezési struktúra messze
van a pontok véletlenszerű egyenletes eloszlásától.

Az algoritmus a következőképpen működik:

\begin{enumerate}
\def\labelenumi{\arabic{enumi}.}
\tightlist
\item
  Klaszterezzük a megfigyelt adatokat úgy, hogy a klaszterek számát
  \(k=1, \dots, k_{max}\) értékkel változtatjuk, és számítsuk ki a
  \(W_k\) klasztereken belüli összvariabilitást.
\item
  Generáljunk B referencia adatkészletet véletlenszerű egyenletes
  eloszlással. Végezzünk klaszterezést ezen a referenciaadatkészleten
  változó számú klaszterrel \(k=1, \dots, k_{max}\) és számítsuk ki a
  megfelelő teljes klasztereken belüli összvariabilitást (\(W_{kb}\)).
\item
  Számítsuk ki a becsült gap-statisztikát:
  \(Gap(k)=\frac{1}{B}\sum_{b=1}^Blog(W_{kb})−log(W_k)\). Számítsuk ki a
  statisztika szórását is.\\
  1.Válasszuk meg a klaszterek számát a legkisebb olyan \(k\)-ra, amely
  kielégíti a következő feltételt: \(Gap(k)\geq Gap(k+1)−s_{k+1}\).
\end{enumerate}

A fenti 3 módszerhez tartozó ábra a \texttt{\{factoextra\}} csomag
\texttt{fviz\_nbclust()} függvényével is megjeleníthető.

\begin{Shaded}
\begin{Highlighting}[]
\FunctionTok{library}\NormalTok{(factoextra)}
\CommentTok{\# Elbow method}
\FunctionTok{fviz\_nbclust}\NormalTok{(}\AttributeTok{x =}\NormalTok{ fogyaszto, kmeans, }\AttributeTok{method =} \StringTok{"wss"}\NormalTok{) }\SpecialCharTok{+} \FunctionTok{geom\_vline}\NormalTok{(}\AttributeTok{xintercept =} \DecValTok{4}\NormalTok{,}
    \AttributeTok{linetype =} \DecValTok{2}\NormalTok{) }\SpecialCharTok{+} \FunctionTok{labs}\NormalTok{(}\AttributeTok{subtitle =} \StringTok{"Elbow method"}\NormalTok{)}

\CommentTok{\# Silhouette method}
\FunctionTok{fviz\_nbclust}\NormalTok{(}\AttributeTok{x =}\NormalTok{ fogyaszto, kmeans, }\AttributeTok{method =} \StringTok{"silhouette"}\NormalTok{) }\SpecialCharTok{+}
    \FunctionTok{labs}\NormalTok{(}\AttributeTok{subtitle =} \StringTok{"Silhouette method"}\NormalTok{)}

\CommentTok{\# Gap statistic nboot = 50 to keep the function speedy.}
\CommentTok{\# recommended value: nboot= 500 for your analysis.  Use}
\CommentTok{\# verbose = FALSE to hide computing progression.}
\FunctionTok{set.seed}\NormalTok{(}\DecValTok{123}\NormalTok{)}
\FunctionTok{fviz\_nbclust}\NormalTok{(}\AttributeTok{x =}\NormalTok{ fogyaszto, kmeans, }\AttributeTok{nstart =} \DecValTok{25}\NormalTok{, }\AttributeTok{method =} \StringTok{"gap\_stat"}\NormalTok{,}
    \AttributeTok{nboot =} \DecValTok{50}\NormalTok{) }\SpecialCharTok{+} \FunctionTok{labs}\NormalTok{(}\AttributeTok{subtitle =} \StringTok{"Gap statistic method"}\NormalTok{)}
\end{Highlighting}
\end{Shaded}

\begin{figure}[H]

{\centering \includegraphics{./sec_klaszter_files/figure-pdf/unnamed-chunk-9-1.pdf}

}

\end{figure}

\begin{figure}[H]

{\centering \includegraphics{./sec_klaszter_files/figure-pdf/unnamed-chunk-9-2.pdf}

}

\end{figure}

\begin{figure}[H]

{\centering \includegraphics{./sec_klaszter_files/figure-pdf/unnamed-chunk-9-3.pdf}

}

\end{figure}

A könyök módszer 4, a sziluett módszer és a gap-statisztikán alapuló
módszer 3 klasztert tart optimálisnak.

\hypertarget{a-klaszterek-uxe9rtelmezuxe9se-uxe9s-jellemzuxe9se}{%
\subsection{5. A klaszterek értelmezése és
jellemzése}\label{a-klaszterek-uxe9rtelmezuxe9se-uxe9s-jellemzuxe9se}}

A klaszterek értelmezését és jellemzését a klasztercentroidok (átlagok)
értelmezésével végezzük. A centroidokon a klaszterekbe tartozó
megfigyelési egységeknek a az összes változó alapján számított átlagát
értjük. A centroidok lehetővé teszik, hogy mindegyik klaszterhez egy
nevet vagy címkét rendeljünk.

Az 1. klaszteren viszonylag magas az értéke az 1. változónak (a vásárlás
szórakozás) és a 3. változónak (a vásárlást étteremlátogatással köti
össze). Alacsony az értéke az 5. változónak (nem érdekel a vásárlás). Az
1. klaszter a ``Szórakozáskedvelő, érdeklődő vásárlók'' névvel
jelölhető. Ez a klaszter az 1, 3, 6, 7, 8, 12, 15 és a 17 eseteket
tartalmazza. A 2. klaszter éppen az előző klaszter ellentettje, mivel
alacsony az 1. és a 3. változó értéke, és magas az 5. változó átlaga,
ezért ``Apatikus vásárlónak'' nevezhetjük el. A 2. klaszter tagjait a 2,
5, 9, 11, 13 és 20 elemek alkotják. A 3. klaszternél a 2. változó (a
vásárlás megterheli a pénztárcát), a 4. változó (vásárláskor a legjobb
vételt akarom csinálni) és a 6. változó (az árak összehasonlításával
sokat lehet megtakarítani) értéke magas. Ennek következtében ez a
klaszter a ``Takarékos vásárlók'' nevet kapta. A klasztereket a 4, 10,
14, 16, 18 és 19 eseteket foglalja magában.

Gyakran segít a klaszterek értelmezésében és jellemzésében olyan
változók bevonása, amelyeket nem használtunk fel a klaszterelemzésben.
Ezek lehetnek például demográfiai adatok.

\hypertarget{a-klaszterelemzuxe9s-megbuxedzhatuxf3suxe1guxe1nak-uxe9s-uxe9rvuxe9nyessuxe9guxe9nek-ellenux151rzuxe9se}{%
\subsection{6. A klaszterelemzés megbízhatóságának és érvényességének
ellenőrzése}\label{a-klaszterelemzuxe9s-megbuxedzhatuxf3suxe1guxe1nak-uxe9s-uxe9rvuxe9nyessuxe9guxe9nek-ellenux151rzuxe9se}}

A klasztermegoldás megbízhatóságát és érvényességét is ellenőrizni kell.
Ezek igen komplex eljárások, és teljes mértékben nem igazolhatók. A
következő eljárások jól használhatók a klasztereredmények minőségének
értékelésére.

\begin{itemize}
\tightlist
\item
  A klaszterelemzést elvégezzük ugyanazokkal az adatokkal, de más
  távolságmértéket használunk. A két mérték alapján kapott eredményeket
  összehasonlítjuk.
\item
  Különböző klasztereljárásokat alkalmazunk, és összehasonlítjuk az
  eredményeket.
\item
  Az adatokat véletlenszerűen két csoportra osztjuk. Mindkét részre
  elvégezzük a klaszterelemzést. Összehasonlítjuk a a két alminta
  klaszterátlagait. Véletlenszerűen elhagyunk változókat, és a
  klaszterelemzést a csökkentett számú változók alapján végezzük el.
  Hasonlítsuk össze az eredményeket a teljes változókészlettel kapott
  eredménnyel.
\item
  A nem hierarchikus klaszterelemzésnél a megoldás az esetek
  adatbázisban elfoglalt sorrendjétől is függhet. Futtassuk az elemzést
  az esetek különböző sorrendjével, amíg a megoldás nem stabilizálódik.
\end{itemize}

\begin{Shaded}
\begin{Highlighting}[]
\FunctionTok{table}\NormalTok{(fogyaszto}\SpecialCharTok{$}\NormalTok{group\_1, fogyaszto}\SpecialCharTok{$}\NormalTok{group\_2)}
\CommentTok{\#\textgreater{}    }
\CommentTok{\#\textgreater{}     1 2 3}
\CommentTok{\#\textgreater{}   1 0 8 0}
\CommentTok{\#\textgreater{}   2 0 0 6}
\CommentTok{\#\textgreater{}   3 6 0 0}
\end{Highlighting}
\end{Shaded}

A fenti kétdimenziós gyakorisági táblázatból kiolvasható, hogy a
hierarchikus klaszterelemzés és a K-középpontú klaszterelemzés ugyanazt
az eredményt szolgáltatja (kivéve a csoportok elnevezését).

\hypertarget{puxe9lda-vuxe1llalatok-vizsguxe1lata}{%
\section{Példa: Vállalatok
vizsgálata}\label{puxe9lda-vuxe1llalatok-vizsguxe1lata}}

A következő problémában különböző vállalatokat próbálunk meg
klaszterezni. A vállalatokat számtalan jellemző mentén mérhetjük,
vizsgálhatjuk, ezáltal többféleképpen is csoportosíthatjuk őket. A
csoportosítás alapjául mi most a vállalat nagyságát, a hatalmi
távolságot és a vállalat szemléletében jelen levő konzervativizmus
mértékét választottuk. Az adatok, ahogyan az az 5.27. R-forráskódon is
látható.

\begin{Shaded}
\begin{Highlighting}[]
\NormalTok{vallalat }\OtherTok{\textless{}{-}}\NormalTok{ rio}\SpecialCharTok{::}\FunctionTok{import}\NormalTok{(}\AttributeTok{file =} \StringTok{"adat/klaszter\_vallalatok.xlsx"}\NormalTok{)}
\FunctionTok{str}\NormalTok{(vallalat)}
\CommentTok{\#\textgreater{} \textquotesingle{}data.frame\textquotesingle{}:    10 obs. of  4 variables:}
\CommentTok{\#\textgreater{}  $ NEV     : chr  "A vallalat" "B vallalat" "C vall...}
\CommentTok{\#\textgreater{}  $ MERET   : num  75 1500 2000 21 1000 900 1000 35 ...}
\CommentTok{\#\textgreater{}  $ HATALMIT: num  1 10 11 3 10 11 10 4 2 5}
\CommentTok{\#\textgreater{}  $ KONZERVA: num  2 9 8 4 9 8 11 3 2 4}
\NormalTok{psych}\SpecialCharTok{::}\FunctionTok{headTail}\NormalTok{(vallalat)}
\CommentTok{\#\textgreater{}            NEV MERET HATALMIT KONZERVA}
\CommentTok{\#\textgreater{} 1   A vallalat    75        1        2}
\CommentTok{\#\textgreater{} 2   B vallalat  1500       10        9}
\CommentTok{\#\textgreater{} 3   C vallalat  2000       11        8}
\CommentTok{\#\textgreater{} 4   D vallalat    21        3        4}
\CommentTok{\#\textgreater{} ...       \textless{}NA\textgreater{}   ...      ...      ...}
\CommentTok{\#\textgreater{} 7   G vallalat  1000       10       11}
\CommentTok{\#\textgreater{} 8   H vallalat    35        4        3}
\CommentTok{\#\textgreater{} 9   I vallalat   120        2        2}
\CommentTok{\#\textgreater{} 10  J vallalat   100        5        4}
\end{Highlighting}
\end{Shaded}

Végezzünk hierarchikus klaszterelemzést. Ehhez először a
távolságmátrixot határozzuk meg. Ehhez első lépésként másoljuk a
sornevekbe a vállalatok nevét, mert akkor a kapott dendrogram levelein a
vállalatok neveit fogjuk látni és így sokkal áttekinthetőbb ábrát fogunk
kapni és könnyebben tudjuk azonosítani az egyes klasztereket.

\begin{Shaded}
\begin{Highlighting}[]
\CommentTok{\# a vállalatnevek a sornevekbe íródnak}
\FunctionTok{rownames}\NormalTok{(vallalat) }\OtherTok{\textless{}{-}}\NormalTok{ vallalat}\SpecialCharTok{$}\NormalTok{NEV}
\end{Highlighting}
\end{Shaded}

Számítsuk ki a távolságmátrixot, használjunk Euklideszi távolságot.

\begin{Shaded}
\begin{Highlighting}[]
\NormalTok{tavolsagmatrix }\OtherTok{\textless{}{-}} \FunctionTok{dist}\NormalTok{(vallalat[}\DecValTok{2}\SpecialCharTok{:}\DecValTok{4}\NormalTok{], }\AttributeTok{method =} \StringTok{"euclidean"}\NormalTok{)}
\FunctionTok{print}\NormalTok{(tavolsagmatrix, }\AttributeTok{digits =} \DecValTok{1}\NormalTok{)}
\CommentTok{\#\textgreater{}            A vallalat B vallalat C vallalat D vallalat}
\CommentTok{\#\textgreater{} B vallalat       1425                                 }
\CommentTok{\#\textgreater{} C vallalat       1925        500                      }
\CommentTok{\#\textgreater{} D vallalat         54       1479       1979           }
\CommentTok{\#\textgreater{} E vallalat        925        500       1000        979}
\CommentTok{\#\textgreater{} F vallalat        825        600       1100        879}
\CommentTok{\#\textgreater{} G vallalat        925        500       1000        979}
\CommentTok{\#\textgreater{} H vallalat         40       1465       1965         14}
\CommentTok{\#\textgreater{} I vallalat         45       1380       1880         99}
\CommentTok{\#\textgreater{} J vallalat         25       1400       1900         79}
\CommentTok{\#\textgreater{}            E vallalat F vallalat G vallalat H vallalat}
\CommentTok{\#\textgreater{} B vallalat                                            }
\CommentTok{\#\textgreater{} C vallalat                                            }
\CommentTok{\#\textgreater{} D vallalat                                            }
\CommentTok{\#\textgreater{} E vallalat                                            }
\CommentTok{\#\textgreater{} F vallalat        100                                 }
\CommentTok{\#\textgreater{} G vallalat          2        100                      }
\CommentTok{\#\textgreater{} H vallalat        965        865        965           }
\CommentTok{\#\textgreater{} I vallalat        880        780        880         85}
\CommentTok{\#\textgreater{} J vallalat        900        800        900         65}
\CommentTok{\#\textgreater{}            I vallalat}
\CommentTok{\#\textgreater{} B vallalat           }
\CommentTok{\#\textgreater{} C vallalat           }
\CommentTok{\#\textgreater{} D vallalat           }
\CommentTok{\#\textgreater{} E vallalat           }
\CommentTok{\#\textgreater{} F vallalat           }
\CommentTok{\#\textgreater{} G vallalat           }
\CommentTok{\#\textgreater{} H vallalat           }
\CommentTok{\#\textgreater{} I vallalat           }
\CommentTok{\#\textgreater{} J vallalat         20}
\end{Highlighting}
\end{Shaded}

A távolságmátrix birtokában már futtathatunk egy klaszteranalízist, az
egyszerű lánc módszert használjuk a klaszterképzéshez.

\begin{Shaded}
\begin{Highlighting}[]
\NormalTok{klaszter }\OtherTok{\textless{}{-}} \FunctionTok{hclust}\NormalTok{(tavolsagmatrix, }\AttributeTok{method =} \StringTok{"single"}\NormalTok{)}
\FunctionTok{plot}\NormalTok{(klaszter)}
\end{Highlighting}
\end{Shaded}

\begin{figure}[H]

{\centering \includegraphics{./sec_klaszter_files/figure-pdf/unnamed-chunk-14-1.pdf}

}

\end{figure}

A fenti dendrogramon látható, hogy alapvetően két nagy csoportja van a
vizsgált vállalatoknak. Az egyikbe tartoznak a \texttt{D}, \texttt{H},
\texttt{A}, \texttt{I} és a \texttt{J} vállalatok, míg a másikba az
\texttt{F}, \texttt{E}, \texttt{G} és egy kicsit távolabb a \texttt{B}
és a \texttt{C}. A \texttt{B} és a \texttt{C} vállalat akár önálló
klasztert is alkothat.

\begin{Shaded}
\begin{Highlighting}[]
\NormalTok{klaszter}\SpecialCharTok{$}\NormalTok{merge  }\CommentTok{\# az összevonások lépései: az egyes lépésekben miket vont össze: negatív szám elem, pozitív klaszter}
\CommentTok{\#\textgreater{}       [,1] [,2]}
\CommentTok{\#\textgreater{}  [1,]   {-}5   {-}7}
\CommentTok{\#\textgreater{}  [2,]   {-}4   {-}8}
\CommentTok{\#\textgreater{}  [3,]   {-}9  {-}10}
\CommentTok{\#\textgreater{}  [4,]   {-}1    3}
\CommentTok{\#\textgreater{}  [5,]    2    4}
\CommentTok{\#\textgreater{}  [6,]   {-}6    1}
\CommentTok{\#\textgreater{}  [7,]   {-}2    6}
\CommentTok{\#\textgreater{}  [8,]   {-}3    7}
\CommentTok{\#\textgreater{}  [9,]    5    8}
\NormalTok{klaszter}\SpecialCharTok{$}\NormalTok{height  }\CommentTok{\# az egyes összevonások milyen távolság esetén történtek meg}
\CommentTok{\#\textgreater{} [1]   2.00000  14.07125  20.32240  25.39685  40.12481}
\CommentTok{\#\textgreater{} [6] 100.01000 500.00000 500.00200 780.07500}
\FunctionTok{cbind}\NormalTok{(klaszter}\SpecialCharTok{$}\NormalTok{merge, klaszter}\SpecialCharTok{$}\NormalTok{height)  }\CommentTok{\# együtt a két fenti információ}
\CommentTok{\#\textgreater{}       [,1] [,2]      [,3]}
\CommentTok{\#\textgreater{}  [1,]   {-}5   {-}7   2.00000}
\CommentTok{\#\textgreater{}  [2,]   {-}4   {-}8  14.07125}
\CommentTok{\#\textgreater{}  [3,]   {-}9  {-}10  20.32240}
\CommentTok{\#\textgreater{}  [4,]   {-}1    3  25.39685}
\CommentTok{\#\textgreater{}  [5,]    2    4  40.12481}
\CommentTok{\#\textgreater{}  [6,]   {-}6    1 100.01000}
\CommentTok{\#\textgreater{}  [7,]   {-}2    6 500.00000}
\CommentTok{\#\textgreater{}  [8,]   {-}3    7 500.00200}
\CommentTok{\#\textgreater{}  [9,]    5    8 780.07500}
\end{Highlighting}
\end{Shaded}

A fenti elemzés jamovi-ban a
\texttt{snowCluster\ /\ Hierarchical\ Clustering} vagy
\texttt{snowCluster\ /\ Clustering\ Dendrogram} menüpontjaival is
elvégezhető.

\begin{figure}

{\centering \includegraphics{./images/klaszter_vallalat_01.jpg}

}

\caption{Vállalatok hierarchikus klaszterelemzése: snowCluster /
Hierarchical Clustering}

\end{figure}

\begin{figure}

{\centering \includegraphics{./images/klaszter_vallalat_02.jpg}

}

\caption{Vállalatok hierarchikus klaszterelemzése: snowCluster /
Clustering Dendrogram}

\end{figure}

\hypertarget{puxe9lda-uxe9tteremluxe1togatuxe1ssal-kapcsolatos-attitux171duxf6k-vizsguxe1lata}{%
\section{Példa: Étteremlátogatással kapcsolatos attitűdök
vizsgálata}\label{puxe9lda-uxe9tteremluxe1togatuxe1ssal-kapcsolatos-attitux171duxf6k-vizsguxe1lata}}

A vásárláshoz hasonlóan az étteremlátogatás is viszonylag megosztja az
embereket. Vannak, akik felesleges kiadásnak tartják, és inkább otthon,
saját maguk főznek. Vannak, akik igyekeznek kímélni magukat az
ilyesfajta házimunkáktól - vagy egyszerűen nem tudnak főzni - és ebből
kifolyólag az éttermek rendszeres látogatói. Megint mások csupán
praktikus okokból járnak étterembe: ünnepek alkalmával, baráti
összejövetelekkor stb. A következőkben a klaszteranalízis segítségével
az étteremlátogatással kapcsolatos attitűdöket fogjuk szemügyre venni.

A vizsgálathoz szükséges adatok a
\texttt{klaszter\_etteremlatogatas.xlsx} állományban találhatok.

\begin{Shaded}
\begin{Highlighting}[]
\NormalTok{etterem }\OtherTok{\textless{}{-}}\NormalTok{ rio}\SpecialCharTok{::}\FunctionTok{import}\NormalTok{(}\AttributeTok{file =} \StringTok{"adat/klaszter\_etteremlatogatas.xlsx"}\NormalTok{)}
\FunctionTok{str}\NormalTok{(etterem)}
\CommentTok{\#\textgreater{} \textquotesingle{}data.frame\textquotesingle{}:    20 obs. of  8 variables:}
\CommentTok{\#\textgreater{}  $ V1: num  6 2 7 4 1 6 5 7 2 3 ...}
\CommentTok{\#\textgreater{}  $ V2: num  4 3 2 6 3 4 3 3 4 5 ...}
\CommentTok{\#\textgreater{}  $ V3: num  7 1 6 4 2 6 6 7 3 3 ...}
\CommentTok{\#\textgreater{}  $ V4: num  3 4 4 5 2 2 3 4 3 6 ...}
\CommentTok{\#\textgreater{}  $ V5: num  2 5 1 3 6 3 3 1 6 4 ...}
\CommentTok{\#\textgreater{}  $ V6: num  4 4 3 6 4 4 4 4 3 6 ...}
\CommentTok{\#\textgreater{}  $ V7: num  2 5 1 3 7 3 3 1 6 4 ...}
\CommentTok{\#\textgreater{}  $ V8: num  5 4 3 6 4 4 5 4 3 6 ...}
\NormalTok{psych}\SpecialCharTok{::}\FunctionTok{headTail}\NormalTok{(etterem)}
\CommentTok{\#\textgreater{}      V1  V2  V3  V4  V5  V6  V7  V8}
\CommentTok{\#\textgreater{} 1     6   4   7   3   2   4   2   5}
\CommentTok{\#\textgreater{} 2     2   3   1   4   5   4   5   4}
\CommentTok{\#\textgreater{} 3     7   2   6   4   1   3   1   3}
\CommentTok{\#\textgreater{} 4     4   6   4   5   3   6   3   6}
\CommentTok{\#\textgreater{} ... ... ... ... ... ... ... ... ...}
\CommentTok{\#\textgreater{} 17    4   4   7   2   2   5   2   5}
\CommentTok{\#\textgreater{} 18    3   7   2   6   4   3   4   3}
\CommentTok{\#\textgreater{} 19    4   6   3   7   2   7   2   7}
\CommentTok{\#\textgreater{} 20    3   3   2   4   7   2   5   3}
\end{Highlighting}
\end{Shaded}

Az egyes itemek a következők:

\begin{itemize}
\tightlist
\item
  V1: Ha csak tehetem, étteremben ebédelek.
\item
  V2: Munkahelyemen szívesen választom a munkahelyi étkezdét.
\item
  V3: Szerintem éttermek nélkül nem is lenne kerek a világ.
\item
  V4: Családi alkalmak, ünnepek esetén szívesen étkezem étteremben.
\item
  V5: Szerintem étteremben étkezni merő pénzpocséklás.
\item
  V6: Időnként szívesen étkezem házon kívül.
\item
  V7: Előnyben részesítem a saját főztömet.
\item
  V8: Szívesen járok korrekt árakkal dolgozó éttermekbe.
\end{itemize}

Végezzünk K-közép klaszterelemzést!

Első lépésként most is a csoporton belüli négyzetösszegeket ábrázoljuk a
lehetséges klaszterszámok függvényében, hogy el tudjuk dönteni, hány
klaszteres megoldás lenne a megfelelő az adatokra.

\begin{Shaded}
\begin{Highlighting}[]
\NormalTok{n }\OtherTok{\textless{}{-}} \FunctionTok{length}\NormalTok{(etterem}\SpecialCharTok{$}\NormalTok{V1)}
\NormalTok{wss1 }\OtherTok{\textless{}{-}}\NormalTok{ (n }\SpecialCharTok{{-}} \DecValTok{1}\NormalTok{) }\SpecialCharTok{*} \FunctionTok{sum}\NormalTok{(}\FunctionTok{apply}\NormalTok{(etterem, }\DecValTok{2}\NormalTok{, var))  }\CommentTok{\# teljes variabilitás}
\NormalTok{wss }\OtherTok{\textless{}{-}} \FunctionTok{numeric}\NormalTok{(}\DecValTok{0}\NormalTok{)}
\CommentTok{\# 2{-}6 klaszteres megoldás kipróbálása}
\ControlFlowTok{for}\NormalTok{ (i }\ControlFlowTok{in} \DecValTok{2}\SpecialCharTok{:}\DecValTok{6}\NormalTok{) \{}
\NormalTok{    W }\OtherTok{\textless{}{-}} \FunctionTok{sum}\NormalTok{(}\FunctionTok{kmeans}\NormalTok{(etterem, i)}\SpecialCharTok{$}\NormalTok{withinss)}
\NormalTok{    wss }\OtherTok{\textless{}{-}} \FunctionTok{c}\NormalTok{(wss, W)}
\NormalTok{\}}
\NormalTok{wss }\OtherTok{\textless{}{-}} \FunctionTok{c}\NormalTok{(wss1, wss)}
\FunctionTok{plot}\NormalTok{(}\DecValTok{1}\SpecialCharTok{:}\DecValTok{6}\NormalTok{, wss, }\AttributeTok{type =} \StringTok{"l"}\NormalTok{, }\AttributeTok{xlab =} \StringTok{"Csoportok száma"}\NormalTok{, }\AttributeTok{ylab =} \StringTok{"Csoporton belüli}
\StringTok{négyzetösszegek"}\NormalTok{,}
    \AttributeTok{lwd =} \DecValTok{2}\NormalTok{)}
\end{Highlighting}
\end{Shaded}

\begin{figure}[H]

{\centering \includegraphics{./sec_klaszter_files/figure-pdf/unnamed-chunk-17-1.pdf}

}

\end{figure}

A fenti ábrán láthatjuk, hogy a hármas értéknél van éles törés a görbén,
ez alapján a háromklaszteres megoldást fogjuk vizsgálni K-középpontú
klaszteranalízissel.

\begin{Shaded}
\begin{Highlighting}[]
\NormalTok{kkozep }\OtherTok{\textless{}{-}} \FunctionTok{kmeans}\NormalTok{(}\AttributeTok{x =}\NormalTok{ etterem, }\AttributeTok{centers =} \DecValTok{3}\NormalTok{)}
\NormalTok{kkozep}
\CommentTok{\#\textgreater{} K{-}means clustering with 3 clusters of sizes 7, 5, 8}
\CommentTok{\#\textgreater{} }
\CommentTok{\#\textgreater{} Cluster means:}
\CommentTok{\#\textgreater{}      V1       V2       V3       V4       V5       V6}
\CommentTok{\#\textgreater{} 1 2.000 3.571429 1.857143 3.857143 5.285714 3.285714}
\CommentTok{\#\textgreater{} 2 3.600 5.600000 3.600000 6.400000 3.400000 6.600000}
\CommentTok{\#\textgreater{} 3 6.125 3.625000 6.000000 3.000000 2.000000 4.000000}
\CommentTok{\#\textgreater{}         V7       V8}
\CommentTok{\#\textgreater{} 1 5.142857 3.428571}
\CommentTok{\#\textgreater{} 2 3.400000 6.600000}
\CommentTok{\#\textgreater{} 3 2.000000 4.250000}
\CommentTok{\#\textgreater{} }
\CommentTok{\#\textgreater{} Clustering vector:}
\CommentTok{\#\textgreater{}  1  2  3  4  5  6  7  8  9 10 11 12 13 14 15 16 17 ...}
\CommentTok{\#\textgreater{}  3  1  3  2  1  3  3  3  1  2  1  3  1  2  3  2  3 ...}
\CommentTok{\#\textgreater{} 20 }
\CommentTok{\#\textgreater{}  1 }
\CommentTok{\#\textgreater{} }
\CommentTok{\#\textgreater{} Within cluster sum of squares by cluster:}
\CommentTok{\#\textgreater{} [1] 52.85714 17.60000 44.25000}
\CommentTok{\#\textgreater{}  (between\_SS / total\_SS =  73.8 \%)}
\CommentTok{\#\textgreater{} }
\CommentTok{\#\textgreater{} Available components:}
\CommentTok{\#\textgreater{} }
\CommentTok{\#\textgreater{} [1] "cluster"      "centers"      "totss"       }
\CommentTok{\#\textgreater{} [4] "withinss"     "tot.withinss" "betweenss"   }
\CommentTok{\#\textgreater{} [7] "size"         "iter"         "ifault"}
\NormalTok{kkozep}\SpecialCharTok{$}\NormalTok{betweenss}
\CommentTok{\#\textgreater{} [1] 323.2429}
\end{Highlighting}
\end{Shaded}

A fenti eredmény a klaszteranalízis eredményét mutatja. Az első sor
(\texttt{K-means\ clustering\ with\ 3\ clusters\ of\ sizes\ 7,5,8})
arról ad információt, hogy háromklaszteres megoldásunk van, melyek
mérete 7, illetve 5 és 8 elemszám. Hogy az egyes elemek melyik
klaszterbe esnek, arról a \texttt{Clustering\ vector} ad információt. Az
első sor az egyes elemeket, a második pedig a csoporttagságot mutatja.

A \texttt{Cluster\ means} az egyes klaszterek átlagos tagjának,
centroidjának a jellemzőit mutatják. Az első klaszter átlagos tagja
kiválóan érzi magát éttermek nélkül is, csak pénzpocséklásnak tartja
azokat és inkább saját maga főz. A második klaszter átlagos alkalomadtán
jár éttermekben (családi ünnepek esetén például), esetleg a munkahelyi
étkezdét használja, előnyben részesíti a mérsékeltebb árakat. Míg a
harmadik klaszter átlagos tagja szívesen jár éttermekbe, nem is nagyon
szeret főzni.

A \texttt{Within\ cluster\ sum\ of\ squares\ by\ cluster} értékei a
klaszteren belüli eltérések négyzetösszegét mutatja.

Az eredmények alapján vannak olyan emberek, akik szeretnek étterembe
járni, az számukra egy életforma, s vannak olyanok, akik az otthoni
konyhát részesítik előnyben. Ugyanakkor vannak megfontoltabb emberek is,
akik igyekeznek energiájukkal takarékoskodni, ezért munkahelyen vagy
valamilyen nagyobb összejövetel esetén szívesen étkeznek házon kívül.

A fenti K-közép klaszterelemzés jamovi-ban is elvégezhető a
\texttt{snowCluster\ /\ K-means\ Clustering} menüponttal.

\begin{figure}

{\centering \includegraphics{./images/klaszter_etterem_01.jpg}

}

\caption{Étteremlátogatás - snowCluster / K-means Clustering}

\end{figure}

\hypertarget{puxe9lda-vuxe1suxe1rluxf3i-attitux171duxf6k-vizsguxe1lata}{%
\section{Példa: Vásárlói attitűdök
vizsgálata}\label{puxe9lda-vuxe1suxe1rluxf3i-attitux171duxf6k-vizsguxe1lata}}

Vásárolni mindenki szokott. Van, akinek szenvedélye a vásárlás, mások,
pedig ha csak lehet, kerülik az üzleteket. Ebben a példában annak fogunk
utánajárni, hogy milyen tipikus vásárlási attitűdök vannak. A
vizsgálathoz szükséges adatokat a
\texttt{klaszter\_vasarloi\_attitudok.xlsx} tartalmazza.

\begin{Shaded}
\begin{Highlighting}[]
\NormalTok{vasarlok }\OtherTok{\textless{}{-}}\NormalTok{ rio}\SpecialCharTok{::}\FunctionTok{import}\NormalTok{(}\AttributeTok{file =} \StringTok{"adat/klaszter\_vasarloi\_attitudok.xlsx"}\NormalTok{)}
\FunctionTok{str}\NormalTok{(vasarlok)}
\CommentTok{\#\textgreater{} \textquotesingle{}data.frame\textquotesingle{}:    20 obs. of  6 variables:}
\CommentTok{\#\textgreater{}  $ V1: num  6 2 7 4 1 6 5 7 2 3 ...}
\CommentTok{\#\textgreater{}  $ V2: num  4 3 2 6 3 4 3 3 4 5 ...}
\CommentTok{\#\textgreater{}  $ V3: num  7 1 6 4 2 6 6 7 3 3 ...}
\CommentTok{\#\textgreater{}  $ V4: num  3 4 4 5 2 2 3 4 3 6 ...}
\CommentTok{\#\textgreater{}  $ V5: num  2 5 1 3 6 3 3 1 6 4 ...}
\CommentTok{\#\textgreater{}  $ V6: num  4 4 3 6 4 4 4 4 3 6 ...}
\NormalTok{psych}\SpecialCharTok{::}\FunctionTok{headTail}\NormalTok{(vasarlok)}
\CommentTok{\#\textgreater{}      V1  V2  V3  V4  V5  V6}
\CommentTok{\#\textgreater{} 1     6   4   7   3   2   4}
\CommentTok{\#\textgreater{} 2     2   3   1   4   5   4}
\CommentTok{\#\textgreater{} 3     7   2   6   4   1   3}
\CommentTok{\#\textgreater{} 4     4   6   4   5   3   6}
\CommentTok{\#\textgreater{} ... ... ... ... ... ... ...}
\CommentTok{\#\textgreater{} 17    4   4   7   2   2   5}
\CommentTok{\#\textgreater{} 18    3   7   2   6   4   3}
\CommentTok{\#\textgreater{} 19    4   6   3   7   2   7}
\CommentTok{\#\textgreater{} 20    3   3   2   4   7   2}
\end{Highlighting}
\end{Shaded}

A fenti outputban lévő változók jelentése a következő:

\begin{itemize}
\tightlist
\item
  V1: Általában igyekszem diszkont áruházakban vásárolni.
\item
  V2: Imádok vásárolgatni.
\item
  V3: Mindig figyelem az árleszállításokat.
\item
  V4: A vásárlás számomra szinte egy hobbi.
\item
  V5: Ha csak tehetem, nem én vásárolok.
\item
  V6: Szívesen járom az üzleteket baráti társaságban.
\end{itemize}

Első lépésként a csoporton belüli négyzetösszegeket ábrázoljuk a
lehetséges klaszterszámok függvényében, hogy el tudjuk dönteni, hány
klaszteres megoldás lenne a megfelelő az adatokra.

\begin{Shaded}
\begin{Highlighting}[]
\NormalTok{n }\OtherTok{\textless{}{-}} \FunctionTok{length}\NormalTok{(vasarlok}\SpecialCharTok{$}\NormalTok{V1)}
\NormalTok{wss1 }\OtherTok{\textless{}{-}}\NormalTok{ (n }\SpecialCharTok{{-}} \DecValTok{1}\NormalTok{) }\SpecialCharTok{*} \FunctionTok{sum}\NormalTok{(}\FunctionTok{apply}\NormalTok{(vasarlok, }\DecValTok{2}\NormalTok{, var))}
\NormalTok{wss }\OtherTok{\textless{}{-}} \FunctionTok{numeric}\NormalTok{(}\DecValTok{0}\NormalTok{)}
\ControlFlowTok{for}\NormalTok{ (i }\ControlFlowTok{in} \DecValTok{2}\SpecialCharTok{:}\DecValTok{6}\NormalTok{) \{}
\NormalTok{    W }\OtherTok{\textless{}{-}} \FunctionTok{sum}\NormalTok{(}\FunctionTok{kmeans}\NormalTok{(vasarlok, i)}\SpecialCharTok{$}\NormalTok{withinss)}
\NormalTok{    wss }\OtherTok{\textless{}{-}} \FunctionTok{c}\NormalTok{(wss, W)}
\NormalTok{\}}
\NormalTok{wss }\OtherTok{\textless{}{-}} \FunctionTok{c}\NormalTok{(wss1, wss)}
\FunctionTok{plot}\NormalTok{(}\DecValTok{1}\SpecialCharTok{:}\DecValTok{6}\NormalTok{, wss, }\AttributeTok{type =} \StringTok{"l"}\NormalTok{, }\AttributeTok{xlab =} \StringTok{"Csoportok száma"}\NormalTok{, }\AttributeTok{ylab =} \StringTok{"Csoporton belüli}
\StringTok{négyzetösszegek"}\NormalTok{,}
    \AttributeTok{lwd =} \DecValTok{2}\NormalTok{)}
\end{Highlighting}
\end{Shaded}

\begin{figure}[H]

{\centering \includegraphics{./sec_klaszter_files/figure-pdf/unnamed-chunk-20-1.pdf}

}

\end{figure}

A fenti képen láthatjuk, hogy a hármas értéknél van törés a görbén, ez
alapján a háromklaszteres megoldást fogjuk vizsgálni K-középpontú
klaszteranalízissel.

\begin{Shaded}
\begin{Highlighting}[]
\NormalTok{kkozep }\OtherTok{\textless{}{-}} \FunctionTok{kmeans}\NormalTok{(vasarlok, }\DecValTok{3}\NormalTok{)}
\NormalTok{kkozep}
\CommentTok{\#\textgreater{} K{-}means clustering with 3 clusters of sizes 6, 8, 6}
\CommentTok{\#\textgreater{} }
\CommentTok{\#\textgreater{} Cluster means:}
\CommentTok{\#\textgreater{}         V1       V2       V3       V4  V5       V6}
\CommentTok{\#\textgreater{} 1 3.500000 5.833333 3.333333 6.333333 3.5 6.000000}
\CommentTok{\#\textgreater{} 2 6.125000 3.625000 6.000000 3.000000 2.0 4.000000}
\CommentTok{\#\textgreater{} 3 1.833333 3.000000 1.833333 3.500000 5.5 3.333333}
\CommentTok{\#\textgreater{} }
\CommentTok{\#\textgreater{} Clustering vector:}
\CommentTok{\#\textgreater{}  1  2  3  4  5  6  7  8  9 10 11 12 13 14 15 16 17 ...}
\CommentTok{\#\textgreater{}  2  3  2  1  3  2  2  2  3  1  3  2  3  1  2  1  2 ...}
\CommentTok{\#\textgreater{} 20 }
\CommentTok{\#\textgreater{}  3 }
\CommentTok{\#\textgreater{} }
\CommentTok{\#\textgreater{} Within cluster sum of squares by cluster:}
\CommentTok{\#\textgreater{} [1] 28.50 36.75 22.00}
\CommentTok{\#\textgreater{}  (between\_SS / total\_SS =  75.1 \%)}
\CommentTok{\#\textgreater{} }
\CommentTok{\#\textgreater{} Available components:}
\CommentTok{\#\textgreater{} }
\CommentTok{\#\textgreater{} [1] "cluster"      "centers"      "totss"       }
\CommentTok{\#\textgreater{} [4] "withinss"     "tot.withinss" "betweenss"   }
\CommentTok{\#\textgreater{} [7] "size"         "iter"         "ifault"}
\NormalTok{kkozep}\SpecialCharTok{$}\NormalTok{totss}
\CommentTok{\#\textgreater{} [1] 350.05}
\end{Highlighting}
\end{Shaded}

A fenti output a klaszteranalízis eredményét mutatja. Az első sor
(\texttt{K-means\ clustering\ with\ 3\ clusters\ of\ sizes\ 8,6,6})
arról ad információt, hogy háromklaszteres megoldásunk van, melyek
mérete 8, illetve 6, 6 elemszám. Hogy az egyes elemek melyik klaszterbe
esnek, arról a \texttt{Clustering\ vector} ad információt. Az első sor
az egyes elemeket, a második pedig a csoporttagságot mutatja.

A \texttt{Cluster\ means} az egyes klaszterek átlagos tagjának,
centroidjának a jellemzőit mutatják. Az első klaszter átlagos tagja
igyekszik diszkontáruházakban és árleszállításokon vásárolni, minél több
pénzt megtakarítani. A második klaszter átlagos tagja ha csak teheti,
másokkal vásároltat be. Míg a harmadik klaszter átlagos tagja
szenvedélyes vásárló, baráti társaságokkal is szívesen járja az
üzleteket.

A \texttt{Within\ cluster\ sum\ of\ squares\ by\ cluster} értékei a
klaszteren belüli eltérések négyzetösszegét mutatja.

Az eredmények alapján vannak olyan emberek, akik nem szeretnek
vásárolni, s vannak olyanok, akiknek egyfajta hobbi a vásárlás.
Ugyanakkor vannak megfontoltabb emberek is, akik igyekeznek
takarékossági szempontokat is figyelembe venni, és minél olcsóbban
elintézni a bevásárlásokat.

A fenti K-közép klaszterelemzés jamovi-ban is elvégezhető a
\texttt{snowCluster\ /\ K-means\ Clustering} menüponttal.

\begin{figure}

{\centering \includegraphics{./images/klaszter_vasarloi_attitudok_01.jpg}

}

\caption{Vásárlói attitűdök - snowCluster / K-means Clustering}

\end{figure}

\hypertarget{puxe9lda-csokoluxe1duxe9muxe1rkuxe1k-vizsguxe1lata}{%
\section{Példa: Csokoládémárkák
vizsgálata}\label{puxe9lda-csokoluxe1duxe9muxe1rkuxe1k-vizsguxe1lata}}

A klaszteranalízissel elemzett jelen problémában csokoládémárkákat
vizsgálunk. Összesen tíz csokimárkát ítéltek meg a személyek a csoki
nagysága, krémességének és töménységének tekintetében. Azt fogjuk
megvizsgálni, hogy mely csokoládék állnak a vizsgálati személyek szerint
közel egymáshoz. Ennek ismerete hasznos lehet marketing szempontból -
például mely csokoládékat érdemes közel tenni egymáshoz a polcon.

Az adatokat a \texttt{klaszter\_csokolademarkak.xlsx} tartalmazza.

\begin{Shaded}
\begin{Highlighting}[]
\NormalTok{csokolade }\OtherTok{\textless{}{-}}\NormalTok{ rio}\SpecialCharTok{::}\FunctionTok{import}\NormalTok{(}\AttributeTok{file =} \StringTok{"adat/klaszter\_csokolademarkak.xlsx"}\NormalTok{)}
\FunctionTok{str}\NormalTok{(csokolade)}
\CommentTok{\#\textgreater{} \textquotesingle{}data.frame\textquotesingle{}:    10 obs. of  4 variables:}
\CommentTok{\#\textgreater{}  $ CSOKI  : chr  "Boci" "Milka" "Tibi" "Balaton" ...}
\CommentTok{\#\textgreater{}  $ NAGYSAG: num  10 10 9 4 4 3 2 5 5 4}
\CommentTok{\#\textgreater{}  $ KREMES : num  9 10 8 5 2 7 5 7 7 8}
\CommentTok{\#\textgreater{}  $ TOMENY : num  7 8 6 6 4 7 7 8 9 9}
\NormalTok{psych}\SpecialCharTok{::}\FunctionTok{headTail}\NormalTok{(csokolade)}
\CommentTok{\#\textgreater{}         CSOKI NAGYSAG KREMES TOMENY}
\CommentTok{\#\textgreater{} 1        Boci      10      9      7}
\CommentTok{\#\textgreater{} 2       Milka      10     10      8}
\CommentTok{\#\textgreater{} 3        Tibi       9      8      6}
\CommentTok{\#\textgreater{} 4     Balaton       4      5      6}
\CommentTok{\#\textgreater{} ...      \textless{}NA\textgreater{}     ...    ...    ...}
\CommentTok{\#\textgreater{} 7   Kapuciner       2      5      7}
\CommentTok{\#\textgreater{} 8        Mars       5      7      8}
\CommentTok{\#\textgreater{} 9    Snickers       5      7      9}
\CommentTok{\#\textgreater{} 10  Sportszel       4      8      9}
\end{Highlighting}
\end{Shaded}

Most sem távolságmátrixszal dolgozunk, hanem az „eredeti'' változókkal.
Az adatokból számított távolságmátrixot természetesen elkészíthetjük az
R segítségével is. Ehhez első lépésként a \texttt{CSOKI} változót
másoljuk át az adatbázis sorneveibe, mert így a kapott dendrogram
levelein a csokoládék neveit fogjuk látni, nem pedig számokat. Ezáltal
sokkal áttekinthetőbb ábrát fogunk kapni és könnyebben tudjuk
azonosítani az egyes klasztereket.

\begin{Shaded}
\begin{Highlighting}[]
\FunctionTok{rownames}\NormalTok{(csokolade) }\OtherTok{\textless{}{-}}\NormalTok{ csokolade}\SpecialCharTok{$}\NormalTok{CSOKI}
\end{Highlighting}
\end{Shaded}

Ezután kiszámíttathatjuk a távolságmátrixot az adatokra, használjuk az
Euklideszi távolságot.

\begin{Shaded}
\begin{Highlighting}[]
\NormalTok{tavolsagmatrix }\OtherTok{\textless{}{-}} \FunctionTok{dist}\NormalTok{(csokolade[}\DecValTok{2}\SpecialCharTok{:}\DecValTok{4}\NormalTok{])}
\FunctionTok{print}\NormalTok{(tavolsagmatrix, }\AttributeTok{digits =} \DecValTok{1}\NormalTok{)}
\CommentTok{\#\textgreater{}           Boci Milka Tibi Balaton Müzliszel 3{-}bit}
\CommentTok{\#\textgreater{} Milka        1                                   }
\CommentTok{\#\textgreater{} Tibi         2     3                             }
\CommentTok{\#\textgreater{} Balaton      7     8    6                        }
\CommentTok{\#\textgreater{} Müzliszel   10    11    8       4                }
\CommentTok{\#\textgreater{} 3{-}bit        7     8    6       2         6      }
\CommentTok{\#\textgreater{} Kapuciner    9     9    8       2         5     2}
\CommentTok{\#\textgreater{} Mars         5     6    5       3         6     2}
\CommentTok{\#\textgreater{} Snickers     6     6    5       4         7     3}
\CommentTok{\#\textgreater{} Sportszel    6     6    6       4         8     2}
\CommentTok{\#\textgreater{}           Kapuciner Mars Snickers}
\CommentTok{\#\textgreater{} Milka                            }
\CommentTok{\#\textgreater{} Tibi                             }
\CommentTok{\#\textgreater{} Balaton                          }
\CommentTok{\#\textgreater{} Müzliszel                        }
\CommentTok{\#\textgreater{} 3{-}bit                            }
\CommentTok{\#\textgreater{} Kapuciner                        }
\CommentTok{\#\textgreater{} Mars              4              }
\CommentTok{\#\textgreater{} Snickers          4    1         }
\CommentTok{\#\textgreater{} Sportszel         4    2        1}
\end{Highlighting}
\end{Shaded}

A távolságmátrix birtokában már futtathatunk egy klaszteranalízist, az
egyszerű lánc módszert használva a klaszterképzéshez.

\begin{Shaded}
\begin{Highlighting}[]
\NormalTok{klaszter }\OtherTok{\textless{}{-}} \FunctionTok{hclust}\NormalTok{(tavolsagmatrix, }\AttributeTok{method =} \StringTok{"single"}\NormalTok{)}
\FunctionTok{plot}\NormalTok{(klaszter)}
\end{Highlighting}
\end{Shaded}

\begin{figure}[H]

{\centering \includegraphics{./sec_klaszter_files/figure-pdf/unnamed-chunk-25-1.pdf}

}

\end{figure}

\begin{Shaded}
\begin{Highlighting}[]
\NormalTok{klaszter}\SpecialCharTok{$}\NormalTok{merge}
\CommentTok{\#\textgreater{}       [,1] [,2]}
\CommentTok{\#\textgreater{}  [1,]   {-}8   {-}9}
\CommentTok{\#\textgreater{}  [2,]   {-}1   {-}2}
\CommentTok{\#\textgreater{}  [3,]  {-}10    1}
\CommentTok{\#\textgreater{}  [4,]   {-}3    2}
\CommentTok{\#\textgreater{}  [5,]   {-}4   {-}7}
\CommentTok{\#\textgreater{}  [6,]   {-}6    5}
\CommentTok{\#\textgreater{}  [7,]    3    6}
\CommentTok{\#\textgreater{}  [8,]   {-}5    7}
\CommentTok{\#\textgreater{}  [9,]    4    8}
\NormalTok{klaszter}\SpecialCharTok{$}\NormalTok{height}
\CommentTok{\#\textgreater{} [1] 1.000000 1.414214 1.414214 1.732051 2.236068 2....}
\CommentTok{\#\textgreater{} [7] 2.236068 3.605551 4.582576}
\end{Highlighting}
\end{Shaded}

Láthatjuk, hogy a klaszterek kialakítása 9 lépésben történt és legutolsó
elemet 4,58-as távolságnál vontuk be a klaszterbe.

A dendrogram látható, hogy alapvetően két nagy csoportja van a vizsgált
csokoládéknak. Az egyikbe tartoznak a táblás csokoládék (Tibi, Milka és
a Boci), míg a másikba a szeletes csokik. Az utóbbiba vonta be a módszer
a müzliszeletet is, bár meglehetősen távol van a többi csokoládétól.

A fenti elemzés jamovi-ban a
\texttt{snowCluster\ /\ Hierarchical\ Clustering} vagy
\texttt{snowCluster\ /\ Clustering\ Dendrogram} menüpontjaival is
elvégezhető.

\begin{figure}

{\centering \includegraphics{./images/klaszter_csokolade_01.jpg}

}

\caption{Csokoládémárkák hierarchikus klaszterelemzése: snowCluster /
Hierarchical Clustering}

\end{figure}

\begin{figure}

{\centering \includegraphics{./images/klaszter_csokolade_02.jpg}

}

\caption{Csokoládémárkák hierarchikus klaszterelemzése: snowCluster /
Clustering Dendrogram}

\end{figure}

\bookmarksetup{startatroot}

\hypertarget{sec-diszkriminancia-elemzes}{%
\chapter{Diszkriminancia elemzés}\label{sec-diszkriminancia-elemzes}}

A diszkriminancia analízisben azt a problémát járjuk körül, hogyan lehet
az emberek egyes csoportjait valamilyen vizsgált jellemzők alapján
szétválasztani, az egyes csoportokat azonosítani, valamint a
csoporttagságokat az előbb említett vizsgált jellemzők alapján
előrejelezni.

Képzeljünk el, hogy szalagmunkásokkal végeznek alkalmasság-vizsgálatot.
A szalagmunka általában sok figyelmet igényel, ugyanakkor meglehetősen
monoton munka, éppen ezért jó figyelmi képességek és monotónia tűrés
szükséges hozzá. A lenti \texttt{szalagmunka} adatmátrix 10 személy
adatát tartalmazza.

\begin{Shaded}
\begin{Highlighting}[]
\NormalTok{szalagmunka }\OtherTok{\textless{}{-}}\NormalTok{ rio}\SpecialCharTok{::}\FunctionTok{import}\NormalTok{(}\AttributeTok{file =} \StringTok{"adat/diszkriminancia\_alkalmassag.xlsx"}\NormalTok{)}
\NormalTok{szalagmunka}\SpecialCharTok{$}\NormalTok{bevalt }\OtherTok{\textless{}{-}} \FunctionTok{factor}\NormalTok{(szalagmunka}\SpecialCharTok{$}\NormalTok{bevalt, }\AttributeTok{labels =} \FunctionTok{c}\NormalTok{(}\StringTok{"nem"}\NormalTok{,}
    \StringTok{"igen"}\NormalTok{))}
\NormalTok{szalagmunka}
\CommentTok{\#\textgreater{}    bevalt figyelem monotonia\_tures}
\CommentTok{\#\textgreater{} 1    igen        1               2}
\CommentTok{\#\textgreater{} 2    igen        1               5}
\CommentTok{\#\textgreater{} 3     nem        2               1}
\CommentTok{\#\textgreater{} 4    igen        2               3}
\CommentTok{\#\textgreater{} 5     nem        3               2}
\CommentTok{\#\textgreater{} 6    igen        3               4}
\CommentTok{\#\textgreater{} 7     nem        4               3}
\CommentTok{\#\textgreater{} 8     nem        4               1}
\CommentTok{\#\textgreater{} 9    igen        4               6}
\CommentTok{\#\textgreater{} 10    nem        6               5}
\end{Highlighting}
\end{Shaded}

A fenti adatok egy részét a szalagmunkára való jelentkezéskor
gyűjtötték:

\begin{itemize}
\tightlist
\item
  \texttt{figyelem}: figyelmi képességükre és
\item
  \texttt{monotonia\_tures}: monotónia-tűrésükre vonatkozó információk.
\end{itemize}

Mindkét változót 7 fokú skálán értékeltek (ahol a magasabb érték jobb
képességeket jelent), valamint azt az információt is láthatjuk, hogy
később beváltak-e vagy sem (\texttt{bevalt}).

Azt kellene megmutatnunk, hogy a figyelem és a monotónia-tűrés
pontszámai alapján valóban lehet következtetést levonni a személy
beválását illetően. Ha ezt sikerül egy objektív statisztikai módszerrel
is igazolnunk, akkor az ezt követően szalagmunkára jelentkezőket
figyelem és monotónia tűréssel vizsgálva tesztelhetjük, és egész jól ki
lehet válogatni az alkalmasabb jelölteket.

Ha pontdiagramon ábrázoljuk az adatokat, és színezéssel jelöljük a
beválást, akkor a két csoport szemmel láthatóan szétválik egymástól, ám
sem a függőleges, sem a vízszintes tengely mentén nem lehet elkülöníteni
a csoportokat.

\begin{Shaded}
\begin{Highlighting}[]
\FunctionTok{library}\NormalTok{(ggplot2)}
\FunctionTok{ggplot}\NormalTok{(szalagmunka, }\FunctionTok{aes}\NormalTok{(}\AttributeTok{x =}\NormalTok{ figyelem, }\AttributeTok{y =}\NormalTok{ monotonia\_tures, }\AttributeTok{colour =}\NormalTok{ bevalt)) }\SpecialCharTok{+}
    \FunctionTok{geom\_point}\NormalTok{(}\AttributeTok{size =} \DecValTok{4}\NormalTok{)}
\end{Highlighting}
\end{Shaded}

\begin{figure}[H]

{\centering \includegraphics{./sec_diszkrimninancia_files/figure-pdf/unnamed-chunk-3-1.pdf}

}

\end{figure}

\textbf{A diszkriminancia elemzés sajátossága, hogy a csoportokat a
magyarázó változók együttes figyelembevételével tudja szétválasztani.}
Ennek megfelelően ha önmagában tekintjük az egyik (például
\texttt{figyelem}) vagy másik (\texttt{monotonia\_tures}) magyarázó
változókat, akkor nem tudunk szignifikáns különbséget kimutatni a
\texttt{bevalt} változó két csoportjában \((p = 0,108; p=0,159)\).

\begin{Shaded}
\begin{Highlighting}[]
\FunctionTok{t.test}\NormalTok{(figyelem }\SpecialCharTok{\textasciitilde{}}\NormalTok{ bevalt, }\AttributeTok{data =}\NormalTok{ szalagmunka)}\SpecialCharTok{$}\NormalTok{p.value}
\CommentTok{\#\textgreater{} [1] 0.1082333}
\FunctionTok{t.test}\NormalTok{(monotonia\_tures }\SpecialCharTok{\textasciitilde{}}\NormalTok{ bevalt, }\AttributeTok{data =}\NormalTok{ szalagmunka)}\SpecialCharTok{$}\NormalTok{p.value}
\CommentTok{\#\textgreater{} [1] 0.1588974}
\end{Highlighting}
\end{Shaded}

Nézzük meg, hogy többváltozós variancia-analízissel (MANOVA) szét
tudjuk-e választani a csoportokat, amikor a két magyarázó változót
egyszerre vesszük figyelembe. Továbbra is arra keressük a választ, hogy
beválás tekintetében valóban létezik-e a munkások két csoportja.

\begin{Shaded}
\begin{Highlighting}[]
\NormalTok{man\_1 }\OtherTok{\textless{}{-}} \FunctionTok{manova}\NormalTok{(}\FunctionTok{cbind}\NormalTok{(figyelem, monotonia\_tures) }\SpecialCharTok{\textasciitilde{}}\NormalTok{ bevalt, }\AttributeTok{data =}\NormalTok{ szalagmunka)}
\FunctionTok{summary}\NormalTok{(man\_1, }\AttributeTok{test =} \StringTok{"Wilks"}\NormalTok{)}
\CommentTok{\#\textgreater{}           Df  Wilks approx F num Df den Df  Pr(\textgreater{}F)  }
\CommentTok{\#\textgreater{} bevalt     1 0.2708   9.4247      2      7 0.01033 *}
\CommentTok{\#\textgreater{} Residuals  8                                        }
\CommentTok{\#\textgreater{} {-}{-}{-}}
\CommentTok{\#\textgreater{} Signif. codes:  }
\CommentTok{\#\textgreater{} 0 \textquotesingle{}***\textquotesingle{} 0.001 \textquotesingle{}**\textquotesingle{} 0.01 \textquotesingle{}*\textquotesingle{} 0.05 \textquotesingle{}.\textquotesingle{} 0.1 \textquotesingle{} \textquotesingle{} 1}
\end{Highlighting}
\end{Shaded}

A fenti output alapján megnyugodhatunk, a bevált és a nem bevált
munkások csoportja valóban különbözik egymástól \((p=0,010)\).

A diszkriminancia elemzéstől azonban ettől többet fogunk várni, például
predikciót is végezhetünk a kapott modellben.

Ha diszkriminancia elemzést szeretnénk végrehajtani, akkor a
\texttt{\{MASS\}} csomag \texttt{lda()} függvényét használhatjuk.

\begin{Shaded}
\begin{Highlighting}[]
\NormalTok{lda\_1 }\OtherTok{\textless{}{-}}\NormalTok{ MASS}\SpecialCharTok{::}\FunctionTok{lda}\NormalTok{(bevalt }\SpecialCharTok{\textasciitilde{}}\NormalTok{ figyelem }\SpecialCharTok{+}\NormalTok{ monotonia\_tures, }\AttributeTok{data =}\NormalTok{ szalagmunka)}
\NormalTok{lda\_1}
\CommentTok{\#\textgreater{} Call:}
\CommentTok{\#\textgreater{} lda(bevalt \textasciitilde{} figyelem + monotonia\_tures, data = sza...}
\CommentTok{\#\textgreater{} }
\CommentTok{\#\textgreater{} Prior probabilities of groups:}
\CommentTok{\#\textgreater{}  nem igen }
\CommentTok{\#\textgreater{}  0.5  0.5 }
\CommentTok{\#\textgreater{} }
\CommentTok{\#\textgreater{} Group means:}
\CommentTok{\#\textgreater{}      figyelem monotonia\_tures}
\CommentTok{\#\textgreater{} nem       3.8             2.4}
\CommentTok{\#\textgreater{} igen      2.2             4.0}
\CommentTok{\#\textgreater{} }
\CommentTok{\#\textgreater{} Coefficients of linear discriminants:}
\CommentTok{\#\textgreater{}                        LD1}
\CommentTok{\#\textgreater{} figyelem        {-}0.9981001}
\CommentTok{\#\textgreater{} monotonia\_tures  0.8365579}
\end{Highlighting}
\end{Shaded}

A fenti outputban látható, hogy akik nem váltak be, a monotónia-tűrés
tesztben gyengébb teljesítményt nyújtottak, a figyelem tesztben pedig
egy jobbat, míg akik beváltak, a monotónia-tűrés tesztben igen jó
teljesítményt értek el, a figyelem tesztben pedig valamivel gyengébbet.
A kanonikus diszkriminancia együtthatókat is láthatjuk, melyek alapján
felírhatjuk a kanonikus diszkriminancia-függvényt a következő módon:

\begin{Shaded}
\begin{Highlighting}[]
\NormalTok{Z = {-}0,998 * figyelem + 0,837 * monotonia\_tures}
\end{Highlighting}
\end{Shaded}

Az eddig vizsgált 10 személyről tudjuk, hogy bevált-e vagy sem, vagyis
ismertük a tényleges csoporttagságát. Ám a diszkriminancia elemzés
fontos célja, hogy előre jelezzük a csoporttagságokat, vagyis a figyelem
és monotónia-tűrés ismeretében megmondjuk egy személyről, hogy nagy
valószínűséggel beválik-e vagy sem.

Tegyük fel, hogy az első személy a figyelem teszten 2, míg a
monotónia-tűrés teszten 4 pontot kapott, míg a második személy
pontszámai ebben a sorrendben 6 és 1.

\begin{Shaded}
\begin{Highlighting}[]
\NormalTok{newdata }\OtherTok{\textless{}{-}} \FunctionTok{data.frame}\NormalTok{(}\AttributeTok{figyelem =} \FunctionTok{c}\NormalTok{(}\DecValTok{2}\NormalTok{, }\DecValTok{6}\NormalTok{), }\AttributeTok{monotonia\_tures =} \FunctionTok{c}\NormalTok{(}\DecValTok{4}\NormalTok{,}
    \DecValTok{1}\NormalTok{))}
\NormalTok{newdata}
\CommentTok{\#\textgreater{}   figyelem monotonia\_tures}
\CommentTok{\#\textgreater{} 1        2               4}
\CommentTok{\#\textgreater{} 2        6               1}
\NormalTok{lda\_1\_pred }\OtherTok{\textless{}{-}} \FunctionTok{predict}\NormalTok{(lda\_1, }\AttributeTok{newdata =}\NormalTok{ newdata)}
\NormalTok{lda\_1\_pred}
\CommentTok{\#\textgreater{} $class}
\CommentTok{\#\textgreater{} [1] igen nem }
\CommentTok{\#\textgreater{} Levels: nem igen}
\CommentTok{\#\textgreater{} }
\CommentTok{\#\textgreater{} $posterior}
\CommentTok{\#\textgreater{}          nem         igen}
\CommentTok{\#\textgreater{} 1 0.00743262 9.925674e{-}01}
\CommentTok{\#\textgreater{} 2 0.99999931 6.861874e{-}07}
\CommentTok{\#\textgreater{} }
\CommentTok{\#\textgreater{} $x}
\CommentTok{\#\textgreater{}         LD1}
\CommentTok{\#\textgreater{} 1  1.667346}
\CommentTok{\#\textgreater{} 2 {-}4.834728}
\end{Highlighting}
\end{Shaded}

A fenti output \texttt{class} és \texttt{posterior} része alapján
láthatjuk, hogy az első személyt nagy valószínűséggel alkalmasnak, míg a
másodikat alkalmatlannak ítélhetjük a szalagmunkára.

Utolsó lépésként összevetjük a tényleges és a becsült csoporttagságot,
és megállapítjuk, az adatok mekkora részét tudjuk helyesen besorolni az
alkotott modell alapján. Ezzel magát a modellt értékeljük.

\begin{Shaded}
\begin{Highlighting}[]
\NormalTok{lda\_2\_pred }\OtherTok{\textless{}{-}} \FunctionTok{predict}\NormalTok{(lda\_1, }\AttributeTok{method =} \StringTok{"plug{-}in"}\NormalTok{)}
\NormalTok{tab\_1 }\OtherTok{\textless{}{-}} \FunctionTok{table}\NormalTok{(lda\_2\_pred}\SpecialCharTok{$}\NormalTok{class, szalagmunka}\SpecialCharTok{$}\NormalTok{bevalt)}
\NormalTok{tab\_1}
\CommentTok{\#\textgreater{}       }
\CommentTok{\#\textgreater{}        nem igen}
\CommentTok{\#\textgreater{}   nem    5    0}
\CommentTok{\#\textgreater{}   igen   0    5}
\end{Highlighting}
\end{Shaded}

Mivel az összes adat a főátlóban van, így megállapíthatjuk, hogy a
modell alapján az összes adatot helyesen kategorizáltuk. A helyes
besorolás arányát százalékosan is kiszámíthatjuk, ez az arány 100\%.

\begin{Shaded}
\begin{Highlighting}[]
\DecValTok{100} \SpecialCharTok{*} \FunctionTok{sum}\NormalTok{(}\FunctionTok{diag}\NormalTok{(tab\_1))}\SpecialCharTok{/}\FunctionTok{sum}\NormalTok{(tab\_1)}
\CommentTok{\#\textgreater{} [1] 100}
\end{Highlighting}
\end{Shaded}

\hypertarget{puxe9lda-kikbux151l-lesznek-a-balesetezux151k}{%
\section{Példa: Kikből lesznek a
balesetezők?}\label{puxe9lda-kikbux151l-lesznek-a-balesetezux151k}}

Ebben a példában azt vizsgáljuk, mely tényezők járulnak hozzá a
balesetekhez.

\begin{Shaded}
\begin{Highlighting}[]
\NormalTok{baleset }\OtherTok{\textless{}{-}}\NormalTok{ rio}\SpecialCharTok{::}\FunctionTok{import}\NormalTok{(}\AttributeTok{file =} \StringTok{"adat/diszkriminancia\_baleset.xlsx"}\NormalTok{)}
\NormalTok{baleset}\SpecialCharTok{$}\NormalTok{baleset }\OtherTok{\textless{}{-}} \FunctionTok{factor}\NormalTok{(baleset}\SpecialCharTok{$}\NormalTok{baleset, }\AttributeTok{labels =} \FunctionTok{c}\NormalTok{(}\StringTok{"nem volt baleste"}\NormalTok{,}
    \StringTok{"volt baleste"}\NormalTok{))}
\FunctionTok{str}\NormalTok{(baleset)}
\CommentTok{\#\textgreater{} \textquotesingle{}data.frame\textquotesingle{}:    36 obs. of  5 variables:}
\CommentTok{\#\textgreater{}  $ baleset   : Factor w/ 2 levels "nem volt baleste...}
\CommentTok{\#\textgreater{}  $ megosztott: num  7 6 5 6 7 3 6 7 5 6 ...}
\CommentTok{\#\textgreater{}  $ pontossag : num  6 6 5 6 7 3 5 7 5 6 ...}
\CommentTok{\#\textgreater{}  $ kockazat  : num  2 3 1 2 4 7 2 1 3 2 ...}
\CommentTok{\#\textgreater{}  $ eszleles  : num  7 6 5 6 7 7 7 6 3 7 ...}
\NormalTok{psych}\SpecialCharTok{::}\FunctionTok{headTail}\NormalTok{(baleset)}
\CommentTok{\#\textgreater{}              baleset megosztott pontossag kockazat}
\CommentTok{\#\textgreater{} 1   nem volt baleste          7         6        2}
\CommentTok{\#\textgreater{} 2   nem volt baleste          6         6        3}
\CommentTok{\#\textgreater{} 3   nem volt baleste          5         5        1}
\CommentTok{\#\textgreater{} 4   nem volt baleste          6         6        2}
\CommentTok{\#\textgreater{} ...             \textless{}NA\textgreater{}        ...       ...      ...}
\CommentTok{\#\textgreater{} 33      volt baleste          3         3        5}
\CommentTok{\#\textgreater{} 34      volt baleste          2         2        7}
\CommentTok{\#\textgreater{} 35      volt baleste          3         3        4}
\CommentTok{\#\textgreater{} 36      volt baleste          4         4        6}
\CommentTok{\#\textgreater{}     eszleles}
\CommentTok{\#\textgreater{} 1          7}
\CommentTok{\#\textgreater{} 2          6}
\CommentTok{\#\textgreater{} 3          5}
\CommentTok{\#\textgreater{} 4          6}
\CommentTok{\#\textgreater{} ...      ...}
\CommentTok{\#\textgreater{} 33         4}
\CommentTok{\#\textgreater{} 34         1}
\CommentTok{\#\textgreater{} 35         4}
\CommentTok{\#\textgreater{} 36         4}
\end{Highlighting}
\end{Shaded}

Az adatbázisban a \texttt{baleset} változó azt rögzíti, hogy volt-e már
balesete a személynek vagy sem. Ez lesz tehát a csoportosító változó. A
többi változó, melyek segítségével próbáljuk a csoportok közötti
különbséget jellemezni, olyan dolgot mérnek, mint a megosztott figyelem
(\texttt{megosztott} változó), a figyelem pontossága
(\texttt{pontossag}), kockázatvállalási hajlandóság (\texttt{kockazat})
és az észlelés gyorsasága (\texttt{eszleles}).

A diszkriminancia-analízisben az első lépés annak megállapítása, vajon
valóban szét lehet-e választani a balesetezők és a nem balesetezők
csoportját az adott változók alapján. Ehhez a Wilks-lambda tesztet
használjuk a többváltozós variancia-analízis keretein belül.

\begin{Shaded}
\begin{Highlighting}[]
\NormalTok{man\_1 }\OtherTok{\textless{}{-}} \FunctionTok{manova}\NormalTok{(}\FunctionTok{cbind}\NormalTok{(megosztott, pontossag, kockazat, eszleles) }\SpecialCharTok{\textasciitilde{}}
\NormalTok{    baleset, }\AttributeTok{data =}\NormalTok{ baleset)}
\FunctionTok{summary}\NormalTok{(man\_1, }\AttributeTok{test =} \StringTok{"Wilks"}\NormalTok{)}
\CommentTok{\#\textgreater{}           Df   Wilks approx F num Df den Df    Pr(\textgreater{}F)}
\CommentTok{\#\textgreater{} baleset    1 0.27605   20.324      4     31 2.645e{-}08}
\CommentTok{\#\textgreater{} Residuals 34                                         }
\CommentTok{\#\textgreater{}              }
\CommentTok{\#\textgreater{} baleset   ***}
\CommentTok{\#\textgreater{} Residuals    }
\CommentTok{\#\textgreater{} {-}{-}{-}}
\CommentTok{\#\textgreater{} Signif. codes:  }
\CommentTok{\#\textgreater{} 0 \textquotesingle{}***\textquotesingle{} 0.001 \textquotesingle{}**\textquotesingle{} 0.01 \textquotesingle{}*\textquotesingle{} 0.05 \textquotesingle{}.\textquotesingle{} 0.1 \textquotesingle{} \textquotesingle{} 1}
\end{Highlighting}
\end{Shaded}

A fenti output tesztstatisztikájának szignifikanciaszintje azt mutatja,
hogy a csoportok közötti különbségek szignifikánsak, vagyis valóban van
különbség a balesetet szenvedett és a balesetmentes autóvezetők között.

Futtassuk le a diszkriminancia-analízis.

\begin{Shaded}
\begin{Highlighting}[]
\NormalTok{lda\_1 }\OtherTok{\textless{}{-}}\NormalTok{ MASS}\SpecialCharTok{::}\FunctionTok{lda}\NormalTok{(baleset }\SpecialCharTok{\textasciitilde{}}\NormalTok{ megosztott }\SpecialCharTok{+}\NormalTok{ pontossag }\SpecialCharTok{+}\NormalTok{ kockazat }\SpecialCharTok{+}
\NormalTok{    eszleles, }\AttributeTok{data =}\NormalTok{ baleset)}
\NormalTok{lda\_1}
\CommentTok{\#\textgreater{} Call:}
\CommentTok{\#\textgreater{} lda(baleset \textasciitilde{} megosztott + pontossag + kockazat + e...}
\CommentTok{\#\textgreater{} }
\CommentTok{\#\textgreater{} Prior probabilities of groups:}
\CommentTok{\#\textgreater{} nem volt baleste     volt baleste }
\CommentTok{\#\textgreater{}        0.4722222        0.5277778 }
\CommentTok{\#\textgreater{} }
\CommentTok{\#\textgreater{} Group means:}
\CommentTok{\#\textgreater{}                  megosztott pontossag kockazat eszl...}
\CommentTok{\#\textgreater{} nem volt baleste   5.941176  5.647059 2.588235 5.94...}
\CommentTok{\#\textgreater{} volt baleste       2.842105  2.684211 5.578947 3.10...}
\CommentTok{\#\textgreater{} }
\CommentTok{\#\textgreater{} Coefficients of linear discriminants:}
\CommentTok{\#\textgreater{}                    LD1}
\CommentTok{\#\textgreater{} megosztott {-}0.25764616}
\CommentTok{\#\textgreater{} pontossag  {-}0.07708289}
\CommentTok{\#\textgreater{} kockazat    0.36270659}
\CommentTok{\#\textgreater{} eszleles   {-}0.36702363}
\end{Highlighting}
\end{Shaded}

A fenti output alapján az előzetes valószínűsége annak, hogy valakinek
még nem volt balesete 0,472, míg annak a valószínűsége, hogy már volt
balesete a személynek 0,528. Ezután vizsgálhatjuk a csoportátlagokat. A
balesetmentes vezetők esetében magasabb a megosztott figyelem, a
figyelem pontosságának és az észlelés változójának az átlaga, míg a
kockázatvállalásé alacsonyabb. Ugyanakkor a másik csoport esetében a
kockázatvállalás változójának az átlaga magasabb, míg a másik három
képesség változójának átlaga alacsonyabb. Vagyis a balesetmentes vezetők
gyorsabban képesek észlelni és jobban meg tudják osztani a figyelmüket,
figyelmük pontosabb. A balesetet szenvedett vezetők esetében ezek a
képességek gyengébbek, míg jobban szeretnek kockázatot vállalni.

Végül a kanonikus diszkriminancia együtthatók segítségével felírhatjuk a
kanonikus diszkriminancia-függvényt a következő módon:

\begin{Shaded}
\begin{Highlighting}[]
\NormalTok{Z = 0,3627 * kockázat {-} 0,367 * észlelés {-} 0,2567 * megosztott{-}0,0771 * pontosság}
\end{Highlighting}
\end{Shaded}

Utolsó lépésként pedig megnézhetjük, mennyire hatékony a
diszkriminancia-analízis vagyis összevethetjük az eredeti
csoporttagságokat a modell alapján alkotott besorolásokkal.

\begin{Shaded}
\begin{Highlighting}[]
\NormalTok{lda\_1\_pred }\OtherTok{\textless{}{-}} \FunctionTok{predict}\NormalTok{(lda\_1, }\AttributeTok{method =} \StringTok{"plug{-}in"}\NormalTok{)}
\NormalTok{tab\_1 }\OtherTok{\textless{}{-}} \FunctionTok{table}\NormalTok{(lda\_1\_pred}\SpecialCharTok{$}\NormalTok{class, baleset}\SpecialCharTok{$}\NormalTok{baleset)}
\NormalTok{tab\_1}
\CommentTok{\#\textgreater{}                   }
\CommentTok{\#\textgreater{}                    nem volt baleste volt baleste}
\CommentTok{\#\textgreater{}   nem volt baleste               16            2}
\CommentTok{\#\textgreater{}   volt baleste                    1           17}
\DecValTok{100} \SpecialCharTok{*} \FunctionTok{sum}\NormalTok{(}\FunctionTok{diag}\NormalTok{(tab\_1))}\SpecialCharTok{/}\FunctionTok{sum}\NormalTok{(tab\_1)}
\CommentTok{\#\textgreater{} [1] 91.66667}
\end{Highlighting}
\end{Shaded}

A fenti sorok elkészík a predikciót, majd egy táblázatban reprezentálják
az eredeti és a becsült csoportba tartozásokat. A legtöbb adat a
főátlóban helyezkedik el, ami igen magas helyes besorolási arányra utal.
A helyes besorolások aránya 91,7\%.

A példában a gépjárműbalesetek emberi okait vizsgáltuk. Az eredmények
alapján a balesetmentes vezetők gyorsabban képesek észlelni és jobban
meg tudják osztani a figyelmüket, figyelmük pontosabb is. Ellenben a
balesetet szenvedett vezetők esetében ezek a képességek gyengébbek, míg
jobban szeretnek kockázatot vállalni.

\hypertarget{puxe9lda-a-szuxfcluxe9s-utuxe1ni-depressziuxf3-vizsguxe1lata}{%
\section{Példa: A szülés utáni depresszió
vizsgálata}\label{puxe9lda-a-szuxfcluxe9s-utuxe1ni-depressziuxf3-vizsguxe1lata}}

Ebben a példában a szülés utáni depresszió pszichés és szociális
hátterét vizsgáljuk meg a diszkriminancia-analízis segítségével.

\begin{Shaded}
\begin{Highlighting}[]
\NormalTok{depresszio }\OtherTok{\textless{}{-}}\NormalTok{ rio}\SpecialCharTok{::}\FunctionTok{import}\NormalTok{(}\AttributeTok{file =} \StringTok{"adat/diszkriminancia\_depresszio.xlsx"}\NormalTok{)}
\NormalTok{depresszio}\SpecialCharTok{$}\NormalTok{ppdepresszio }\OtherTok{\textless{}{-}} \FunctionTok{factor}\NormalTok{(depresszio}\SpecialCharTok{$}\NormalTok{ppdepresszio, }\AttributeTok{labels =} \FunctionTok{c}\NormalTok{(}\StringTok{"nincs depresszió"}\NormalTok{,}
    \StringTok{"van depresszió"}\NormalTok{))}
\FunctionTok{str}\NormalTok{(depresszio)}
\CommentTok{\#\textgreater{} \textquotesingle{}data.frame\textquotesingle{}:    20 obs. of  5 variables:}
\CommentTok{\#\textgreater{}  $ ppdepresszio: Factor w/ 2 levels "nincs depressz...}
\CommentTok{\#\textgreater{}  $ szeretet    : num  7 6 2 6 7 4 6 7 6 7 ...}
\CommentTok{\#\textgreater{}  $ tulvedes    : num  4 2 8 3 9 5 3 5 3 5 ...}
\CommentTok{\#\textgreater{}  $ kor         : num  24 20 19 22 23 25 26 18 19 22...}
\CommentTok{\#\textgreater{}  $ iskola      : num  12 17 8 16 17 17 12 17 16 17 ...}
\NormalTok{psych}\SpecialCharTok{::}\FunctionTok{headTail}\NormalTok{(depresszio)}
\CommentTok{\#\textgreater{}         ppdepresszio szeretet tulvedes kor iskola}
\CommentTok{\#\textgreater{} 1   nincs depresszió        7        4  24     12}
\CommentTok{\#\textgreater{} 2   nincs depresszió        6        2  20     17}
\CommentTok{\#\textgreater{} 3   nincs depresszió        2        8  19      8}
\CommentTok{\#\textgreater{} 4   nincs depresszió        6        3  22     16}
\CommentTok{\#\textgreater{} ...             \textless{}NA\textgreater{}      ...      ... ...    ...}
\CommentTok{\#\textgreater{} 17    van depresszió        4        7  30      6}
\CommentTok{\#\textgreater{} 18    van depresszió        4        6  24      8}
\CommentTok{\#\textgreater{} 19    van depresszió        3        7  21      9}
\CommentTok{\#\textgreater{} 20    van depresszió        2        8  18     10}
\end{Highlighting}
\end{Shaded}

Az adatbázisban a \texttt{ppdepresszio} változó mutatja a depresszió
jelenlétét, vagy hiányát. A magyarázó változók között a következő
változók szerepelnek: a szeretet skála (\texttt{szeretet} változó),
amely azt méri, hogy a személyek mennyire érzik, hogy a szüleik szeretik
őket; \texttt{tulvedes}-sel jelölt túlvédés iránti tendencia azt
mutatja, hogy mennyire hajlamosak arra a személyek, hogy túlságosan is
burokban tartsák, túlvédjék gyerekeiket, illetve szeretteiket; ezeken
kívül két szociológiai adat is a rendelkezésünkre áll, nevezetesen az
életkor (\texttt{kor} változó) és az elvégzett iskolai osztályok száma
(\texttt{iskola}).

A diszkriminancia-analízisben az első lépésében megvizsgáljuk, vajon
valóban szét lehet-e választani a depressziósok és a nem depressziósok
csoportját az adott változók alapján. Ehhez a Wilks-lambda tesztet
használjuk a többváltozós variancia-analízis keretein belül.

\begin{Shaded}
\begin{Highlighting}[]
\NormalTok{man\_1 }\OtherTok{\textless{}{-}} \FunctionTok{manova}\NormalTok{(}\FunctionTok{cbind}\NormalTok{(szeretet, tulvedes, kor, iskola) }\SpecialCharTok{\textasciitilde{}}\NormalTok{ ppdepresszio,}
    \AttributeTok{data =}\NormalTok{ depresszio)}
\FunctionTok{summary}\NormalTok{(man\_1, }\AttributeTok{test =} \StringTok{"Wilks"}\NormalTok{)}
\CommentTok{\#\textgreater{}              Df   Wilks approx F num Df den Df    P...}
\CommentTok{\#\textgreater{} ppdepresszio  1 0.29985   8.7561      4     15 0.00...}
\CommentTok{\#\textgreater{} Residuals    18                                    ...}
\CommentTok{\#\textgreater{}                 }
\CommentTok{\#\textgreater{} ppdepresszio ***}
\CommentTok{\#\textgreater{} Residuals       }
\CommentTok{\#\textgreater{} {-}{-}{-}}
\CommentTok{\#\textgreater{} Signif. codes:  }
\CommentTok{\#\textgreater{} 0 \textquotesingle{}***\textquotesingle{} 0.001 \textquotesingle{}**\textquotesingle{} 0.01 \textquotesingle{}*\textquotesingle{} 0.05 \textquotesingle{}.\textquotesingle{} 0.1 \textquotesingle{} \textquotesingle{} 1}
\end{Highlighting}
\end{Shaded}

A fenti output tesztstatisztikájának szignifikanciaszintje azt mutatja,
hogy a csoportok közötti különbségek szignifikánsak, vagyis valóban van
különbség a depressziós és a nem depressziós nők között az adott
változókat vizsgálva.

Végezzük el a diszkriminancia elemzést!

\begin{Shaded}
\begin{Highlighting}[]
\FunctionTok{library}\NormalTok{(MASS)}
\NormalTok{lda\_1 }\OtherTok{\textless{}{-}} \FunctionTok{lda}\NormalTok{(ppdepresszio }\SpecialCharTok{\textasciitilde{}}\NormalTok{ szeretet }\SpecialCharTok{+}\NormalTok{ tulvedes }\SpecialCharTok{+}\NormalTok{ kor }\SpecialCharTok{+}\NormalTok{ iskola,}
    \AttributeTok{data =}\NormalTok{ depresszio)}
\NormalTok{lda\_1}
\CommentTok{\#\textgreater{} Call:}
\CommentTok{\#\textgreater{} lda(ppdepresszio \textasciitilde{} szeretet + tulvedes + kor + isko...}
\CommentTok{\#\textgreater{} }
\CommentTok{\#\textgreater{} Prior probabilities of groups:}
\CommentTok{\#\textgreater{} nincs depresszió   van depresszió }
\CommentTok{\#\textgreater{}              0.5              0.5 }
\CommentTok{\#\textgreater{} }
\CommentTok{\#\textgreater{} Group means:}
\CommentTok{\#\textgreater{}                  szeretet tulvedes  kor iskola}
\CommentTok{\#\textgreater{} nincs depresszió      5.8      4.7 21.8   14.9}
\CommentTok{\#\textgreater{} van depresszió        3.3      7.5 24.0    8.3}
\CommentTok{\#\textgreater{} }
\CommentTok{\#\textgreater{} Coefficients of linear discriminants:}
\CommentTok{\#\textgreater{}                  LD1}
\CommentTok{\#\textgreater{} szeretet {-}0.21900671}
\CommentTok{\#\textgreater{} tulvedes  0.18422053}
\CommentTok{\#\textgreater{} kor       0.03467147}
\CommentTok{\#\textgreater{} iskola   {-}0.26661705}
\end{Highlighting}
\end{Shaded}

Az fenti output alapján az előzetes valószínűségek egyenlőek. A
csoportátlagok közötti különbségek azt mutatják, hogy a nem
depressziósok átlaga szeretet tekintetében magasabb (5,8), mint a
depressziósoké (3,3), az iskolai végzettségük is magasabb (14,9), mint a
depressziósoké (8,3). Ellenben a túlvédésnél a depressziósok értek el
magasabb átlagot, ők az idősebbek is (24). Vagyis azok, akik postpartum
depresszióban szenvednek, úgy érzik, a szüleik kevésbé szeretik őket,
túlvédőbbek a gyerekeikkel szemben, idősebbek, és az iskolai
végzettségük is alacsonyabb. A kanonikus diszkriminancia egyenlet
alakja:

\begin{Shaded}
\begin{Highlighting}[]
\NormalTok{Z =0,1842 * túlvédés + 0,0347 * kor {-} 0,2666 * iskola {-} 0,219 * szeretet}
\end{Highlighting}
\end{Shaded}

Utolsó momentumként az analízis értékelésére van még szükség.

\begin{Shaded}
\begin{Highlighting}[]
\NormalTok{lda\_1\_pred }\OtherTok{\textless{}{-}} \FunctionTok{predict}\NormalTok{(lda\_1, }\AttributeTok{method =} \StringTok{"plug{-}in"}\NormalTok{)}
\NormalTok{tab\_1 }\OtherTok{\textless{}{-}} \FunctionTok{table}\NormalTok{(lda\_1\_pred}\SpecialCharTok{$}\NormalTok{class, depresszio}\SpecialCharTok{$}\NormalTok{ppdepresszio)}
\NormalTok{tab\_1}
\CommentTok{\#\textgreater{}                   }
\CommentTok{\#\textgreater{}                    nincs depresszió van depresszió}
\CommentTok{\#\textgreater{}   nincs depresszió                9              0}
\CommentTok{\#\textgreater{}   van depresszió                  1             10}
\DecValTok{100} \SpecialCharTok{*} \FunctionTok{sum}\NormalTok{(}\FunctionTok{diag}\NormalTok{(tab\_1))}\SpecialCharTok{/}\FunctionTok{sum}\NormalTok{(tab\_1)}
\CommentTok{\#\textgreater{} [1] 95}
\end{Highlighting}
\end{Shaded}

Láthatjuk, hogy a valódi és a modell alapján becsült csoporttagságok
mátrixában a legtöbb adat a főátlóban helyezkedik el. Ez arra utal, hogy
a becsült csoporttagságok nagyjából lefedik az eredetit, az arány 95\%.

Vagyis azok, akik postpartum depresszióban szenvednek, úgy érzik, a
szüleik kevésbé szeretik őket, túlvédőbbek a gyerekeikkel szemben,
idősebbek, és az iskolai végzettségük is alacsonyabb.

\hypertarget{puxe9lda-pszichoszomatikus-megbetegeduxe9sekre-hajlamosuxedtuxf3-tuxe9nyezux151k}{%
\section{Példa: Pszichoszomatikus megbetegedésekre hajlamosító
tényezők}\label{puxe9lda-pszichoszomatikus-megbetegeduxe9sekre-hajlamosuxedtuxf3-tuxe9nyezux151k}}

Ebben a példában a pszichoszomatikus megbetegedéseket vizsgáljuk a
diszkriminancia-analízis segítségével.

\begin{Shaded}
\begin{Highlighting}[]
\NormalTok{pszichoszomatikus }\OtherTok{\textless{}{-}}\NormalTok{ rio}\SpecialCharTok{::}\FunctionTok{import}\NormalTok{(}\AttributeTok{file =} \StringTok{"adat/diszkriminancia\_pszichoszomatika.xlsx"}\NormalTok{)}
\NormalTok{pszichoszomatikus}\SpecialCharTok{$}\NormalTok{pszichoszomatika }\OtherTok{\textless{}{-}} \FunctionTok{factor}\NormalTok{(pszichoszomatikus}\SpecialCharTok{$}\NormalTok{pszichoszomatika,}
    \AttributeTok{labels =} \FunctionTok{c}\NormalTok{(}\StringTok{"pszichoszomatikus megbetegedése van"}\NormalTok{, }\StringTok{"egészséges"}\NormalTok{))}
\FunctionTok{str}\NormalTok{(pszichoszomatikus)}
\CommentTok{\#\textgreater{} \textquotesingle{}data.frame\textquotesingle{}:    36 obs. of  4 variables:}
\CommentTok{\#\textgreater{}  $ pszichoszomatika: Factor w/ 2 levels "pszichoszo...}
\CommentTok{\#\textgreater{}  $ stressz         : num  5 6 5 6 7 3 6 7 5 6 ...}
\CommentTok{\#\textgreater{}  $ szorongas       : num  6 6 5 6 7 3 3 7 5 6 ...}
\CommentTok{\#\textgreater{}  $ coping          : num  2 3 1 2 4 7 2 1 3 2 ...}
\NormalTok{psych}\SpecialCharTok{::}\FunctionTok{headTail}\NormalTok{(pszichoszomatikus)}
\CommentTok{\#\textgreater{}                        pszichoszomatika stressz}
\CommentTok{\#\textgreater{} 1   pszichoszomatikus megbetegedése van       5}
\CommentTok{\#\textgreater{} 2   pszichoszomatikus megbetegedése van       6}
\CommentTok{\#\textgreater{} 3   pszichoszomatikus megbetegedése van       5}
\CommentTok{\#\textgreater{} 4   pszichoszomatikus megbetegedése van       6}
\CommentTok{\#\textgreater{} ...                                \textless{}NA\textgreater{}     ...}
\CommentTok{\#\textgreater{} 33                           egészséges       3}
\CommentTok{\#\textgreater{} 34                           egészséges       2}
\CommentTok{\#\textgreater{} 35                           egészséges       3}
\CommentTok{\#\textgreater{} 36                           egészséges       4}
\CommentTok{\#\textgreater{}     szorongas coping}
\CommentTok{\#\textgreater{} 1           6      2}
\CommentTok{\#\textgreater{} 2           6      3}
\CommentTok{\#\textgreater{} 3           5      1}
\CommentTok{\#\textgreater{} 4           6      2}
\CommentTok{\#\textgreater{} ...       ...    ...}
\CommentTok{\#\textgreater{} 33          3      5}
\CommentTok{\#\textgreater{} 34          2      7}
\CommentTok{\#\textgreater{} 35          3      4}
\CommentTok{\#\textgreater{} 36          4      6}
\end{Highlighting}
\end{Shaded}

Az adatbázisban most a \texttt{pszichoszomatika} változó jelzi, hogy
valamilyen pszichoszomatikus megbetegedése van vagy nincs a személynek.
A változók közt szerepel a személyt ért stressz mértéke
(\texttt{stressz}), a szorongási szintje (\texttt{szorongás}) és a
megküzdési stratégiáinak hatékonysága (\texttt{coping}).

A diszkriminancia-analízisben az első lépésében megvizsgáljuk, vajon
valóban szét lehet-e választani a pszichoszomatikusok és a nem
pszichoszomatikusok csoportját az adott változók alapján. Ehhez a
Wilks-lambda tesztet használjuk a többváltozós variancia-analízis
keretein belül.

\begin{Shaded}
\begin{Highlighting}[]
\NormalTok{man\_1 }\OtherTok{\textless{}{-}} \FunctionTok{manova}\NormalTok{(}\FunctionTok{cbind}\NormalTok{(stressz, szorongas, coping) }\SpecialCharTok{\textasciitilde{}}\NormalTok{ pszichoszomatika,}
    \AttributeTok{data =}\NormalTok{ pszichoszomatikus)}
\FunctionTok{summary}\NormalTok{(man\_1, }\AttributeTok{test =} \StringTok{"Wilks"}\NormalTok{)}
\CommentTok{\#\textgreater{}                  Df   Wilks approx F num Df den Df}
\CommentTok{\#\textgreater{} pszichoszomatika  1 0.37974   17.423      3     32}
\CommentTok{\#\textgreater{} Residuals        34                               }
\CommentTok{\#\textgreater{}                    Pr(\textgreater{}F)    }
\CommentTok{\#\textgreater{} pszichoszomatika 6.92e{-}07 ***}
\CommentTok{\#\textgreater{} Residuals                    }
\CommentTok{\#\textgreater{} {-}{-}{-}}
\CommentTok{\#\textgreater{} Signif. codes:  }
\CommentTok{\#\textgreater{} 0 \textquotesingle{}***\textquotesingle{} 0.001 \textquotesingle{}**\textquotesingle{} 0.01 \textquotesingle{}*\textquotesingle{} 0.05 \textquotesingle{}.\textquotesingle{} 0.1 \textquotesingle{} \textquotesingle{} 1}
\end{Highlighting}
\end{Shaded}

A fenti output tesztstatisztikájának szignifikanciaszintje azt mutatja,
hogy a csoportok közötti különbségek szignifikánsak, vagyis valóban van
különbség a két csoport között az adott változókat vizsgálva.

Végezzük el a diszkriminancia elemzést.

\begin{Shaded}
\begin{Highlighting}[]
\FunctionTok{library}\NormalTok{(MASS)}
\NormalTok{lda\_1 }\OtherTok{\textless{}{-}} \FunctionTok{lda}\NormalTok{(pszichoszomatika }\SpecialCharTok{\textasciitilde{}}\NormalTok{ stressz }\SpecialCharTok{+}\NormalTok{ coping }\SpecialCharTok{+}\NormalTok{ szorongas,}
    \AttributeTok{data =}\NormalTok{ pszichoszomatikus)}
\NormalTok{lda\_1}
\CommentTok{\#\textgreater{} Call:}
\CommentTok{\#\textgreater{} lda(pszichoszomatika \textasciitilde{} stressz + coping + szorongas...}
\CommentTok{\#\textgreater{} }
\CommentTok{\#\textgreater{} Prior probabilities of groups:}
\CommentTok{\#\textgreater{} pszichoszomatikus megbetegedése van }
\CommentTok{\#\textgreater{}                           0.4722222 }
\CommentTok{\#\textgreater{}                          egészséges }
\CommentTok{\#\textgreater{}                           0.5277778 }
\CommentTok{\#\textgreater{} }
\CommentTok{\#\textgreater{} Group means:}
\CommentTok{\#\textgreater{}                                      stressz   coping}
\CommentTok{\#\textgreater{} pszichoszomatikus megbetegedése van 5.764706 2.588235}
\CommentTok{\#\textgreater{} egészséges                          2.842105 5.578947}
\CommentTok{\#\textgreater{}                                     szorongas}
\CommentTok{\#\textgreater{} pszichoszomatikus megbetegedése van  5.529412}
\CommentTok{\#\textgreater{} egészséges                           2.684211}
\CommentTok{\#\textgreater{} }
\CommentTok{\#\textgreater{} Coefficients of linear discriminants:}
\CommentTok{\#\textgreater{}                   LD1}
\CommentTok{\#\textgreater{} stressz   {-}0.31309547}
\CommentTok{\#\textgreater{} coping     0.46637406}
\CommentTok{\#\textgreater{} szorongas {-}0.06258674}
\end{Highlighting}
\end{Shaded}

A fenti outputból láthatjuk, hogy a csoporttagságok előzetes
valószínűsége a pszichoszomatikusok esetében kicsit kisebb (0,472). A
két csoport összevetésénél azt láthatjuk, hogy a stressz és a szorongás
változó átlaga a pszichoszomatikusok esetében, míg a coping változó
átlaga az egészségesen esetében magasabb. Vagyis az egészséges
személyeket kevesebb stressz éri, és azzal hatékonyabban is tudnak
megküzdeni, mint a pszichoszomatikusok, illetve kevesebbet is
szoronganak.

A kanonikus diszkriminancia egyenlet pedig a következő módon alakul:

\begin{Shaded}
\begin{Highlighting}[]
\NormalTok{Z = 0.4664 * coping {-} 0,3131 * stressz {-} 0,0626 * szorongas}
\end{Highlighting}
\end{Shaded}

Utolsó lépésként az analízis értékelésére van még szükség.

\begin{Shaded}
\begin{Highlighting}[]
\NormalTok{lda\_1\_pred }\OtherTok{\textless{}{-}} \FunctionTok{predict}\NormalTok{(lda\_1, }\AttributeTok{method =} \StringTok{"plug{-}in"}\NormalTok{)}
\NormalTok{tab\_1 }\OtherTok{\textless{}{-}} \FunctionTok{table}\NormalTok{(lda\_1\_pred}\SpecialCharTok{$}\NormalTok{class, pszichoszomatikus}\SpecialCharTok{$}\NormalTok{pszichoszomatika)}
\NormalTok{tab\_1}
\CommentTok{\#\textgreater{}                                      }
\CommentTok{\#\textgreater{}                                       pszichoszomat...}
\CommentTok{\#\textgreater{}   pszichoszomatikus megbetegedése van              ...}
\CommentTok{\#\textgreater{}   egészséges                                       ...}
\CommentTok{\#\textgreater{}                                      }
\CommentTok{\#\textgreater{}                                       egészséges}
\CommentTok{\#\textgreater{}   pszichoszomatikus megbetegedése van          2}
\CommentTok{\#\textgreater{}   egészséges                                  17}
\DecValTok{100} \SpecialCharTok{*} \FunctionTok{sum}\NormalTok{(}\FunctionTok{diag}\NormalTok{(tab\_1))}\SpecialCharTok{/}\FunctionTok{sum}\NormalTok{(tab\_1)}
\CommentTok{\#\textgreater{} [1] 91.66667}
\end{Highlighting}
\end{Shaded}

Az eredményen láthatjuk, hogy a valódi és a modell alapján becsült
csoporttagságok mátrixában a legtöbb adat a főátlóban helyezkedik el. Ez
arra utal, hogy a becsült csoporttagságok nagyjából lefedik az eredetit.
Ez az arány 91,7\%.

Ebben a példában a pszichoszomatikus megbetegedések lelki okait
vizsgáltuk. Az diszkriminancia-analízis eredménye szerint az egészséges
személyeket kevesebb stressz éri, és azzal hatékonyabban is tudnak
megküzdeni, mint a pszichoszomatikusok, valamint kevesebbet is
szoronganak.

\hypertarget{puxe9lda-kik-vuxe1suxe1rolnak-gyakran-bio-termuxe9keket}{%
\section{Példa: Kik vásárolnak gyakran bio
termékeket?}\label{puxe9lda-kik-vuxe1suxe1rolnak-gyakran-bio-termuxe9keket}}

Utolsó példánk a marketingkutatás területére kalauzol minket. Azt
próbáljuk megvizsgálni, hogy főként kik vásárolnak bio termékeket.

\begin{Shaded}
\begin{Highlighting}[]
\NormalTok{bio }\OtherTok{\textless{}{-}}\NormalTok{ rio}\SpecialCharTok{::}\FunctionTok{import}\NormalTok{(}\AttributeTok{file =} \StringTok{"adat/diszkriminancia\_bio.xlsx"}\NormalTok{)}
\NormalTok{bio}\SpecialCharTok{$}\NormalTok{vasarlas }\OtherTok{\textless{}{-}} \FunctionTok{factor}\NormalTok{(bio}\SpecialCharTok{$}\NormalTok{vasarlas, }\AttributeTok{labels =} \FunctionTok{c}\NormalTok{(}\StringTok{"soha nem vesz"}\NormalTok{,}
    \StringTok{"időnként vesz"}\NormalTok{, }\StringTok{"gyakran vesz"}\NormalTok{))}
\FunctionTok{table}\NormalTok{(bio}\SpecialCharTok{$}\NormalTok{vasarlas)}
\CommentTok{\#\textgreater{} }
\CommentTok{\#\textgreater{} soha nem vesz időnként vesz  gyakran vesz }
\CommentTok{\#\textgreater{}            10            10            10}
\FunctionTok{str}\NormalTok{(bio)}
\CommentTok{\#\textgreater{} \textquotesingle{}data.frame\textquotesingle{}:    30 obs. of  5 variables:}
\CommentTok{\#\textgreater{}  $ vasarlas: Factor w/ 3 levels "soha nem vesz",..:...}
\CommentTok{\#\textgreater{}  $ ertek   : num  2 4 2 3 1 1 2 3 4 6 ...}
\CommentTok{\#\textgreater{}  $ attitud : num  2 4 2 3 1 3 5 3 1 2 ...}
\CommentTok{\#\textgreater{}  $ fizetes : num  55 67 89 78 99 112 132 78 95 64 ...}
\CommentTok{\#\textgreater{}  $ kor     : num  32 56 59 48 44 39 37 40 44 43 ...}
\NormalTok{psych}\SpecialCharTok{::}\FunctionTok{headTail}\NormalTok{(bio)}
\CommentTok{\#\textgreater{}          vasarlas ertek attitud fizetes kor}
\CommentTok{\#\textgreater{} 1   soha nem vesz     2       2      55  32}
\CommentTok{\#\textgreater{} 2   soha nem vesz     4       4      67  56}
\CommentTok{\#\textgreater{} 3   soha nem vesz     2       2      89  59}
\CommentTok{\#\textgreater{} 4   soha nem vesz     3       3      78  48}
\CommentTok{\#\textgreater{} ...          \textless{}NA\textgreater{}   ...     ...     ... ...}
\CommentTok{\#\textgreater{} 27   gyakran vesz     6       6      62  19}
\CommentTok{\#\textgreater{} 28   gyakran vesz     9       9      69  27}
\CommentTok{\#\textgreater{} 29   gyakran vesz     9       9      78  28}
\CommentTok{\#\textgreater{} 30   gyakran vesz     8       8      73  30}
\end{Highlighting}
\end{Shaded}

Az adatbázisban a \texttt{vasarlas} változó mutatja a biotermékek
vásárlásának gyakoriságát, amely három értéket vehet fel: a személy
szinte soha nem vesz ilyen termékeket, időnként vesz, illetve gyakran
vesz. A vásárlás gyakoriságát a következő változókkal próbáljuk előre
jelezni: milyen értékeket tulajdonít ezeknek a termékeknek
(\texttt{ertek} változó, minél nagyobb pontszámot kap a skálán, annál
jobban értékeli a személy a bio termékeket); az \texttt{attitud} skála a
termékek iránti attitűdöt méri, a magasabb értékek itt is kedvezőbb
atttitűdöt jeleznek; ezen túl szerepel még a személy életkora
(\texttt{kor} változó) és a fizetése is (\texttt{fizetes}).

A diszkriminancia-analízisben az első lépésében megvizsgáljuk, vajon
valóban szét lehet-e választani a bio termékeket vásárlók három
csoportját az adott változók alapján. Ehhez a Wilks-lambda tesztet
használjuk a többváltozós variancia-analízis keretein belül.

\begin{Shaded}
\begin{Highlighting}[]
\NormalTok{man\_1 }\OtherTok{\textless{}{-}} \FunctionTok{manova}\NormalTok{(}\FunctionTok{cbind}\NormalTok{(ertek, attitud, fizetes, kor) }\SpecialCharTok{\textasciitilde{}}\NormalTok{ vasarlas,}
    \AttributeTok{data =}\NormalTok{ bio)}
\FunctionTok{summary}\NormalTok{(man\_1, }\AttributeTok{test =} \StringTok{"Wilks"}\NormalTok{)}
\CommentTok{\#\textgreater{}           Df   Wilks approx F num Df den Df    Pr(\textgreater{}F)}
\CommentTok{\#\textgreater{} vasarlas   2 0.12109   11.242      8     48 8.599e{-}09}
\CommentTok{\#\textgreater{} Residuals 27                                         }
\CommentTok{\#\textgreater{}              }
\CommentTok{\#\textgreater{} vasarlas  ***}
\CommentTok{\#\textgreater{} Residuals    }
\CommentTok{\#\textgreater{} {-}{-}{-}}
\CommentTok{\#\textgreater{} Signif. codes:  }
\CommentTok{\#\textgreater{} 0 \textquotesingle{}***\textquotesingle{} 0.001 \textquotesingle{}**\textquotesingle{} 0.01 \textquotesingle{}*\textquotesingle{} 0.05 \textquotesingle{}.\textquotesingle{} 0.1 \textquotesingle{} \textquotesingle{} 1}
\FunctionTok{summary.aov}\NormalTok{(man\_1)}
\CommentTok{\#\textgreater{}  Response ertek :}
\CommentTok{\#\textgreater{}             Df Sum Sq Mean Sq F value    Pr(\textgreater{}F)    }
\CommentTok{\#\textgreater{} vasarlas     2 146.07  73.033  27.656 2.915e{-}07 ***}
\CommentTok{\#\textgreater{} Residuals   27  71.30   2.641                      }
\CommentTok{\#\textgreater{} {-}{-}{-}}
\CommentTok{\#\textgreater{} Signif. codes:  }
\CommentTok{\#\textgreater{} 0 \textquotesingle{}***\textquotesingle{} 0.001 \textquotesingle{}**\textquotesingle{} 0.01 \textquotesingle{}*\textquotesingle{} 0.05 \textquotesingle{}.\textquotesingle{} 0.1 \textquotesingle{} \textquotesingle{} 1}
\CommentTok{\#\textgreater{} }
\CommentTok{\#\textgreater{}  Response attitud :}
\CommentTok{\#\textgreater{}             Df Sum Sq Mean Sq F value    Pr(\textgreater{}F)    }
\CommentTok{\#\textgreater{} vasarlas     2 211.67 105.833  57.495 1.853e{-}10 ***}
\CommentTok{\#\textgreater{} Residuals   27  49.70   1.841                      }
\CommentTok{\#\textgreater{} {-}{-}{-}}
\CommentTok{\#\textgreater{} Signif. codes:  }
\CommentTok{\#\textgreater{} 0 \textquotesingle{}***\textquotesingle{} 0.001 \textquotesingle{}**\textquotesingle{} 0.01 \textquotesingle{}*\textquotesingle{} 0.05 \textquotesingle{}.\textquotesingle{} 0.1 \textquotesingle{} \textquotesingle{} 1}
\CommentTok{\#\textgreater{} }
\CommentTok{\#\textgreater{}  Response fizetes :}
\CommentTok{\#\textgreater{}             Df  Sum Sq Mean Sq F value  Pr(\textgreater{}F)  }
\CommentTok{\#\textgreater{} vasarlas     2  6427.5  3213.7  3.1857 0.05726 .}
\CommentTok{\#\textgreater{} Residuals   27 27237.5  1008.8                  }
\CommentTok{\#\textgreater{} {-}{-}{-}}
\CommentTok{\#\textgreater{} Signif. codes:  }
\CommentTok{\#\textgreater{} 0 \textquotesingle{}***\textquotesingle{} 0.001 \textquotesingle{}**\textquotesingle{} 0.01 \textquotesingle{}*\textquotesingle{} 0.05 \textquotesingle{}.\textquotesingle{} 0.1 \textquotesingle{} \textquotesingle{} 1}
\CommentTok{\#\textgreater{} }
\CommentTok{\#\textgreater{}  Response kor :}
\CommentTok{\#\textgreater{}             Df Sum Sq Mean Sq F value   Pr(\textgreater{}F)   }
\CommentTok{\#\textgreater{} vasarlas     2 1449.3  724.63  7.3001 0.002922 **}
\CommentTok{\#\textgreater{} Residuals   27 2680.1   99.26                    }
\CommentTok{\#\textgreater{} {-}{-}{-}}
\CommentTok{\#\textgreater{} Signif. codes:  }
\CommentTok{\#\textgreater{} 0 \textquotesingle{}***\textquotesingle{} 0.001 \textquotesingle{}**\textquotesingle{} 0.01 \textquotesingle{}*\textquotesingle{} 0.05 \textquotesingle{}.\textquotesingle{} 0.1 \textquotesingle{} \textquotesingle{} 1}
\end{Highlighting}
\end{Shaded}

A fenti elemzés tesztstatisztikájának szignifikanciaszintje azt mutatja,
hogy a csoportok közötti különbségek szignifikánsak, vagyis valóban van
különbség a három csoport között az adott változókat vizsgálva.

Végezzük el a diszkriminancia elemzést.

\begin{Shaded}
\begin{Highlighting}[]
\FunctionTok{library}\NormalTok{(MASS)}
\NormalTok{lda\_1 }\OtherTok{\textless{}{-}} \FunctionTok{lda}\NormalTok{(vasarlas }\SpecialCharTok{\textasciitilde{}}\NormalTok{ ertek }\SpecialCharTok{+}\NormalTok{ attitud }\SpecialCharTok{+}\NormalTok{ fizetes }\SpecialCharTok{+}\NormalTok{ kor, }\AttributeTok{data =}\NormalTok{ bio)}
\NormalTok{lda\_1}
\CommentTok{\#\textgreater{} Call:}
\CommentTok{\#\textgreater{} lda(vasarlas \textasciitilde{} ertek + attitud + fizetes + kor, dat...}
\CommentTok{\#\textgreater{} }
\CommentTok{\#\textgreater{} Prior probabilities of groups:}
\CommentTok{\#\textgreater{} soha nem vesz időnként vesz  gyakran vesz }
\CommentTok{\#\textgreater{}     0.3333333     0.3333333     0.3333333 }
\CommentTok{\#\textgreater{} }
\CommentTok{\#\textgreater{} Group means:}
\CommentTok{\#\textgreater{}               ertek attitud fizetes  kor}
\CommentTok{\#\textgreater{} soha nem vesz   2.8     2.6    86.9 44.2}
\CommentTok{\#\textgreater{} időnként vesz   5.3     5.6   106.5 34.9}
\CommentTok{\#\textgreater{} gyakran vesz    8.2     9.1    70.7 27.2}
\CommentTok{\#\textgreater{} }
\CommentTok{\#\textgreater{} Coefficients of linear discriminants:}
\CommentTok{\#\textgreater{}                  LD1          LD2}
\CommentTok{\#\textgreater{} ertek    0.278041839 {-}0.175361797}
\CommentTok{\#\textgreater{} attitud  0.578431017  0.066376341}
\CommentTok{\#\textgreater{} fizetes {-}0.003687657 {-}0.032311789}
\CommentTok{\#\textgreater{} kor     {-}0.019282287  0.003972177}
\CommentTok{\#\textgreater{} }
\CommentTok{\#\textgreater{} Proportion of trace:}
\CommentTok{\#\textgreater{}    LD1    LD2 }
\CommentTok{\#\textgreater{} 0.9687 0.0313}
\CommentTok{\# klaR::greedy.wilks(vasarlas\textasciitilde{}ertek+attitud+fizetes+kor,data=bio,}
\CommentTok{\# niveau = 0.15)}
\end{Highlighting}
\end{Shaded}

A fenti output alapján az elemzés elején a három vásárlási gyakoriság
valószínűsége egyenlő (0,333). Ha a csoportátlagokat vizsgáljuk akkor
láthatjuk, hogy mind az értékek, mind az attitűd változójának
tekintetében a soha sem vásárolók átlaga a legalacsonyabb (3 mindkét
változó esetében), az időnként bio termékeket vásárlók csoport átlaga
középen helyezkedik el mind a két változó esetében (5 és 6), és a
gyakran vásárlók átlaga a legmagasabb (8 és 9). Életkor tekintetében egy
kissé másképpen alakulnak a csoportok. A legidősebbek szinte sohasem
vásárolnak bio termékeket, a legfiatalabbak pedig igen gyakran
vásárolnak. Fizetés tekintetében nem figyelhető meg jól magyarázható
összefüggés: a legalacsonyabb fizetésűek gyakran, míg a közepes
fizetésűek szinte soha sem vásárolnak bio termékeket.

A két kanonikus diszkriminancia-egyenlet a következőképpen alakul:

\begin{Shaded}
\begin{Highlighting}[]
\NormalTok{Z1 = 0,278 * ertek + 0,578 * attitud {-} 0,019 * kor {-} 0,004 *fizetes}
\NormalTok{Z2 = {-}0,175 * ertek + 0,066 * attitud + 0,004 * kor {-} 0,032 *fizetes}
\end{Highlighting}
\end{Shaded}

Utolsó lépésként az analízis értékelésére van még szükség.

\begin{Shaded}
\begin{Highlighting}[]
\NormalTok{lda\_1\_pred }\OtherTok{\textless{}{-}} \FunctionTok{predict}\NormalTok{(lda\_1, }\AttributeTok{method =} \StringTok{"plug{-}in"}\NormalTok{)}
\NormalTok{tab\_1 }\OtherTok{\textless{}{-}} \FunctionTok{table}\NormalTok{(lda\_1\_pred}\SpecialCharTok{$}\NormalTok{class, bio}\SpecialCharTok{$}\NormalTok{vasarlas)}
\NormalTok{tab\_1}
\CommentTok{\#\textgreater{}                }
\CommentTok{\#\textgreater{}                 soha nem vesz időnként vesz gyakran...}
\CommentTok{\#\textgreater{}   soha nem vesz             9             1        ...}
\CommentTok{\#\textgreater{}   időnként vesz             1             7        ...}
\CommentTok{\#\textgreater{}   gyakran vesz              0             2        ...}
\DecValTok{100} \SpecialCharTok{*} \FunctionTok{sum}\NormalTok{(}\FunctionTok{diag}\NormalTok{(tab\_1))}\SpecialCharTok{/}\FunctionTok{sum}\NormalTok{(tab\_1)}
\CommentTok{\#\textgreater{} [1] 83.33333}
\end{Highlighting}
\end{Shaded}

A fenti eredményben láthatjuk, hogy a valódi és a modell alapján becsült
csoporttagságok mátrixában a legtöbb adat a főátlóban helyezkedik el. Ez
arra utal, hogy a becsült csoporttagságok nagyjából lefedik az eredetit.
Ez az arány 83\%.

Az utolsó probléma körében a bio termékek vásárlásának gyakoriságát
vizsgáltuk. A kapott eredményeink alapján azok, akik gyakran vásárolnak
ilyen termékeket, pozitívabbak értékelik és pozitívabb attitűdökkel
rendelkeznek a bio termékek irányában, fiatalabbak, fizetésük viszont
alacsonyabb.

\hypertarget{puxe9lda-vezetuxe9si-programok}{%
\section{Példa: Vezetési
programok}\label{puxe9lda-vezetuxe9si-programok}}

Egy vállalat menedzsmentje szeretné megvizsgálni különböző vezetési
programok hatását, ezért három különböző vezetési programot vezetett be
három különböző stratégiai üzleti egységben (SÜE). Az első SÜE-ben
bevezetett program az egyenlőséget és az individualizmust hangsúlyozta.
A második SÜE-ben az egyenlőséget és a csoportmunkát helyzeték
középpontba. A harmadik SÜE-ben a bevezetett program egy nagyon
hierarchikus vezetési elvet alkalmazott. Később mindhárom SÜE
dolgozóinak körében felmérést végeztek, és a kérdések között szerepelt a
szervezettel való elkötelezettség mértéke (\texttt{szelkot}), a
szervezettel való elégedettség nagysága (\texttt{elegedett}), illetve a
rendszer egalitárius vagy tekintélyelvű (autokrata) jellege
(\texttt{rendszer}).

\begin{Shaded}
\begin{Highlighting}[]
\NormalTok{vezetes }\OtherTok{\textless{}{-}}\NormalTok{ rio}\SpecialCharTok{::}\FunctionTok{import}\NormalTok{(}\AttributeTok{file =} \StringTok{"adat/diszkriminancia\_vezetesi\_program.xlsx"}\NormalTok{)}
\NormalTok{vezetes}\SpecialCharTok{$}\NormalTok{SUE }\OtherTok{\textless{}{-}} \FunctionTok{factor}\NormalTok{(vezetes}\SpecialCharTok{$}\NormalTok{SUE)}
\FunctionTok{str}\NormalTok{(vezetes)}
\CommentTok{\#\textgreater{} \textquotesingle{}data.frame\textquotesingle{}:    30 obs. of  4 variables:}
\CommentTok{\#\textgreater{}  $ SUE      : Factor w/ 3 levels "1","2","3": 1 1 1...}
\CommentTok{\#\textgreater{}  $ szelkot  : num  4 4 5 5 3 3 3 5 3 5 ...}
\CommentTok{\#\textgreater{}  $ elegedett: num  1 3 4 1 2 2 4 2 1 1 ...}
\CommentTok{\#\textgreater{}  $ rendszer : num  2 1 2 4 4 3 2 3 4 2 ...}
\NormalTok{psych}\SpecialCharTok{::}\FunctionTok{headTail}\NormalTok{(vezetes)}
\CommentTok{\#\textgreater{}      SUE szelkot elegedett rendszer}
\CommentTok{\#\textgreater{} 1      1       4         1        2}
\CommentTok{\#\textgreater{} 2      1       4         3        1}
\CommentTok{\#\textgreater{} 3      1       5         4        2}
\CommentTok{\#\textgreater{} 4      1       5         1        4}
\CommentTok{\#\textgreater{} ... \textless{}NA\textgreater{}     ...       ...      ...}
\CommentTok{\#\textgreater{} 27     3       1         1        5}
\CommentTok{\#\textgreater{} 28     3       5         1        4}
\CommentTok{\#\textgreater{} 29     3       5         1        5}
\CommentTok{\#\textgreater{} 30     3       5         3        4}
\end{Highlighting}
\end{Shaded}

Végezzük el a többváltozós variancia elemzést.

\begin{Shaded}
\begin{Highlighting}[]
\NormalTok{man\_1 }\OtherTok{\textless{}{-}} \FunctionTok{manova}\NormalTok{(}\FunctionTok{cbind}\NormalTok{(szelkot, elegedett, rendszer) }\SpecialCharTok{\textasciitilde{}}\NormalTok{ SUE, }\AttributeTok{data =}\NormalTok{ vezetes)}
\FunctionTok{summary}\NormalTok{(man\_1, }\AttributeTok{test =} \StringTok{"Wilks"}\NormalTok{)}
\CommentTok{\#\textgreater{}           Df  Wilks approx F num Df den Df    Pr(\textgreater{}F...}
\CommentTok{\#\textgreater{} SUE        2 0.1894   10.815      6     50 1.102e{-}0...}
\CommentTok{\#\textgreater{} Residuals 27                                       ...}
\CommentTok{\#\textgreater{} {-}{-}{-}}
\CommentTok{\#\textgreater{} Signif. codes:  }
\CommentTok{\#\textgreater{} 0 \textquotesingle{}***\textquotesingle{} 0.001 \textquotesingle{}**\textquotesingle{} 0.01 \textquotesingle{}*\textquotesingle{} 0.05 \textquotesingle{}.\textquotesingle{} 0.1 \textquotesingle{} \textquotesingle{} 1}
\FunctionTok{summary.aov}\NormalTok{(man\_1, }\AttributeTok{test =} \StringTok{"Wilks"}\NormalTok{)}
\CommentTok{\#\textgreater{}  Response szelkot :}
\CommentTok{\#\textgreater{}             Df Sum Sq Mean Sq F value Pr(\textgreater{}F)}
\CommentTok{\#\textgreater{} SUE          2  2.867  1.4333  1.1057 0.3455}
\CommentTok{\#\textgreater{} Residuals   27 35.000  1.2963               }
\CommentTok{\#\textgreater{} }
\CommentTok{\#\textgreater{}  Response elegedett :}
\CommentTok{\#\textgreater{}             Df Sum Sq Mean Sq F value   Pr(\textgreater{}F)   }
\CommentTok{\#\textgreater{} SUE          2 17.267  8.6333  8.4152 0.001444 **}
\CommentTok{\#\textgreater{} Residuals   27 27.700  1.0259                    }
\CommentTok{\#\textgreater{} {-}{-}{-}}
\CommentTok{\#\textgreater{} Signif. codes:  }
\CommentTok{\#\textgreater{} 0 \textquotesingle{}***\textquotesingle{} 0.001 \textquotesingle{}**\textquotesingle{} 0.01 \textquotesingle{}*\textquotesingle{} 0.05 \textquotesingle{}.\textquotesingle{} 0.1 \textquotesingle{} \textquotesingle{} 1}
\CommentTok{\#\textgreater{} }
\CommentTok{\#\textgreater{}  Response rendszer :}
\CommentTok{\#\textgreater{}             Df Sum Sq Mean Sq F value    Pr(\textgreater{}F)    }
\CommentTok{\#\textgreater{} SUE          2 52.267 26.1333      48 1.287e{-}09 ***}
\CommentTok{\#\textgreater{} Residuals   27 14.700  0.5444                      }
\CommentTok{\#\textgreater{} {-}{-}{-}}
\CommentTok{\#\textgreater{} Signif. codes:  }
\CommentTok{\#\textgreater{} 0 \textquotesingle{}***\textquotesingle{} 0.001 \textquotesingle{}**\textquotesingle{} 0.01 \textquotesingle{}*\textquotesingle{} 0.05 \textquotesingle{}.\textquotesingle{} 0.1 \textquotesingle{} \textquotesingle{} 1}

\NormalTok{lm\_1 }\OtherTok{\textless{}{-}} \FunctionTok{lm}\NormalTok{(szelkot }\SpecialCharTok{\textasciitilde{}}\NormalTok{ SUE, }\AttributeTok{data =}\NormalTok{ vezetes)}
\NormalTok{car}\SpecialCharTok{::}\FunctionTok{Anova}\NormalTok{(lm\_1, }\AttributeTok{test.statistic =} \FunctionTok{c}\NormalTok{(}\StringTok{"Wilks"}\NormalTok{))}
\CommentTok{\#\textgreater{} Anova Table (Type II tests)}
\CommentTok{\#\textgreater{} }
\CommentTok{\#\textgreater{} Response: szelkot}
\CommentTok{\#\textgreater{}           Sum Sq Df F value Pr(\textgreater{}F)}
\CommentTok{\#\textgreater{} SUE        2.867  2  1.1057 0.3455}
\CommentTok{\#\textgreater{} Residuals 35.000 27}
\DecValTok{1} \SpecialCharTok{{-}} \FunctionTok{summary}\NormalTok{(lm\_1)}\SpecialCharTok{$}\NormalTok{r.squared  }\CommentTok{\# Wilks lambda}
\CommentTok{\#\textgreater{} [1] 0.9242958}

\NormalTok{lm\_1 }\OtherTok{\textless{}{-}} \FunctionTok{lm}\NormalTok{(elegedett }\SpecialCharTok{\textasciitilde{}}\NormalTok{ SUE, }\AttributeTok{data =}\NormalTok{ vezetes)}
\NormalTok{car}\SpecialCharTok{::}\FunctionTok{Anova}\NormalTok{(lm\_1, }\AttributeTok{test.statistic =} \FunctionTok{c}\NormalTok{(}\StringTok{"Wilks"}\NormalTok{))}
\CommentTok{\#\textgreater{} Anova Table (Type II tests)}
\CommentTok{\#\textgreater{} }
\CommentTok{\#\textgreater{} Response: elegedett}
\CommentTok{\#\textgreater{}           Sum Sq Df F value   Pr(\textgreater{}F)   }
\CommentTok{\#\textgreater{} SUE       17.267  2  8.4152 0.001444 **}
\CommentTok{\#\textgreater{} Residuals 27.700 27                    }
\CommentTok{\#\textgreater{} {-}{-}{-}}
\CommentTok{\#\textgreater{} Signif. codes:  }
\CommentTok{\#\textgreater{} 0 \textquotesingle{}***\textquotesingle{} 0.001 \textquotesingle{}**\textquotesingle{} 0.01 \textquotesingle{}*\textquotesingle{} 0.05 \textquotesingle{}.\textquotesingle{} 0.1 \textquotesingle{} \textquotesingle{} 1}
\DecValTok{1} \SpecialCharTok{{-}} \FunctionTok{summary}\NormalTok{(lm\_1)}\SpecialCharTok{$}\NormalTok{r.squared  }\CommentTok{\# Wilks lambda}
\CommentTok{\#\textgreater{} [1] 0.6160119}

\NormalTok{lm\_1 }\OtherTok{\textless{}{-}} \FunctionTok{lm}\NormalTok{(rendszer }\SpecialCharTok{\textasciitilde{}}\NormalTok{ SUE, }\AttributeTok{data =}\NormalTok{ vezetes)}
\NormalTok{car}\SpecialCharTok{::}\FunctionTok{Anova}\NormalTok{(lm\_1, }\AttributeTok{test.statistic =} \FunctionTok{c}\NormalTok{(}\StringTok{"Wilks"}\NormalTok{))}
\CommentTok{\#\textgreater{} Anova Table (Type II tests)}
\CommentTok{\#\textgreater{} }
\CommentTok{\#\textgreater{} Response: rendszer}
\CommentTok{\#\textgreater{}           Sum Sq Df F value    Pr(\textgreater{}F)    }
\CommentTok{\#\textgreater{} SUE       52.267  2      48 1.287e{-}09 ***}
\CommentTok{\#\textgreater{} Residuals 14.700 27                      }
\CommentTok{\#\textgreater{} {-}{-}{-}}
\CommentTok{\#\textgreater{} Signif. codes:  }
\CommentTok{\#\textgreater{} 0 \textquotesingle{}***\textquotesingle{} 0.001 \textquotesingle{}**\textquotesingle{} 0.01 \textquotesingle{}*\textquotesingle{} 0.05 \textquotesingle{}.\textquotesingle{} 0.1 \textquotesingle{} \textquotesingle{} 1}
\DecValTok{1} \SpecialCharTok{{-}} \FunctionTok{summary}\NormalTok{(lm\_1)}\SpecialCharTok{$}\NormalTok{r.squared  }\CommentTok{\# Wilks lambda}
\CommentTok{\#\textgreater{} [1] 0.2195122}
\end{Highlighting}
\end{Shaded}

\begin{Shaded}
\begin{Highlighting}[]
\NormalTok{biotools}\SpecialCharTok{::}\FunctionTok{boxM}\NormalTok{(}\AttributeTok{data =}\NormalTok{ vezetes[}\FunctionTok{c}\NormalTok{(}\StringTok{"szelkot"}\NormalTok{, }\StringTok{"elegedett"}\NormalTok{, }\StringTok{"rendszer"}\NormalTok{)],}
    \AttributeTok{grouping =}\NormalTok{ vezetes}\SpecialCharTok{$}\NormalTok{SUE)}
\CommentTok{\#\textgreater{} }
\CommentTok{\#\textgreater{}  Box\textquotesingle{}s M{-}test for Homogeneity of Covariance}
\CommentTok{\#\textgreater{}  Matrices}
\CommentTok{\#\textgreater{} }
\CommentTok{\#\textgreater{} data:  vezetes[c("szelkot", "elegedett", "rendszer")]}
\CommentTok{\#\textgreater{} Chi{-}Sq (approx.) = 19.607, df = 12, p{-}value =}
\CommentTok{\#\textgreater{} 0.0749}
\end{Highlighting}
\end{Shaded}

\begin{Shaded}
\begin{Highlighting}[]
\FunctionTok{library}\NormalTok{(MASS)}
\NormalTok{lda\_1 }\OtherTok{\textless{}{-}} \FunctionTok{lda}\NormalTok{(SUE }\SpecialCharTok{\textasciitilde{}}\NormalTok{ szelkot }\SpecialCharTok{+}\NormalTok{ elegedett }\SpecialCharTok{+}\NormalTok{ rendszer, }\AttributeTok{data =}\NormalTok{ vezetes)}
\NormalTok{lda\_1}
\CommentTok{\#\textgreater{} Call:}
\CommentTok{\#\textgreater{} lda(SUE \textasciitilde{} szelkot + elegedett + rendszer, data = ve...}
\CommentTok{\#\textgreater{} }
\CommentTok{\#\textgreater{} Prior probabilities of groups:}
\CommentTok{\#\textgreater{}         1         2         3 }
\CommentTok{\#\textgreater{} 0.3333333 0.3333333 0.3333333 }
\CommentTok{\#\textgreater{} }
\CommentTok{\#\textgreater{} Group means:}
\CommentTok{\#\textgreater{}   szelkot elegedett rendszer}
\CommentTok{\#\textgreater{} 1     4.0       2.1      2.7}
\CommentTok{\#\textgreater{} 2     4.7       3.4      1.5}
\CommentTok{\#\textgreater{} 3     4.1       1.6      4.7}
\CommentTok{\#\textgreater{} }
\CommentTok{\#\textgreater{} Coefficients of linear discriminants:}
\CommentTok{\#\textgreater{}                   LD1       LD2}
\CommentTok{\#\textgreater{} szelkot   {-}0.07056752 0.4959142}
\CommentTok{\#\textgreater{} elegedett  0.10193972 0.8596451}
\CommentTok{\#\textgreater{} rendszer  {-}1.32360357 0.5392379}
\CommentTok{\#\textgreater{} }
\CommentTok{\#\textgreater{} Proportion of trace:}
\CommentTok{\#\textgreater{}    LD1    LD2 }
\CommentTok{\#\textgreater{} 0.9616 0.0384}
\CommentTok{\# klaR::greedy.wilks(vasarlas\textasciitilde{}ertek+attitud+fizetes+kor,data=bio,}
\CommentTok{\# niveau = 0.15)}
\end{Highlighting}
\end{Shaded}

\begin{Shaded}
\begin{Highlighting}[]
\NormalTok{lda\_1\_pred }\OtherTok{\textless{}{-}} \FunctionTok{predict}\NormalTok{(lda\_1, }\AttributeTok{method =} \StringTok{"plug{-}in"}\NormalTok{)}
\NormalTok{tab\_1 }\OtherTok{\textless{}{-}} \FunctionTok{table}\NormalTok{(lda\_1\_pred}\SpecialCharTok{$}\NormalTok{class, vezetes}\SpecialCharTok{$}\NormalTok{SUE)}
\NormalTok{tab\_1}
\CommentTok{\#\textgreater{}    }
\CommentTok{\#\textgreater{}      1  2  3}
\CommentTok{\#\textgreater{}   1  4  1  0}
\CommentTok{\#\textgreater{}   2  3  9  0}
\CommentTok{\#\textgreater{}   3  3  0 10}
\DecValTok{100} \SpecialCharTok{*} \FunctionTok{sum}\NormalTok{(}\FunctionTok{diag}\NormalTok{(tab\_1))}\SpecialCharTok{/}\FunctionTok{sum}\NormalTok{(tab\_1)}
\CommentTok{\#\textgreater{} [1] 76.66667}
\end{Highlighting}
\end{Shaded}

\hypertarget{megjegyzuxe9sek}{%
\section{Megjegyzések}\label{megjegyzuxe9sek}}

Diszkriminancia analízis esetén az adatokat nem szükséges
standardizálni, ennek oka, hogy az analízis eredményét nem befolyásolja
jelentős mértékben az egyes változók mértékegysége.

A függő változónk tehát kategorikus, a függetlenek pedig numerikusak.
Arra vagyunk kíváncsiak, hogy a függő változó által meghatározott
csoportok mely független változókban különböznek egymástól, melyek
különböztetik meg egy egymástól a függő változó kategóriáit.

Ha a kategorikus függő változónk csupán kétértékű, akkor kétváltozós
diszkriminancia elemzésről beszélünk, több szint esetén többváltozós
diszkiriminancia elemzésről.

\hypertarget{az-alkalmazuxe1si-feltuxe9telek}{%
\section{Az alkalmazási
feltételek}\label{az-alkalmazuxe1si-feltuxe9telek}}

A fűggő változó kategorikus két vagy több szinttel. A független változók
intervallum vagy arány skálájú változók, de használhatunk dichotóm
változókat és a legalább 5 fokú likert skálán mért értékeket is. A függő
változó csoportjaiban nagyjából azanosnak kell lennie a
csoportnagyságnak, minden csoportnak legalább két adatsort tartalmaznia
kell. A mintanagyságra is figyelnünk kell, a független változók számának
kisebb kell lenni, mint a legkisebb csoport esetszáma, a teljes
mintanagyság legalább 10-szer nagyobb a független változók számánál. A
diszkriminancia elemzés feltételezi a független változók közötti
lineáris kapcsolatot.

Az egyváltozós normalitás vizsgálatára a kiugró értékek vizsgálata
javasolt, illetve megfelelő mérési skála (például nem dichotóm változó
esetén) a Shapiro--Wilk-próbát is használhatjuk. A többváltozós
normalitás vizsgálatához

A csoportok szétválasztásának egyik megközelítése a Mahalanobis-féle
távolságot használja. Az eljárás lényege, hogy az \(m\) csoportot
tartalmazó minta átlagvektorával becsüljük a csoportok valódi
átlagvektorát. Az egyes személyek csoportközéptől való átlagát számoljuk
ki a Mahalanobis-féle távolsággal, és minden személyt abba a csoportba
sorolunk be ez alapján, amelyhez közelebb esik. Ez lehet az a csoport,
amelybe a személy valóban beletartozik, de lehet másik is. A helyes
besorolások aránya világosan megmutatja, hogy mennyire jól lehet a
csoportokat szétválasztani a használt változók alapján.

\begin{Shaded}
\begin{Highlighting}[]
\CommentTok{\# remotes::install\_github(\textquotesingle{}hyunsooseol/snowCluster\textquotesingle{})}
\end{Highlighting}
\end{Shaded}

\begin{Shaded}
\begin{Highlighting}[]
\FunctionTok{library}\NormalTok{(snowCluster)}
\NormalTok{vallalat }\OtherTok{\textless{}{-}}\NormalTok{ rio}\SpecialCharTok{::}\FunctionTok{import}\NormalTok{(}\AttributeTok{file =} \StringTok{"adat/diszkriminancia\_vezetesi\_program.sav"}\NormalTok{)}
\NormalTok{snowCluster}\SpecialCharTok{::}\FunctionTok{disc}\NormalTok{(}
    \AttributeTok{data =}\NormalTok{ vallalat,}
    \AttributeTok{dep =}\NormalTok{ SUE,}
    \AttributeTok{covs =} \FunctionTok{vars}\NormalTok{(szelköt, elégedett, rendszer),}
    \AttributeTok{gm =} \ConstantTok{TRUE}\NormalTok{,}
    \AttributeTok{coef =} \ConstantTok{TRUE}\NormalTok{,}
    \AttributeTok{prop =} \ConstantTok{TRUE}\NormalTok{,}
    \AttributeTok{tes =} \ConstantTok{TRUE}\NormalTok{,}
    \AttributeTok{plot =} \ConstantTok{TRUE}\NormalTok{,}
    \AttributeTok{plot1 =} \ConstantTok{TRUE}\NormalTok{,}
    \AttributeTok{plot2 =} \ConstantTok{TRUE}\NormalTok{)}


\FunctionTok{str}\NormalTok{(vallalat)}
\NormalTok{vallalat}\SpecialCharTok{$}\NormalTok{SUE }\OtherTok{\textless{}{-}} \FunctionTok{factor}\NormalTok{(vallalat}\SpecialCharTok{$}\NormalTok{SUE)}

\NormalTok{lda\_1 }\OtherTok{\textless{}{-}}\NormalTok{ MASS}\SpecialCharTok{::}\FunctionTok{lda}\NormalTok{(SUE }\SpecialCharTok{\textasciitilde{}}\NormalTok{ szelköt }\SpecialCharTok{+}\NormalTok{ elégedett }\SpecialCharTok{+}\NormalTok{ rendszer, }\AttributeTok{data =}\NormalTok{ vallalat)}
\NormalTok{lda\_1}

\NormalTok{man\_1 }\OtherTok{\textless{}{-}}\NormalTok{ stats}\SpecialCharTok{::}\FunctionTok{manova}\NormalTok{(}\FunctionTok{cbind}\NormalTok{(szelköt, elégedett, rendszer)}\SpecialCharTok{\textasciitilde{}}\NormalTok{SUE, }\AttributeTok{data=}\NormalTok{vallalat)}
\NormalTok{man\_1}
\FunctionTok{summary}\NormalTok{(man\_1, }\AttributeTok{test=}\StringTok{"Wilks"}\NormalTok{)}
\FunctionTok{summary.aov}\NormalTok{(man\_1)}

\NormalTok{F }\OtherTok{\textless{}{-}} \FloatTok{1.1057} \CommentTok{\# F próbastatisztika érték}
\NormalTok{p }\OtherTok{\textless{}{-}} \DecValTok{1} \CommentTok{\# függő változók száma}
\NormalTok{n }\OtherTok{\textless{}{-}} \DecValTok{30} \CommentTok{\# mintaelemszám}
\NormalTok{k }\OtherTok{\textless{}{-}} \DecValTok{3}  \CommentTok{\# a független változó csoportjainak a száma}
  
\NormalTok{Wilks\_1 }\OtherTok{\textless{}{-}}  \DecValTok{1} \SpecialCharTok{/}\NormalTok{ (}\DecValTok{1} \SpecialCharTok{+}\NormalTok{ (F }\SpecialCharTok{*}\NormalTok{ p) }\SpecialCharTok{/}\NormalTok{ (n }\SpecialCharTok{{-}}\NormalTok{ k }\SpecialCharTok{{-}} \DecValTok{1} \SpecialCharTok{{-}}\NormalTok{ p))}
\NormalTok{Wilks\_1}

 \DecValTok{1} \SpecialCharTok{{-}}\NormalTok{ (F }\SpecialCharTok{/}\NormalTok{ (F }\SpecialCharTok{+} \DecValTok{27}\NormalTok{))}

\NormalTok{ahol F az F}\SpecialCharTok{{-}}\NormalTok{érték, df1 pedig az első szab}
\FunctionTok{anova}\NormalTok{(}\FunctionTok{lm}\NormalTok{(szelköt}\SpecialCharTok{\textasciitilde{}}\NormalTok{SUE, }\AttributeTok{data=}\NormalTok{vallalat), }\AttributeTok{test=}\StringTok{"Wilks"}\NormalTok{)}


\NormalTok{man\_1 }\OtherTok{\textless{}{-}} \FunctionTok{manova}\NormalTok{(}\FunctionTok{cbind}\NormalTok{(szelköt, elégedett, rendszer)}\SpecialCharTok{\textasciitilde{}}\NormalTok{SUE, }\AttributeTok{data=}\NormalTok{vallalat)}
\FunctionTok{summary}\NormalTok{(man\_1, }\AttributeTok{test=}\StringTok{"Wilks"}\NormalTok{)}
\FunctionTok{summary}\NormalTok{(man\_1)}


\FunctionTok{install.packages}\NormalTok{(}\StringTok{"klaR"}\NormalTok{)}
\NormalTok{gw\_1 }\OtherTok{\textless{}{-}}\NormalTok{ klaR}\SpecialCharTok{::}\FunctionTok{greedy.wilks}\NormalTok{(SUE }\SpecialCharTok{\textasciitilde{}}\NormalTok{ szelköt }\SpecialCharTok{+}\NormalTok{ elégedett }\SpecialCharTok{+}\NormalTok{ rendszer, }\AttributeTok{data =}\NormalTok{ vallalat, }\AttributeTok{output=}\NormalTok{T)}
\FunctionTok{unclass}\NormalTok{(gw\_1)}
\FunctionTok{plot}\NormalTok{(gw\_1)}

\NormalTok{jmv}\SpecialCharTok{::}\FunctionTok{mancova}\NormalTok{(}
    \AttributeTok{data =}\NormalTok{ vallalat,}
    \AttributeTok{deps =} \FunctionTok{vars}\NormalTok{(szelköt, elégedett, rendszer),}
    \AttributeTok{factors =}\NormalTok{ SUE,}
    \AttributeTok{multivar =} \StringTok{"wilks"}\NormalTok{,}
    \AttributeTok{boxM =} \ConstantTok{TRUE}\NormalTok{,}
    \AttributeTok{shapiro =} \ConstantTok{TRUE}\NormalTok{)}

\NormalTok{??}\StringTok{\textquotesingle{}Wilk\textquotesingle{}}
\NormalTok{rrcov}\SpecialCharTok{::}\FunctionTok{Wilks.test}\NormalTok{(SUE }\SpecialCharTok{\textasciitilde{}}\NormalTok{ szelköt }\SpecialCharTok{+}\NormalTok{ elégedett }\SpecialCharTok{+}\NormalTok{ rendszer, }\AttributeTok{data =}\NormalTok{ vallalat)}
\NormalTok{rrcov}\SpecialCharTok{::}\FunctionTok{Wilks.test}\NormalTok{(}\AttributeTok{x =}\NormalTok{ vallalat[}\DecValTok{2}\SpecialCharTok{:}\DecValTok{4}\NormalTok{], }\AttributeTok{grouping=}\NormalTok{vallalat}\SpecialCharTok{$}\NormalTok{SUE)}

\FunctionTok{library}\NormalTok{(klaR)}
\FunctionTok{data}\NormalTok{(iris)}
\FunctionTok{library}\NormalTok{(MASS)}
\NormalTok{iris.d }\OtherTok{\textless{}{-}}\NormalTok{ iris[,}\DecValTok{1}\SpecialCharTok{:}\DecValTok{4}\NormalTok{]  }\CommentTok{\# the data    }
\NormalTok{iris.c }\OtherTok{\textless{}{-}}\NormalTok{ iris[,}\DecValTok{5}\NormalTok{]    }\CommentTok{\# the classes }
\NormalTok{sc\_obj }\OtherTok{\textless{}{-}} \FunctionTok{stepclass}\NormalTok{(iris.d, iris.c, }\StringTok{"lda"}\NormalTok{, }\AttributeTok{start.vars =} \StringTok{"Sepal.Width"}\NormalTok{)}
\NormalTok{sc\_obj}
\FunctionTok{plot}\NormalTok{(sc\_obj)}

\DocumentationTok{\#\# or using formulas:}
\NormalTok{sc\_obj }\OtherTok{\textless{}{-}} \FunctionTok{stepclass}\NormalTok{(Species }\SpecialCharTok{\textasciitilde{}}\NormalTok{ ., }\AttributeTok{data =}\NormalTok{ iris, }\AttributeTok{method =} \StringTok{"qda"}\NormalTok{, }
    \AttributeTok{start.vars =} \StringTok{"Sepal.Width"}\NormalTok{, }\AttributeTok{criterion =} \StringTok{"AS"}\NormalTok{)  }\CommentTok{\# same as above }
\NormalTok{sc\_obj}
\end{Highlighting}
\end{Shaded}

\begin{Shaded}
\begin{Highlighting}[]
\NormalTok{data }\OtherTok{\textless{}{-}}\NormalTok{ rio}\SpecialCharTok{::}\FunctionTok{import}\NormalTok{(}\AttributeTok{file =} \StringTok{"adat/diszkriminancia\_alkalmassag.xlsx"}\NormalTok{)}
\NormalTok{data }\OtherTok{\textless{}{-}}\NormalTok{ rio}\SpecialCharTok{::}\FunctionTok{import}\NormalTok{(}\AttributeTok{file =} \StringTok{"adat/diszkriminancia\_baleset.xlsx"}\NormalTok{)}
\NormalTok{snowCluster}\SpecialCharTok{::}\FunctionTok{disc}\NormalTok{(}\AttributeTok{data =}\NormalTok{ data, }\AttributeTok{dep =}\NormalTok{ baleset, }\AttributeTok{covs =} \FunctionTok{vars}\NormalTok{(megosztott,}
\NormalTok{    pontossag, kockazat, eszleles), }\AttributeTok{gm =} \ConstantTok{TRUE}\NormalTok{, }\AttributeTok{coef =} \ConstantTok{TRUE}\NormalTok{, }\AttributeTok{prop =} \ConstantTok{TRUE}\NormalTok{,}
    \AttributeTok{tra =} \ConstantTok{TRUE}\NormalTok{, }\AttributeTok{plot =} \ConstantTok{TRUE}\NormalTok{, }\AttributeTok{plot1 =} \ConstantTok{TRUE}\NormalTok{, }\AttributeTok{plot2 =} \ConstantTok{TRUE}\NormalTok{)}


\NormalTok{iris}





\NormalTok{snowCluster}\SpecialCharTok{::}\FunctionTok{disc}\NormalTok{(}\AttributeTok{data =}\NormalTok{ iris, }\AttributeTok{dep =}\NormalTok{ Species, }\AttributeTok{covs =} \FunctionTok{vars}\NormalTok{(Sepal.Length,}
\NormalTok{    Sepal.Width, Petal.Length, Petal.Width), }\AttributeTok{gm =} \ConstantTok{TRUE}\NormalTok{, }\AttributeTok{coef =} \ConstantTok{TRUE}\NormalTok{,}
    \AttributeTok{prop =} \ConstantTok{TRUE}\NormalTok{, }\AttributeTok{tra =} \ConstantTok{TRUE}\NormalTok{, }\AttributeTok{plot =} \ConstantTok{TRUE}\NormalTok{, }\AttributeTok{plot1 =} \ConstantTok{TRUE}\NormalTok{, }\AttributeTok{plot2 =} \ConstantTok{TRUE}\NormalTok{)}
\end{Highlighting}
\end{Shaded}

\bookmarksetup{startatroot}

\hypertarget{sec-tobbvaltozos-varianciaelemzes}{%
\chapter{Többváltozós
varianciaelemzés}\label{sec-tobbvaltozos-varianciaelemzes}}

\hypertarget{elmuxe9leti-huxe1ttuxe9r}{%
\section{Elméleti háttér}\label{elmuxe9leti-huxe1ttuxe9r}}

A MANOVA a többváltozós varianciaelemzés angol megfelelőjéből képzett
betűszó (Multivariate ANOVA vagy Multivariate Analysis of Variance). A
szokásos ANOVA kiterjesztésének tekinthető, ahol nem egy, hanem kettő
vagy több függő változóval dolgozhatunk, de a cél ugyanaz: a független
változó több csoportja közötti különbségek elemzése.

Felmerülhet bennünk, hogy ha több függő változónk van, akkor mindegyikre
végezzünk el külön egy-egy hagyományos ANOVA-t, azonban ez az elsőfajú
hiba emelkedéséhez vezet. A MANOVA olyan megoldást kínál, amivel több
függő változó kombinált információi alapján képes kimutatni a
csoportkülönbségeket.

Mivel a MANOVA egynél több függő változót használ, a null- és az
ellenhipotézisek kissé megváltoznak:

\begin{itemize}
\tightlist
\item
  \(H_0\): A csoportok várható érték vektorai minden csoportban
  azonosak.
\item
  \(H_1\): A csoportok várható érték vektorainak legalább egyike eltér
  egy másiktól.
\end{itemize}

\hypertarget{puxe9lda-vezetuxe9si-programok-1}{%
\section{Példa: Vezetési
programok}\label{puxe9lda-vezetuxe9si-programok-1}}

Egy vállalat menedzsmentje szeretné megvizsgálni különböző vezetési
programok hatását, ezért három különböző vezetési programot vezetett be
három különböző stratégiai üzleti egységben (SÜE). Az első SÜE-ben
bevezetett program az egyenlőséget és az individualizmust hangsúlyozta.
A második SÜE-ben az egyenlőséget és a csoportmunkát helyzeték
középpontba. A harmadik SÜE-ben a bevezetett program egy nagyon
hierarchikus vezetési elvet alkalmazott. Később mindhárom SÜE
dolgozóinak körében felmérést végeztek, és a kérdések között szerepelt a
szervezettel való elkötelezettség mértéke (\texttt{szelkot}), a
szervezettel való elégedettség nagysága (\texttt{elegedett}), illetve a
rendszer egalitárius vagy tekintélyelvű (autokrata) jellege
(\texttt{rendszer}). Vizsgáljuk meg, hogy a SÜE három csoportja
azonosnak tekinthető-e a vizsgált 3 kérdésre (\texttt{szelkot},
\texttt{elegedett} és \texttt{rendszer}) adott válaszok tekintetében.

\begin{Shaded}
\begin{Highlighting}[]
\NormalTok{vezetes }\OtherTok{\textless{}{-}}\NormalTok{ rio}\SpecialCharTok{::}\FunctionTok{import}\NormalTok{(}\AttributeTok{file =} \StringTok{"adat/manova\_vezetesi\_program.xlsx"}\NormalTok{)}
\NormalTok{vezetes}\SpecialCharTok{$}\NormalTok{SUE }\OtherTok{\textless{}{-}} \FunctionTok{factor}\NormalTok{(vezetes}\SpecialCharTok{$}\NormalTok{SUE)}
\FunctionTok{str}\NormalTok{(vezetes)}
\CommentTok{\#\textgreater{} \textquotesingle{}data.frame\textquotesingle{}:    30 obs. of  4 variables:}
\CommentTok{\#\textgreater{}  $ SUE      : Factor w/ 3 levels "1","2","3": 1 1 1...}
\CommentTok{\#\textgreater{}  $ szelkot  : num  4 4 5 5 3 3 3 5 3 5 ...}
\CommentTok{\#\textgreater{}  $ elegedett: num  1 3 4 1 2 2 4 2 1 1 ...}
\CommentTok{\#\textgreater{}  $ rendszer : num  2 1 2 4 4 3 2 3 4 2 ...}
\NormalTok{psych}\SpecialCharTok{::}\FunctionTok{headTail}\NormalTok{(vezetes)}
\CommentTok{\#\textgreater{}      SUE szelkot elegedett rendszer}
\CommentTok{\#\textgreater{} 1      1       4         1        2}
\CommentTok{\#\textgreater{} 2      1       4         3        1}
\CommentTok{\#\textgreater{} 3      1       5         4        2}
\CommentTok{\#\textgreater{} 4      1       5         1        4}
\CommentTok{\#\textgreater{} ... \textless{}NA\textgreater{}     ...       ...      ...}
\CommentTok{\#\textgreater{} 27     3       1         1        5}
\CommentTok{\#\textgreater{} 28     3       5         1        4}
\CommentTok{\#\textgreater{} 29     3       5         1        5}
\CommentTok{\#\textgreater{} 30     3       5         3        4}
\end{Highlighting}
\end{Shaded}

Mivel a MANOVA a három stratégiai üzleti egységben (SÜE) a kérdőívek
pontszámainak (vagyis a függő változók) átlagainak különbségire kérdez
rá, így készítsünk dobozdiagramot mindhárom csoportban

\begin{Shaded}
\begin{Highlighting}[]
\FunctionTok{library}\NormalTok{(ggplot2)}
\NormalTok{p1 }\OtherTok{\textless{}{-}} \FunctionTok{ggplot}\NormalTok{(vezetes, }\FunctionTok{aes}\NormalTok{(}\AttributeTok{x =}\NormalTok{ SUE, }\AttributeTok{y =}\NormalTok{ szelkot, }\AttributeTok{fill =}\NormalTok{ SUE)) }\SpecialCharTok{+}
    \FunctionTok{geom\_boxplot}\NormalTok{() }\SpecialCharTok{+} \FunctionTok{theme}\NormalTok{(}\AttributeTok{legend.position =} \StringTok{"top"}\NormalTok{)}
\NormalTok{p2 }\OtherTok{\textless{}{-}} \FunctionTok{ggplot}\NormalTok{(vezetes, }\FunctionTok{aes}\NormalTok{(}\AttributeTok{x =}\NormalTok{ SUE, }\AttributeTok{y =}\NormalTok{ elegedett, }\AttributeTok{fill =}\NormalTok{ SUE)) }\SpecialCharTok{+}
    \FunctionTok{geom\_boxplot}\NormalTok{() }\SpecialCharTok{+} \FunctionTok{theme}\NormalTok{(}\AttributeTok{legend.position =} \StringTok{"top"}\NormalTok{)}
\NormalTok{p3 }\OtherTok{\textless{}{-}} \FunctionTok{ggplot}\NormalTok{(vezetes, }\FunctionTok{aes}\NormalTok{(}\AttributeTok{x =}\NormalTok{ SUE, }\AttributeTok{y =}\NormalTok{ rendszer, }\AttributeTok{fill =}\NormalTok{ SUE)) }\SpecialCharTok{+}
    \FunctionTok{geom\_boxplot}\NormalTok{() }\SpecialCharTok{+} \FunctionTok{theme}\NormalTok{(}\AttributeTok{legend.position =} \StringTok{"top"}\NormalTok{)}
\NormalTok{gridExtra}\SpecialCharTok{::}\FunctionTok{grid.arrange}\NormalTok{(p1, p2, p3, }\AttributeTok{nrow =} \DecValTok{1}\NormalTok{)}
\end{Highlighting}
\end{Shaded}

\begin{figure}[H]

{\centering \includegraphics{./sec_tobbvaltozos_variancia_files/figure-pdf/unnamed-chunk-3-1.pdf}

}

\end{figure}

Úgy tűnik, hogy a mindhárom csoport eléggé különbözik egymástól.

Végezzük el az egyszempontos többváltozós variancia elemzést. A
\texttt{manova()} függvény a \texttt{formula=} argumentumában a kettő
vagy több numerikus függő változót és legalább egy független változót
vár. A függő változókat most a \texttt{cbind()} függvénnyel fűztük
egymás mellé, a független változónk pedig a 3 szintű kategorikus
\texttt{SUE}.

\begin{Shaded}
\begin{Highlighting}[]
\NormalTok{man\_1 }\OtherTok{\textless{}{-}} \FunctionTok{manova}\NormalTok{(}\AttributeTok{formula =} \FunctionTok{cbind}\NormalTok{(szelkot, elegedett, rendszer) }\SpecialCharTok{\textasciitilde{}}
\NormalTok{    SUE, }\AttributeTok{data =}\NormalTok{ vezetes)}
\FunctionTok{summary}\NormalTok{(man\_1)}
\CommentTok{\#\textgreater{}           Df  Pillai approx F num Df den Df    Pr(\textgreater{}F)}
\CommentTok{\#\textgreater{} SUE        2 0.90947   7.2278      6     52 1.211e{-}05}
\CommentTok{\#\textgreater{} Residuals 27                                         }
\CommentTok{\#\textgreater{}              }
\CommentTok{\#\textgreater{} SUE       ***}
\CommentTok{\#\textgreater{} Residuals    }
\CommentTok{\#\textgreater{} {-}{-}{-}}
\CommentTok{\#\textgreater{} Signif. codes:  }
\CommentTok{\#\textgreater{} 0 \textquotesingle{}***\textquotesingle{} 0.001 \textquotesingle{}**\textquotesingle{} 0.01 \textquotesingle{}*\textquotesingle{} 0.05 \textquotesingle{}.\textquotesingle{} 0.1 \textquotesingle{} \textquotesingle{} 1}
\end{Highlighting}
\end{Shaded}

Alapértelmezés szerint a MANOVA az R-ben a Pillai-féle
tesztstatisztikáit használja. A p-érték gyakorlatilag nulla, ami azt
jelenti, hogy nyugodtan elvethetjük a nullhipotézist: legalább egy
csoportátlagvektor eltér a többitől.

Használhat más teszteket is, mint például a Wilk-lambda, a Roy-féle vagy
a Hotelling-Lawley statisztikákat, de a Pillai-féle a legrobusztosabb.

\begin{Shaded}
\begin{Highlighting}[]
\FunctionTok{summary}\NormalTok{(man\_1, }\AttributeTok{test =} \StringTok{"Wilks"}\NormalTok{)}
\CommentTok{\#\textgreater{}           Df  Wilks approx F num Df den Df    Pr(\textgreater{}F...}
\CommentTok{\#\textgreater{} SUE        2 0.1894   10.815      6     50 1.102e{-}0...}
\CommentTok{\#\textgreater{} Residuals 27                                       ...}
\CommentTok{\#\textgreater{} {-}{-}{-}}
\CommentTok{\#\textgreater{} Signif. codes:  }
\CommentTok{\#\textgreater{} 0 \textquotesingle{}***\textquotesingle{} 0.001 \textquotesingle{}**\textquotesingle{} 0.01 \textquotesingle{}*\textquotesingle{} 0.05 \textquotesingle{}.\textquotesingle{} 0.1 \textquotesingle{} \textquotesingle{} 1}
\FunctionTok{summary}\NormalTok{(man\_1, }\AttributeTok{test =} \StringTok{"Hotelling{-}Lawley"}\NormalTok{)}
\CommentTok{\#\textgreater{}           Df Hotelling{-}Lawley approx F num Df den Df}
\CommentTok{\#\textgreater{} SUE        2           3.7577   15.031      6     48}
\CommentTok{\#\textgreater{} Residuals 27                                        }
\CommentTok{\#\textgreater{}              Pr(\textgreater{}F)    }
\CommentTok{\#\textgreater{} SUE       1.375e{-}09 ***}
\CommentTok{\#\textgreater{} Residuals              }
\CommentTok{\#\textgreater{} {-}{-}{-}}
\CommentTok{\#\textgreater{} Signif. codes:  }
\CommentTok{\#\textgreater{} 0 \textquotesingle{}***\textquotesingle{} 0.001 \textquotesingle{}**\textquotesingle{} 0.01 \textquotesingle{}*\textquotesingle{} 0.05 \textquotesingle{}.\textquotesingle{} 0.1 \textquotesingle{} \textquotesingle{} 1}
\FunctionTok{summary}\NormalTok{(man\_1, }\AttributeTok{test =} \StringTok{"Roy"}\NormalTok{)}
\CommentTok{\#\textgreater{}           Df    Roy approx F num Df den Df    Pr(\textgreater{}F...}
\CommentTok{\#\textgreater{} SUE        2 3.6133   31.315      3     26 8.724e{-}0...}
\CommentTok{\#\textgreater{} Residuals 27                                       ...}
\CommentTok{\#\textgreater{} {-}{-}{-}}
\CommentTok{\#\textgreater{} Signif. codes:  }
\CommentTok{\#\textgreater{} 0 \textquotesingle{}***\textquotesingle{} 0.001 \textquotesingle{}**\textquotesingle{} 0.01 \textquotesingle{}*\textquotesingle{} 0.05 \textquotesingle{}.\textquotesingle{} 0.1 \textquotesingle{} \textquotesingle{} 1}
\end{Highlighting}
\end{Shaded}

A hatásnagyság kiszámítására MANOVA esetében a parciális Eta négyzet
\((\eta_p^2)\) mutatót használhatjuk. Azt méri, hogy a független változó
milyen hatással van a függő változókra. Ha az érték 0,14 vagy nagyobb,
akkor azt mondhatjuk, hogy a hatás mérete nagy. Ez most 0,45, ami azt
jelenti, hogy a hatás mérete nagy.

\begin{Shaded}
\begin{Highlighting}[]
\NormalTok{effectsize}\SpecialCharTok{::}\FunctionTok{eta\_squared}\NormalTok{(man\_1, }\AttributeTok{partial =}\NormalTok{ T)}
\CommentTok{\#\textgreater{} \# Effect Size for ANOVA (Type I)}
\CommentTok{\#\textgreater{} }
\CommentTok{\#\textgreater{} Parameter | Eta2 (partial) |       95\% CI}
\CommentTok{\#\textgreater{} {-}{-}{-}{-}{-}{-}{-}{-}{-}{-}{-}{-}{-}{-}{-}{-}{-}{-}{-}{-}{-}{-}{-}{-}{-}{-}{-}{-}{-}{-}{-}{-}{-}{-}{-}{-}{-}{-}{-}{-}{-}}
\CommentTok{\#\textgreater{} SUE       |           0.45 | [0.24, 1.00]}
\CommentTok{\#\textgreater{} }
\CommentTok{\#\textgreater{} {-} One{-}sided CIs: upper bound fixed at [1.00].}
\NormalTok{effectsize}\SpecialCharTok{::}\FunctionTok{interpret\_eta\_squared}\NormalTok{(}\FloatTok{0.45}\NormalTok{, }\AttributeTok{partial =}\NormalTok{ T)}
\CommentTok{\#\textgreater{} [1] "large"}
\CommentTok{\#\textgreater{} (Rules: field2013)}
\end{Highlighting}
\end{Shaded}

Mivel a MANOVA szignifikáns lett, további kérdés, hogy melyik csoport
átlagvektora különbözik a többitől? Post-hoc tesztet kell végeznünk,
amely esetünkben a lineáris diszkriminancia elemzés és az egyváltozós
ANOVA lesz.

\hypertarget{post-hoc-teszt-lda}{%
\section{Post-hoc teszt: LDA}\label{post-hoc-teszt-lda}}

A lineáris diszkriminancia elemzés (LDA) célja, hogy változók olyan
lineáris kombinációját találja meg, amely a legjobban elválaszt két vagy
több csoportot. Ezáltal képesek leszünk egy olyan pontdiagramot
megjeleníteni, amely az X és Y tengely két lineáris diszkriminánst
jeleníti meg, a pontokat pedig a független változónak (\texttt{SUE})
megfelelően fogjuk színezni.

A lineáris diszkriminancia elemzést R-ben a \texttt{\{MASS\}} csomag
\texttt{lda()} függvényével végzünk.

\begin{Shaded}
\begin{Highlighting}[]
\FunctionTok{library}\NormalTok{(MASS)}
\NormalTok{lda\_1 }\OtherTok{\textless{}{-}} \FunctionTok{lda}\NormalTok{(SUE }\SpecialCharTok{\textasciitilde{}}\NormalTok{ szelkot }\SpecialCharTok{+}\NormalTok{ elegedett }\SpecialCharTok{+}\NormalTok{ rendszer, }\AttributeTok{data =}\NormalTok{ vezetes)}
\NormalTok{lda\_1}
\CommentTok{\#\textgreater{} Call:}
\CommentTok{\#\textgreater{} lda(SUE \textasciitilde{} szelkot + elegedett + rendszer, data = ve...}
\CommentTok{\#\textgreater{} }
\CommentTok{\#\textgreater{} Prior probabilities of groups:}
\CommentTok{\#\textgreater{}         1         2         3 }
\CommentTok{\#\textgreater{} 0.3333333 0.3333333 0.3333333 }
\CommentTok{\#\textgreater{} }
\CommentTok{\#\textgreater{} Group means:}
\CommentTok{\#\textgreater{}   szelkot elegedett rendszer}
\CommentTok{\#\textgreater{} 1     4.0       2.1      2.7}
\CommentTok{\#\textgreater{} 2     4.7       3.4      1.5}
\CommentTok{\#\textgreater{} 3     4.1       1.6      4.7}
\CommentTok{\#\textgreater{} }
\CommentTok{\#\textgreater{} Coefficients of linear discriminants:}
\CommentTok{\#\textgreater{}                   LD1       LD2}
\CommentTok{\#\textgreater{} szelkot   {-}0.07056752 0.4959142}
\CommentTok{\#\textgreater{} elegedett  0.10193972 0.8596451}
\CommentTok{\#\textgreater{} rendszer  {-}1.32360357 0.5392379}
\CommentTok{\#\textgreater{} }
\CommentTok{\#\textgreater{} Proportion of trace:}
\CommentTok{\#\textgreater{}    LD1    LD2 }
\CommentTok{\#\textgreater{} 0.9616 0.0384}
\CommentTok{\# klaR::greedy.wilks(vasarlas\textasciitilde{}ertek+attitud+fizetes+kor,data=bio,}
\CommentTok{\# niveau = 0.15)}
\end{Highlighting}
\end{Shaded}

A fenti együtthatókból megtudhatjuk hogyan használják fel a függő
változókat az LDA döntési szabályának kialakítására. Az LD1 a
következőképpen számítható ki:

A \texttt{vezetes} adatmátrix numerikus változóira magunk is
kiszámolhatjuk az LD1 és LDA2 értékét a \texttt{predict()} függvénnyel:

\begin{Shaded}
\begin{Highlighting}[]
\NormalTok{lda\_1\_pred }\OtherTok{\textless{}{-}} \FunctionTok{predict}\NormalTok{(lda\_1, }\AttributeTok{method =} \StringTok{"plug{-}in"}\NormalTok{)}
\NormalTok{psych}\SpecialCharTok{::}\FunctionTok{headTail}\NormalTok{(lda\_1\_pred}\SpecialCharTok{$}\NormalTok{x)}
\CommentTok{\#\textgreater{}       LD1   LD2}
\CommentTok{\#\textgreater{} 1    1.16 {-}1.83}
\CommentTok{\#\textgreater{} 2    2.69 {-}0.65}
\CommentTok{\#\textgreater{} 3    1.39  1.25}
\CommentTok{\#\textgreater{} 4   {-}1.56 {-}0.25}
\CommentTok{\#\textgreater{} ...   ...   ...}
\CommentTok{\#\textgreater{} 27   {-}2.6  {-}1.7}
\CommentTok{\#\textgreater{} 28  {-}1.56 {-}0.25}
\CommentTok{\#\textgreater{} 29  {-}2.88  0.29}
\CommentTok{\#\textgreater{} 30  {-}1.35  1.47}
\end{Highlighting}
\end{Shaded}

A post-hoc teszt utolsó lépése a fenti a pontdiagram megjelenítése.
Ideális esetben egy vagy több csoport kiemelkedik:

\begin{Shaded}
\begin{Highlighting}[]
\NormalTok{d }\OtherTok{\textless{}{-}} \FunctionTok{data.frame}\NormalTok{(lda\_1\_pred}\SpecialCharTok{$}\NormalTok{x, }\AttributeTok{SUE =}\NormalTok{ vezetes}\SpecialCharTok{$}\NormalTok{SUE)}
\NormalTok{psych}\SpecialCharTok{::}\FunctionTok{headTail}\NormalTok{(d)}
\CommentTok{\#\textgreater{}       LD1   LD2  SUE}
\CommentTok{\#\textgreater{} 1    1.16 {-}1.83    1}
\CommentTok{\#\textgreater{} 2    2.69 {-}0.65    1}
\CommentTok{\#\textgreater{} 3    1.39  1.25    1}
\CommentTok{\#\textgreater{} 4   {-}1.56 {-}0.25    1}
\CommentTok{\#\textgreater{} ...   ...   ... \textless{}NA\textgreater{}}
\CommentTok{\#\textgreater{} 27   {-}2.6  {-}1.7    3}
\CommentTok{\#\textgreater{} 28  {-}1.56 {-}0.25    3}
\CommentTok{\#\textgreater{} 29  {-}2.88  0.29    3}
\CommentTok{\#\textgreater{} 30  {-}1.35  1.47    3}
\end{Highlighting}
\end{Shaded}

\begin{Shaded}
\begin{Highlighting}[]
\FunctionTok{ggplot}\NormalTok{(d, }\FunctionTok{aes}\NormalTok{(}\AttributeTok{x =}\NormalTok{ LD1, }\AttributeTok{y =}\NormalTok{ LD2, }\AttributeTok{colour =}\NormalTok{ SUE)) }\SpecialCharTok{+} \FunctionTok{geom\_point}\NormalTok{(}\AttributeTok{size =} \DecValTok{4}\NormalTok{)}
\end{Highlighting}
\end{Shaded}

\begin{figure}[H]

{\centering \includegraphics{./sec_tobbvaltozos_variancia_files/figure-pdf/unnamed-chunk-10-1.pdf}

}

\end{figure}

A képen látható, hogy a harmadik SÜE csoport eltér mindkét másik
csoporttól, míg a az első két csoport eltérése egymástól nem mondható
markánsnak. Könnyen elképzelhető, hogy a SÜE harmadik csoportja volt a
legnagyobb hatással a nullhipotézis elutasítására.

\hypertarget{post-hoc-test-egyvuxe1ltozuxf3s-vizsguxe1latok}{%
\section{Post-hoc test: egyváltozós
vizsgálatok}\label{post-hoc-test-egyvuxe1ltozuxf3s-vizsguxe1latok}}

A statisztikailag szignifikáns egyszempontos MANOVA után az egyváltozós
egyszempontos ANOVA-val is vizsgálódhatunk, amely minden függő változót
külön-külön vizsgál. A cél az, hogy azonosítsuk azokat a konkrét függő
változókat, amelyek hozzájárultak a jelentős globális hatáshoz. A
klasszikus ANOVA mellett a Welch-féle változat és a
Kruskal--Wallis-próba is használható, a feltételek egyre nagyobb
csorbulása esetén. Most a nemparaméteres Kruskal--Wallis-próbát
használjuk.

\begin{Shaded}
\begin{Highlighting}[]
\FunctionTok{kruskal.test}\NormalTok{(szelkot }\SpecialCharTok{\textasciitilde{}}\NormalTok{ SUE, }\AttributeTok{data =}\NormalTok{ vezetes)}\SpecialCharTok{$}\NormalTok{p.value}
\CommentTok{\#\textgreater{} [1] 0.2499069}
\FunctionTok{kruskal.test}\NormalTok{(elegedett }\SpecialCharTok{\textasciitilde{}}\NormalTok{ SUE, }\AttributeTok{data =}\NormalTok{ vezetes)}\SpecialCharTok{$}\NormalTok{p.value}
\CommentTok{\#\textgreater{} [1] 0.003003241}
\FunctionTok{kruskal.test}\NormalTok{(rendszer }\SpecialCharTok{\textasciitilde{}}\NormalTok{ SUE, }\AttributeTok{data =}\NormalTok{ vezetes)}\SpecialCharTok{$}\NormalTok{p.value}
\CommentTok{\#\textgreater{} [1] 1.690362e{-}05}
\end{Highlighting}
\end{Shaded}

Látjuk, hogy az \texttt{elegedett} és a \texttt{rendszer} függő
változókban nem egyeznek a várható értékek a SÜE egyes csoportjaiban.
Megjegyezzük, hogy mivel 3 függő változónk van, a Bonferroni-féle
többszörös tesztelési korrekciót alkalmaznunk kell, vagyis a
statisztikai szignifikancia szintet csökkenteni kell. Ez úgy történik,
hogy a klasszikus alfa szintet (0,05) elosztjuk a tesztek (vagy függő
változók, itt 3) számával. Ez p \textless{} 0,017-es szignifikancia
elfogadási kritériumhoz vezet. A fenti próbák szignifikáns voltán ez
most nem változtat.

A statisztikailag szignifikáns egyváltozós ANOVA-t (esetünkben
Kruskal--Wallis-próbát) többszörös páronkénti összehasonlítás követi
annak meghatározására, hogy mely csoportok különböznek egymástól. Most a
Kruskal--Wallis-próba szokásos utóvizsgálatát a Dunn-próbát fogjuk
használni.

\begin{Shaded}
\begin{Highlighting}[]
\FunctionTok{library}\NormalTok{(DescTools)}
\FunctionTok{DunnTest}\NormalTok{(}\AttributeTok{formula =}\NormalTok{ szelkot }\SpecialCharTok{\textasciitilde{}}\NormalTok{ SUE, }\AttributeTok{data =}\NormalTok{ vezetes, }\AttributeTok{method =} \StringTok{"holm"}\NormalTok{)}
\CommentTok{\#\textgreater{} }
\CommentTok{\#\textgreater{}  Dunn\textquotesingle{}s test of multiple comparisons using rank sum...}
\CommentTok{\#\textgreater{} }
\CommentTok{\#\textgreater{}     mean.rank.diff   pval    }
\CommentTok{\#\textgreater{} 2{-}1            5.6 0.3179    }
\CommentTok{\#\textgreater{} 3{-}1            4.0 0.4964    }
\CommentTok{\#\textgreater{} 3{-}2           {-}1.6 0.6442    }
\CommentTok{\#\textgreater{} {-}{-}{-}}
\CommentTok{\#\textgreater{} Signif. codes:  0 \textquotesingle{}***\textquotesingle{} 0.001 \textquotesingle{}**\textquotesingle{} 0.01 \textquotesingle{}*\textquotesingle{} 0.05 \textquotesingle{}....}
\FunctionTok{DunnTest}\NormalTok{(}\AttributeTok{formula =}\NormalTok{ elegedett }\SpecialCharTok{\textasciitilde{}}\NormalTok{ SUE, }\AttributeTok{data =}\NormalTok{ vezetes, }\AttributeTok{method =} \StringTok{"holm"}\NormalTok{)}
\CommentTok{\#\textgreater{} }
\CommentTok{\#\textgreater{}  Dunn\textquotesingle{}s test of multiple comparisons using rank sum...}
\CommentTok{\#\textgreater{} }
\CommentTok{\#\textgreater{}     mean.rank.diff   pval    }
\CommentTok{\#\textgreater{} 2{-}1           8.75 0.0410 *  }
\CommentTok{\#\textgreater{} 3{-}1          {-}3.80 0.3143    }
\CommentTok{\#\textgreater{} 3{-}2         {-}12.55 0.0027 ** }
\CommentTok{\#\textgreater{} {-}{-}{-}}
\CommentTok{\#\textgreater{} Signif. codes:  0 \textquotesingle{}***\textquotesingle{} 0.001 \textquotesingle{}**\textquotesingle{} 0.01 \textquotesingle{}*\textquotesingle{} 0.05 \textquotesingle{}....}
\FunctionTok{DunnTest}\NormalTok{(}\AttributeTok{formula =}\NormalTok{ rendszer }\SpecialCharTok{\textasciitilde{}}\NormalTok{ SUE, }\AttributeTok{data =}\NormalTok{ vezetes, }\AttributeTok{method =} \StringTok{"holm"}\NormalTok{)}
\CommentTok{\#\textgreater{} }
\CommentTok{\#\textgreater{}  Dunn\textquotesingle{}s test of multiple comparisons using rank sum...}
\CommentTok{\#\textgreater{} }
\CommentTok{\#\textgreater{}     mean.rank.diff    pval    }
\CommentTok{\#\textgreater{} 2{-}1          {-}6.95  0.0694 .  }
\CommentTok{\#\textgreater{} 3{-}1          10.85  0.0092 ** }
\CommentTok{\#\textgreater{} 3{-}2          17.80 9.9e{-}06 ***}
\CommentTok{\#\textgreater{} {-}{-}{-}}
\CommentTok{\#\textgreater{} Signif. codes:  0 \textquotesingle{}***\textquotesingle{} 0.001 \textquotesingle{}**\textquotesingle{} 0.01 \textquotesingle{}*\textquotesingle{} 0.05 \textquotesingle{}....}
\end{Highlighting}
\end{Shaded}

A fenti utóvizsgálatok világossá teszik, hogy a harmadik SÜE csoport a
\texttt{rendszer} változó esetén mindkét másik csoporttól, az
\texttt{elegedett} változó esetén pedig a második csoporttól
szignifikánsan eltér. Legjelentősebb mértékben tehűt a harmadik csoport
különül el a másik két csoporttól, tehát ez okozza a MANOVA
nullhipotézisének elvetését.

\hypertarget{elemzuxe9s-jamovi-ban}{%
\section{Elemzés jamovi-ban}\label{elemzuxe9s-jamovi-ban}}

A fenti elemzés jamovi-ban is elvégezhető az \texttt{ANOVA\ /\ MANCOVA}
menüpontok kiválasztásával.

\begin{figure}

{\centering \includegraphics{./images/manova_kep_01.jpg}

}

\caption{MANOVA jamovi-ban}

\end{figure}

\hypertarget{alkalmazuxe1si-feltuxe9telek-vizsguxe1lata}{%
\section{Alkalmazási feltételek
vizsgálata}\label{alkalmazuxe1si-feltuxe9telek-vizsguxe1lata}}

A MANOVA statisztikai próbának számos szigorú alkalmazási feltétele van.
Néhány az ANOVA-ból jön, például a megfigyelések függetlensége vagy a
variancia homogenitása, azonban vannak újdonságok is.

\begin{itemize}
\tightlist
\item
  \textbf{Megfelelő mintanagyság.} Ökölszabály: a mintaelemszám
  mindegyik független változó csoportban nagyobb az függő változók
  számánál.
\end{itemize}

\begin{Shaded}
\begin{Highlighting}[]
\NormalTok{summarytools}\SpecialCharTok{::}\FunctionTok{freq}\NormalTok{(vezetes}\SpecialCharTok{$}\NormalTok{SUE, }\AttributeTok{cumul =} \ConstantTok{FALSE}\NormalTok{)}
\CommentTok{\#\textgreater{} Frequencies  }
\CommentTok{\#\textgreater{} vezetes$SUE  }
\CommentTok{\#\textgreater{} Type: Factor  }
\CommentTok{\#\textgreater{} }
\CommentTok{\#\textgreater{}               Freq   \% Valid   \% Total}
\CommentTok{\#\textgreater{} {-}{-}{-}{-}{-}{-}{-}{-}{-}{-}{-} {-}{-}{-}{-}{-}{-} {-}{-}{-}{-}{-}{-}{-}{-}{-} {-}{-}{-}{-}{-}{-}{-}{-}{-}}
\CommentTok{\#\textgreater{}           1     10     33.33     33.33}
\CommentTok{\#\textgreater{}           2     10     33.33     33.33}
\CommentTok{\#\textgreater{}           3     10     33.33     33.33}
\CommentTok{\#\textgreater{}        \textless{}NA\textgreater{}      0                0.00}
\CommentTok{\#\textgreater{}       Total     30    100.00    100.00}
\end{Highlighting}
\end{Shaded}

Látható, hogy a függő változók számát (3) minden csoport elemszáma (10)
meghaladja.

\begin{itemize}
\item
  \textbf{A megfigyelések függetlensége.} Minden személynek csak egy
  csoportba kell tartoznia. Az egyes csoportok megfigyelései között
  nincs kapcsolat. Az ismételt mérések nem megengedettek. A minta
  kiválasztásának teljesen véletlenszerűnek kell lennie.
\item
  \textbf{Az egyváltozós vagy többváltozós kiugró értékek hiánya.}
\end{itemize}

Az egydimenziós kiugró értékek dobozdiagramokkal is ellenőrizhetők, ezt
korábban elvégeztük, láttuk csak egyetlen részcsoportban van kiugró
értékek (a \texttt{szelkot} változó változó esetén a SÜE harmadik
csoportjában). Használhatjuk a kényelmes
\texttt{rstatix::identify\_outliers()} függvényt is.

\begin{Shaded}
\begin{Highlighting}[]
\FunctionTok{library}\NormalTok{(tidyverse)}
\NormalTok{vezetes }\SpecialCharTok{\%\textgreater{}\%}
    \FunctionTok{group\_by}\NormalTok{(SUE) }\SpecialCharTok{\%\textgreater{}\%}
\NormalTok{    rstatix}\SpecialCharTok{::}\FunctionTok{identify\_outliers}\NormalTok{(szelkot)}
\CommentTok{\#\textgreater{} \# A tibble: 2 x 6}
\CommentTok{\#\textgreater{}   SUE   szelkot elegedett rendszer is.outlier is.ex...}
\CommentTok{\#\textgreater{}   \textless{}fct\textgreater{}   \textless{}dbl\textgreater{}     \textless{}dbl\textgreater{}    \textless{}dbl\textgreater{} \textless{}lgl\textgreater{}      \textless{}lgl\textgreater{}...}
\CommentTok{\#\textgreater{} 1 3           1         1        5 TRUE       TRUE ...}
\CommentTok{\#\textgreater{} 2 3           1         1        5 TRUE       TRUE}
\NormalTok{vezetes }\SpecialCharTok{\%\textgreater{}\%}
    \FunctionTok{group\_by}\NormalTok{(SUE) }\SpecialCharTok{\%\textgreater{}\%}
\NormalTok{    rstatix}\SpecialCharTok{::}\FunctionTok{identify\_outliers}\NormalTok{(elegedett)}
\CommentTok{\#\textgreater{} [1] SUE        szelkot    elegedett  rendszer  }
\CommentTok{\#\textgreater{} [5] is.outlier is.extreme}
\CommentTok{\#\textgreater{} \textless{}0 rows\textgreater{} (or 0{-}length row.names)}
\NormalTok{vezetes }\SpecialCharTok{\%\textgreater{}\%}
    \FunctionTok{group\_by}\NormalTok{(SUE) }\SpecialCharTok{\%\textgreater{}\%}
\NormalTok{    rstatix}\SpecialCharTok{::}\FunctionTok{identify\_outliers}\NormalTok{(rendszer)}
\CommentTok{\#\textgreater{} [1] SUE        szelkot    elegedett  rendszer  }
\CommentTok{\#\textgreater{} [5] is.outlier is.extreme}
\CommentTok{\#\textgreater{} \textless{}0 rows\textgreater{} (or 0{-}length row.names)}
\end{Highlighting}
\end{Shaded}

A többváltozós kiugró értékek olyan adatpontok, amelyek szokatlan
értékkombinációt tartalmaznak a kimeneti (vagy függő) változókon. A
Mahalanobis távolságot általában a többváltozós kiugró értékek
észlelésére használják. A távolság megmondja, milyen messze van egy
megfigyelés a felhő középpontjától, figyelembe véve a felhő alakját
(kovariancia) is. A \texttt{rstatix::mahalanobis\_distance()} függvény
könnyen használható a Mahalanobis-távolság kiszámítására és a
többváltozós kiugró értékek megjelölésére. A Mahalanobis-távolságot
csoportonként kell kiszámítani:

\begin{Shaded}
\begin{Highlighting}[]
\NormalTok{vezetes }\SpecialCharTok{\%\textgreater{}\%}
    \FunctionTok{group\_by}\NormalTok{(SUE) }\SpecialCharTok{\%\textgreater{}\%}
\NormalTok{    rstatix}\SpecialCharTok{::}\FunctionTok{mahalanobis\_distance}\NormalTok{() }\SpecialCharTok{\%\textgreater{}\%}
    \FunctionTok{filter}\NormalTok{(is.outlier }\SpecialCharTok{==} \ConstantTok{TRUE}\NormalTok{) }\SpecialCharTok{\%\textgreater{}\%}
    \FunctionTok{as.data.frame}\NormalTok{()}
\CommentTok{\#\textgreater{} [1] szelkot    elegedett  rendszer   mahal.dist}
\CommentTok{\#\textgreater{} [5] is.outlier}
\CommentTok{\#\textgreater{} \textless{}0 rows\textgreater{} (or 0{-}length row.names)}
\end{Highlighting}
\end{Shaded}

Látható, hogy nincs többváltozós kiugró érték az adatbázisban.

\begin{itemize}
\tightlist
\item
  \textbf{Többváltozós normalitás.}
\end{itemize}

A többváltozós normalitás Shapiro-Wilk tesztjének végrehajtása:

\begin{Shaded}
\begin{Highlighting}[]
\NormalTok{rstatix}\SpecialCharTok{::}\FunctionTok{mshapiro\_test}\NormalTok{(vezetes[}\FunctionTok{c}\NormalTok{(}\StringTok{"szelkot"}\NormalTok{, }\StringTok{"elegedett"}\NormalTok{, }\StringTok{"rendszer"}\NormalTok{)])}
\CommentTok{\#\textgreater{} \# A tibble: 1 x 2}
\CommentTok{\#\textgreater{}   statistic   p.value}
\CommentTok{\#\textgreater{}       \textless{}dbl\textgreater{}     \textless{}dbl\textgreater{}}
\CommentTok{\#\textgreater{} 1     0.772 0.0000214}
\end{Highlighting}
\end{Shaded}

Látható, hogy ez az alkalmazási feltétel nem teljesül.

Az egyváltozós normalitásokat is érdemes lehet tesztelni:

\begin{Shaded}
\begin{Highlighting}[]
\CommentTok{\# egyváltozós Shapiro–Wilk próba több csoportra}
\FunctionTok{library}\NormalTok{(onewaytests)}
\FunctionTok{nor.test}\NormalTok{(}\AttributeTok{formula =}\NormalTok{ szelkot }\SpecialCharTok{\textasciitilde{}}\NormalTok{ SUE, }\AttributeTok{data =}\NormalTok{ vezetes, }\AttributeTok{method =} \StringTok{"SW"}\NormalTok{)}
\CommentTok{\#\textgreater{} }
\CommentTok{\#\textgreater{}   Shapiro{-}Wilk Normality Test (alpha = 0.05) }
\CommentTok{\#\textgreater{} {-}{-}{-}{-}{-}{-}{-}{-}{-}{-}{-}{-}{-}{-}{-}{-}{-}{-}{-}{-}{-}{-}{-}{-}{-}{-}{-}{-}{-}{-}{-}{-}{-}{-}{-}{-}{-}{-}{-}{-}{-}{-}{-}{-}{-}{-}{-}{-}{-}{-} }
\CommentTok{\#\textgreater{}   data : szelkot and SUE }
\CommentTok{\#\textgreater{} }
\CommentTok{\#\textgreater{}   Level Statistic      p.value   Normality}
\CommentTok{\#\textgreater{} 1     1 0.7685823 6.009970e{-}03      Reject}
\CommentTok{\#\textgreater{} 2     2 0.5941735 4.713464e{-}05      Reject}
\CommentTok{\#\textgreater{} 3     3 0.5876023 3.936679e{-}05      Reject}
\CommentTok{\#\textgreater{} {-}{-}{-}{-}{-}{-}{-}{-}{-}{-}{-}{-}{-}{-}{-}{-}{-}{-}{-}{-}{-}{-}{-}{-}{-}{-}{-}{-}{-}{-}{-}{-}{-}{-}{-}{-}{-}{-}{-}{-}{-}{-}{-}{-}{-}{-}{-}{-}{-}{-}}
\FunctionTok{nor.test}\NormalTok{(}\AttributeTok{formula =}\NormalTok{ elegedett }\SpecialCharTok{\textasciitilde{}}\NormalTok{ SUE, }\AttributeTok{data =}\NormalTok{ vezetes, }\AttributeTok{method =} \StringTok{"SW"}\NormalTok{)}
\CommentTok{\#\textgreater{} }
\CommentTok{\#\textgreater{}   Shapiro{-}Wilk Normality Test (alpha = 0.05) }
\CommentTok{\#\textgreater{} {-}{-}{-}{-}{-}{-}{-}{-}{-}{-}{-}{-}{-}{-}{-}{-}{-}{-}{-}{-}{-}{-}{-}{-}{-}{-}{-}{-}{-}{-}{-}{-}{-}{-}{-}{-}{-}{-}{-}{-}{-}{-}{-}{-}{-}{-}{-}{-}{-}{-} }
\CommentTok{\#\textgreater{}   data : elegedett and SUE }
\CommentTok{\#\textgreater{} }
\CommentTok{\#\textgreater{}   Level Statistic      p.value   Normality}
\CommentTok{\#\textgreater{} 1     1 0.8236140 2.802300e{-}02      Reject}
\CommentTok{\#\textgreater{} 2     2 0.7172415 1.425861e{-}03      Reject}
\CommentTok{\#\textgreater{} 3     3 0.5941735 4.713464e{-}05      Reject}
\CommentTok{\#\textgreater{} {-}{-}{-}{-}{-}{-}{-}{-}{-}{-}{-}{-}{-}{-}{-}{-}{-}{-}{-}{-}{-}{-}{-}{-}{-}{-}{-}{-}{-}{-}{-}{-}{-}{-}{-}{-}{-}{-}{-}{-}{-}{-}{-}{-}{-}{-}{-}{-}{-}{-}}
\FunctionTok{nor.test}\NormalTok{(}\AttributeTok{formula =}\NormalTok{ rendszer }\SpecialCharTok{\textasciitilde{}}\NormalTok{ SUE, }\AttributeTok{data =}\NormalTok{ vezetes, }\AttributeTok{method =} \StringTok{"SW"}\NormalTok{)}
\CommentTok{\#\textgreater{} }
\CommentTok{\#\textgreater{}   Shapiro{-}Wilk Normality Test (alpha = 0.05) }
\CommentTok{\#\textgreater{} {-}{-}{-}{-}{-}{-}{-}{-}{-}{-}{-}{-}{-}{-}{-}{-}{-}{-}{-}{-}{-}{-}{-}{-}{-}{-}{-}{-}{-}{-}{-}{-}{-}{-}{-}{-}{-}{-}{-}{-}{-}{-}{-}{-}{-}{-}{-}{-}{-}{-} }
\CommentTok{\#\textgreater{}   data : rendszer and SUE }
\CommentTok{\#\textgreater{} }
\CommentTok{\#\textgreater{}   Level Statistic      p.value   Normality}
\CommentTok{\#\textgreater{} 1     1 0.8737456 1.105101e{-}01  Not reject}
\CommentTok{\#\textgreater{} 2     2 0.6552710 2.539627e{-}04      Reject}
\CommentTok{\#\textgreater{} 3     3 0.5941735 4.713464e{-}05      Reject}
\CommentTok{\#\textgreater{} {-}{-}{-}{-}{-}{-}{-}{-}{-}{-}{-}{-}{-}{-}{-}{-}{-}{-}{-}{-}{-}{-}{-}{-}{-}{-}{-}{-}{-}{-}{-}{-}{-}{-}{-}{-}{-}{-}{-}{-}{-}{-}{-}{-}{-}{-}{-}{-}{-}{-}}
\end{Highlighting}
\end{Shaded}

\begin{figure}[H]

{\centering \includegraphics{./sec_tobbvaltozos_variancia_files/figure-pdf/unnamed-chunk-17-1.pdf}

}

\end{figure}

\begin{figure}[H]

{\centering \includegraphics{./sec_tobbvaltozos_variancia_files/figure-pdf/unnamed-chunk-17-2.pdf}

}

\end{figure}

\begin{figure}[H]

{\centering \includegraphics{./sec_tobbvaltozos_variancia_files/figure-pdf/unnamed-chunk-17-3.pdf}

}

\end{figure}

\begin{itemize}
\tightlist
\item
  \textbf{A multikollinearitás hiánya.} A függő (eredmény) változók nem
  korrelálhatnak túlságosan egymással. Egyetlen korreláció sem lehet r =
  0,90 feletti.
\end{itemize}

Ideális esetben az eredményváltozók közötti korreláció mérsékelt, nem
túl magas. A 0,9 feletti korreláció a multikollinearitást jelzi, ami a
MANOVA esetében problematikus. Másrészt, ha a korreláció túl alacsony,
fontolóra kell vennie külön egyszempontos ANOVA futtatását minden függő
változóra.

Számítsuk ki a páronkénti Pearson-korrelációs együtthatókat a függő
változók között.

\begin{Shaded}
\begin{Highlighting}[]
\FunctionTok{cor}\NormalTok{(vezetes[}\FunctionTok{c}\NormalTok{(}\StringTok{"szelkot"}\NormalTok{, }\StringTok{"elegedett"}\NormalTok{, }\StringTok{"rendszer"}\NormalTok{)])}
\CommentTok{\#\textgreater{}              szelkot  elegedett   rendszer}
\CommentTok{\#\textgreater{} szelkot    1.0000000  0.1954881 {-}0.2528624}
\CommentTok{\#\textgreater{} elegedett  0.1954881  1.0000000 {-}0.6129079}
\CommentTok{\#\textgreater{} rendszer  {-}0.2528624 {-}0.6129079  1.0000000}
\end{Highlighting}
\end{Shaded}

Látható, hogy a korrelációs együtthatók nem támogatják a
multikollinearitás tényét.

\begin{itemize}
\tightlist
\item
  \textbf{Linearitás az összes függő változó között minden csoportban.}
\end{itemize}

Mivel a függő változók közötti páronkénti kapcsolatnak lineárisnak kell
lennie minden csoport esetében, ezért érdemes ezt a feltételt vizuálisan
ellenőrizni. A \texttt{\{GGally\}} csomag \texttt{ggpairs()} függvényét
használhatjuk.

\begin{Shaded}
\begin{Highlighting}[]
\FunctionTok{library}\NormalTok{(GGally)}
\NormalTok{res }\OtherTok{\textless{}{-}}\NormalTok{ vezetes }\SpecialCharTok{\%\textgreater{}\%}
    \FunctionTok{select}\NormalTok{(SUE, szelkot, elegedett, rendszer) }\SpecialCharTok{\%\textgreater{}\%}
    \FunctionTok{group\_by}\NormalTok{(SUE) }\SpecialCharTok{\%\textgreater{}\%}
\NormalTok{    rstatix}\SpecialCharTok{::}\FunctionTok{doo}\NormalTok{(}\SpecialCharTok{\textasciitilde{}}\FunctionTok{ggpairs}\NormalTok{(.) }\SpecialCharTok{+} \FunctionTok{theme\_bw}\NormalTok{(), }\AttributeTok{result =} \StringTok{"plots"}\NormalTok{)}
\NormalTok{res}\SpecialCharTok{$}\NormalTok{plots}
\CommentTok{\#\textgreater{} [[1]]}
\CommentTok{\#\textgreater{} }
\CommentTok{\#\textgreater{} [[2]]}
\CommentTok{\#\textgreater{} }
\CommentTok{\#\textgreater{} [[3]]}
\end{Highlighting}
\end{Shaded}

\begin{figure}[H]

{\centering \includegraphics{./sec_tobbvaltozos_variancia_files/figure-pdf/unnamed-chunk-19-1.pdf}

}

\end{figure}

\begin{figure}[H]

{\centering \includegraphics{./sec_tobbvaltozos_variancia_files/figure-pdf/unnamed-chunk-19-2.pdf}

}

\end{figure}

\begin{figure}[H]

{\centering \includegraphics{./sec_tobbvaltozos_variancia_files/figure-pdf/unnamed-chunk-19-3.pdf}

}

\end{figure}

A fenti ábrák megkérdőjelezik a páronkénti lineáris kapcsolatok
létezését.

\begin{itemize}
\tightlist
\item
  \textbf{A varianciák homogenitása.} A Levene-próba használható a
  csoportok közötti varianciák egyenlőségének tesztelésére. A
  Levene-próba nem szignifikáns értékei a varianciák homogenitását
  támogatják.
\end{itemize}

Az egyszempontos MANOVA mindegyik függő változó esetében azt
feltételezi, hogy a csoportok között egyenlők a varianciák.

\begin{Shaded}
\begin{Highlighting}[]
\NormalTok{DescTools}\SpecialCharTok{::}\FunctionTok{LeveneTest}\NormalTok{(szelkot }\SpecialCharTok{\textasciitilde{}}\NormalTok{ SUE, }\AttributeTok{data =}\NormalTok{ vezetes)}
\CommentTok{\#\textgreater{} Levene\textquotesingle{}s Test for Homogeneity of Variance (center =...}
\CommentTok{\#\textgreater{}       Df F value Pr(\textgreater{}F)}
\CommentTok{\#\textgreater{} group  2  0.9755 0.3899}
\CommentTok{\#\textgreater{}       27}
\NormalTok{DescTools}\SpecialCharTok{::}\FunctionTok{LeveneTest}\NormalTok{(elegedett }\SpecialCharTok{\textasciitilde{}}\NormalTok{ SUE, }\AttributeTok{data =}\NormalTok{ vezetes)}
\CommentTok{\#\textgreater{} Levene\textquotesingle{}s Test for Homogeneity of Variance (center =...}
\CommentTok{\#\textgreater{}       Df F value Pr(\textgreater{}F)}
\CommentTok{\#\textgreater{} group  2  0.4112  0.667}
\CommentTok{\#\textgreater{}       27}
\NormalTok{DescTools}\SpecialCharTok{::}\FunctionTok{LeveneTest}\NormalTok{(rendszer }\SpecialCharTok{\textasciitilde{}}\NormalTok{ SUE, }\AttributeTok{data =}\NormalTok{ vezetes)}
\CommentTok{\#\textgreater{} Levene\textquotesingle{}s Test for Homogeneity of Variance (center =...}
\CommentTok{\#\textgreater{}       Df F value   Pr(\textgreater{}F)   }
\CommentTok{\#\textgreater{} group  2     5.6 0.009238 **}
\CommentTok{\#\textgreater{}       27                    }
\CommentTok{\#\textgreater{} {-}{-}{-}}
\CommentTok{\#\textgreater{} Signif. codes:  }
\CommentTok{\#\textgreater{} 0 \textquotesingle{}***\textquotesingle{} 0.001 \textquotesingle{}**\textquotesingle{} 0.01 \textquotesingle{}*\textquotesingle{} 0.05 \textquotesingle{}.\textquotesingle{} 0.1 \textquotesingle{} \textquotesingle{} 1}
\end{Highlighting}
\end{Shaded}

Látható, hogy a szóráshomogenitás a \texttt{rendszer} változó
kivételével teljesül.

\begin{itemize}
\tightlist
\item
  \textbf{Variancia-kovariancia mátrixok homogenitása.} A BoxM-próba
  használható a csoportok közötti kovariancia egyenlőségének
  ellenőrzésére. Ez egyenértékű a variancia többváltozós
  homogenitásával. Ez a teszt rendkívül érzékenynek tekinthető. Ezért
  ennek a tesztnek a szignifikanciáját alfa = 0,001 értéknél határozzuk
  meg. A \texttt{\{biotools\}} csomag megvalósított \texttt{boxM()}
  függvényét használhatjuk.
\end{itemize}

\begin{Shaded}
\begin{Highlighting}[]
\NormalTok{biotools}\SpecialCharTok{::}\FunctionTok{boxM}\NormalTok{(}\AttributeTok{data =}\NormalTok{ vezetes[}\FunctionTok{c}\NormalTok{(}\StringTok{"szelkot"}\NormalTok{, }\StringTok{"elegedett"}\NormalTok{, }\StringTok{"rendszer"}\NormalTok{)],}
    \AttributeTok{grouping =}\NormalTok{ vezetes}\SpecialCharTok{$}\NormalTok{SUE)}
\CommentTok{\#\textgreater{} }
\CommentTok{\#\textgreater{}  Box\textquotesingle{}s M{-}test for Homogeneity of Covariance}
\CommentTok{\#\textgreater{}  Matrices}
\CommentTok{\#\textgreater{} }
\CommentTok{\#\textgreater{} data:  vezetes[c("szelkot", "elegedett", "rendszer")]}
\CommentTok{\#\textgreater{} Chi{-}Sq (approx.) = 19.607, df = 12, p{-}value =}
\CommentTok{\#\textgreater{} 0.0749}
\end{Highlighting}
\end{Shaded}

A teszt statisztikailag nem szignifikáns (azaz p \textgreater{} 0,001),
tehát az adatok nem sértették meg a variancia-kovariancia mátrixok
homogenitásának feltételezését.

Kiegyensúlyozott a csoportelemszámok esetén nem probléma a
variancia-kovariancia mátrixok homogenitásának megsértése miatt, de
kiegyensúlyozatlan kialakításnál már problémás lehet.

\bookmarksetup{startatroot}

\hypertarget{sec-logisztikus-regresszio}{%
\chapter{Logisztikus regresszió}\label{sec-logisztikus-regresszio}}

A logisztikus regresszió céljait tekintve megegyezik a diszkriminancia
elemzéssel, de sokkal robusztusabb, azaz kevesebb alkalmazási
feltétellel rendelkezik. Használható a logisztikus regresszió akkor is,
ha a független változók között kategorikus változók is előfordulnak,
illetve a normalitásra és homoszkedaszticitásra vonatkozó feltétel
megsértésre sem érzékeny a módszer.

A logisztikus regressziónak 3 típusa van:

\begin{itemize}
\tightlist
\item
  binomiális logisztikus regresszió: a függő változónk dichotóm, csak
  két értéke van,
\item
  multinominális logisztikus regresszió: a függő változónk olyan
  kategorikus változó, amelynek kettőnél több értéke van,
\item
  ordinális logisztikus regresszió: a függő változó ordinális skálán
  mért.
\end{itemize}

\begin{Shaded}
\begin{Highlighting}[]
\NormalTok{d }\OtherTok{\textless{}{-}}\NormalTok{ rio}\SpecialCharTok{::}\FunctionTok{import}\NormalTok{(}\AttributeTok{file =} \StringTok{"adat/logreg\_tanulo.sav"}\NormalTok{)}
\end{Highlighting}
\end{Shaded}

\begin{Shaded}
\begin{Highlighting}[]
\CommentTok{\# summarytools::ctable(x = d$HIVO2, y = d$HIVO01)}
\NormalTok{DescTools}\SpecialCharTok{::}\FunctionTok{Desc}\NormalTok{(NEME2 }\SpecialCharTok{\textasciitilde{}}\NormalTok{ HIVO01, }\AttributeTok{data =}\NormalTok{ d, }\AttributeTok{plotit =}\NormalTok{ F)}
\CommentTok{\#\textgreater{} {-}{-}{-}{-}{-}{-}{-}{-}{-}{-}{-}{-}{-}{-}{-}{-}{-}{-}{-}{-}{-}{-}{-}{-}{-}{-}{-}{-}{-}{-}{-}{-}{-}{-}{-}{-}{-}{-}{-}{-}{-}{-}{-}{-}{-}{-}{-}{-}{-}{-}{-}...}
\CommentTok{\#\textgreater{} NEME2 \textasciitilde{} HIVO01 (d)}
\CommentTok{\#\textgreater{} }
\CommentTok{\#\textgreater{} Summary: }
\CommentTok{\#\textgreater{} n: 1\textquotesingle{}717, rows: 2, columns: 2}
\CommentTok{\#\textgreater{} }
\CommentTok{\#\textgreater{} Pearson\textquotesingle{}s Chi{-}squared test (cont. adj):}
\CommentTok{\#\textgreater{}   X{-}squared = 42.748, df = 1, p{-}value = 6.225e{-}11}
\CommentTok{\#\textgreater{} Fisher\textquotesingle{}s exact test p{-}value = 5.219e{-}11}
\CommentTok{\#\textgreater{} McNemar\textquotesingle{}s chi{-}squared = 1.8435, df = 1, p{-}value = 0...}
\CommentTok{\#\textgreater{} }
\CommentTok{\#\textgreater{}                     estimate lwr.ci upr.ci\textquotesingle{}}
\CommentTok{\#\textgreater{}                                           }
\CommentTok{\#\textgreater{} odds ratio             1.925  1.583  2.341}
\CommentTok{\#\textgreater{} rel. risk (col1)       1.465  1.308  1.641}
\CommentTok{\#\textgreater{} rel. risk (col2)       0.761  0.699  0.828}
\CommentTok{\#\textgreater{} }
\CommentTok{\#\textgreater{} }
\CommentTok{\#\textgreater{} Contingency Coeff.     0.157}
\CommentTok{\#\textgreater{} Cramer\textquotesingle{}s V             0.159}
\CommentTok{\#\textgreater{} Kendall Tau{-}b          0.159}
\CommentTok{\#\textgreater{} }
\CommentTok{\#\textgreater{}                                    }
\CommentTok{\#\textgreater{}         HIVO01      0      1    Sum}
\CommentTok{\#\textgreater{} NEME2                              }
\CommentTok{\#\textgreater{}                                    }
\CommentTok{\#\textgreater{} 0       freq      366    370    736}
\CommentTok{\#\textgreater{}         perc    21.3\%  21.5\%  42.9\%}
\CommentTok{\#\textgreater{}         p.row   49.7\%  50.3\%      .}
\CommentTok{\#\textgreater{}         p.col   52.4\%  36.3\%      .}
\CommentTok{\#\textgreater{}                                    }
\CommentTok{\#\textgreater{} 1       freq      333    648    981}
\CommentTok{\#\textgreater{}         perc    19.4\%  37.7\%  57.1\%}
\CommentTok{\#\textgreater{}         p.row   33.9\%  66.1\%      .}
\CommentTok{\#\textgreater{}         p.col   47.6\%  63.7\%      .}
\CommentTok{\#\textgreater{}                                    }
\CommentTok{\#\textgreater{} Sum     freq      699  1\textquotesingle{}018  1\textquotesingle{}717}
\CommentTok{\#\textgreater{}         perc    40.7\%  59.3\% 100.0\%}
\CommentTok{\#\textgreater{}         p.row       .      .      .}
\CommentTok{\#\textgreater{}         p.col       .      .      .}
\CommentTok{\#\textgreater{}                                    }
\CommentTok{\#\textgreater{} }
\CommentTok{\#\textgreater{} {-}{-}{-}{-}{-}{-}{-}{-}{-}{-}}
\CommentTok{\#\textgreater{} \textquotesingle{} 95\% conf. level}
\end{Highlighting}
\end{Shaded}

\begin{Shaded}
\begin{Highlighting}[]
\NormalTok{lm\_1 }\OtherTok{\textless{}{-}} \FunctionTok{lm}\NormalTok{(HIVO01 }\SpecialCharTok{\textasciitilde{}}\NormalTok{ NEME2, }\AttributeTok{data =}\NormalTok{ d)}
\FunctionTok{summary}\NormalTok{(lm\_1)}
\CommentTok{\#\textgreater{} }
\CommentTok{\#\textgreater{} Call:}
\CommentTok{\#\textgreater{} lm(formula = HIVO01 \textasciitilde{} NEME2, data = d)}
\CommentTok{\#\textgreater{} }
\CommentTok{\#\textgreater{} Residuals:}
\CommentTok{\#\textgreater{}     Min      1Q  Median      3Q     Max }
\CommentTok{\#\textgreater{} {-}0.6605 {-}0.5027  0.3394  0.3394  0.4973 }
\CommentTok{\#\textgreater{} }
\CommentTok{\#\textgreater{} Coefficients:}
\CommentTok{\#\textgreater{}             Estimate Std. Error t value Pr(\textgreater{}|t|)    }
\CommentTok{\#\textgreater{} (Intercept)  0.50272    0.01789  28.101  \textless{} 2e{-}16 ***}
\CommentTok{\#\textgreater{} NEME2        0.15783    0.02367   6.669 3.47e{-}11 ***}
\CommentTok{\#\textgreater{} {-}{-}{-}}
\CommentTok{\#\textgreater{} Signif. codes:  }
\CommentTok{\#\textgreater{} 0 \textquotesingle{}***\textquotesingle{} 0.001 \textquotesingle{}**\textquotesingle{} 0.01 \textquotesingle{}*\textquotesingle{} 0.05 \textquotesingle{}.\textquotesingle{} 0.1 \textquotesingle{} \textquotesingle{} 1}
\CommentTok{\#\textgreater{} }
\CommentTok{\#\textgreater{} Residual standard error: 0.4853 on 1715 degrees of ...}
\CommentTok{\#\textgreater{} Multiple R{-}squared:  0.02528,    Adjusted R{-}squared:  ...}
\CommentTok{\#\textgreater{} F{-}statistic: 44.47 on 1 and 1715 DF,  p{-}value: 3.46...}
\end{Highlighting}
\end{Shaded}

Mindhárom fenti esetben a független változóink lehetnek kategorikusak és
folytonosak is.

\bookmarksetup{startatroot}

\hypertarget{sec-tobbdimenzios-skalazas}{%
\chapter{Többdimenziós skálázás}\label{sec-tobbdimenzios-skalazas}}

A többdimenziós skálázás (MDS) egy olyan többváltozós adatelemzési
módszer, amelyet a minták közötti hasonlóság/különbség megjelenítésére
használnak.

Az MDS a kiinduló \(p\) változó közötti hasonlóságot alacsonyabb, \(k\)
dimenziós térben ábrázolja. Legtöbbször a \(k = 2\) az optimális
választás, mert ekkor az objektumok helyét egy kétdimenziós
pontdiagramon láthatjuk.

Az MDS-algoritmus bemenetként az objektumpárok közötti távolságokat
reprezentáló hasonlósági mátrixot vár. Ezt R-ben a \texttt{dist()}
függvénnyel állíthatjuk elő.

\hypertarget{az-mds-algoritmusok-tuxedpusai}{%
\section{Az MDS algoritmusok
típusai}\label{az-mds-algoritmusok-tuxedpusai}}

Különféle MDS-algoritmusok léteznek:

\begin{itemize}
\item
  \textbf{Klasszikus többdimenziós skálázás.} Ez a módszer a lehető
  legjobban őrzi az eredeti távolságmértéket a pontok között. Az MDS
  térképen lévő illesztett távolságok és az eredeti távolságok
  ugyanabban a mértékegységben vannak kifejezve. A klasszikus MDS az
  úgynevezett metrikus többdimenziós skálázás kategóriába tartozik.
  Szokták főtengely-elemzésnek is nevezni, és jellemzően kvantitatív
  adatokra alkalmazzuk.
\item
  \textbf{Nem metrikus többdimenziós skálázás.} Ordinális MDS néven is
  ismert. Itt nem a távolságérték mérőszáma a fontos vagy értelmes,
  hanem az, hogy a többi objektumpár közötti távolságokhoz képes ez
  kisebb vagy nagyobb. Az ordinális MDS olyan illesztett távolságokat
  konstruál, amelyek az eredeti távolságokkal azonos rangsorrendben
  helyednek el. Például, ha az 1. és 5. objektumok távolsága az ötödik
  helyen áll az eredeti távolságadatokban, akkor az MDS-konfigurációban
  is az ötödik helyen kell szerepelniük. Ezt az algoritmust általában
  kategorikus adatokra alkalmazzuk.
\end{itemize}

\hypertarget{puxe9lda-svuxe1jci-tartomuxe1nyok}{%
\section{Példa: Svájci
tartományok}\label{puxe9lda-svuxe1jci-tartomuxe1nyok}}

A \texttt{swiss} adatbázis Svájc 47 francia nyelvű tartományának
termékenységi és társadalmi-gazdasági adatait tartalmazza. Az adatbázis
a \texttt{\{datasets\}} csomagból származik, további információ:
\texttt{?swiss}. A példa ötlete
\href{http://www.sthda.com/english/articles/31-principal-component-methods-in-r-practical-guide/122-multidimensional-scaling-essentials-algorithms-and-r-code/}{innen}
származik.

\begin{Shaded}
\begin{Highlighting}[]
\NormalTok{d }\OtherTok{\textless{}{-}}\NormalTok{ rio}\SpecialCharTok{::}\FunctionTok{import}\NormalTok{(}\AttributeTok{file =} \StringTok{"adat/mds\_swiss.xlsx"}\NormalTok{)}
\FunctionTok{str}\NormalTok{(d)}
\CommentTok{\#\textgreater{} \textquotesingle{}data.frame\textquotesingle{}:    47 obs. of  7 variables:}
\CommentTok{\#\textgreater{}  $ Fertility       : num  80.2 83.1 92.5 85.8 76.9 ...}
\CommentTok{\#\textgreater{}  $ Agriculture     : num  17 45.1 39.7 36.5 43.5 35...}
\CommentTok{\#\textgreater{}  $ Examination     : num  15 6 5 12 17 9 16 14 12 1...}
\CommentTok{\#\textgreater{}  $ Education       : num  12 9 5 7 15 7 7 8 7 13 ...}
\CommentTok{\#\textgreater{}  $ Catholic        : num  9.96 84.84 93.4 33.77 5.1...}
\CommentTok{\#\textgreater{}  $ Infant.Mortality: num  22.2 22.2 20.2 20.3 20.6 ...}
\CommentTok{\#\textgreater{}  $ province        : chr  "Courtelary" "Delemont" "...}
\NormalTok{psych}\SpecialCharTok{::}\FunctionTok{headTail}\NormalTok{(d)}
\CommentTok{\#\textgreater{}     Fertility Agriculture Examination Education Cat...}
\CommentTok{\#\textgreater{} 1        80.2          17          15        12    ...}
\CommentTok{\#\textgreater{} 2        83.1        45.1           6         9    ...}
\CommentTok{\#\textgreater{} 3        92.5        39.7           5         5    ...}
\CommentTok{\#\textgreater{} 4        85.8        36.5          12         7    ...}
\CommentTok{\#\textgreater{} ...       ...         ...         ...       ...    ...}
\CommentTok{\#\textgreater{} 44       67.6        18.7          25         7    ...}
\CommentTok{\#\textgreater{} 45         35         1.2          37        53    ...}
\CommentTok{\#\textgreater{} 46       44.7        46.6          16        29    ...}
\CommentTok{\#\textgreater{} 47       42.8        27.7          22        29    ...}
\CommentTok{\#\textgreater{}     Infant.Mortality     province}
\CommentTok{\#\textgreater{} 1               22.2   Courtelary}
\CommentTok{\#\textgreater{} 2               22.2     Delemont}
\CommentTok{\#\textgreater{} 3               20.2 Franches{-}Mnt}
\CommentTok{\#\textgreater{} 4               20.3      Moutier}
\CommentTok{\#\textgreater{} ...              ...         \textless{}NA\textgreater{}}
\CommentTok{\#\textgreater{} 44              19.5 ValdeTravers}
\CommentTok{\#\textgreater{} 45                18 V. De Geneve}
\CommentTok{\#\textgreater{} 46              18.2  Rive Droite}
\CommentTok{\#\textgreater{} 47              19.3  Rive Gauche}
\end{Highlighting}
\end{Shaded}

Az R-ben több függvény is rendelkezésre áll:

\begin{itemize}
\tightlist
\item
  \texttt{cmdscale()} - Klasszikus (metrikus) többdimenziós skálázás
  kiszámítása.
\item
  \texttt{MASS::isoMDS()} - A Kruskal nem metrikus többdimenziós
  skálázásának kiszámítása (a nem metrikus MDS egyik formája).
\item
  \texttt{MASS::sammon()} - Sammon nemlineáris leképezésének kiszámítása
  (a nem metrikus MDS egyik formája).
\end{itemize}

A fenti függvények egy távolságobjektumot várnak argumentumként, és
\texttt{k=} a kívánt dimenziószámot jelenti. Alapértelmezés szerint
kétdimenziós megoldással térnek vissza, de ezt meg tudjuk változtatni.

\begin{Shaded}
\begin{Highlighting}[]
\NormalTok{dist\_1 }\OtherTok{\textless{}{-}} \FunctionTok{dist}\NormalTok{(}\AttributeTok{x =}\NormalTok{ d, }\AttributeTok{method =} \StringTok{"euclidean"}\NormalTok{)}
\NormalTok{mds\_1 }\OtherTok{\textless{}{-}} \FunctionTok{cmdscale}\NormalTok{(dist\_1, }\AttributeTok{k =} \DecValTok{2}\NormalTok{)}
\NormalTok{mds\_1 }\OtherTok{\textless{}{-}} \FunctionTok{as.data.frame}\NormalTok{(mds\_1)}
\FunctionTok{names}\NormalTok{(mds\_1) }\OtherTok{\textless{}{-}} \FunctionTok{c}\NormalTok{(}\StringTok{"Dim.1"}\NormalTok{, }\StringTok{"Dim.2"}\NormalTok{)}
\CommentTok{\# Plot MDS}
\NormalTok{ggpubr}\SpecialCharTok{::}\FunctionTok{ggscatter}\NormalTok{(mds\_1, }\AttributeTok{x =} \StringTok{"Dim.1"}\NormalTok{, }\AttributeTok{y =} \StringTok{"Dim.2"}\NormalTok{, }\AttributeTok{label =}\NormalTok{ d}\SpecialCharTok{$}\NormalTok{province,}
    \AttributeTok{size =} \DecValTok{1}\NormalTok{, }\AttributeTok{repel =} \ConstantTok{TRUE}\NormalTok{)}
\end{Highlighting}
\end{Shaded}

\begin{figure}[H]

{\centering \includegraphics{./sec_tobbdimenzios_skalazas_files/figure-pdf/unnamed-chunk-3-1.pdf}

}

\end{figure}

Hozzunk létre 3 csoportot a k-közép eljárással.

\begin{Shaded}
\begin{Highlighting}[]
\FunctionTok{library}\NormalTok{(magrittr)}
\CommentTok{\# K{-}közep klaszter}
\NormalTok{mds\_1}\SpecialCharTok{$}\NormalTok{groups }\OtherTok{\textless{}{-}} \FunctionTok{kmeans}\NormalTok{(mds\_1, }\DecValTok{3}\NormalTok{)}\SpecialCharTok{$}\NormalTok{cluster }\SpecialCharTok{\%\textgreater{}\%}
    \FunctionTok{as.factor}\NormalTok{()}
\CommentTok{\# Plot and color by groups}
\NormalTok{ggpubr}\SpecialCharTok{::}\FunctionTok{ggscatter}\NormalTok{(mds\_1, }\AttributeTok{x =} \StringTok{"Dim.1"}\NormalTok{, }\AttributeTok{y =} \StringTok{"Dim.2"}\NormalTok{, }\AttributeTok{label =} \FunctionTok{rownames}\NormalTok{(swiss),}
    \AttributeTok{color =} \StringTok{"groups"}\NormalTok{, }\AttributeTok{palette =} \StringTok{"jco"}\NormalTok{, }\AttributeTok{size =} \DecValTok{1}\NormalTok{, }\AttributeTok{ellipse =} \ConstantTok{TRUE}\NormalTok{,}
    \AttributeTok{ellipse.type =} \StringTok{"convex"}\NormalTok{, }\AttributeTok{repel =} \ConstantTok{TRUE}\NormalTok{)}
\end{Highlighting}
\end{Shaded}

\begin{figure}[H]

{\centering \includegraphics{./sec_tobbdimenzios_skalazas_files/figure-pdf/unnamed-chunk-4-1.pdf}

}

\end{figure}

Jamovi-ban a fenti lépések végrehajtásáshoz a \texttt{snowCluster}
csomagot kell telepíteni, majd a megjelenő \texttt{snowCluster} menüből
a \texttt{Multidimensional\ Scaling\ Plot} almenüpontot kell
kiválasztani.

\begin{figure}

{\centering \includegraphics{./images/mds_swiss_kep_01.jpg}

}

\caption{Többdimenziós skálázás jamovi-ban}

\end{figure}

\hypertarget{korreluxe1ciuxf3s-muxe1trix-megjelenuxedtuxe9se-tuxf6bbdimenziuxf3s-skuxe1luxe1zuxe1ssal}{%
\section{Korrelációs mátrix megjelenítése többdimenziós
skálázással}\label{korreluxe1ciuxf3s-muxe1trix-megjelenuxedtuxe9se-tuxf6bbdimenziuxf3s-skuxe1luxe1zuxe1ssal}}

Az MDS a korrelációs mátrix vizsgálatára is alkalmas, rejtett mintázat
felfedésére is használhatjuk.

A korreláció valójában a hasonlóságot méri, de könnyen átalakítható az
eltérés (távolság) jellegű mértékké. Az objektumok közötti távolság:
\texttt{1\ -\ res.cor}. A példában a
\texttt{faktor\_real\_human\_targyak.xlsx} adatbázist használjuk
(Münnich és mtsai., 2006) {[}3.7.1 Probléma{]}.

\begin{Shaded}
\begin{Highlighting}[]
\NormalTok{d }\OtherTok{\textless{}{-}}\NormalTok{ rio}\SpecialCharTok{::}\FunctionTok{import}\NormalTok{(}\AttributeTok{file =} \StringTok{"adat/faktor\_real\_human\_targyak.xlsx"}\NormalTok{)}
\FunctionTok{str}\NormalTok{(d)}
\CommentTok{\#\textgreater{} \textquotesingle{}data.frame\textquotesingle{}:    30 obs. of  6 variables:}
\CommentTok{\#\textgreater{}  $ matek      : num  5 4 3 2 5 1 5 2 5 5 ...}
\CommentTok{\#\textgreater{}  $ informatika: num  4 4 4 2 5 1 5 2 5 4 ...}
\CommentTok{\#\textgreater{}  $ kemia      : num  5 5 3 3 5 1 5 3 5 5 ...}
\CommentTok{\#\textgreater{}  $ irodalom   : num  5 4 2 5 3 5 3 5 4 2 ...}
\CommentTok{\#\textgreater{}  $ nyelvtan   : num  4 4 2 5 3 5 3 5 5 2 ...}
\CommentTok{\#\textgreater{}  $ angol      : num  5 5 3 5 3 5 3 5 5 2 ...}
\NormalTok{psych}\SpecialCharTok{::}\FunctionTok{headTail}\NormalTok{(d)}
\CommentTok{\#\textgreater{}     matek informatika kemia irodalom nyelvtan angol}
\CommentTok{\#\textgreater{} 1       5           4     5        5        4     5}
\CommentTok{\#\textgreater{} 2       4           4     5        4        4     5}
\CommentTok{\#\textgreater{} 3       3           4     3        2        2     3}
\CommentTok{\#\textgreater{} 4       2           2     3        5        5     5}
\CommentTok{\#\textgreater{} ...   ...         ...   ...      ...      ...   ...}
\CommentTok{\#\textgreater{} 27      5           5     5        2        2     3}
\CommentTok{\#\textgreater{} 28      5           5     5        4        4     4}
\CommentTok{\#\textgreater{} 29      2           2     3        4        5     5}
\CommentTok{\#\textgreater{} 30      5           5     5        4        5     5}
\end{Highlighting}
\end{Shaded}

\begin{Shaded}
\begin{Highlighting}[]
\NormalTok{res.cor }\OtherTok{\textless{}{-}} \FunctionTok{cor}\NormalTok{(d, }\AttributeTok{method =} \StringTok{"spearman"}\NormalTok{)}
\NormalTok{mds.cor }\OtherTok{\textless{}{-}}\NormalTok{ (}\DecValTok{1} \SpecialCharTok{{-}}\NormalTok{ res.cor) }\SpecialCharTok{\%\textgreater{}\%}
    \FunctionTok{cmdscale}\NormalTok{() }\SpecialCharTok{\%\textgreater{}\%}
    \FunctionTok{as.data.frame}\NormalTok{()}
\FunctionTok{colnames}\NormalTok{(mds.cor) }\OtherTok{\textless{}{-}} \FunctionTok{c}\NormalTok{(}\StringTok{"Dim.1"}\NormalTok{, }\StringTok{"Dim.2"}\NormalTok{)}
\NormalTok{ggpubr}\SpecialCharTok{::}\FunctionTok{ggscatter}\NormalTok{(mds.cor, }\AttributeTok{x =} \StringTok{"Dim.1"}\NormalTok{, }\AttributeTok{y =} \StringTok{"Dim.2"}\NormalTok{, }\AttributeTok{size =} \DecValTok{1}\NormalTok{,}
    \AttributeTok{label =} \FunctionTok{colnames}\NormalTok{(res.cor), }\AttributeTok{repel =} \ConstantTok{TRUE}\NormalTok{)}
\end{Highlighting}
\end{Shaded}

\begin{figure}[H]

{\centering \includegraphics{./sec_tobbdimenzios_skalazas_files/figure-pdf/unnamed-chunk-6-1.pdf}

}

\end{figure}

A pozitívan korreláló objektumok közel vannak egymáshoz, ugyanazon
oldalon (bal vagy jobb).

\hypertarget{az-mds-uxe9s-a-pca-uxf6sszehasonluxedtuxe1sa}{%
\section{Az MDS és a PCA
összehasonlítása}\label{az-mds-uxe9s-a-pca-uxf6sszehasonluxedtuxe1sa}}

Az MDS és a dimenzió-redukciós módszerek (például a főkomponens elemzés
és a faktoranalízis) között matematikailag és fogalmilag szoros
összefüggés van .

A PCA inkább magukra a dimenziókra összpontosít, és a megmagyarázott
variancia maximalizálására törekszik, míg az MDS inkább a skálázott
objektumok közötti kapcsolatokra összpontosít.

Az MDS n-dimenziós adatpontokat vetít ki egy (általában) 2-dimenziós
síkba úgy, hogy az n-dimenziós térben lévő hasonló objektumok közel
lesznek egymáshoz a kétdimenziós diagramon is, míg a PCA többdimenziós
teret vetít a maximális variancia irányába a korrelációs/kovariancia
mátrix elemzésével.

\hypertarget{puxe9lda-magyar-vuxe1rosok}{%
\section{Példa: Magyar városok}\label{puxe9lda-magyar-vuxe1rosok}}

A példa a magyar városokat elhelyezkedését vizsgálja a térképen (Münnich
és mtsai., 2006) {[}6.1. R-forráskód{]}. Az adatok már eleve
távolságmátrixban vannak reprezentálva
(\texttt{mds\_varos\_tavolsagmatrix.xlsx}), ahol az egyes cellák a
városok közti légvonalbeli távolságot listázzák. Mivel a többdimenziós
skálázást eredetileg a térképészetben használták térképek rajzolására,
ebben a példában Magyarország nagyobb városait jelenítjük meg egy
kétdimenziós térképen.

\begin{Shaded}
\begin{Highlighting}[]
\NormalTok{d }\OtherTok{\textless{}{-}}\NormalTok{ rio}\SpecialCharTok{::}\FunctionTok{import}\NormalTok{(}\AttributeTok{file =} \StringTok{"adat/mds\_varos\_tavolsagmatrix.xlsx"}\NormalTok{)}
\FunctionTok{str}\NormalTok{(d)}
\CommentTok{\#\textgreater{} \textquotesingle{}data.frame\textquotesingle{}:    10 obs. of  11 variables:}
\CommentTok{\#\textgreater{}  $ VAROSNEV: chr  "Budapest" "Gyor" "Tatab" "Szhely...}
\CommentTok{\#\textgreater{}  $ BUDAPEST: num  0 114 52 185 190 160 157 190 204 135}
\CommentTok{\#\textgreater{}  $ GYŐR    : num  NA 0 60 95 113 148 248 304 305 233}
\CommentTok{\#\textgreater{}  $ TATAB   : num  NA NA 0 144 144 140 193 243 251 182}
\CommentTok{\#\textgreater{}  $ SZHELY  : num  NA NA NA 0 45 126 289 381 392 330}
\CommentTok{\#\textgreater{}  $ ZALAE   : num  NA NA NA NA 0 90 262 375 391 330}
\CommentTok{\#\textgreater{}  $ KAPOSVAR: num  NA NA NA NA NA 0 183 322 345 297}
\CommentTok{\#\textgreater{}  $ SZEGED  : num  NA NA NA NA NA NA 0 179 220 208}
\CommentTok{\#\textgreater{}  $ DEBRECEN: num  NA NA NA NA NA NA NA 0 44 91}
\CommentTok{\#\textgreater{}  $ NYHAZA  : num  NA NA NA NA NA NA NA NA 0 72}
\CommentTok{\#\textgreater{}  $ MISKOLC : num  NA NA NA NA NA NA NA NA NA 0}
\NormalTok{psych}\SpecialCharTok{::}\FunctionTok{headTail}\NormalTok{(d)}
\CommentTok{\#\textgreater{}     VAROSNEV BUDAPEST GYŐR TATAB SZHELY ZALAE KAPOSVAR}
\CommentTok{\#\textgreater{} 1   Budapest        0 \textless{}NA\textgreater{}  \textless{}NA\textgreater{}   \textless{}NA\textgreater{}  \textless{}NA\textgreater{}     \textless{}NA\textgreater{}}
\CommentTok{\#\textgreater{} 2       Gyor      114    0  \textless{}NA\textgreater{}   \textless{}NA\textgreater{}  \textless{}NA\textgreater{}     \textless{}NA\textgreater{}}
\CommentTok{\#\textgreater{} 3      Tatab       52   60     0   \textless{}NA\textgreater{}  \textless{}NA\textgreater{}     \textless{}NA\textgreater{}}
\CommentTok{\#\textgreater{} 4     Szhely      185   95   144      0  \textless{}NA\textgreater{}     \textless{}NA\textgreater{}}
\CommentTok{\#\textgreater{} ...     \textless{}NA\textgreater{}      ...  ...   ...    ...   ...      ...}
\CommentTok{\#\textgreater{} 7     Szeged      157  248   193    289   262      183}
\CommentTok{\#\textgreater{} 8   Debrecen      190  304   243    381   375      322}
\CommentTok{\#\textgreater{} 9     Nyhaza      204  305   251    392   391      345}
\CommentTok{\#\textgreater{} 10   Miskolc      135  233   182    330   330      297}
\CommentTok{\#\textgreater{}     SZEGED DEBRECEN NYHAZA MISKOLC}
\CommentTok{\#\textgreater{} 1     \textless{}NA\textgreater{}     \textless{}NA\textgreater{}   \textless{}NA\textgreater{}    \textless{}NA\textgreater{}}
\CommentTok{\#\textgreater{} 2     \textless{}NA\textgreater{}     \textless{}NA\textgreater{}   \textless{}NA\textgreater{}    \textless{}NA\textgreater{}}
\CommentTok{\#\textgreater{} 3     \textless{}NA\textgreater{}     \textless{}NA\textgreater{}   \textless{}NA\textgreater{}    \textless{}NA\textgreater{}}
\CommentTok{\#\textgreater{} 4     \textless{}NA\textgreater{}     \textless{}NA\textgreater{}   \textless{}NA\textgreater{}    \textless{}NA\textgreater{}}
\CommentTok{\#\textgreater{} ...    ...      ...    ...     ...}
\CommentTok{\#\textgreater{} 7        0     \textless{}NA\textgreater{}   \textless{}NA\textgreater{}    \textless{}NA\textgreater{}}
\CommentTok{\#\textgreater{} 8      179        0   \textless{}NA\textgreater{}    \textless{}NA\textgreater{}}
\CommentTok{\#\textgreater{} 9      220       44      0    \textless{}NA\textgreater{}}
\CommentTok{\#\textgreater{} 10     208       91     72       0}
\end{Highlighting}
\end{Shaded}

\begin{Shaded}
\begin{Highlighting}[]
\NormalTok{dist }\OtherTok{\textless{}{-}} \FunctionTok{as.dist}\NormalTok{(d[}\DecValTok{2}\SpecialCharTok{:}\DecValTok{11}\NormalTok{])}
\NormalTok{dist}
\CommentTok{\#\textgreater{}      1   2   3   4   5   6   7   8   9}
\CommentTok{\#\textgreater{} 2  114                                }
\CommentTok{\#\textgreater{} 3   52  60                            }
\CommentTok{\#\textgreater{} 4  185  95 144                        }
\CommentTok{\#\textgreater{} 5  190 113 144  45                    }
\CommentTok{\#\textgreater{} 6  160 148 140 126  90                }
\CommentTok{\#\textgreater{} 7  157 248 193 289 262 183            }
\CommentTok{\#\textgreater{} 8  190 304 243 381 375 322 179        }
\CommentTok{\#\textgreater{} 9  204 305 251 392 391 345 220  44    }
\CommentTok{\#\textgreater{} 10 135 233 182 330 330 297 208  91  72}
\end{Highlighting}
\end{Shaded}

Az elkészült távolságmátrix ismeretében már lefuttathatjuk a nem
metrikus többdimenziós skálázást az R statisztikai program segítségével.

\begin{Shaded}
\begin{Highlighting}[]
\NormalTok{mds\_1 }\OtherTok{\textless{}{-}}\NormalTok{ MASS}\SpecialCharTok{::}\FunctionTok{isoMDS}\NormalTok{(dist, }\AttributeTok{k =} \DecValTok{2}\NormalTok{)}
\CommentTok{\#\textgreater{} initial  value 0.364001 }
\CommentTok{\#\textgreater{} iter   5 value 0.114745}
\CommentTok{\#\textgreater{} iter  10 value 0.037146}
\CommentTok{\#\textgreater{} final  value 0.005866 }
\CommentTok{\#\textgreater{} converged}
\NormalTok{mds\_1}\SpecialCharTok{$}\NormalTok{points}
\CommentTok{\#\textgreater{}           [,1]      [,2]}
\CommentTok{\#\textgreater{} 1     9.578911 {-}19.29945}
\CommentTok{\#\textgreater{} 2   {-}95.520939 {-}68.48619}
\CommentTok{\#\textgreater{} 3   {-}39.393870 {-}37.74982}
\CommentTok{\#\textgreater{} 4  {-}180.860153 {-}32.75667}
\CommentTok{\#\textgreater{} 5  {-}178.931803  11.19081}
\CommentTok{\#\textgreater{} 6  {-}121.302621  77.85193}
\CommentTok{\#\textgreater{} 7    55.462602 130.86356}
\CommentTok{\#\textgreater{} 8   196.837303  18.00145}
\CommentTok{\#\textgreater{} 9   210.243678 {-}22.43017}
\CommentTok{\#\textgreater{} 10  143.886892 {-}57.18545}
\end{Highlighting}
\end{Shaded}

A fenti output mutatja a kapott kétdimenziós megoldást. Az egyes
oszlopok az elemek első illetve második dimenzióbeli értékeit mutatja.
Mivel a többdimenziós skálázásban fontos cél az adatok grafikus
ábrázolása is, ezeket az értékeket kezelhetjük koordinátákként, melyek
segítségével rajzolhatunk egy kétdimenziós térképet.

\begin{Shaded}
\begin{Highlighting}[]
\NormalTok{mds\_data }\OtherTok{\textless{}{-}} \FunctionTok{as.data.frame}\NormalTok{(mds\_1}\SpecialCharTok{$}\NormalTok{points)}
\NormalTok{psych}\SpecialCharTok{::}\FunctionTok{headTail}\NormalTok{(mds\_data)}
\CommentTok{\#\textgreater{}          V1     V2}
\CommentTok{\#\textgreater{} 1      9.58  {-}19.3}
\CommentTok{\#\textgreater{} 2    {-}95.52 {-}68.49}
\CommentTok{\#\textgreater{} 3    {-}39.39 {-}37.75}
\CommentTok{\#\textgreater{} 4   {-}180.86 {-}32.76}
\CommentTok{\#\textgreater{} ...     ...    ...}
\CommentTok{\#\textgreater{} 7     55.46 130.86}
\CommentTok{\#\textgreater{} 8    196.84     18}
\CommentTok{\#\textgreater{} 9    210.24 {-}22.43}
\CommentTok{\#\textgreater{} 10   143.89 {-}57.19}
\end{Highlighting}
\end{Shaded}

\begin{Shaded}
\begin{Highlighting}[]
\FunctionTok{library}\NormalTok{(ggplot2)}
\FunctionTok{library}\NormalTok{(ggrepel)}
\FunctionTok{ggplot}\NormalTok{(mds\_data, }\FunctionTok{aes}\NormalTok{(}\AttributeTok{x =}\NormalTok{ V1, }\AttributeTok{y =}\NormalTok{ V2)) }\SpecialCharTok{+} \FunctionTok{geom\_point}\NormalTok{() }\SpecialCharTok{+} \FunctionTok{geom\_label\_repel}\NormalTok{(}\AttributeTok{label =}\NormalTok{ d}\SpecialCharTok{$}\NormalTok{VAROSNEV)}
\end{Highlighting}
\end{Shaded}

\begin{figure}[H]

{\centering \includegraphics{./sec_tobbdimenzios_skalazas_files/figure-pdf/unnamed-chunk-11-1.pdf}

}

\end{figure}

A fenti outputban a magát a térképet kaphatjuk meg. Az egyetlen
furcsaság a kapott térképen az, hogy az észak-dél irány fordítva van.
Ennek oka, hogy a módszer az egyes objektumok egymáshoz való viszonyát
modellezi, ám a koordináta-tengelyek iránya és helye változhat.
Természetesen megkaphatjuk a ``valódi'' Magyarország térképet is.

Ennek megoldásához csupán meg kell szoroznunk a második dimenziót
(-1)-gyel.

\begin{Shaded}
\begin{Highlighting}[]
\NormalTok{mds\_data}\SpecialCharTok{$}\NormalTok{V2 }\OtherTok{\textless{}{-}} \SpecialCharTok{{-}}\DecValTok{1} \SpecialCharTok{*}\NormalTok{ mds\_data}\SpecialCharTok{$}\NormalTok{V2}
\FunctionTok{ggplot}\NormalTok{(mds\_data, }\FunctionTok{aes}\NormalTok{(}\AttributeTok{x =}\NormalTok{ V1, }\AttributeTok{y =}\NormalTok{ V2)) }\SpecialCharTok{+} \FunctionTok{geom\_point}\NormalTok{() }\SpecialCharTok{+} \FunctionTok{geom\_label\_repel}\NormalTok{(}\AttributeTok{label =}\NormalTok{ d}\SpecialCharTok{$}\NormalTok{VAROSNEV)}
\end{Highlighting}
\end{Shaded}

\begin{figure}[H]

{\centering \includegraphics{./sec_tobbdimenzios_skalazas_files/figure-pdf/unnamed-chunk-12-1.pdf}

}

\end{figure}

A kapott geometriai reprezentáció igen jól interpretálható. Ám emellett
szükség van objektív mérőszámokra is, melyek információt adnak a kapott
távolságok illeszkedésére vonatkozóan. A következőkben ilyen
mérőszámokat mutatunk be.

\begin{itemize}
\tightlist
\item
  \textbf{A Stress-érték.} Az első illeszkedés jóságát mutató mérőszám a
  Stress-érték. Az információveszteség mértékét méri. Minél kisebb,
  annál jobb. Az értéke 0,05 alatt igazán jó.
\end{itemize}

\begin{Shaded}
\begin{Highlighting}[]
\NormalTok{mds\_1}\SpecialCharTok{$}\NormalTok{stress}
\CommentTok{\#\textgreater{} [1] 0.005865683}
\end{Highlighting}
\end{Shaded}

\begin{itemize}
\tightlist
\item
  \textbf{Shepard-diagram.} Grafikus információt adhat a kétdimenziós
  (vagy bármilyen más) megoldás jóságáról. Ehhez első lépésként
  elkészítjük a kétdimenziós megoldás távolságmátrixát (\texttt{delta}
  objektum). Ezután ábrázolhatjuk az eredeti távolságok és a
  kétdimenziós távolságok kapcsolatát egy pontdiagram segítségével.
  Minél jobb a kétdimenziós megoldás, annál inkább egy egyenesre
  illeszkednek az adatok.
\end{itemize}

\begin{Shaded}
\begin{Highlighting}[]
\NormalTok{delta }\OtherTok{\textless{}{-}} \FunctionTok{dist}\NormalTok{(mds\_1}\SpecialCharTok{$}\NormalTok{points)}
\FunctionTok{plot}\NormalTok{(delta, dist)}
\end{Highlighting}
\end{Shaded}

\begin{figure}[H]

{\centering \includegraphics{./sec_tobbdimenzios_skalazas_files/figure-pdf/unnamed-chunk-14-1.pdf}

}

\end{figure}

\hypertarget{puxe9lda-uxfcduxedtux151-italok}{%
\section{Példa: Üdítő italok}\label{puxe9lda-uxfcduxedtux151-italok}}

Ebben a példában különböző üdítőitalokat vizsgálunk meg a többdimenziós
skálázás segítségével (Münnich és mtsai., 2006) {[}6.5.1 probléma{]}. Az
adatbázis a különböző üdítőitalok távolságmátrixát tartalmazza
(\texttt{mds\_uditok\_tavolsagmatrix.xlsx}). Az embereknek azt kellett
megítélni, hogy az egyes üdítők mennyire különböznek egymástól. A 0
érték azt jelenti, hogy teljesen egyformák az italok, míg az 1 a lehető
legnagyobb mértékű különbözőséget jelzi.

\begin{Shaded}
\begin{Highlighting}[]
\NormalTok{d }\OtherTok{\textless{}{-}}\NormalTok{ rio}\SpecialCharTok{::}\FunctionTok{import}\NormalTok{(}\AttributeTok{file =} \StringTok{"adat/mds\_uditok\_tavolsagmatrix.xlsx"}\NormalTok{)}
\FunctionTok{str}\NormalTok{(d)}
\CommentTok{\#\textgreater{} \textquotesingle{}data.frame\textquotesingle{}:    7 obs. of  8 variables:}
\CommentTok{\#\textgreater{}  $ NEVEK       : chr  "szorp" "hohesC" "savm\_asv\_vi...}
\CommentTok{\#\textgreater{}  $ szorp       : num  0 0.48 0.66 0.19 0.72 0.94 0.89}
\CommentTok{\#\textgreater{}  $ hohesC      : num  0.48 0 0.45 0.3 0.32 0.4 0.56}
\CommentTok{\#\textgreater{}  $ savm\_asv\_viz: num  0.66 0.45 0 0.44 0.38 0.56 0.38}
\CommentTok{\#\textgreater{}  $ szobi       : num  0.19 0.3 0.44 0 0.48 0.68 0.67}
\CommentTok{\#\textgreater{}  $ traubi      : num  0.72 0.32 0.38 0.48 0 0.2 0.3}
\CommentTok{\#\textgreater{}  $ fantanarancs: num  0.94 0.4 0.56 0.68 0.2 0 0.45}
\CommentTok{\#\textgreater{}  $ asv\_viz     : num  0.89 0.56 0.38 0.67 0.3 0.45 0}
\NormalTok{psych}\SpecialCharTok{::}\FunctionTok{headTail}\NormalTok{(d)}
\CommentTok{\#\textgreater{}            NEVEK szorp hohesC savm\_asv\_viz szobi tr...}
\CommentTok{\#\textgreater{} 1          szorp     0   0.48         0.66  0.19   ...}
\CommentTok{\#\textgreater{} 2         hohesC  0.48      0         0.45   0.3   ...}
\CommentTok{\#\textgreater{} 3   savm\_asv\_viz  0.66   0.45            0  0.44   ...}
\CommentTok{\#\textgreater{} 4          szobi  0.19    0.3         0.44     0   ...}
\CommentTok{\#\textgreater{} ...         \textless{}NA\textgreater{}   ...    ...          ...   ...   ...}
\CommentTok{\#\textgreater{} 41         szobi  0.19    0.3         0.44     0   ...}
\CommentTok{\#\textgreater{} 5         traubi  0.72   0.32         0.38  0.48   ...}
\CommentTok{\#\textgreater{} 6   fantanarancs  0.94    0.4         0.56  0.68   ...}
\CommentTok{\#\textgreater{} 7        asv\_viz  0.89   0.56         0.38  0.67   ...}
\CommentTok{\#\textgreater{}     fantanarancs asv\_viz}
\CommentTok{\#\textgreater{} 1           0.94    0.89}
\CommentTok{\#\textgreater{} 2            0.4    0.56}
\CommentTok{\#\textgreater{} 3           0.56    0.38}
\CommentTok{\#\textgreater{} 4           0.68    0.67}
\CommentTok{\#\textgreater{} ...          ...     ...}
\CommentTok{\#\textgreater{} 41          0.68    0.67}
\CommentTok{\#\textgreater{} 5            0.2     0.3}
\CommentTok{\#\textgreater{} 6              0    0.45}
\CommentTok{\#\textgreater{} 7           0.45       0}
\end{Highlighting}
\end{Shaded}

\begin{Shaded}
\begin{Highlighting}[]
\NormalTok{dist }\OtherTok{\textless{}{-}} \FunctionTok{as.dist}\NormalTok{(d[}\DecValTok{2}\SpecialCharTok{:}\DecValTok{8}\NormalTok{])}
\NormalTok{dist}
\CommentTok{\#\textgreater{}      1    2    3    4    5    6}
\CommentTok{\#\textgreater{} 2 0.48                         }
\CommentTok{\#\textgreater{} 3 0.66 0.45                    }
\CommentTok{\#\textgreater{} 4 0.19 0.30 0.44               }
\CommentTok{\#\textgreater{} 5 0.72 0.32 0.38 0.48          }
\CommentTok{\#\textgreater{} 6 0.94 0.40 0.56 0.68 0.20     }
\CommentTok{\#\textgreater{} 7 0.89 0.56 0.38 0.67 0.30 0.45}
\end{Highlighting}
\end{Shaded}

\begin{Shaded}
\begin{Highlighting}[]
\NormalTok{mds\_1 }\OtherTok{\textless{}{-}}\NormalTok{ MASS}\SpecialCharTok{::}\FunctionTok{isoMDS}\NormalTok{(dist, }\AttributeTok{k =} \DecValTok{2}\NormalTok{)}
\CommentTok{\#\textgreater{} initial  value 1.573115 }
\CommentTok{\#\textgreater{} iter   5 value 0.164954}
\CommentTok{\#\textgreater{} iter  10 value 0.022263}
\CommentTok{\#\textgreater{} final  value 0.002023 }
\CommentTok{\#\textgreater{} converged}
\NormalTok{mds\_1}\SpecialCharTok{$}\NormalTok{points}
\CommentTok{\#\textgreater{}          [,1]        [,2]}
\CommentTok{\#\textgreater{} 1 {-}0.53868137  0.03041114}
\CommentTok{\#\textgreater{} 2 {-}0.06571066  0.18616088}
\CommentTok{\#\textgreater{} 3  0.02345226 {-}0.25210840}
\CommentTok{\#\textgreater{} 4 {-}0.31947272  0.03166493}
\CommentTok{\#\textgreater{} 5  0.17978098  0.01382719}
\CommentTok{\#\textgreater{} 6  0.37258175  0.22327908}
\CommentTok{\#\textgreater{} 7  0.34804977 {-}0.23323482}
\end{Highlighting}
\end{Shaded}

A fenti output mutatja a kapott kétdimenziós megoldást. Az egyes
oszlopok az elemek első illetve második dimenzióbeli értékeit mutatja.
Mivel a többdimenziós skálázásban fontos cél az adatok grafikus
ábrázolása is, ezeket az értékeket kezelhetjük koordinátákként, melyek
segítségével rajzolhatunk egy kétdimenziós térképet.

\begin{Shaded}
\begin{Highlighting}[]
\NormalTok{mds\_data }\OtherTok{\textless{}{-}} \FunctionTok{as.data.frame}\NormalTok{(mds\_1}\SpecialCharTok{$}\NormalTok{points)}
\NormalTok{psych}\SpecialCharTok{::}\FunctionTok{headTail}\NormalTok{(mds\_data)}
\CommentTok{\#\textgreater{}        V1    V2}
\CommentTok{\#\textgreater{} 1   {-}0.54  0.03}
\CommentTok{\#\textgreater{} 2   {-}0.07  0.19}
\CommentTok{\#\textgreater{} 3    0.02 {-}0.25}
\CommentTok{\#\textgreater{} 4   {-}0.32  0.03}
\CommentTok{\#\textgreater{} ...   ...   ...}
\CommentTok{\#\textgreater{} 41  {-}0.32  0.03}
\CommentTok{\#\textgreater{} 5    0.18  0.01}
\CommentTok{\#\textgreater{} 6    0.37  0.22}
\CommentTok{\#\textgreater{} 7    0.35 {-}0.23}
\end{Highlighting}
\end{Shaded}

\begin{Shaded}
\begin{Highlighting}[]
\FunctionTok{library}\NormalTok{(ggplot2)}
\FunctionTok{library}\NormalTok{(ggrepel)}
\FunctionTok{ggplot}\NormalTok{(mds\_data, }\FunctionTok{aes}\NormalTok{(}\AttributeTok{x =}\NormalTok{ V1, }\AttributeTok{y =}\NormalTok{ V2)) }\SpecialCharTok{+} \FunctionTok{geom\_point}\NormalTok{() }\SpecialCharTok{+} \FunctionTok{geom\_label\_repel}\NormalTok{(}\AttributeTok{label =}\NormalTok{ d}\SpecialCharTok{$}\NormalTok{NEVEK)}
\end{Highlighting}
\end{Shaded}

\begin{figure}[H]

{\centering \includegraphics{./sec_tobbdimenzios_skalazas_files/figure-pdf/unnamed-chunk-19-1.pdf}

}

\end{figure}

\begin{Shaded}
\begin{Highlighting}[]
\NormalTok{mds\_1}\SpecialCharTok{$}\NormalTok{stress}
\CommentTok{\#\textgreater{} [1] 0.002022712}
\end{Highlighting}
\end{Shaded}

A Magyarország városait bemutató példában egyértelmű volt az egyes
koordinátatengelyek, dimenziók elnevezése. Ám egy ilyen példánál már
nagyobb gondot okozhat. A fenti eredmények alapján láthatjuk, hogy az
első dimenzióban az ásványvíz és a Fanta Narancs szerepel magas
értékekkel, viszonylag kis értéke van a Szobi gyümölcslének és a
szörpnek. A második dimenzióiban is magas értékkel szerepel a Fanta és a
Hohes C, viszont extrém alacsonnyal az ásványvíz. Ezek alapján az első
dimenzió képviselheti a szénsavtartalmat, míg a második a
gyümölcstartalmat. A feladatban szereplő emberek fejében ez a két
szempont tűnt fontosnak az üdítőitalok különbözőségének megítélése
során.

\hypertarget{puxe9lda-autuxf3muxe1rkuxe1k}{%
\section{Példa: Autómárkák}\label{puxe9lda-autuxf3muxe1rkuxe1k}}

Ebben a példában autómárkák közötti hasonlóságokat ítéltetünk meg a
személyekkel (Münnich és mtsai., 2006) {[}6.5.2 probléma{]}. Az 1 érték
jelenti a márkák teljes hasonlóságát, míg a 0 a hasonlóság hiányát.

\begin{Shaded}
\begin{Highlighting}[]
\NormalTok{d }\OtherTok{\textless{}{-}}\NormalTok{ rio}\SpecialCharTok{::}\FunctionTok{import}\NormalTok{(}\AttributeTok{file =} \StringTok{"adat/mds\_autok\_tavolsagmatrix.xlsx"}\NormalTok{)}
\FunctionTok{str}\NormalTok{(d)}
\CommentTok{\#\textgreater{} \textquotesingle{}data.frame\textquotesingle{}:    8 obs. of  9 variables:}
\CommentTok{\#\textgreater{}  $ AUTOK   : chr  "toyota celica" "audi a3" "seat i...}
\CommentTok{\#\textgreater{}  $ TOYOTACE: num  1 0.4 0.25 0.12 0.67 0.39 0.26 0.19}
\CommentTok{\#\textgreater{}  $ AUDIA3  : num  0.4 1 0.31 0.39 0.5 0.24 0.18 0.52}
\CommentTok{\#\textgreater{}  $ SEATIBIZ: num  0.25 0.31 1 0.46 0.28 0.38 0.42 0.49}
\CommentTok{\#\textgreater{}  $ SKODAOCT: num  0.12 0.39 0.46 1 0.2 0.14 0.29 0.55}
\CommentTok{\#\textgreater{}  $ MAZDAMX6: num  0.67 0.5 0.28 0.2 1 0.38 0.26 0.26}
\CommentTok{\#\textgreater{}  $ NISSANM : num  0.39 0.24 0.38 0.14 0.38 1 0.4 0.22}
\CommentTok{\#\textgreater{}  $ SEATLEON: num  0.26 0.18 0.42 0.29 0.26 0.4 1 0.25}
\CommentTok{\#\textgreater{}  $ FORDMOND: num  0.19 0.52 0.49 0.55 0.26 0.22 0.25 1}
\NormalTok{d}
\CommentTok{\#\textgreater{}           AUTOK TOYOTACE AUDIA3 SEATIBIZ SKODAOCT}
\CommentTok{\#\textgreater{} 1 toyota celica     1.00   0.40     0.25     0.12}
\CommentTok{\#\textgreater{} 2       audi a3     0.40   1.00     0.31     0.39}
\CommentTok{\#\textgreater{} 3    seat ibiza     0.25   0.31     1.00     0.46}
\CommentTok{\#\textgreater{} 4 skoda octavia     0.12   0.39     0.46     1.00}
\CommentTok{\#\textgreater{} 5     mazda mx6     0.67   0.50     0.28     0.20}
\CommentTok{\#\textgreater{} 6  nissan micra     0.39   0.24     0.38     0.14}
\CommentTok{\#\textgreater{} 7     seat leon     0.26   0.18     0.42     0.29}
\CommentTok{\#\textgreater{} 8   ford mondeo     0.19   0.52     0.49     0.55}
\CommentTok{\#\textgreater{}   MAZDAMX6 NISSANM SEATLEON FORDMOND}
\CommentTok{\#\textgreater{} 1     0.67    0.39     0.26     0.19}
\CommentTok{\#\textgreater{} 2     0.50    0.24     0.18     0.52}
\CommentTok{\#\textgreater{} 3     0.28    0.38     0.42     0.49}
\CommentTok{\#\textgreater{} 4     0.20    0.14     0.29     0.55}
\CommentTok{\#\textgreater{} 5     1.00    0.38     0.26     0.26}
\CommentTok{\#\textgreater{} 6     0.38    1.00     0.40     0.22}
\CommentTok{\#\textgreater{} 7     0.26    0.40     1.00     0.25}
\CommentTok{\#\textgreater{} 8     0.26    0.22     0.25     1.00}
\end{Highlighting}
\end{Shaded}

A fenti outputban egy hasonlósági mátrixot láthatunk. A többdimenziós
skálázás előtt a hasonlósági mátrixot távolságmátrixszá kell
transzformálni. Ezt egyszerűbben és pontosabban is megtehetjük:

Az egyszerűbb eset az, hogy minden hasonlósági értéket kivonunk 1-ből,
így a kis értékek közeli hasonlóságot jelentenek (kis távolság), a nagy
értékek távoli hasonlóságot jelentenek (bagy távolság).

\begin{Shaded}
\begin{Highlighting}[]
\NormalTok{d\_1 }\OtherTok{\textless{}{-}} \DecValTok{1} \SpecialCharTok{{-}}\NormalTok{ d[}\DecValTok{2}\SpecialCharTok{:}\DecValTok{9}\NormalTok{]}
\NormalTok{d\_1}
\CommentTok{\#\textgreater{}   TOYOTACE AUDIA3 SEATIBIZ SKODAOCT MAZDAMX6 NISSANM}
\CommentTok{\#\textgreater{} 1     0.00   0.60     0.75     0.88     0.33    0.61}
\CommentTok{\#\textgreater{} 2     0.60   0.00     0.69     0.61     0.50    0.76}
\CommentTok{\#\textgreater{} 3     0.75   0.69     0.00     0.54     0.72    0.62}
\CommentTok{\#\textgreater{} 4     0.88   0.61     0.54     0.00     0.80    0.86}
\CommentTok{\#\textgreater{} 5     0.33   0.50     0.72     0.80     0.00    0.62}
\CommentTok{\#\textgreater{} 6     0.61   0.76     0.62     0.86     0.62    0.00}
\CommentTok{\#\textgreater{} 7     0.74   0.82     0.58     0.71     0.74    0.60}
\CommentTok{\#\textgreater{} 8     0.81   0.48     0.51     0.45     0.74    0.78}
\CommentTok{\#\textgreater{}   SEATLEON FORDMOND}
\CommentTok{\#\textgreater{} 1     0.74     0.81}
\CommentTok{\#\textgreater{} 2     0.82     0.48}
\CommentTok{\#\textgreater{} 3     0.58     0.51}
\CommentTok{\#\textgreater{} 4     0.71     0.45}
\CommentTok{\#\textgreater{} 5     0.74     0.74}
\CommentTok{\#\textgreater{} 6     0.60     0.78}
\CommentTok{\#\textgreater{} 7     0.00     0.75}
\CommentTok{\#\textgreater{} 8     0.75     0.00}
\end{Highlighting}
\end{Shaded}

\begin{Shaded}
\begin{Highlighting}[]
\NormalTok{dist\_1 }\OtherTok{\textless{}{-}} \FunctionTok{as.dist}\NormalTok{(d\_1)}
\NormalTok{dist\_1}
\CommentTok{\#\textgreater{}      1    2    3    4    5    6    7}
\CommentTok{\#\textgreater{} 2 0.60                              }
\CommentTok{\#\textgreater{} 3 0.75 0.69                         }
\CommentTok{\#\textgreater{} 4 0.88 0.61 0.54                    }
\CommentTok{\#\textgreater{} 5 0.33 0.50 0.72 0.80               }
\CommentTok{\#\textgreater{} 6 0.61 0.76 0.62 0.86 0.62          }
\CommentTok{\#\textgreater{} 7 0.74 0.82 0.58 0.71 0.74 0.60     }
\CommentTok{\#\textgreater{} 8 0.81 0.48 0.51 0.45 0.74 0.78 0.75}
\end{Highlighting}
\end{Shaded}

\begin{Shaded}
\begin{Highlighting}[]
\NormalTok{mds\_1 }\OtherTok{\textless{}{-}}\NormalTok{ MASS}\SpecialCharTok{::}\FunctionTok{isoMDS}\NormalTok{(dist\_1, }\AttributeTok{k =} \DecValTok{2}\NormalTok{)}
\CommentTok{\#\textgreater{} initial  value 5.826508 }
\CommentTok{\#\textgreater{} iter   5 value 2.950170}
\CommentTok{\#\textgreater{} final  value 2.891815 }
\CommentTok{\#\textgreater{} converged}
\NormalTok{mds\_1}\SpecialCharTok{$}\NormalTok{points}
\CommentTok{\#\textgreater{}           [,1]        [,2]}
\CommentTok{\#\textgreater{} 1 {-}0.361055519 {-}0.17243204}
\CommentTok{\#\textgreater{} 2  0.006764815 {-}0.30716195}
\CommentTok{\#\textgreater{} 3  0.216130743  0.18832931}
\CommentTok{\#\textgreater{} 4  0.412741547 {-}0.08945592}
\CommentTok{\#\textgreater{} 5 {-}0.324654056 {-}0.19618092}
\CommentTok{\#\textgreater{} 6 {-}0.288487909  0.28343789}
\CommentTok{\#\textgreater{} 7 {-}0.007968750  0.43461843}
\CommentTok{\#\textgreater{} 8  0.346529128 {-}0.14115478}
\NormalTok{mds\_1}\SpecialCharTok{$}\NormalTok{stress}
\CommentTok{\#\textgreater{} [1] 2.891815}
\end{Highlighting}
\end{Shaded}

A távolságmátrix birtokában lefuttattuk a többdimenziós skálázást. Ez
alkalommal háromdimenziós megoldást érdemes kérni, mivel a kétdimenziós
megoldás Stress-értéke túl nagy.

\begin{Shaded}
\begin{Highlighting}[]
\NormalTok{mds\_1 }\OtherTok{\textless{}{-}}\NormalTok{ MASS}\SpecialCharTok{::}\FunctionTok{isoMDS}\NormalTok{(dist\_1, }\AttributeTok{k =} \DecValTok{3}\NormalTok{)}
\CommentTok{\#\textgreater{} initial  value 2.196611 }
\CommentTok{\#\textgreater{} iter   5 value 0.256820}
\CommentTok{\#\textgreater{} iter  10 value 0.151733}
\CommentTok{\#\textgreater{} iter  15 value 0.107927}
\CommentTok{\#\textgreater{} iter  20 value 0.019342}
\CommentTok{\#\textgreater{} final  value 0.000000 }
\CommentTok{\#\textgreater{} converged}
\NormalTok{mds\_1}\SpecialCharTok{$}\NormalTok{points}
\CommentTok{\#\textgreater{}          [,1]        [,2]        [,3]}
\CommentTok{\#\textgreater{} 1 {-}0.46448409 {-}0.15747133 {-}0.07985979}
\CommentTok{\#\textgreater{} 2  0.03707153 {-}0.40270153 {-}0.00671623}
\CommentTok{\#\textgreater{} 3  0.24092363  0.24184414  0.09663242}
\CommentTok{\#\textgreater{} 4  0.49812765 {-}0.05451923 {-}0.12227626}
\CommentTok{\#\textgreater{} 5 {-}0.36659549 {-}0.21506553 {-}0.04539081}
\CommentTok{\#\textgreater{} 6 {-}0.32944563  0.30522587  0.23017071}
\CommentTok{\#\textgreater{} 7 {-}0.00870002  0.45983004 {-}0.21064227}
\CommentTok{\#\textgreater{} 8  0.39310242 {-}0.17714243  0.13808223}
\NormalTok{mds\_1}\SpecialCharTok{$}\NormalTok{stress}
\CommentTok{\#\textgreater{} [1] 4.343565e{-}14}
\end{Highlighting}
\end{Shaded}

\begin{Shaded}
\begin{Highlighting}[]
\FunctionTok{library}\NormalTok{(plotly)}
\CommentTok{\# 3D ábrázolás}
\NormalTok{d\_3d }\OtherTok{\textless{}{-}} \FunctionTok{as.data.frame}\NormalTok{(mds\_1}\SpecialCharTok{$}\NormalTok{points)}
\NormalTok{d\_3d}\SpecialCharTok{$}\NormalTok{AUTOK }\OtherTok{\textless{}{-}}\NormalTok{ d}\SpecialCharTok{$}\NormalTok{AUTOK}
\FunctionTok{plot\_ly}\NormalTok{(d\_3d, }\AttributeTok{x =} \SpecialCharTok{\textasciitilde{}}\NormalTok{V1, }\AttributeTok{y =} \SpecialCharTok{\textasciitilde{}}\NormalTok{V2, }\AttributeTok{z =} \SpecialCharTok{\textasciitilde{}}\NormalTok{V3) }\SpecialCharTok{\%\textgreater{}\%}
    \FunctionTok{add\_text}\NormalTok{(}\AttributeTok{text =} \SpecialCharTok{\textasciitilde{}}\NormalTok{AUTOK) }\SpecialCharTok{\%\textgreater{}\%}
    \FunctionTok{add\_markers}\NormalTok{(}\AttributeTok{color =} \SpecialCharTok{\textasciitilde{}}\NormalTok{AUTOK)}
\end{Highlighting}
\end{Shaded}

A hasonlósági mátrix távolságmátrixszá alakításának pontosabb módja a
(Münnich és mtsai., 2006)
\href{https://psycho.unideb.hu/statisztika/pages/p_6_12.html}{6.4
fejezetében} olvasható.

\begin{Shaded}
\begin{Highlighting}[]
\NormalTok{d\_2 }\OtherTok{\textless{}{-}} \FunctionTok{sqrt}\NormalTok{(}\DecValTok{2} \SpecialCharTok{*}\NormalTok{ (}\DecValTok{1} \SpecialCharTok{{-}}\NormalTok{ d[}\DecValTok{2}\SpecialCharTok{:}\DecValTok{9}\NormalTok{]))}
\end{Highlighting}
\end{Shaded}

\begin{Shaded}
\begin{Highlighting}[]
\NormalTok{dist\_2 }\OtherTok{\textless{}{-}} \FunctionTok{as.dist}\NormalTok{(d\_2)}
\NormalTok{dist\_2}
\CommentTok{\#\textgreater{}           1         2         3         4         5}
\CommentTok{\#\textgreater{} 2 1.0954451                                        }
\CommentTok{\#\textgreater{} 3 1.2247449 1.1747340                              }
\CommentTok{\#\textgreater{} 4 1.3266499 1.1045361 1.0392305                    }
\CommentTok{\#\textgreater{} 5 0.8124038 1.0000000 1.2000000 1.2649111          }
\CommentTok{\#\textgreater{} 6 1.1045361 1.2328828 1.1135529 1.3114877 1.1135529}
\CommentTok{\#\textgreater{} 7 1.2165525 1.2806248 1.0770330 1.1916375 1.2165525}
\CommentTok{\#\textgreater{} 8 1.2727922 0.9797959 1.0099505 0.9486833 1.2165525}
\CommentTok{\#\textgreater{}           6         7}
\CommentTok{\#\textgreater{} 2                    }
\CommentTok{\#\textgreater{} 3                    }
\CommentTok{\#\textgreater{} 4                    }
\CommentTok{\#\textgreater{} 5                    }
\CommentTok{\#\textgreater{} 6                    }
\CommentTok{\#\textgreater{} 7 1.0954451          }
\CommentTok{\#\textgreater{} 8 1.2489996 1.2247449}
\end{Highlighting}
\end{Shaded}

\begin{Shaded}
\begin{Highlighting}[]
\NormalTok{mds\_2 }\OtherTok{\textless{}{-}}\NormalTok{ MASS}\SpecialCharTok{::}\FunctionTok{isoMDS}\NormalTok{(dist\_2, }\AttributeTok{k =} \DecValTok{2}\NormalTok{)}
\CommentTok{\#\textgreater{} initial  value 6.389472 }
\CommentTok{\#\textgreater{} iter   5 value 5.192764}
\CommentTok{\#\textgreater{} iter  10 value 3.416955}
\CommentTok{\#\textgreater{} iter  15 value 2.949710}
\CommentTok{\#\textgreater{} final  value 2.891118 }
\CommentTok{\#\textgreater{} converged}
\NormalTok{mds\_2}\SpecialCharTok{$}\NormalTok{points}
\CommentTok{\#\textgreater{}         [,1]       [,2]}
\CommentTok{\#\textgreater{} 1  1.4484081  0.8945686}
\CommentTok{\#\textgreater{} 2 {-}0.1610748  1.3129618}
\CommentTok{\#\textgreater{} 3 {-}0.8393349 {-}0.9062232}
\CommentTok{\#\textgreater{} 4 {-}1.7845996  0.2177189}
\CommentTok{\#\textgreater{} 5  1.3049511  0.9653162}
\CommentTok{\#\textgreater{} 6  1.3352331 {-}1.0778650}
\CommentTok{\#\textgreater{} 7  0.2295031 {-}1.8533287}
\CommentTok{\#\textgreater{} 8 {-}1.5330862  0.4468514}
\NormalTok{mds\_2}\SpecialCharTok{$}\NormalTok{stress}
\CommentTok{\#\textgreater{} [1] 2.891118}
\end{Highlighting}
\end{Shaded}

A távolságmátrix birtokában lefuttattuk a többdimenziós skálázást. Ez
alkalommal háromdimenziós megoldást érdemes kérni, mivel a kétdimenziós
megoldás Stress-értéke túl nagy.

\begin{Shaded}
\begin{Highlighting}[]
\NormalTok{mds\_2 }\OtherTok{\textless{}{-}}\NormalTok{ MASS}\SpecialCharTok{::}\FunctionTok{isoMDS}\NormalTok{(dist\_2, }\AttributeTok{k =} \DecValTok{3}\NormalTok{)}
\CommentTok{\#\textgreater{} initial  value 5.751803 }
\CommentTok{\#\textgreater{} iter   5 value 2.166360}
\CommentTok{\#\textgreater{} iter  10 value 1.472838}
\CommentTok{\#\textgreater{} iter  15 value 0.710732}
\CommentTok{\#\textgreater{} iter  20 value 0.271743}
\CommentTok{\#\textgreater{} final  value 0.008855 }
\CommentTok{\#\textgreater{} converged}
\NormalTok{mds\_2}\SpecialCharTok{$}\NormalTok{points}
\CommentTok{\#\textgreater{}          [,1]        [,2]       [,3]}
\CommentTok{\#\textgreater{} 1  1.13442661  0.31620329  0.2158284}
\CommentTok{\#\textgreater{} 2 {-}0.10372721  0.94488131  0.2158065}
\CommentTok{\#\textgreater{} 3 {-}0.56639006 {-}0.51061522 {-}0.3193788}
\CommentTok{\#\textgreater{} 4 {-}1.22334085  0.08201655  0.2935412}
\CommentTok{\#\textgreater{} 5  0.86269111  0.49684620  0.2687634}
\CommentTok{\#\textgreater{} 6  0.78134789 {-}0.62616329 {-}0.7513363}
\CommentTok{\#\textgreater{} 7  0.06291958 {-}1.18359998  0.3083139}
\CommentTok{\#\textgreater{} 8 {-}0.94792707  0.48043115 {-}0.2315383}
\NormalTok{mds\_2}\SpecialCharTok{$}\NormalTok{stress}
\CommentTok{\#\textgreater{} [1] 0.008855227}
\end{Highlighting}
\end{Shaded}

\begin{Shaded}
\begin{Highlighting}[]
\FunctionTok{library}\NormalTok{(plotly)}
\CommentTok{\# 3D ábrázolás}
\NormalTok{d\_3d }\OtherTok{\textless{}{-}} \FunctionTok{as.data.frame}\NormalTok{(mds\_2}\SpecialCharTok{$}\NormalTok{points)}
\NormalTok{d\_3d}\SpecialCharTok{$}\NormalTok{AUTOK }\OtherTok{\textless{}{-}}\NormalTok{ d}\SpecialCharTok{$}\NormalTok{AUTOK}
\FunctionTok{plot\_ly}\NormalTok{(d\_3d, }\AttributeTok{x =} \SpecialCharTok{\textasciitilde{}}\NormalTok{V1, }\AttributeTok{y =} \SpecialCharTok{\textasciitilde{}}\NormalTok{V2, }\AttributeTok{z =} \SpecialCharTok{\textasciitilde{}}\NormalTok{V3) }\SpecialCharTok{\%\textgreater{}\%}
    \FunctionTok{add\_text}\NormalTok{(}\AttributeTok{text =} \SpecialCharTok{\textasciitilde{}}\NormalTok{AUTOK) }\SpecialCharTok{\%\textgreater{}\%}
    \FunctionTok{add\_markers}\NormalTok{(}\AttributeTok{color =} \SpecialCharTok{\textasciitilde{}}\NormalTok{AUTOK)}
\end{Highlighting}
\end{Shaded}

\begin{Shaded}
\begin{Highlighting}[]
\NormalTok{dist }\OtherTok{\textless{}{-}} \FunctionTok{as.dist}\NormalTok{(d[}\DecValTok{2}\SpecialCharTok{:}\DecValTok{9}\NormalTok{])}
\NormalTok{dist}
\CommentTok{\#\textgreater{}      1    2    3    4    5    6    7}
\CommentTok{\#\textgreater{} 2 0.40                              }
\CommentTok{\#\textgreater{} 3 0.25 0.31                         }
\CommentTok{\#\textgreater{} 4 0.12 0.39 0.46                    }
\CommentTok{\#\textgreater{} 5 0.67 0.50 0.28 0.20               }
\CommentTok{\#\textgreater{} 6 0.39 0.24 0.38 0.14 0.38          }
\CommentTok{\#\textgreater{} 7 0.26 0.18 0.42 0.29 0.26 0.40     }
\CommentTok{\#\textgreater{} 8 0.19 0.52 0.49 0.55 0.26 0.22 0.25}
\end{Highlighting}
\end{Shaded}

\begin{Shaded}
\begin{Highlighting}[]
\NormalTok{mds\_1 }\OtherTok{\textless{}{-}}\NormalTok{ MASS}\SpecialCharTok{::}\FunctionTok{isoMDS}\NormalTok{(dist, }\AttributeTok{k =} \DecValTok{2}\NormalTok{)}
\CommentTok{\#\textgreater{} initial  value 33.207938 }
\CommentTok{\#\textgreater{} iter   5 value 20.383903}
\CommentTok{\#\textgreater{} iter   5 value 20.375967}
\CommentTok{\#\textgreater{} iter   5 value 20.375494}
\CommentTok{\#\textgreater{} final  value 20.375494 }
\CommentTok{\#\textgreater{} converged}
\NormalTok{mds\_1}\SpecialCharTok{$}\NormalTok{points}
\CommentTok{\#\textgreater{}          [,1]        [,2]}
\CommentTok{\#\textgreater{} 1  0.21725475 {-}0.07057062}
\CommentTok{\#\textgreater{} 2  0.23169840  0.16729709}
\CommentTok{\#\textgreater{} 3  0.02516126  0.14398696}
\CommentTok{\#\textgreater{} 4 {-}0.07964438  0.22492857}
\CommentTok{\#\textgreater{} 5 {-}0.27770841 {-}0.02595944}
\CommentTok{\#\textgreater{} 6 {-}0.07736511 {-}0.03519537}
\CommentTok{\#\textgreater{} 7  0.04465412 {-}0.09341317}
\CommentTok{\#\textgreater{} 8 {-}0.08405063 {-}0.31107402}
\NormalTok{mds\_1}\SpecialCharTok{$}\NormalTok{stress}
\CommentTok{\#\textgreater{} [1] 20.37549}
\end{Highlighting}
\end{Shaded}

A távolságmátrix birtokában lefuttattuk a többdimenziós skálázást. Ez
alkalommal háromdimenziós megoldást érdemes kérni, mivel a kétdimenziós
megoldás Stress-értéke túl nagy.

\begin{Shaded}
\begin{Highlighting}[]
\NormalTok{mds\_1 }\OtherTok{\textless{}{-}}\NormalTok{ MASS}\SpecialCharTok{::}\FunctionTok{isoMDS}\NormalTok{(dist, }\AttributeTok{k =} \DecValTok{3}\NormalTok{)}
\CommentTok{\#\textgreater{} initial  value 20.475699 }
\CommentTok{\#\textgreater{} iter   5 value 13.683747}
\CommentTok{\#\textgreater{} iter  10 value 12.173787}
\CommentTok{\#\textgreater{} iter  15 value 11.582671}
\CommentTok{\#\textgreater{} iter  20 value 10.623521}
\CommentTok{\#\textgreater{} iter  25 value 10.205554}
\CommentTok{\#\textgreater{} final  value 10.087435 }
\CommentTok{\#\textgreater{} converged}
\NormalTok{mds\_1}\SpecialCharTok{$}\NormalTok{points}
\CommentTok{\#\textgreater{}          [,1]        [,2]       [,3]}
\CommentTok{\#\textgreater{} 1  0.13460836  0.12178456 {-}0.2459601}
\CommentTok{\#\textgreater{} 2  0.26513691  0.11656555  0.1311542}
\CommentTok{\#\textgreater{} 3 {-}0.11581558  0.20541786  0.2225079}
\CommentTok{\#\textgreater{} 4 {-}0.07304790  0.17140624 {-}0.1649379}
\CommentTok{\#\textgreater{} 5 {-}0.24115264 {-}0.20298696  0.1827992}
\CommentTok{\#\textgreater{} 6 {-}0.19159160  0.01802297 {-}0.1309034}
\CommentTok{\#\textgreater{} 7  0.18865375 {-}0.07207344  0.1366108}
\CommentTok{\#\textgreater{} 8  0.03320869 {-}0.35813678 {-}0.1312707}
\NormalTok{mds\_1}\SpecialCharTok{$}\NormalTok{stress}
\CommentTok{\#\textgreater{} [1] 10.08743}
\end{Highlighting}
\end{Shaded}

\begin{Shaded}
\begin{Highlighting}[]
\NormalTok{mds\_data }\OtherTok{\textless{}{-}} \FunctionTok{as.data.frame}\NormalTok{(mds\_1}\SpecialCharTok{$}\NormalTok{points)}
\NormalTok{mds\_data}
\CommentTok{\#\textgreater{}            V1          V2         V3}
\CommentTok{\#\textgreater{} 1  0.13460836  0.12178456 {-}0.2459601}
\CommentTok{\#\textgreater{} 2  0.26513691  0.11656555  0.1311542}
\CommentTok{\#\textgreater{} 3 {-}0.11581558  0.20541786  0.2225079}
\CommentTok{\#\textgreater{} 4 {-}0.07304790  0.17140624 {-}0.1649379}
\CommentTok{\#\textgreater{} 5 {-}0.24115264 {-}0.20298696  0.1827992}
\CommentTok{\#\textgreater{} 6 {-}0.19159160  0.01802297 {-}0.1309034}
\CommentTok{\#\textgreater{} 7  0.18865375 {-}0.07207344  0.1366108}
\CommentTok{\#\textgreater{} 8  0.03320869 {-}0.35813678 {-}0.1312707}
\end{Highlighting}
\end{Shaded}

Az eredmények vizsgálata alapján elnevezhetjük az egyes dimenziókat. A
kapott konfigurációban az első dimenziót, vagyis az x-tengelyt
nevezhetjük az ár tengelyének. A második tengely, az y, a családbarát
jellegre vonatozik, míg az utolsó, z dimenzió a sportosságot
képviselheti.

Az MDS és főkomponens analízis összehasonlításáról itt olvashatunk:
\href{https://www.geeksforgeeks.org/multidimensional-scaling-using-r/}{MDS
using R}

A hasonlósági-távolságmátrix átalakítás két módját egy ábrán is
összehasonlíthatjuk.

\begin{Shaded}
\begin{Highlighting}[]
\NormalTok{r\_ertekek }\OtherTok{\textless{}{-}} \FunctionTok{seq}\NormalTok{(}\AttributeTok{from =} \DecValTok{0}\NormalTok{, }\AttributeTok{to =} \DecValTok{1}\NormalTok{, }\AttributeTok{by =} \FloatTok{0.01}\NormalTok{)}
\NormalTok{d\_egyszeru }\OtherTok{\textless{}{-}} \DecValTok{1} \SpecialCharTok{{-}}\NormalTok{ r\_ertekek}
\NormalTok{d\_pontos }\OtherTok{\textless{}{-}} \FunctionTok{sqrt}\NormalTok{(}\DecValTok{2} \SpecialCharTok{*}\NormalTok{ (}\DecValTok{1} \SpecialCharTok{{-}}\NormalTok{ r\_ertekek))}
\NormalTok{d }\OtherTok{\textless{}{-}} \FunctionTok{data.frame}\NormalTok{(}\AttributeTok{x =} \FunctionTok{c}\NormalTok{(r\_ertekek, r\_ertekek), }\AttributeTok{y =} \FunctionTok{c}\NormalTok{(d\_egyszeru,}
\NormalTok{    d\_pontos), }\AttributeTok{szamitas =} \FunctionTok{rep}\NormalTok{(}\FunctionTok{c}\NormalTok{(}\StringTok{"egyszerű"}\NormalTok{, }\StringTok{"pontos"}\NormalTok{), }\AttributeTok{each =} \FunctionTok{length}\NormalTok{(r\_ertekek)))}
\FunctionTok{str}\NormalTok{(d)}
\CommentTok{\#\textgreater{} \textquotesingle{}data.frame\textquotesingle{}:    202 obs. of  3 variables:}
\CommentTok{\#\textgreater{}  $ x       : num  0 0.01 0.02 0.03 0.04 0.05 0.06 0...}
\CommentTok{\#\textgreater{}  $ y       : num  1 0.99 0.98 0.97 0.96 0.95 0.94 0...}
\CommentTok{\#\textgreater{}  $ szamitas: chr  "egyszerű" "egyszerű" "egyszerű" ...}
\NormalTok{psych}\SpecialCharTok{::}\FunctionTok{headTail}\NormalTok{(d)}
\CommentTok{\#\textgreater{}        x    y szamitas}
\CommentTok{\#\textgreater{} 1      0    1 egyszerű}
\CommentTok{\#\textgreater{} 2   0.01 0.99 egyszerű}
\CommentTok{\#\textgreater{} 3   0.02 0.98 egyszerű}
\CommentTok{\#\textgreater{} 4   0.03 0.97 egyszerű}
\CommentTok{\#\textgreater{} ...  ...  ...     \textless{}NA\textgreater{}}
\CommentTok{\#\textgreater{} 199 0.97 0.24   pontos}
\CommentTok{\#\textgreater{} 200 0.98  0.2   pontos}
\CommentTok{\#\textgreater{} 201 0.99 0.14   pontos}
\CommentTok{\#\textgreater{} 202    1    0   pontos}
\end{Highlighting}
\end{Shaded}

\begin{Shaded}
\begin{Highlighting}[]
\FunctionTok{library}\NormalTok{(ggplot2)}
\FunctionTok{ggplot}\NormalTok{(d, }\FunctionTok{aes}\NormalTok{(}\AttributeTok{x =}\NormalTok{ x, }\AttributeTok{y =}\NormalTok{ y, }\AttributeTok{color =}\NormalTok{ szamitas, }\AttributeTok{group =}\NormalTok{ szamitas)) }\SpecialCharTok{+}
    \FunctionTok{geom\_line}\NormalTok{(}\AttributeTok{linewidth =} \DecValTok{1}\NormalTok{) }\SpecialCharTok{+} \FunctionTok{labs}\NormalTok{(}\AttributeTok{x =} \StringTok{"r (korreláció vagy hasonlóság)"}\NormalTok{,}
    \AttributeTok{y =} \StringTok{"Távolság"}\NormalTok{, }\AttributeTok{color =} \StringTok{"Számítás"}\NormalTok{)}
\end{Highlighting}
\end{Shaded}

\begin{figure}[H]

{\centering \includegraphics{./sec_tobbdimenzios_skalazas_files/figure-pdf/unnamed-chunk-37-1.pdf}

}

\end{figure}

\hypertarget{puxe9lda-vuxe1llalatok}{%
\section{Példa: Vállalatok}\label{puxe9lda-vuxe1llalatok}}

A vállalatokat számtalan jellemző mentén mérhetjük, most a vállalat
nagyságát, a hatalmi távolságot és a vállalat szemléletében jelen levő
konzervativizmus mértékét választottuk.

\begin{Shaded}
\begin{Highlighting}[]
\NormalTok{d }\OtherTok{\textless{}{-}}\NormalTok{ rio}\SpecialCharTok{::}\FunctionTok{import}\NormalTok{(}\AttributeTok{file =} \StringTok{"adat/mds\_vallalatok.xlsx"}\NormalTok{)}
\FunctionTok{str}\NormalTok{(d)}
\CommentTok{\#\textgreater{} \textquotesingle{}data.frame\textquotesingle{}:    10 obs. of  4 variables:}
\CommentTok{\#\textgreater{}  $ NEV     : chr  "A vallalat" "B vallalat" "C vall...}
\CommentTok{\#\textgreater{}  $ MERET   : num  75 1500 2000 21 1000 900 1000 35 ...}
\CommentTok{\#\textgreater{}  $ HATALMIT: num  1 10 11 3 10 11 10 4 2 5}
\CommentTok{\#\textgreater{}  $ KONZERVA: num  2 9 8 4 9 8 11 3 2 4}
\NormalTok{d}
\CommentTok{\#\textgreater{}           NEV MERET HATALMIT KONZERVA}
\CommentTok{\#\textgreater{} 1  A vallalat    75        1        2}
\CommentTok{\#\textgreater{} 2  B vallalat  1500       10        9}
\CommentTok{\#\textgreater{} 3  C vallalat  2000       11        8}
\CommentTok{\#\textgreater{} 4  D vallalat    21        3        4}
\CommentTok{\#\textgreater{} 5  E vallalat  1000       10        9}
\CommentTok{\#\textgreater{} 6  F vallalat   900       11        8}
\CommentTok{\#\textgreater{} 7  G vallalat  1000       10       11}
\CommentTok{\#\textgreater{} 8  H vallalat    35        4        3}
\CommentTok{\#\textgreater{} 9  I vallalat   120        2        2}
\CommentTok{\#\textgreater{} 10 J vallalat   100        5        4}
\end{Highlighting}
\end{Shaded}

Ebben a példában nem távolságmátrixból indulunk ki.Ez tehát olyan
példája a többdimenziós skálázásnak, amikor nyers adatokból indulunk ki.
Állítsuk elő a távolságmátrixot.

\begin{Shaded}
\begin{Highlighting}[]
\NormalTok{dist\_1 }\OtherTok{\textless{}{-}} \FunctionTok{dist}\NormalTok{(d[}\DecValTok{2}\SpecialCharTok{:}\DecValTok{4}\NormalTok{])}
\NormalTok{dist\_1}
\CommentTok{\#\textgreater{}             1          2          3          4}
\CommentTok{\#\textgreater{} 2  1425.04561                                 }
\CommentTok{\#\textgreater{} 3  1925.03532  500.00200                      }
\CommentTok{\#\textgreater{} 4    54.07402 1479.02502 1979.02021           }
\CommentTok{\#\textgreater{} 5   925.07027  500.00000 1000.00100  979.03779}
\CommentTok{\#\textgreater{} 6   825.08242  600.00167 1100.00000  879.04551}
\CommentTok{\#\textgreater{} 7   925.08756  500.00400 1000.00500  979.05005}
\CommentTok{\#\textgreater{} 8    40.12481 1465.02457 1965.01883   14.07125}
\CommentTok{\#\textgreater{} 9    45.01111 1380.04094 1880.03112   99.02525}
\CommentTok{\#\textgreater{} 10   25.39685 1400.01786 1900.01368   79.02531}
\CommentTok{\#\textgreater{}             5          6          7          8}
\CommentTok{\#\textgreater{} 2                                             }
\CommentTok{\#\textgreater{} 3                                             }
\CommentTok{\#\textgreater{} 4                                             }
\CommentTok{\#\textgreater{} 5                                             }
\CommentTok{\#\textgreater{} 6   100.01000                                 }
\CommentTok{\#\textgreater{} 7     2.00000  100.04999                      }
\CommentTok{\#\textgreater{} 8   965.03730  865.04277  965.05181           }
\CommentTok{\#\textgreater{} 9   880.06420  780.07500  880.08238   85.02941}
\CommentTok{\#\textgreater{} 10  900.02778  800.03250  900.04111   65.01538}
\CommentTok{\#\textgreater{}             9}
\CommentTok{\#\textgreater{} 2            }
\CommentTok{\#\textgreater{} 3            }
\CommentTok{\#\textgreater{} 4            }
\CommentTok{\#\textgreater{} 5            }
\CommentTok{\#\textgreater{} 6            }
\CommentTok{\#\textgreater{} 7            }
\CommentTok{\#\textgreater{} 8            }
\CommentTok{\#\textgreater{} 9            }
\CommentTok{\#\textgreater{} 10   20.32240}
\end{Highlighting}
\end{Shaded}

\begin{Shaded}
\begin{Highlighting}[]
\NormalTok{mds\_1 }\OtherTok{\textless{}{-}}\NormalTok{ MASS}\SpecialCharTok{::}\FunctionTok{isoMDS}\NormalTok{(}\AttributeTok{d =}\NormalTok{ dist\_1, }\AttributeTok{k =} \DecValTok{2}\NormalTok{)}
\CommentTok{\#\textgreater{} initial  value 0.000000 }
\CommentTok{\#\textgreater{} final  value 0.000000 }
\CommentTok{\#\textgreater{} converged}
\NormalTok{mds\_1}\SpecialCharTok{$}\NormalTok{points}
\CommentTok{\#\textgreater{}          [,1]        [,2]}
\CommentTok{\#\textgreater{} 1    600.1319 {-}3.12622014}
\CommentTok{\#\textgreater{} 2   {-}824.9116 {-}0.67138875}
\CommentTok{\#\textgreater{} 3  {-}1324.9030 {-}3.78267877}
\CommentTok{\#\textgreater{} 4    654.1132  0.03765570}
\CommentTok{\#\textgreater{} 5   {-}324.9215  2.43526416}
\CommentTok{\#\textgreater{} 6   {-}224.9247  3.05195762}
\CommentTok{\#\textgreater{} 7   {-}324.9291  3.85407994}
\CommentTok{\#\textgreater{} 8    640.1123 {-}0.05396771}
\CommentTok{\#\textgreater{} 9    555.1277 {-}2.70104813}
\CommentTok{\#\textgreater{} 10   575.1048  0.95634608}
\NormalTok{mds\_1}\SpecialCharTok{$}\NormalTok{stress}
\CommentTok{\#\textgreater{} [1] 7.337118e{-}14}
\end{Highlighting}
\end{Shaded}

A Stress-érték megfelelően kicsi, így elfogadhatjuk a kétdimenziós
megoldást.

A fenti output segítségével pedig pontosan megtudhatjuk, hogy az egyes
vállalatok milyen értékekkel szerepelnek az egyes dimenziókban.

\begin{Shaded}
\begin{Highlighting}[]
\NormalTok{mds\_data }\OtherTok{\textless{}{-}} \FunctionTok{as.data.frame}\NormalTok{(mds\_1}\SpecialCharTok{$}\NormalTok{points)}
\NormalTok{psych}\SpecialCharTok{::}\FunctionTok{headTail}\NormalTok{(mds\_data)}
\CommentTok{\#\textgreater{}          V1    V2}
\CommentTok{\#\textgreater{} 1    600.13 {-}3.13}
\CommentTok{\#\textgreater{} 2   {-}824.91 {-}0.67}
\CommentTok{\#\textgreater{} 3   {-}1324.9 {-}3.78}
\CommentTok{\#\textgreater{} 4    654.11  0.04}
\CommentTok{\#\textgreater{} ...     ...   ...}
\CommentTok{\#\textgreater{} 7   {-}324.93  3.85}
\CommentTok{\#\textgreater{} 8    640.11 {-}0.05}
\CommentTok{\#\textgreater{} 9    555.13  {-}2.7}
\CommentTok{\#\textgreater{} 10    575.1  0.96}
\end{Highlighting}
\end{Shaded}

\begin{Shaded}
\begin{Highlighting}[]
\FunctionTok{library}\NormalTok{(ggplot2)}
\FunctionTok{library}\NormalTok{(ggrepel)}
\FunctionTok{ggplot}\NormalTok{(mds\_data, }\FunctionTok{aes}\NormalTok{(}\AttributeTok{x =}\NormalTok{ V1, }\AttributeTok{y =}\NormalTok{ V2)) }\SpecialCharTok{+} \FunctionTok{geom\_point}\NormalTok{() }\SpecialCharTok{+} \FunctionTok{geom\_label\_repel}\NormalTok{(}\AttributeTok{label =}\NormalTok{ d}\SpecialCharTok{$}\NormalTok{NEV)}
\end{Highlighting}
\end{Shaded}

\begin{figure}[H]

{\centering \includegraphics{./sec_tobbdimenzios_skalazas_files/figure-pdf/unnamed-chunk-42-1.pdf}

}

\end{figure}

A kétdimenziós geometriai reprezentáció megmutatja, hogy az emberek
véleménye szerint az egyes vállalatok hogyan helyezkednek el egymáshoz
képest.

\appendix
\addcontentsline{toc}{part}{Függelék}

\hypertarget{sec-elerheto-videok}{%
\chapter{Elérhető videók}\label{sec-elerheto-videok}}

\begin{itemize}
\item
  Alapozo jamovi és R videók

  \begin{itemize}
  \tightlist
  \item
    \href{https://www.youtube.com/watch?v=sZummF3Sd90\&list=PLnmeQHnHYqv5_FoVOvX9tqE90EqoOW01o}{Jamovi
    a gyakorlatban}
  \item
    \href{https://www.youtube.com/watch?v=UBeiIKzE8VA\&list=PLnmeQHnHYqv6ENGrdTXiE9YJrvHuxH2C9}{R
    a gyakorlatban}
  \end{itemize}
\item
  Magyar nyelvű ismertető statisztikai eljárásokról:
  \href{https://www.youtube.com/@stathelp}{Soltész-Várhelyi Klára}
\item
  Jamovi tutorial videók az összes tanult eljáráshoz

  \begin{itemize}
  \tightlist
  \item
    \href{https://www.youtube.com/@datalabcc/videos}{datalab.cc}
  \item
    \href{https://www.youtube.com/@AlexanderSwan/videos}{Alexander Swan}
  \end{itemize}
\item
  Lineáris regresszió

  \begin{itemize}
  \tightlist
  \item
    \href{https://www.youtube.com/watch?v=m88h75F3Rl8}{The linear
    regression model}
  \item
    \href{https://www.youtube.com/watch?v=nk2CQITm_eo}{Linear
    Regression, Clearly Explained}
  \item
    \href{https://www.youtube.com/watch?v=2AQKmw14mHM}{R-squared,
    Clearly Explained}
  \item
    \href{https://www.youtube.com/watch?v=rUMWjRE8L1U}{Simple linear
    regression in Jamovi}
  \end{itemize}
\item
  Főkomponens elemzés

  \begin{itemize}
  \tightlist
  \item
    \href{https://www.youtube.com/watch?v=g-Hb26agBFg}{Principal
    Component Analysis (PCA)}
  \end{itemize}
\item
  Klaszterelemzés

  \begin{itemize}
  \tightlist
  \item
    \href{https://www.youtube.com/watch?v=ijUMKMC4f9I}{Flat and
    Hierarchical Clustering \textbar{} The Dendrogram Explained}
  \item
    \href{https://www.youtube.com/watch?v=YIGtalP1mv0}{K Means
    Clustering: Pros and Cons of K Means Clustering}
  \item
    \href{https://www.youtube.com/watch?v=TnFfAQpq0o4}{Hierarchical
    Cluster analysis in Jamovi}
  \item
    \href{https://www.youtube.com/watch?v=wW7TNlam77A}{K Means Cluster
    analysis in Jamovi}
  \end{itemize}
\item
  Többszempontos varianciaelemzés

  \begin{itemize}
  \tightlist
  \item
    \href{https://www.youtube.com/watch?v=pECfK5IfB_o}{One Way ANOVA
    Post hoc test in Jamovi}
  \item
    \href{https://www.youtube.com/watch?v=O6BiwVRj3BQ}{One Way Repeated
    Measure ANOVA Repeated Measure ANOVA Within Subject ANOVA in Jamovi}
  \item
    \href{https://www.youtube.com/watch?v=l6ClJFgxNXs}{Two Way ANOVA
    Post hoc test in Jamovi}
  \item
    \href{https://www.youtube.com/watch?v=WW8ALE6x3uQ}{Three Way ANOVA
    Post hoc test in Jamovi}
  \end{itemize}
\end{itemize}

\hypertarget{sec-adatbazisok}{%
\chapter{Adatbázisok}\label{sec-adatbazisok}}

Jelen jegyzetben ``könyv'' alatt a következőt értjük:

\begin{itemize}
\tightlist
\item
  Münnich Ákos, Nagy Ágnes, Abari Kálmán. \emph{Többváltozós statisztika
  pszichológus hallgatók számára.} Bölcsész Konzorcium, Debrecen, 2006.
  (http://psycho.unideb.hu/statisztika) ISBN 963 9704 04 0
\end{itemize}

\hypertarget{megbuxedzhatuxf3suxe1g-elemzuxe9s}{%
\section{Megbízhatóság
elemzés}\label{megbuxedzhatuxf3suxe1g-elemzuxe9s}}

\begin{itemize}
\item
  \texttt{megbizhatosag\_tantargyak.xlsx} - fiktív adatbázis 9 tanuló
  iskolai jegyeivel (Münnich és mtsai. (2006), 2.2. táblázat)

  \begin{itemize}
  \item
    Az adatbázis szerkezete:

    \begin{itemize}
    \tightlist
    \item
      \texttt{matek} - matematika érdemjegy (numerikus: 1-5)
    \item
      \texttt{fizika} - fizika érdemjegy (numerikus: 1-5)
    \item
      \texttt{informatika} - informatika érdemjegy (numerikus: 1-5)
    \item
      \texttt{kemia} - kémia érdemjegy (numerikus: 1-5)
    \end{itemize}
  \item
    Kapcsolódó állományok:

    \begin{itemize}
    \tightlist
    \item
      \texttt{megbizhatosag\_tantargyak.omv} - megbízhatóság elemzés
      jamovi-ban
    \end{itemize}
  \end{itemize}
\end{itemize}

\hypertarget{tuxf6bbvuxe1ltozuxf3s-varianciaelemzuxe9s}{%
\section{Többváltozós
varianciaelemzés}\label{tuxf6bbvuxe1ltozuxf3s-varianciaelemzuxe9s}}

\begin{itemize}
\item
  \texttt{manova\_vezetesi\_program.xlsx} - A szervezeti
  elkötelezettség, a szervezeti kultúra és az elégedettség eltér a
  vállalat 3 különböző vezetési irányelvét valló egységében? Az
  adatbázis Sajtos és Mitev (2007, o. 332) könyvéből származik.\\
  Egy vállalat menedzsmentje szeretné megvizsgálni különböző vezetési
  programok hatását, ezért három különböző vezetési programot vezetett
  be három különböző stratégiai üzleti egységben (SÜE). Az első SÜE-ben
  bevezetett program az egyenlőséget és az individualizmust
  hangsúlyozta. A második SÜE-ben az egyenlőséget és a csoportmunkát
  helyzeték középpontba. A harmadik SÜE-ben a bevezetett program egy
  nagyon hierarchikus vezetési elvet alkalmazott. Később mindhárom SÜE
  dolgozóinak körében felmérést végeztek, és a kérdések között szerepelt
  a szervezettel való elkötelezettség mértéke (\texttt{szelkot}), a
  szervezettel való elégedettség nagysága (\texttt{elegedett}), illetve
  a rendszer egalitárius vagy tekintélyelvű (autokrata) jellege
  (\texttt{rendszer}).

  \begin{itemize}
  \item
    Az adatbázis szerkezete:

    \begin{itemize}
    \tightlist
    \item
      \texttt{SUE} - a három különböző vezetési irányelvet követő
      stratégiai üzleti egység (nominális: 1-3)
    \item
      \texttt{szelkot} - szervezet elkötelezettség mértéke (likert: 1-5)
    \item
      \texttt{elegedett} - szervezettel való elégedettség (likert: 1-5)
    \item
      \texttt{rendszer} - szervezet tekintélyelvű jellege (likert: 1-5)
    \end{itemize}
  \item
    Kapcsolódó állományok:

    \begin{itemize}
    \tightlist
    \item
      \texttt{manova\_vezetesi\_program.omv} - Többváltozós
      varianciaelemzés jamovi-ban
    \end{itemize}
  \end{itemize}
\end{itemize}

\hypertarget{diszkriminancia-elemzuxe9s}{%
\section{Diszkriminancia elemzés}\label{diszkriminancia-elemzuxe9s}}

\begin{itemize}
\item
  \texttt{diszkriminancia\_alkalmassag.xlsx} - szalagmunkások adatai
  (Münnich és mtsai. (2006), 4.1. táblázat)

  \begin{itemize}
  \item
    Az adatbázis szerkezete:

    \begin{itemize}
    \tightlist
    \item
      \texttt{bevalt} - a munkás beválásával kapcsolatos információ:
      bevált? (nominális: ``igen'', ``nem'')
    \item
      \texttt{figyelem} - a munkás figyelmi képessége (likert: 1-7, a
      magasabb érték jobb képességeket jelent)
    \item
      \texttt{monotonia\_tures} - a munkás monotónia tűrése (likert:
      1-7, a magasabb érték jobb képességeket jelent)
    \end{itemize}
  \end{itemize}
\item
  \texttt{diszkriminancia\_baleset.xlsx} - mely tényezők járulnak hozzá
  a balesetekhez (Münnich és mtsai. (2006), 4.11. R-forráskód)

  \begin{itemize}
  \item
    Az adatbázis szerkezete:

    \begin{itemize}
    \tightlist
    \item
      \texttt{baleset} - volt már balesete a személynek vagy sem
      (nominális ``nem volt balesete'', ``volt baleste'')
    \item
      \texttt{megosztott} - megosztott figyelem (intervallum/arány)
    \item
      \texttt{pontossag} - a figyelem pontossága (intervallum/arány)
    \item
      \texttt{kockazat} - kockázatvállalási hajlandóság
      (intervallum/arány)
    \item
      \texttt{eszleles} - észlelés gyorsasága (intervallum/arány)
    \end{itemize}
  \end{itemize}
\item
  \texttt{diszkriminancia\_depresszio.xlsx} - a postpartum depresszió
  pszichés és szociális háttere (Münnich és mtsai. (2006), 4.16.
  R-forráskód)

  \begin{itemize}
  \item
    Az adatbázis szerkezete:

    \begin{itemize}
    \tightlist
    \item
      \texttt{ppdepresszio} - szülés utáni depresszió jelenléte
      (nominális: ``nincs depresszió'', ``van depresszió'')
    \item
      \texttt{szeretet} - a személyek mennyire érzik, hogy a szüleik
      szeretik őket (intervallum/arány)
    \item
      \texttt{tulvedes} - mennyire hajlamosak arra a személyek, hogy
      túlságosan is burokban tartsák, túlvédjék gyerekeiket
      (intervallum/arány)
    \item
      \texttt{kor} - életkor (intervallum/arány)
    \item
      \texttt{iskola} - az elvégzett iskolai osztályok száma
      (intervallum/arány)
    \end{itemize}
  \end{itemize}
\item
  \texttt{diszkriminancia\_pszichoszomatika.xlsx} - a pszichoszomatikus
  megbetegedéseket vizsgálata (Münnich és mtsai. (2006), 4.21.
  R-forráskód)

  \begin{itemize}
  \tightlist
  \item
    Az adatbázis szerkezete:

    \begin{itemize}
    \tightlist
    \item
      \texttt{pszichoszomatika} - van valamilyen pszichoszomatikus
      megbetegedése a személynek? (nominális: ``szichoszomatikus
      megbetegedése van'', '' egészséges'')
    \item
      \texttt{stressz} - a személyt ért stressz mértéke
      (intervallum/arány)
    \item
      \texttt{szorongas} - a szorongási szintje (intervallum/arány)
    \item
      \texttt{coping} - a megküzdési stratégiáinak hatékonysága
      (intervallum/arány)
    \end{itemize}
  \end{itemize}
\item
  \texttt{diszkriminancia\_bio.xlsx} - kik vásárolnak bio termékeket
  (Münnich és mtsai. (2006), 4.26. R-forráskód)

  \begin{itemize}
  \item
    Az adatbázis szerkezete:

    \begin{itemize}
    \tightlist
    \item
      \texttt{vasarlas} - a biotermékek vásárlásának gyakorisága
      (ordinális: ``soha nem vesz'', ``időnként vesz'', ``gyakran
      vesz'')
    \item
      \texttt{ertek} - minél nagyobb pontszámot kap a skálán, annál
      jobban értékeli a személy a bio termékeket (intervallum/arány)
    \item
      \texttt{attitud} - a magasabb értékek kedvezőbb atttitűdöt jelez a
      biotermékek iránt (intervallum/arány)
    \item
      \texttt{fizetes} - a személy fizetésének nagysága
      (intervallum/arány)
    \item
      \texttt{kor} - a személy életkora(intervallum/arány)
    \end{itemize}
  \end{itemize}
\item
  \texttt{diszkriminancia\_vezetesi\_program.xlsx} - A szervezeti
  elkötelezettség, a szervezeti kultúra és az elégedettség alapján
  szétválasztható a vállalat 3 különböző vezetési irányelvét valló
  egysége? Az adatbázis Sajtos és Mitev (2007, o. 332) könyvéből
  származik.\\
  Egy vállalat menedzsmentje szeretné megvizsgálni különböző vezetési
  programok hatását, ezért három különböző vezetési programot vezetett
  be három különböző stratégiai üzleti egységben (SÜE). Az első SÜE-ben
  bevezetett program az egyenlőséget és az individualizmust
  hangsúlyozta. A második SÜE-ben az egyenlőséget és a csoportmunkát
  helyzeték középpontba. A harmadik SÜE-ben a bevezetett program egy
  nagyon hierarchikus vezetési elvet alkalmazott. Később mindhárom SÜE
  dolgozóinak körében felmérést végeztek, és a kérdések között szerepelt
  a szervezettel való elkötelezettség mértéke (\texttt{szelkot}), a
  szervezettel való elégedettség nagysága (\texttt{elegedett}), illetve
  a rendszer egalitárius vagy tekintélyelvű (autokrata) jellege
  (\texttt{rendszer}).

  \begin{itemize}
  \tightlist
  \item
    Az adatbázis szerkezete:

    \begin{itemize}
    \tightlist
    \item
      \texttt{SUE} - a három különböző vezetési irányelvet követő
      stratégiai üzleti egység (nominális: 1-3)
    \item
      \texttt{szelkot} - szervezet elkötelezettség mértéke (likert: 1-5)
    \item
      \texttt{elegedett} - szervezettel való elégedettség (likert: 1-5)
    \item
      \texttt{rendszer} - szervezet tekintélyelvű jellege (likert: 1-5)
    \end{itemize}
  \item
    Kapcsolódó állományok:

    \begin{itemize}
    \tightlist
    \item
      \texttt{diszkriminancia\_vezetesi\_program.omv} - diszkriminancia
      elemzés jamovi-ban
    \end{itemize}
  \end{itemize}
\end{itemize}

\hypertarget{lineuxe1ris-regressziuxf3}{%
\section{Lineáris regresszió}\label{lineuxe1ris-regressziuxf3}}

\begin{itemize}
\tightlist
\item
  \texttt{lin\_reg\_fizetes\_elegedettseg\_01.omv} - konstans
  oszlopokkal nem tudunk számolni (Münnich és mtsai. (2006) 1.1/A
  táblázat)
\item
  \texttt{lin\_reg\_fizetes\_elegedettseg\_02.omv}- az adatpontok szinte
  tökéletesen az egyenesre illeszkednek (Münnich és mtsai. (2006) 1.1/B
  táblázat)
\item
  \texttt{lin\_reg\_kapcsolatok\_01.omv} - nem szisztematikus kapcsolat
  két változó között (Münnich és mtsai. (2006) 1.5. R-forráskód)
\item
  \texttt{lin\_reg\_kapcsolatok\_02.omv} - szisztematikus
  (függvényszerű) kapcsolat két változó között (Münnich és mtsai. (2006)
  1.6. R-forráskód)
\item
  \texttt{lin\_reg\_kapcsolatok\_03.omv} - szisztematikus
  (függvényszerű) kapcsolat két változó között (Münnich és mtsai. (2006)
  1.7. R-forráskód)
\item
  \texttt{lin\_reg\_kapcsolatok\_04.omv} - szisztematikus
  (függvényszerű) kapcsolat két változó között (Münnich és mtsai. (2006)
  1.8. R-forráskód)
\item
  \texttt{lin\_reg\_kapcsolatok\_05.omv} - szisztematikus
  (függvényszerű) kapcsolat két változó között (Münnich és mtsai. (2006)
  1.9. R-forráskód)
\item
  \texttt{lin\_reg\_elegedttseg.omv} - a fizetés és a munkahellyel való
  elégedettség pontdiagramja, egyszerű lineáris regresszió (Münnich és
  mtsai. (2006) 1.10 R-forráskód)
\item
  \texttt{lin\_reg\_fizetes\_eletkor\_eledettseg\_01.omv} - többszörös
  lineáris regresszió, 2 numerikus magyarázó változó (Münnich és mtsai.
  (2006) 1.2. táblázat)
\item
  \texttt{lin\_reg\_intelligencia\_testmagassag\_eletkor\_01.omv} -
  többszörös lineáris regresszió, 2 numerikus magyarázó változó,
  parciális korreláció magyarázata

  \begin{itemize}
  \tightlist
  \item
    minél magasabb valaki, annál intelligensebb
  \item
    ha bevonjuk az életkor változót, akkor eltűnik az intelligencia és a
    testmagasság közötti kapcsolat
  \end{itemize}
\item
  \texttt{lin\_tizproba.omv} - többszörös lineáris regresszió, a legjobb
  modell keresése, sok numerikus magyarázó változó
\item
  \texttt{lin\_college\_success\_02.omv} - többszörös lineáris
  regresszió, sok numerikus magyarázó változó, GPA a függő változó, mi
  magyarázza az egyetemi teljesítményt
\item
  \texttt{lin\_reg\_elegedttseg\_02.omv} - A férfiak vagy a nők
  elégedettebbek a munkahelyükkel? (Münnich és mtsai. (2006) 1.6.3
  probléma), egyetlen kategorikus magyarázó változó 2 értékkel (nem:
  férfi, nő)

  \begin{itemize}
  \tightlist
  \item
    kapcsolat a kétmintás t-próbával
  \end{itemize}
\item
  \texttt{lin\_reg\_magassag\_hajhossz\_nem\_01.omv} - többszörös
  lineáris regresszió, parciális korreláció 1 numerikus és 1 kategorikus
  változóval (Münnich és mtsai. (2006) 1.2. táblázat)

  \begin{itemize}
  \tightlist
  \item
    a testmagasság és a hajhossz között kapcsolat van
  \item
    ha a személyek nemét is figyelembe vesszük, egyáltalán nincs
    kapcsolat a testmagasság és a hajhosszúság között
  \end{itemize}
\item
  \texttt{lin\_auction.omv} - többszörös lineáris regresszió, Simpson
  paradoxon, párhuzamos regresszió, majd interakció bevonása.
\end{itemize}

\hypertarget{fux151komponens-elemzuxe9s}{%
\section{Főkomponens elemzés}\label{fux151komponens-elemzuxe9s}}

\begin{itemize}
\tightlist
\item
  \texttt{fokomp\_elemzes\_tantargyak.omv} - 1 főkomponens létrehozása
  (Münnich és mtsai. (2006) 2.2. táblázat)
\item
  \texttt{fokomp\_real\_targyak.omv} - példa kidolgozása, 1
  főkomponens(Münnich és mtsai. (2006) 2.5.1 Probléma)
\item
  \texttt{fokomp\_kerdoivtervezet.omv} - példa kidolgozása (Münnich és
  mtsai. (2006) 2.5.2 Probléma)
\item
  \texttt{fokomp\_munkahelyi\_tolarencia.omv} - példa kidolgozása
  (Münnich és mtsai. (2006) 2.5.3 Probléma)
\item
  \texttt{fokomp\_munkahelyi\_elegedettseg.omv} - példa kidolgozása
  (Münnich és mtsai. (2006) 2.5.4 Probléma)
\end{itemize}

\hypertarget{faktorelemzuxe9s}{%
\section{Faktorelemzés}\label{faktorelemzuxe9s}}

\begin{itemize}
\tightlist
\item
  \texttt{faktor\_szorongas.omv} - példa (Münnich és mtsai. (2006) 3.1.
  R-forráskód)
\item
  \texttt{faktor\_real\_human\_targyak.omv} - példa (Münnich és mtsai.
  (2006) 3.9. R-forráskód)
\item
  \texttt{faktor\_bigfive.omv} - példa (Münnich és mtsai. (2006) 3.21.
  R-forráskód)
\item
  \texttt{faktor\_kockazat.omv} - példa (Münnich és mtsai. (2006) 3.7.4
  Probléma)
\item
  \texttt{faktor\_fogkrem.omv} - példa (Malhotra és Simon (2008) 617.
  oldal)
\end{itemize}

\hypertarget{feltuxe1ruxf3-faktorelemzuxe9s}{%
\section{Feltáró faktorelemzés}\label{feltuxe1ruxf3-faktorelemzuxe9s}}

\begin{itemize}
\item
  \texttt{faktor\_fogkrem.xlsx} - A kutató arra volt kíváncsi, milyen
  előnyöket keresnek a fogyasztók a fogrémvásárlásnál. Egy 30 fős mintán
  a válaszadókat arra kérték, hogy jelezzék, mennyire értenek egyet a
  következő állításokkal (1 = egyáltalán nem ért egyet; 7 = teljes
  mértékben egyetért)

  \begin{itemize}
  \item
    Az adatbázis szerkezete:

    \begin{itemize}
    \tightlist
    \item
      \texttt{sorszam}: válaszadó sorszáma (id)
    \item
      \texttt{v1}: Fontos, hogy olyan fogkrémet vásároljak, amellyel
      megelőzhető a fogszuvasodás. (likert: 1-7)
    \item
      \texttt{v2}: Az olyan fogkrémeket szeretem, amely fényessé teszi a
      fogaimat. (likert: 1-7)
    \item
      \texttt{v3}: Egy fogkrémnek erősítenie kell a fogínyt. (likert:
      1-7)
    \item
      \texttt{v4}: Az olyan fogkrémeket szeretem, amely friss leheletet
      biztosít. (likert: 1-7)
    \item
      \texttt{v5}: A fog romlásának megelőzése számomra nem fontos
      elvárás. (likert: 1-7)
    \item
      \texttt{v6}: A legfontosabb szempont a fogkrém vásárlásánál a szép
      fog. (likert: 1-7)
    \end{itemize}
  \item
    Kapcsolódó állományok:

    \begin{itemize}
    \tightlist
    \item
      \texttt{efa\_fogkrem.omv} - Feltáró faktorelemzés jamovi-ban
    \end{itemize}
  \end{itemize}
\end{itemize}

\hypertarget{sec-gyakorlo-feladatok}{%
\chapter{Gyakorló feladatok}\label{sec-gyakorlo-feladatok}}

\hypertarget{lineuxe1ris-regressziuxf3-1}{%
\chapter{Lineáris regresszió}\label{lineuxe1ris-regressziuxf3-1}}

\hypertarget{feladat-a-rekluxe1m-hatuxe1sa}{%
\section{Feladat: A reklám hatása}\label{feladat-a-rekluxe1m-hatuxe1sa}}

A feladat forrása: Malhotra és Simon (2008, o. 579)

Egy nagy szupermarketlánc meg akarja határozni a reklám hatását az
egymáshoz viszonyított versenyképességre. 15 állam reklámköltség-
(versenytárs költsége = 100) és értékesítési adatait (versenytárs
értékesítése = 100) kaptuk meg a fő versenytárshoz viszonyítva. Ön azt a
feladatot kapta, hogy választ kell adnia a menedzser kérdésére, van-e
összefüggés a reklámköltségek és az értékesítés között. Az adatok a
\texttt{lin\_reg\_reklam\_hatasa.sav} állományban találhatók.

\begin{enumerate}
\def\labelenumi{\arabic{enumi}.}
\item
  Ábrázolja pontdiagramon a relatív értékesítést (y tengely) és a
  relatív reklámköltséget (x tengely), és értelmezze a diagramot!
\item
  Milyen mutatószámot használna annak megállapítására, hogy van-e
  összefüggés a két változó között? Miért?
\item
  Készítsen egyszerű lineáris regressziót a relatív értékesítés és a
  relatív reklámköltség között!
\item
  Értelmezze a regressziós együtthatókat!
\item
  Szignifikáns a regressziós összefüggés?
\item
  Ha a vállalat a versenytárshoz viszonyítva ugyanannyi pénzt költene
  reklámra (ha a relatív reklámköltség 100 volt), mekkora lenne a
  vállalat relatív értékesítési szintje?
\item
  Értelmezze a kapott \(R^2\)-et?
\end{enumerate}

\hypertarget{feladat-illatszerboltok}{%
\section{Feladat: Illatszerboltok}\label{feladat-illatszerboltok}}

A feladat forrása: Malhotra és Simon (2008, o. 579)

Annak megértése érdekében, hogy a minőség és az ár hogyan befolyásolja
az illatszerboltok törzsvásárlóit, a válaszadókat arra kértük, hogy egy
nagyváros 14 üzletét értékeljék a következő szempontok szerint:

\begin{itemize}
\tightlist
\item
  a bolt iránti preferencia
\item
  az áru minősége és az
\item
  elfogadható ár.
\end{itemize}

Minden értékelést 11 fokozatú skálán végeztek, ahol a magasabb számok
pozitívabb értékítéletet jelentettek. Az adatok a
\texttt{lin\_reg\_illatszerbolt.sav} állományban vannak.

\begin{enumerate}
\def\labelenumi{\arabic{enumi}.}
\item
  Futtasson többváltozós regressziót a bolt iránti preferencia
  vizsgálatára az áru minősége és ára tekintetében!
\item
  Értelmezze a parciális regressziós együtthatót!
\item
  Határozza meg a teljes regresszió szignifikanciáját!
\item
  Határozza meg a parciális regressziós együttható szignifikanciáját!
\item
  A multikollinearitás probléma lehet ebben az esetben? Miért vagy miért
  nem?
\item
  Az egy vagy a két magyarázó változót tartalmazó modell az optimális?
  Miért?
\end{enumerate}

\hypertarget{fux151komponens-elemzuxe9s-1}{%
\chapter{Főkomponens elemzés}\label{fux151komponens-elemzuxe9s-1}}

A feladat forrása: Malhotra és Simon (2008, o. 579)

Egy kutatásban, amely háztartások fogyasztói magatartását vizsgálta, a
következő életstílus-állításokat értékelték hétfokú skálán (1 =
egyetért, 7 = nem ért egyet).

\begin{itemize}
\tightlist
\item
  V1: Inkább eltöltök csendben egy estét otthon, mint hogy szórakozni
  menjek.
\item
  V2: Mindig megnézem az árakat, még a kis értékű árucikkeknél is.
\item
  V3: A magazinok érdekesebbek, mint a mozi.
\item
  V4: Nem vásárolok plakáton hirdetett termékeket.
\item
  V5: Otthon ülő típus vagyok.
\item
  V6: A kuponokat beváltom.
\item
  V7: A vállalatok sok pénzt kidobnak a reklámmal.
\end{itemize}

Egy előtesztben 25 fogyasztót kérdeztek meg, ennek adatai láthatók a
\texttt{faktor\_eletstilus.sav} adatbázisban.

\begin{enumerate}
\def\labelenumi{\arabic{enumi}.}
\tightlist
\item
  Elemezze az adatokat varimax forgatáson alapuló főkomponens
  elemzéssel.
\item
  Értelmezze a kapott fakort/faktorokat!
\item
  Számítsa ki a faktorértéket minden egyes válaszadóra!
\item
  Ha helyettesítő változókat kellene választanunk, melyek lennének azok?
\item
  Vizsgálja meg a modell illeszkedését?
\item
  Vizsgálja meg az alkalmazási feltételeket!
\end{enumerate}

\hypertarget{megbuxedzhatuxf3suxe1g-vizsguxe1lat}{%
\chapter{Megbízhatóság
vizsgálat}\label{megbuxedzhatuxf3suxe1g-vizsguxe1lat}}

A feladat forrása:
\href{https://statisticsbyjim.com/basics/cronbachs-alpha/}{Statistics By
Jim}

A HERTDAQ Bank felméri, hogy az ügyfelei mennyire elégedettek a
szolgáltatások gyorsaságával. A következő négy kérdést dolgozza ki:

\begin{itemize}
\item
  \begin{enumerate}
  \def\labelenumi{\arabic{enumi}.}
  \tightlist
  \item
    item: Telefonos, e-mailes vagy leveles megkeresésemre ésszerű időn
    belül válaszoltak.
  \end{enumerate}
\item
  \begin{enumerate}
  \def\labelenumi{\arabic{enumi}.}
  \setcounter{enumi}{1}
  \tightlist
  \item
    item: Elégedett vagyok a nyújtott szolgáltatás gyorsaságával.
  \end{enumerate}
\item
  \begin{enumerate}
  \def\labelenumi{\arabic{enumi}.}
  \setcounter{enumi}{2}
  \tightlist
  \item
    item: A szolgáltatásokra való várakozás ideje ésszerű volt.
  \end{enumerate}
\item
  \begin{enumerate}
  \def\labelenumi{\arabic{enumi}.}
  \setcounter{enumi}{3}
  \tightlist
  \item
    item: Elégedett vagyok a kapott szolgáltatásokkal.
  \end{enumerate}
\end{itemize}

A fenti itemekre egy 5 fokozatú Likert-skála segítségével lehetett
válaszolni, amelyen 1-től (nagyon nem értek egyet) és 5-ig (nagyon
egyetértek) tartó értékek közül lehet választani. Összesen 60 ügyfelet
kértek fel a felmérés kitöltésére az előzetes vizsgálati szakaszban, még
mielőtt a felmérést szélesebb körben elkezdenék terjeszteni.

\begin{enumerate}
\def\labelenumi{\arabic{enumi}.}
\tightlist
\item
  Végezze el a ``szolgáltatás gyorsasága'' skála megbízhatóság
  vizsgálatát Cronbach-alfa és főkomponens analízis segítségével!
\item
  Javasolja tételek kihagyását? Miért?
\end{enumerate}

\hypertarget{feltuxe1ruxf3-faktorelemzuxe9s-1}{%
\chapter{Feltáró faktorelemzés}\label{feltuxe1ruxf3-faktorelemzuxe9s-1}}

A feladat forrása: Malhotra és Simon (2008, o. 579)

Egy kutatásban, amely háztartások fogyasztói magatartását vizsgálta, a
következő életstílus-állításokat értékelték hétfokú skálán (1 =
egyetért, 7 = nem ért egyet).

\begin{itemize}
\tightlist
\item
  V1: Inkább eltöltök csendben egy estét otthon, mint hogy szórakozni
  menjek.
\item
  V2: Mindig megnézem az árakat, még a kis értékű árucikkeknél is.
\item
  V3: A magazinok érdekesebbek, mint a mozi.
\item
  V4: Nem vásárolok plakáton hirdetett termékeket.
\item
  V5: Otthon ülő típus vagyok.
\item
  V6: A kuponokat beváltom.
\item
  V7: A vállalatok sok pénzt kidobnak a reklámmal.
\end{itemize}

Egy előtesztben 25 fogyasztót kérdeztek meg, ennek adatai láthatók a
\texttt{faktor\_eletstilus.sav} adatbázisban.

\begin{enumerate}
\def\labelenumi{\arabic{enumi}.}
\tightlist
\item
  Elemezze az adatokat Oblimin forgatáson alapuló feltáró
  faktorelemzéssel.
\item
  Értelmezze a kapott fakort/faktorokat!
\item
  Számítsa ki a faktorértéket minden egyes válaszadóra!
\item
  Ha helyettesítő változókat kellene választanunk, melyek lennének azok?
\item
  Vizsgálja meg a modell illeszkedését?
\item
  Vizsgálja meg az alkalmazási feltételeket!
\end{enumerate}

\hypertarget{megerux151suxedtux151-faktorelemzuxe9s}{%
\chapter{Megerősítő
faktorelemzés}\label{megerux151suxedtux151-faktorelemzuxe9s}}

A feladat forrása: Abraham és mtsai. (2020). Az eredeti
\href{https://ars.els-cdn.com/content/image/1-s2.0-S2352340920312993-mmc1.docx}{kérdőív}
és
\href{https://ars.els-cdn.com/content/image/1-s2.0-S2352340920312993-mmc2.xlsx}{adatbázis}
is letölthető.

A kutatók indonéz fiatalok körében szeretnék vizsgálni a digitális
írástudást. Összeállítanak egy kérdőívet, amely összesen 40 kérdést
(itemet) tartalmaz. A kutatók szerint az itemek 6 skála (faktor) köré
csoportosíthatók. A következő listában először a 6 skálát, majd a hozzá
tartozó itemeket nevezzük meg:

\begin{itemize}
\tightlist
\item
  SF skála - A szükséges információk hatékony keresése és megtalálása

  \begin{itemize}
  \tightlist
  \item
    Itemek: SF1 SF2 SF3
  \end{itemize}
\item
  PC A rosszindulatú és redundáns tartalom elleni védekezés

  \begin{itemize}
  \tightlist
  \item
    Itemek: PC1 PC2 PC3 PC4 PC5 PC6 PC7 PC8 PC9
  \end{itemize}
\item
  VI Az információk ellenőrzése és kritikus értékelése alternatív
  információforrások felhasználásával

  \begin{itemize}
  \tightlist
  \item
    Itemek: VI1 VI2 VI3
  \end{itemize}
\item
  PU Az információ megfelelő észlelése és hatékony felhasználása

  \begin{itemize}
  \tightlist
  \item
    Itemek: PU1 PU2 PU3 PU4 PU5 PU6 PU7 PU8 PU9
  \end{itemize}
\item
  CD Az információ hatékony és helyes terjesztése

  \begin{itemize}
  \tightlist
  \item
    Itemek: CD1 CD2 CD3 CD4 CD5
  \end{itemize}
\item
  SS Speciális készségek, képesség új média használatára, internet
  szolgáltatások és technikai eszközök használatára

  \begin{itemize}
  \tightlist
  \item
    Itemek: SS1 SS2 SS3 SS4 SS5 SS6 SS7 SS8 SS9 SS10 SS11
  \end{itemize}
\end{itemize}

Végezzünk megerősítő faktorelemzést, ellenőrizzük le, hogy az adataink
jól illeszkednek a fenti faktorstruktúrára! Az adatok a
\texttt{faktor\_digitalis\_irastudas.xlsx} állományban találhatók.

\begin{enumerate}
\def\labelenumi{\arabic{enumi}.}
\tightlist
\item
  Értékeljük az egyes itemek illeszkedését a faktorokra!
\item
  Értékeljük a faktorok kapcsolatát!
\item
  Értékeljük az illeszkedési mutatókat!
\end{enumerate}

\hypertarget{klaszterelemzuxe9s}{%
\chapter{Klaszterelemzés}\label{klaszterelemzuxe9s}}

A \texttt{nike.xlsx} adatain végezze el a következő elemzéseket. Csak a
következő változókat vegye figyelembe: tudatosság, attitúd, preferencia,
cél és a Nike iránti hűség.

\begin{enumerate}
\def\labelenumi{\arabic{enumi}.}
\tightlist
\item
  Hierarchikus klaszterelemzés segítségével rendezze a válaszadókat
  klaszterekbe a feltárt változók alapján! Használja a Ward-féle
  eljárást és az euklideszi távolságot. Hány klasztert javasol és
  miért?\\
\item
  Rendezze a válaszadókat klaszterekbe a feltárt változók alapján a
  következő hierarchikus módszerekkel, majd határozza meg a klasztereket
  az egyes eljárások során és hasonlítsa össze a kapott
  klaszterstruktúrákat az 1. feladat klaszterstruktúrájával:

  \begin{itemize}
  \tightlist
  \item
    legközelebbi szomszéd (egyszerű lánc)
  \item
    legtávolabbi szomszéd (teljes lánc)
  \item
    átlagos lánc.
  \end{itemize}
\item
  K-középérték klasztereljárás és valamely optimális
  klaszterszámmeghatározó eljárással kapott klaszterszám segítségével
  rendezze a válaszadókat klaszterekbe a feltárt változók alapján.
  Hasonlítsa össze az eredményt az 1. feladatban kapott eredménnyel is.
\end{enumerate}

\hypertarget{diszkriminancia-elemzuxe9s-uxe9s-tuxf6bbvuxe1ltozuxf3s-varianciaelemzuxe9st}{%
\chapter{Diszkriminancia elemzés és többváltozós
varianciaelemzést}\label{diszkriminancia-elemzuxe9s-uxe9s-tuxf6bbvuxe1ltozuxf3s-varianciaelemzuxe9st}}

A \texttt{nike.xlsx} adatain végezze el a következő elemzéseket. A
használt oszlopok a következőek:

\begin{itemize}
\tightlist
\item
  \texttt{termekhasznalat} - kategorikus változó 3 csoporttal: 1 =
  ritkán használja; 2 = közepes; 3 = gyakran használja a terméket.
\item
  \texttt{tudatossag} - tudatosság (1-7 skálán: 1 = nagyon kedvezőtlen,
  7 = nagyon kedvező)
\item
  \texttt{attitud} - attitűd (1-7 skálán: 1 = nagyon kedvezőtlen, 7 =
  nagyon kedvező)
\item
  \texttt{preferencia} - preferencia (1-7 skálán: 1 = nagyon
  kedvezőtlen, 7 = nagyon kedvező)
\item
  \texttt{cel} - szandék, cél (1-7 skálán: 1 = nagyon kedvezőtlen, 7 =
  nagyon kedvező)
\item
  \texttt{huseg} - Nike iránti lojalitás (1-7 skálán: 1 = nagyon
  kedvezőtlen, 7 = nagyon kedvező)
\end{itemize}

Van különbség a három fogyasztói csoport (\texttt{termekhasznalat})
között a tudatosság, attitűd, preferencia, szándék és a Nike iránti
lojalitás szempontjából?

\begin{enumerate}
\def\labelenumi{\arabic{enumi}.}
\item
  Végezzünk többváltozós varianciaelemzést a numerikus változók
  várható-érték vektorainak szétválasztására a 3 csoport mentén.
  Szignifikáns a próba? Ha kell végezzünk utóvizsgálatot egy
  tetszőlegesen választott módszerrel.
\item
  A termékhasználat 3 csoportját az 5 numerikus változó milyen lineáris
  kombinációja segítségével tudjuk a legjobban szétválasztani? Adjuk meg
  ezeket az együtthatókat, valamint validáljuk az eljárás! Adjunk egy
  tetszőleges értékkombinációra előrejelzést!
\end{enumerate}

\hypertarget{irodalomjegyzuxe9k}{%
\chapter{Irodalomjegyzék}\label{irodalomjegyzuxe9k}}

\hypertarget{refs}{}
\begin{CSLReferences}{1}{0}
\leavevmode\vadjust pre{\hypertarget{ref-Abrahametal2020}{}}%
Abraham, J., Ali, M. M., Andangsari, E. W. és Hartanti, L. E. P. (2020).
Confirmatory factor analysis of celebrity worship, digital literacy, and
nostalgia: Dataset of Indonesians. \emph{Data in Brief}, \emph{33},
106417. \url{https://doi.org/10.1016/j.dib.2020.106417}

\leavevmode\vadjust pre{\hypertarget{ref-CarverScheier2006}{}}%
Carver, C. S. és Scheier, M. F. (2006). \emph{Személyiségpszichológia}.
Osiris Kiadó.

\leavevmode\vadjust pre{\hypertarget{ref-Csallner2015}{}}%
Csallner, A. E. (2015). \emph{Bevezetés az SPSS statisztikai
programcsomag használatába}.
\url{http://www.jgypk.hu/tamop15e/tananyag_html/spss/index.html}

\leavevmode\vadjust pre{\hypertarget{ref-Horn1965}{}}%
Horn, J. L. (1965). A rationale and test for the number of factors in
factor analysis. \emph{Psychometrika}, \emph{30}, 179--185.
\url{https://doi.org/10.1007/BF02289447}

\leavevmode\vadjust pre{\hypertarget{ref-Karasz2022}{}}%
Kárász, J. T., Nagy, O. N., Széll, K. és Takács, S. (2022).
Cronbach-alfa: vele vagy nélküle? \emph{Magyar Pszichológiai Szemle},
\emph{77}, 81--98. \url{https://doi.org/10.1556/0016.2022.00004}

\leavevmode\vadjust pre{\hypertarget{ref-KetskemetyIzso2005}{}}%
Ketskeméty, L. és Izsó, L. (2005). \emph{Bevezetés az SPSS
programrendszerbe - Módszertani útmutató és feladatgyűjtemény
statisztikai elemzésekhez}. ELTE Eötvös Kiadó.

\leavevmode\vadjust pre{\hypertarget{ref-MalhotraSimon2008}{}}%
Malhotra, N. K. és Simon, J. (2008). \emph{Marketingkutatás}. Akadémiai
Kiadó.

\leavevmode\vadjust pre{\hypertarget{ref-Malkewitz2023}{}}%
Malkewitz, C. P., Schwall, P., Meesters, C. és Hardt, J. (2023).
Estimating reliability: A comparison of Cronbach's α, McDonald's ωt and
the greatest lower bound. \emph{Social Sciences \& Humanities Open},
\emph{7}, 100368. \url{https://doi.org/10.1016/j.ssaho.2022.100368}

\leavevmode\vadjust pre{\hypertarget{ref-Moksony2006}{}}%
Moksony, F. (2006). \emph{Gondolatok és adatok. Társadalomtudományi
elméletek empirikus ellenőrzése}. Aula Kiadó.

\leavevmode\vadjust pre{\hypertarget{ref-MunnichNagyAbari2006}{}}%
Münnich, Á., Nagy, Á. és Abari, K. (2006). \emph{Többváltozós
statisztika pszichológus hallgatók számára}. Bölcsész Konzorcium.
\url{http://psycho.unideb.hu/statisztika}

\leavevmode\vadjust pre{\hypertarget{ref-Nagybanyai2006}{}}%
Nagy, O. N. (2006). A pszichológiai tesztek reliabilitása. In S. Rózsa,
O. N. Nagy, és A. Oláh (Szerk.), \emph{A pszichológiai mérés alapjai.
Elmélet, módszer és gyakorlati alkalmazás}. Bölcsész Konzorcium.
\url{https://mek.oszk.hu/05500/05536/05536.pdf}

\leavevmode\vadjust pre{\hypertarget{ref-Rozsa2019}{}}%
Rózsa, S., Hupuczi, E., Martin, L., Birkás, B., Hartung, I., Hargitai,
R., Varga, J., Láng, A., Tiringer, I. és Kállai, J. (2019). A Tellegen
Abszorpciós Skála részletes pszichometriai elemzése. \emph{Mentálhigiéné
és Pszichoszomatika}, \emph{20}, 35--77.
\url{https://doi.org/10.1556/0406.20.2019.003}

\leavevmode\vadjust pre{\hypertarget{ref-SajtosMitev2007}{}}%
Sajtos, L. és Mitev, A. (2007). \emph{SPSS kutatási és adatelemzési
kézikönyv}. Alinea Kiadó.

\leavevmode\vadjust pre{\hypertarget{ref-SzekelyiBarna2002}{}}%
Székelyi, M. és Barna, I. (2002). \emph{Túlélőkészlet az SPSS-hez.
Többváltozós elemzési technikáról társadalomkutatók számára}. Typotex
Kiadó.

\leavevmode\vadjust pre{\hypertarget{ref-TakacsBev22017}{}}%
Takács, S. (2017). \emph{Bevezetés a matematikai statisztikába 2.
Többváltozós statisztikai módszerek}. Antarész Kiadó.

\leavevmode\vadjust pre{\hypertarget{ref-Varga2019}{}}%
Varga, A. (2019). \emph{Többváltozós statisztika dióhéjban:
Változó-orientált módszerek}. Pólya Kiadó.

\leavevmode\vadjust pre{\hypertarget{ref-Watkins2018}{}}%
Watkins, M. W. (2018). Exploratory Factor Analysis: A Guide to Best
Practice. \emph{Journal of Black Psychology}, \emph{44}, 219--246.
\url{https://doi.org/10.1177/0095798418771807}

\end{CSLReferences}



\backmatter
\printindex

\end{document}
